{{template "preamble.tex"}} % chktex 18
\begin{document}

\begin{center}
	\begin{tabu}{|X[2]|X|}\tabucline{}
		Vertragsnummer: {{.Contract.ID}} & Datum: \iffalse input fields.date value="{{.Contract.Fields.date}}" type="date" \fi {{.Contract.Fields.date}} \\\tabucline{} %chktex 26
	\end{tabu}
\end{center}

\section{ENERGIEEINSPARUNGSVERTRAG}

\textbf{Der Auftraggeber:}
\begin{center}
	\begin{tabu}{|X|X[2]|}\tabucline{}
		Vorname, Nachname & {{.Contract.Fields.client_name}} \iffalse input fields.client_name value="{{.Contract.Fields.client_name}}" \fi \\\tabucline{}
		Registrierungsnummer / persönliche Identifikationsnummer & {{.Contract.Fields.client_id}} \iffalse input fields.client_id value="{{.Contract.Fields.client_id}}" \fi \\\tabucline{}
		Adresse & {{.Contract.Fields.client_address}} \iffalse input fields.client_address value="{{.Asset.Address}}" \fi \\\tabucline{}
	\end{tabu}
\end{center}

\textbf{Der Auftragnehmer:}
\begin{center}
	\begin{tabu}{|X|X[2]|}\tabucline{}
		Name                    		& {{.ESCo.Name}} \\\tabucline{}
		Registrierungsnummer    		& {{.ESCo.VAT}} \\\tabucline{}
		Umsatzsteuer-Identifikationsnummer 	& {{.ESCo.VAT}} \\\tabucline{}
		Firmenanschrift     			& {{.ESCo.Address}} \\\tabucline{}
    		Gesetzlicher Vertreter			& \iffalse input fields.contractor_representative_name value="{{.Contract.Fields.contractor_representative_name}}" \fi {{.Contract.Fields.contractor_representative_name}} \\\tabucline{}
	\end{tabu}
\end{center}
haben diesen Energieeinsparungsvertrag, im Folgenden als Vertrag bezeichnet, abgeschlossen.

\section{BESONDERE BEDINGUNGEN}
\subsection{VERTRAGSGEGENSTAND}
\begin{enumerate}
  \item Der Gegenstand des Vertrages ist die Durchführung von Sanierungsarbeiten und die Erbringung von Energieeffizienzdienstleistungen, die zu Energieeinsparungen im Gebäude an der folgenden Adresse führen: {{asset_address .Asset.Address}} mit der Katasternummer {{.Asset.Cadastre}}.
  \item Der Baubereich dieses Vertrages und dessen Zustand vor den Sanierungsarbeiten sind in Anhang 1 des Vertrages beschrieben.
  \item Der detaillierte Umfang der Sanierungsarbeiten und der damit verbundenen Maßnahmen ist in Anhang 2 des Vertrages beschrieben.
\end{enumerate}

\subsection{DIENSTLEISTUNGEN}
\begin{enumerate}
  \item Der Auftragnehmer verpflichtet sich, für die Durchführung der Maßnahmen im Gebäude in Bezug auf die Sanierungsarbeiten die technische Planung, Beschaffung, Lieferung, Installation, Abnahme, Inbetriebnahme und Finanzierung zu veranlassen.
  \item Der Auftragnehmer garantiert für die Leistungsperiode des Vertrages die im Anhang 3 des Vertrages beschriebenen Komfortstandards.
  \item Der Auftragnehmer verpflichtet sich, für die Leistungsperiode des Vertrages eine Energieeinsparungsgarantie von {{.Project.GuaranteedSavings}}\% zu garantieren, was {{.Contract.Fields.calculations_qietg}}MWh  gegenüber der in Anhang 4 enthaltenen Baseline entspricht.
  \item Der Auftragnehmer verpflichtet sich, für die Leistungsperiode des Vertrages das Gebäude gemäß den Bestimmungen in Anlage 5 dieses Vertrages zu betreiben und zu warten.
\end{enumerate}

\subsection{VERTRAGSDAUER}
\begin{enumerate}
	\item Die Bauzeit beträgt {{date_diff .Project.ConstructionFrom .Project.ConstructionTo}} Tage und ist mit den folgenden Richtdaten vorbehaltlich von Benachrichtigungen vereinbart:
	\begin{enumerate}
		\item Anfangsdatum:     	  \iffalse input project.construction_from value="{{.Project.ConstructionFrom}}" type="date" \fi {{.Project.ConstructionFrom}}
		\item Datum der Inbetriebnahme:   \iffalse input project.construction_to value="{{.Project.ConstructionTo}}" type="date" \fi {{.Project.ConstructionTo}}
	\end{enumerate}
	\item Die Leistungsperiode des Vertrages beträgt {{mul 12 .Project.ContractTerm}} Monate, beginnend mit dem Datum der Inbetriebnahme der Maßnahmen.
	\item Die Zahlungsfrist des Auftraggebers entspricht der Leistungsperiode des Vertrages.
\end{enumerate}

\subsection{VERRECHNUNG}
\begin{enumerate}
	\item Der Auftragnehmer verrechnet dem Auftraggeber während der Leistungsperiode dieses Vertrages monatliche Gebühren bestehend aus:
	\begin{enumerate}
		\item Wärmeenergiegebühr und Warmwassergebühr berechnet gemäß Anhang 6;
		\item Sanierungsgebühr berechnet und indexiert gemäß Anhang 7;
		\item Betriebs- und Wartungsgebühr berechnet und indexiert gemäß Anhang 8;
		\item allen bei der Erbringung der Dienstleistungen anfallenden Steuern (z. B. Umsatzsteuer).
	\end{enumerate}
	\item Die monatliche Sanierungsgebühr und die Betriebs- und Wartungsgebühr für den ersten Monat der Leistungsperiode werden wie folgt vereinbart:

% table: summary

\begin{center}
	\begin{tabu}{|X|X|X|X|}\tabucline{}\rowfont[c]\bfseries
	{{with translate "au" .Contract.Tables.summary}} % chktex 26
	{{.Columns | column}} \\\tabucline{}
	{{range .Rows}} % chktex 26
	{{.|row}} \\\tabucline{}
	{{end}}
	\bfseries {{total .}} \\\tabucline{} % chktex 26
	{{end}}
	\end{tabu}
\end{center}

\item Dem Verwalter des Auftraggebers werden die Rechnungen monatlich gestellt. Rechnungen des Auftragnehmers an den Verwalter des Auftraggebers sind innerhalb von \iffalse input fields.invoiced_days value="{{.Contract.Fields.invoiced_days}}" \fi {{.Contract.Fields.invoiced_days}} Tagen nach Erhalt fällig.
\end{enumerate}

\subsection{SONSTIGE BESTIMMUNGEN}
\begin{enumerate}
\item Der Vertrag umfasst die Allgemeinen Geschäftsbedingungen, die Besonderen Bedingungen und die Anhänge zu den Besonderen Bedingungen, die integraler Bestandteil des Vertrages sind.
\item Mit der Unterzeichnung dieser Besonderen Bedingungen erkennen die Parteien an, dass sie die Besonderen Bedingungen, die Anhänge der Besonderen Bedingungen und die
  Allgemeinen Geschäftsbedingungen dieses Vertrages gelesen, verstanden und akzeptiert haben und einhalten können.
\end{enumerate}

\vspace{2cm}
{{template "sign.tex"}} % chktex 18

{{template "annex1.tex" .}} % chktex 18
{{template "sign.tex"}} % chktex 18

{{template "annex2.tex" .}} % chktex 18 chktex 26
{{template "sign.tex"}} % chktex 18

{{template "annex3.tex" .}} % chktex 18 chktex 26
{{template "sign.tex"}} % chktex 18

{{template "annex4.tex" .}} % chktex 18 chktex 26
{{template "sign.tex"}} % chktex 18

{{template "annex5.tex" .Contract.Tables}} % chktex 18 chktex 26
{{template "sign.tex"}} % chktex 18

{{template "annex6.tex" .}} % chktex 18
{{template "sign.tex"}} % chktex 18

{{template "annex7.tex" .}} % chktex 18 chktex 26
{{template "sign.tex"}} % chktex 18

{{template "annex8.tex" .}} % chktex 18 chktex 26
{{template "sign.tex"}} % chktex 18

{{template "annex9.tex" .Contract.Fields}} % chktex 18 chktex 26
{{template "sign.tex"}} % chktex 18

\pagebreak
\section{ANNEX №10 {-} PROTOCOLS OF THE GENERAL MEETINGS OF APARTMENT OWNERS OF THE BUILDING}

\begin{center}
\begin{tabu}{ |X|X| }
 \hline
 Aquisition protocol meeting & \url{ {{.Attachments.acquisition_meeting }} } \iffalse attachment value="acquisition meeting" \fi \\
 \hline
 Commitment protocol meeting & \url{ {{.Attachments.commitment_meeting }} } \iffalse attachment value="commitment protocol meeting" \fi \\
 \hline
 Kickoff protocol meeting & \url{ {{.Attachments.kickoff_meeting }} } \iffalse attachment value="kickoff protocol meeting" \fi \\
 \hline
\end{tabu}
\end{center}


\pagebreak
{{read .Markdown}} % chktex 26
\FloatBarrier{}\mbox{}\vfill\pagebreak % make sure no floats (e.g. images) go past here.

{{template "terms.tex"}} % chktex 18
{{template "sign.tex"}} % chktex 18

\end{document}
