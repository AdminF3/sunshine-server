\begin{multicols}{2}
  [\section{ALLGEMEINE GESCHÄFTSBEDINGUNGEN}]

  \subsection{BEGRIFFSBESTIMMUNGEN}

  \begin{itemize}[label={}]
  \item\textbf{Vertrag}: dieser zwischen dem Auftraggeber und dem Auftragnehmer abgeschlossene Energieeinsparungsvertrag einschließlich der Besonderen Bedingungen und seiner Anlagen und der Allgemeinen Geschäftsbedingungen, der über die EPC-Plattform – sunshineplatform.eu – entwickelt und verwaltet wird.
  \item\textbf{Wohnung}: Ein Wohnungseigentum ist eine rechtlich abgetrennte autonome Immobilie in einem Wohnhaus.
  \item\textbf{Wohnungseigentümer}: Ein Wohnungseigentümer ist eine Person, die das Wohnungseigentum erworben und den Rechtsanspruch im Grundbuch bestätigt hat.
  \item\textbf{Bankarbeitstag}: ist ein Tag, an dem an einem Ort, an dem gemäß den Bestimmungen des Vertrages Banküberweisungen auszuführen sind, die Geschäftsbanken allgemeine Bankgeschäfte durchführen.
  \item\textbf{Basiswert}: ist der Verbrauch von Wärmeenergie und Warmwasser des Gebäudes, ausgedrückt als Jahresmittelwert, der während des Basiszeitraums auftritt.
  \item\textbf{Basiszeitraum}: ist ein gemeinsam vereinbarter Zeitraum, der das Funktionieren des Gebäudes vor der Umsetzung der Maßnahmen darstellt.
  \item\textbf{Gebäude}: das Mehrfamilienwohnhaus, in dem der Auftragnehmer die Sanierungsarbeiten und die Leistungen im Rahmen dieses Vertrages erbringt.
  \item\textbf{Werktag}: ein offizieller Werktag, der nach lettischem Recht kein offizieller Feiertag oder offizieller arbeitsfreier Tag ist.
  \item\textbf{Auftraggeber}: Der/die Wohnungseigentümer des Gebäudes oder die bevollmächtigte Person, die im Namen des/der Wohnungseigentümer(s) handelt.
  \item\textbf{Komfortstandards}: die festgelegten Raumklimabedingungen und -parameter, die der Auftragnehmer dem Auftraggeber im Rahmen dieses Vertrages garantiert.
  \item\textbf{Anfangsdatum}: das Datum, an dem die Bauzeit beginnt.
  \item\textbf{Datum der Inbetriebnahme}: das Datum, an dem die Parteien das Liefer- und Abnahmeprotokoll für die Maßnahmen unterzeichnen und das Datum, an dem die Leistungsperiode des Vertrages beginnt.
  \item\textbf{Bauzeit}: die vom Auftragnehmer für die Umsetzung der Maßnahmen vorgesehene Zeitspanne. Die Bauzeit beginnt mit dem Anfangsdatum und endet mit dem Datum der Inbetriebnahme.
  \item\textbf{Auftragnehmer}: eine juristische Person, die sich zu diesem Vertrag und den Sanierungsarbeiten verpflichtet und die Dienstleistungen auf der Grundlage der Bestimmungen des Vertrages erbringt.
  \item\textbf{Liefer- und Abnahmeprotokoll}: das vom Auftragnehmer erstellte Protokoll in Übereinstimmung mit den lettischen Vorschriften und Normen für die endgültige Inbetriebnahme der vom Auftragnehmer im Gebäude durchgeführten Maßnahmen.
  \item\textbf{Warmwassergebühr}: die vom Auftraggeber an den Auftragnehmer gezahlte Gebühr, die für den tatsächlichen Warmwasserverbrauch zum aktuellen Heizenergietarif fällig ist.
  \item\textbf{Energie}: ein Produkt von einem bestimmten Wert – Kraftstoff, Wärmeenergie, erneuerbare Energie, Strom oder jede andere Art von Energie.
  \item\textbf{Energieaudit}: Maßnahmen, die durchgeführt werden, um Informationen über die Energieverbrauchsstruktur von Gebäuden oder Gebäudegruppen, von Abläufen oder Anlagen zu erhalten sowie um Möglichkeiten wirtschaftlich realisierbarer Energieeinsparungen zu messen und zu überprüfen, und deren Ergebnisse in einem Bericht zusammengefasst werden.
  \item\textbf{Energieverbrauchsgarantie}: die Menge der im Gebäude während der Leistungsperiode verbrauchten Wärmeenergie für Raumwärme und Zirkulationsverluste, die auf  Grundlage der vom Auftragnehmer gewährten Energieeinsparungsgarantie erreicht wird und die für die Ermittlung des Flächenwärmeenergieverbrauchs verwendet wird.
  \item\textbf{Energieeffizienzdienstleistungen (Dienstleistungen)}: Tätigkeiten, die vom Auftragnehmer durchgeführt werden, einschließlich der Umsetzung der Maßnahmen im Gebäude, des Betriebs und der Wartung der umgesetzten Maßnahmen, der Analyse von Energieverbrauchsdaten, des Monitorings und der Auswertung des Energieverbrauchs, insbesondere im Zusammenhang mit der Erfüllung der Energieeinsparungsgarantie.
  \item\textbf{Energieeinsparungen}: Menge der eingesparten Energie, die durch Messung und Verifizierung des Verbrauchs vor und nach der Umsetzung einer oder mehrerer Energieeffizienzmaßnahmen ermittelt und die im Gebäude durch die Umsetzung der Maßnahmen und die Gewährleistung der Energieeffizienz erreicht wird.
  \item\textbf{Energieeinsparungsgarantie}: die Mindestmenge an Energieeinsparungen, resultierend aus der Umsetzung der Maßnahmen und der Erbringung von Energieeffizienzdienstleistungen, die im Vertrag durch den Auftragnehmer garantiert und nach einem Measurement and Verification Plan festgelegt wird.
  \item\textbf{Energiepreis}: Gebühr für die Energieeinheit, in der sich das Gebäude befindet.
  \item\textbf{EPC-Plattform – sunshineplatform.eu}: die vielseitige Multi-Stakeholder-Online-Plattform für Energy Performance Contracting auf sunshineplatform.eu, die die Entwicklung und das Management von Gebäudesanierungsprojekten auf Basis von Energy Performance Contracting unterstützt.
  \item\textbf{Gebühr(en)}: monatliche Gebühr(en), die der Auftraggeber dem Auftragnehmer für die Erbringung der im Vertrag vorgeschriebenen Leistungen für die Dauer der Leistungsperiode bezahlt, inkl. Wärmeenergiegebühr, Warmwassergebühr, Sanierungsgebühr und Betriebs- und Wartungsgebühr sowie etwaige fällige Steuern (z. B. Umsatzsteuer).
  \item\textbf{Pauschaler Wärmeenergieverbrauch}: die vom Auftragnehmer kalkulierte Menge an thermischer Energie, auf deren Basis dem Auftraggeber eine fixe monatliche Menge an Wärmeenergie für jede Abrechnungsperiode während der Leistungsperiode berechnet wird.
  \item\textbf{Finanzbeitrag}: der Anteil der Investitionskosten für die Sanierungsarbeiten, die direkt mit Eigenkapital finanziert oder indirekt als Fremdfinanzierung durch den Auftragnehmer abgedeckt wurden und für die der Auftragnehmer die Sanierungsgebühr berechnet.
  \item\textbf{Wärmeenergiegebühr}: die vom Auftraggeber an den Auftragnehmer gezahlte Gebühr für den Energieverbrauch des Gebäudes während der Leistungsperiode, vorbehaltlich Anpassungen und Saldo einmal jährlich am Ende der Abrechnungsperiode, um die tatsächlichen Wetterbedingungen während der Abrechnungsperiode sowie die Messung und Verifizierung der Energieeinsparungsgarantie zu berücksichtigen.
  \item\textbf{Wärmeversorgung}: die Lieferung von thermischer Energie an das Gebäude zu Zwecken der Raumheizung und Warmwasserbereitung.
  \item\textbf{Heizperiode}: der Zeitraum im Jahr, in dem der Auftragnehmer die in diesem Vertrag festgelegten Komfortstandards beginnend ab dem 1. Oktober bis zum 30. April eines jeden Abrechnungsjahres während der Leistungsperiode einhalten muss.
  \item\textbf{Rechnung}: eine Rechnung, die dem Auftraggeber (den Wohnungseigentümern oder einem Vertreter der Wohnungseigentümer) für die erhaltenen Dienstleistungen und für andere dem Auftragnehmer zustehende Zahlungen, die sich aus dem Vertrag ergeben, in voller Übereinstimmung mit allen geltenden gesetzlichen Anforderungen des lettischen Rechts ausgestellt wird.
  \item\textbf{IPMVP}: das von der EVO (Efficiency Valuation Organization), (1629 K Street NW, Suite 300, Washington, DC 20006, USA), erstellte International Protocol of Measurement and Verification über die Auswirkungen der Energieeinsparungen, das für die Messung und Verifizierung im Rahmen des Vertrags angewendet wird.
  \item\textbf{LABEEF}: die Latvian Baltic Energy Efficiency Facility Jsc., die als Aktiengesellschaft tätig ist und ordnungsgemäß im Handelsregister Lettlands unter der Firmennummer 40103960646 eingetragen ist.
  \item\textbf{Latente Bedingungen}: Fehler und Mängel des Gebäudes oder in der Nähe des Gebäudes, von denen der Auftraggeber keine Kenntnis hatte und die der Auftragnehmer bei der Vertragserstellung nicht durch angemessene Beobachtungen und übliche Überprüfungen feststellen konnte.
  \item\textbf{Verwalter}: eine natürliche oder juristische Person, die in Übereinstimmung mit den anwendbaren Bestimmungen des lettischen Gesetzes über die Verwaltung von Wohnraum und auf der Grundlage eines Verwaltungsvertrags Verwaltungs- und Wartungstätigkeiten ausübt, die vom Auftraggeber beauftragt und durch den Vertrag festgelegt wurden.
  \item\textbf{Maßnahmen}, auch als Energieeffizienzmaßnahmen bezeichnet, sind Maßnahmen, die zu einer nachprüfbaren, messbaren oder schätzbaren Steigerung der Energieeffizienz führen sowie andere Bau- und Installationsarbeiten, die auf die Sanierung und Verbesserung des Gebäudes sowohl aus baulichen als auch aus ästhetischen Gründen abzielen.
  \item\textbf{Messung und Verifizierung}: die Prozesse und Aktivitäten, die durchgeführt werden, um die Energieeinsparungen zu ermitteln, die dem Gebäude als Ergebnis der durchgeführten Maßnahmen und erbrachten Dienstleistungen zuzurechnen sind.
  \item\textbf{Betriebs - und Wartungsgebühr}: die vom Auftraggeber an den Auftragnehmer gezahlte Gebühr für die Dienstleistungen im Zusammenhang mit dem Betrieb und der Wartung der Maßnahmen, mit einer jährlichen Indexierung gemäß dem für das jeweilige Jahr geltenden lettischen Verbraucherpreisindex, wie er von Centrala statistikas parvalde (Zentralamt für Statistik) veröffentlicht wurde.
  \item\textbf{Betriebs- und Wartungshandbuch}: ein Handbuch, das den Wartungsplan für die im Rahmen dieses Vertrags durchgeführten Maßnahmen und die unter den Vertrag fallenden operativen Tätigkeiten enthält.
  \item\textbf{Parteien}: der Auftraggeber und der Auftragnehmer gemeinsam.
  \item\textbf{Partei}: der Auftraggeber und der Auftragnehmer jeweils einzeln.
  \item\textbf{Zahlungsplan}: ein vom Auftragnehmer für den Auftraggeber erstelltes Dokument, aus dem die Sanierungsgebühr für die Rückzahlung des Finanzbeitrags hervorgeht, die für jeden Zinsberechnungszeitraum gemäß diesem Vertrag berechnet wird.
  \item\textbf{Ordnungsgemäßes Funktionieren}: das Funktionieren der Maßnahmen in einer Weise, die die Erreichung ihrer vollen Funktionsfähigkeit und Effizienz gewährleistet, und umfasst alle notwendigen Wartungsarbeiten, die vom Auftragnehmer und auf dessen Kosten durchgeführt werden.
  \item\textbf{Regulator}: die Energieaufsichtsbehörde oder eine andere zuständige Behörde, die in den in Lettland geltenden Gesetzen und Verordnungen festgelegt ist und die Tarife für den Handel mit Wärmeenergie in der jeweiligen Lokalverwaltung, in der sich das Gebäude befindet, genehmigt.
  \item\textbf{Sanierungsgebühr}: die vom Auftraggeber an den Auftragnehmer gezahlte indexierte und an den Euribor gebundene Gebühr für den Finanzeitrag des Auftragnehmers.
  \item\textbf{Sanierungsarbeiten}: Tätigkeiten des Auftragnehmers, die für die Durchführung der Maßnahmen im Gebäude erforderlich sind, einschließlich technische Planung, Beschaffung, Lieferung, Installation, Abnahme, Inbetriebnahme und Finanzierung der Maßnahmen.
  \item\textbf{Leistungsperiode}: der Zeitraum, in dem der Auftragnehmer die Dienstleistungen für den Auftraggeber erbringt. Die Leistungsperiode beginnt mit dem Datum der Inbetriebnahme.
  \item\textbf{Abrechnungsperiode}: der Zeitraum von einem Kalenderjahr, der sich während der Leistungsperiode jährlich wiederholt.
  \item\textbf{Erklärung}: ein von den Vertragsparteien unterzeichnetes Dokument zum Nachweis verschiedener am Gebäude vorhandener Parameter, die zum Zeitpunkt der Erstellung dieses Dokuments erfasst wurden.
  \item\textbf{Umsatzsteuer}: die gemäß den in Lettland geltenden Gesetzen und Vorschriften und den Bestimmungen des Vertrags zu zahlende Umsatzsteuer.
  \end{itemize}

  \subsection{ANNAHME DER VERTRAGSBEDINGUNGEN}
  \begin{enumerate}
  \item Der Auftraggeber geht davon aus, dass der Auftragnehmer über die erforderlichen Qualifikationen, Erfahrungen und Fähigkeiten verfügt, um die Sanierungsarbeiten durchzuführen und die Dienstleistungen für den Auftraggeber zu erbringen. Aus diesem Grund ermächtigt der Auftraggeber den Auftragnehmer und wird ihn dabei unterstützen, auf Kosten des Auftragnehmers alle rechtlichen und sachlichen Maßnahmen zur Durchführung des Vertrages zu ergreifen, ohne dass eine ausdrückliche Vollmacht zugunsten des Auftragnehmers erforderlich ist.
  \item Der Auftragnehmer übernimmt die Sanierungsarbeiten und erbringt die Dienstleistungen für den Auftraggeber zu den in diesem Vertrag festgelegten Bedingungen. Der Auftragnehmer erkennt an, dass er sich von der Art, der Lage und dem Standort des Gebäudes und allen anderen Gegebenheiten, die die Erfüllung seiner Verpflichtungen aus dem Vertrag in irgendeiner Weise beeinträchtigen könnten, überzeugt hat. Eine Unterlassung des Auftragnehmers, sich mit dem Bauwerk oder den Baustellenbedingungen gemäß dieser Klausel vertraut zu machen, entbindet ihn nicht von der Verantwortung für die Erfüllung seiner Verpflichtungen aus dem Vertrag.
  \item Der Vertrag bestätigt, dass das in den Besonderen Bedingungen dieses Vertrages enthaltene Budget alle Bauarbeiten, Materialien und Ausrüstungen umfasst, die für die Ausführung der Sanierungsarbeiten gemäß den technischen Spezifikationen und Bedingungen dieses Vertrages erforderlich sind.
  \item Die Begriffsbestimmungen für alle Zwecke des Vertrages, seiner Besonderen Bedingungen, seiner Anhänge und dieser Allgemeinen Geschäftsbedingungen haben die jeweilige Bedeutung, die in Artikel 1 der Allgemeinen Geschäftsbedingungen dieses Vertrages angegeben ist.
  \item Im Falle von Widersprüchen zwischen den Allgemeinen Geschäftsbedingungen sowie den Besonderen Bedingungen und ihren Anhängen haben die zuletzt genannten Vorrang.
  \end{enumerate}

  \subsection{SICHERHEIT, QUALITÄT UND KOMFORT}
  \begin{enumerate}
    \item Die vom Auftragnehmer im Rahmen dieses Vertrages erbrachten Leistungen müssen:
    \begin{enumerate}
      \item mit dem höchsten Standard an Kompetenz und Sorgfalt geliefert werden, wie es von erfahrenen und professionellen Auftragnehmern erwartet wird, die regelmäßig Arbeiten und Dienstleistungen im gleichen oder ähnlichen Umfang und in ähnlicher Komplexität wie in diesem Vertrag ausführen;
      \item mit Materialien und Geräten, durchgeführt werden, die von adäquater Qualität sowie neu und für den vorgesehenen Verwendungszweck geeignet sind;
      \item der Baugesetzgebung und allen anderen anwendbaren Rechtsvorschriften, Verordnungen oder Normen entsprechen, die zum Zeitpunkt der Erbringung der Dienstleistungen in der Republik Lettland gelten;
      \item so ausgeführt werden, dass die Verwendung des Gebäudes durch den Auftraggeber und andere Nutzer des Gebäudes so wenig wie möglich beeinträchtigt wird;
      \end{enumerate}
   \item Die Komfortstandards erfüllen oder übertreffen während der Leistungsperiode des Vertrags das in den Besonderen Bedingungen dieses Vertrages  festgelegte Niveau.
   \item Für die Zeit, in der die Fenster in einer Wohnung des Gebäudes geöffnet sind, und für 2 (zwei) Stunden nach dem Schließen der Fenster garantiert der Auftragnehmer für die jeweilige Wohnung, in der die Fenster geöffnet wurden, nicht die in den Besonderen Vertragsbedingungen vereinbarten Raumtemperaturen.
   \item Der Auftragnehmer hat für eine ausreichende Belüftung der
Wohnungen gemäß den einschlägigen lettischen Vorschriften und Normen zu sorgen.
   \item Der Auftragnehmer hat alle erforderlichen Maßnahmen zu ergreifen, um die Sicherheit und den Gesundheitsschutz der Mitarbeiter am Arbeitsplatz in Übereinstimmung mit dem Arbeitsschutzgesetz und allen relevanten lettischen Vorschriften und Normen zu gewährleisten.
   \item Der Auftragnehmer hat geeignete Schutzmaßnahmen zu ergreifen, um alle Menschen vor Tod oder Verletzung zu schützen, die durch Versäumnis oder grobe Fahrlässigkeit des Auftragnehmers, seiner Mitarbeiter, Beauftragte oder Subunternehmer während der Bauarbeiten und der Leistungsperiode verursacht werden können. Der Auftragnehmer hat auch das gesamte Gebäude vor Schäden im Zusammenhang mit der Umsetzung der Maßnahmen zu schützen.
   \item Der Auftragnehmer muss sicherstellen, dass alle Versorgungsleistungen des Gebäudes nicht aufgrund von Versäumnis oder Fahrlässigkeit des Auftragnehmers zu irgendeinem Zeitpunkt ohne vorherige Ankündigung unterbrochen oder gestört werden. Alle Versorgungsleistungen, die aufgrund von Versäumnis oder Fahrlässigkeit des Auftragnehmers unterbrochen oder gestört werden, sind vom Auftragnehmer auf Kosten des Auftragnehmers unverzüglich wiederherzustellen. Der Auftragnehmer haftet nicht für Fälle, in denen solche Störungen außerhalb der Kontrolle des Auftragnehmers liegen und/oder auf Handlungen oder Unterlassungen der Wartungsfirma, der Energie- und Wasserversorgungsunternehmen oder Dritter zurückzuführen sind, die nicht mit dem Auftragnehmer verbunden sind.
   \item Der Auftragnehmer hat während der Bauzeit für einen angemessenen Schutz des Gebäudes vor Witterungseinflüssen zu sorgen, indem er das Eindringen von Regenwasser und Schäden am Gebäude verhindert. Grundwasserinfiltrationen und Ereignisse höherer Gewalt sind ausgeschlossen.
   \item Der Auftragnehmer hält sich an den Europäischen Verhaltenskodex für Energy Performance Contracting (http://transparense.eu/), der eine Reihe von Werten und Prinzipien enthält, die als grundlegend für die erfolgreiche, professionelle und transparente Umsetzung von Energiespar-Contracting in europäischen Ländern gelten.
  \end{enumerate}

  \subsection{GARANTIEN}
  \begin{enumerate}
   \item Der Auftragnehmer hat dem Auftraggeber während der Leistungsperiode eine Energieeinsparungsgarantie im Rahmen dieses Vertrages zur Verfügung zu stellen, die jährlich einer Messung und  Verifizierung zu unterziehen ist.
   \item Der Auftragnehmer garantiert während der Leistungsperiode die in diesem Vertrag zugesagten Komfortstandards.
   \item Der Auftragnehmer garantiert während der Leistungsperiode auf eigene Kosten das ordnungsgemäße Funktionieren der vom Auftragnehmer installierten oder eingeführten Maßnahmen für die Heizungsanlage, Warmwasserversorgungssysteme, Heizungs-, Lüftungs- und Kühlungssysteme, Anschlüsse und Rohrleitungen gemäß deren Spezifikationen und normaler Abnutzung, während der Laufzeit dieses Vertrages, unter anderem durch Reparatur oder Austausch der Maßnahmen, falls erforderlich.
   \item Der Auftragnehmer garantiert während der Leistungsperiode auf eigene Kosten die Wirkung und Effizienz der vom Auftragnehmer installierten oder eingeführten Dämmstoffe gemäß deren Spezifikationen und normaler Abnutzung während der gesamten Laufzeit des Vertrages, unter anderem durch deren Reparatur oder Austausch, falls erforderlich.
   \item Der Auftragnehmer hat am Ende der Leistungsperiode für das ordnungsgemäße Funktionieren aller durchgeführten Maßnahmen gemäß deren Spezifikationen und normaler Abnutzung sowie unter Berücksichtigung der ordnungsgemäßen Wartung zu sorgen. Der Auftragnehmer hat dem Auftraggeber am Ende der Leistungsperiode alle Nutzungs-, Pflege- und Wartungsanleitungen, Aufzeichnungen, Anweisungen, andere Dokumentationen, Software, Lizenzen für geistiges Eigentum, Spezialwerkzeuge und Protokolle und Verfahren zur Verfügung zu stellen, die für die fortdauernde gute Leistung der Maßnahmen zur Erreichung der Komfortstandards dieses Vertrages erforderlich oder zweckmäßig sind.
   \item Der Auftragnehmer hat dem Auftraggeber vor Beginn der Bauzeit eine Leistungsgarantie eines Kreditinstituts oder einer Versicherungsgesellschaft für die Erfüllung seiner Verpflichtungen in Höhe von 10\% der gesamten Investitionskosten (ohne Umsatztsteuer) wie folgt vorzulegen:
   \begin{enumerate}
   \item wenn der Auftragnehmer gleichzeitig der Generalunternehmer ist, wird diese Leistungsgarantie vom Auftragnehmer zugunsten des Auftraggebers gegen die Bestimmungen dieses Vertrages verrechnet;
   \item wenn der Auftragnehmer den Generalunternehmer beschafft, wird diese Leistungsgarantie vom Generalunternehmer zugunsten des Auftragnehmers und auf der Grundlage der Bestimmungen des Bauvertrags zwischen dem Auftragnehmer und des Generalunternehmers bereitgestellt;
   \item Falls der Auftragnehmer eine solche ursprüngliche Leistungsgarantie für die Bauzeit, die die Ausführung der Arbeiten in der Bauzeit sicherstellt, nicht vorweisen kann, hat der Auftragnehmer kein Recht, die Bauarbeiten zu veranlassen.
   \item Diese Leistungsgarantie gilt für die gesamte Bauzeit. Im Falle einer Verlängerung der Bauzeit hat der Auftragnehmer diese Garantie um den gleichen Zeitraum zu verlängern.
   \end{enumerate}
   \item Der Auftragnehmer hat dem Auftraggeber spätestens 10 (zehn) Tage nach Unterzeichnung des Liefer- und Abnahmeprotokolls eine Leistungsgarantie eines Kreditinstituts oder einer Versicherungsgesellschaft für die Erfüllung seiner Verpflichtungen in Höhe von mindestens 5\% der Investitionskosten (ohne Umsatzsteuer) wie folgt vorzulegen:
   \begin{enumerate}
   \item wenn der Auftragnehmer gleichzeitig der Generalunternehmer ist, wird diese Leistungsgarantie vom Auftragnehmer zugunsten des Auftraggebers gegen die Bestimmungen dieses Vertrages verrechnet;
   \item wenn der Auftragnehmer den Generalunternehmer beschafft, wird diese Leistungsgarantie vom Generalunternehmer zugunsten des Auftragnehmers und auf der Grundlage der Bestimmungen des Bauvertrags zwischen dem Auftragnehmer und dem Generalunternehmer bereitgestellt;
   \item Diese Garantie gilt für 36 (sechsunddreißig) Monate.
   \end{enumerate}
   \item Der Auftraggeber hat das Recht, die in Artikel 4.6. und Artikel 4.7. genannte Leistungsgarantie für die Abwicklung der finanziellen Verpflichtungen des Auftragnehmers oder für behördliche Verordnungen in Anspruch zu nehmen.
   \item Die in diesem Artikel genannte Leistungsgarantie ist von einem in der Republik Lettland oder einem anderen Mitgliedstaat der Europäischen Union oder des Europäischen Wirtschaftsraums eingetragenen Kreditinstitut oder Versicherungsunternehmen auszustellen, das nach dem in den Rechtsakten der Republik Lettland festgelegten Verfahren den Betrieb für die Erbringung von Leistungen im Gebiet der Republik Lettland aufgenommen hat.
   \end{enumerate}

 \subsection{RECHTE UND PFLICHTEN DES AUFTRAGNEHMERS}
  \begin{enumerate}
   \item Der Auftragnehmer verfügt über die erforderlichen beruflichen Qualifikationen, Erfahrungen und Fähigkeiten in der Lieferung der Ausrüstung, des Materials und der Dienstleistungen gemäß dem Vertrag.
   \item Der Auftragnehmer hat alle erforderlichen Zulassungen und Genehmigungen von der Regierung, den kommunalen Einrichtungen und Behörden für die Durchführung der Sanierungsarbeiten und die Erbringung der Dienstleistungen ohne Beteiligung des Auftraggebers einzuholen, es sei denn, dies ist aufgrund der geltenden Gesetze oder Vorschriften erforderlich.
   \item Der Auftragnehmer hat mit der Durchführung der Bau- und Installationsarbeiten der Maßnahmen zum Anfangsdatum zu beginnen und sie innerhalb der vorgesehenen Bauzeit durchzuführen. Der Auftragnehmer hat den Auftraggeber über das vorläufige Anfangsdatum zu informieren, spätestens innerhalb von 20 Werktagen nach Unterzeichnung dieses Vertrages.
   \item Der Auftragnehmer hat den Auftraggeber mindestens 10 (zehn) Werktage im Voraus über das Anfangsdatum der Bau- und Installationsarbeiten der Maßnahmen schriftlich zu benachrichtigen, damit der Auftraggeber die Gemeinschaftsräume des Gebäudes (einschließlich Treppenhäuser, Kellerräume, Dachgeschosse, Dach, Kohle-/Holz- und Gaslagerflächen, Strom- und Telefonverteilerkästen sowie Kesselräume) von Abfällen, zurückgelassenen Objekten und allen anderen dort befindlichen Gegenständen räumen lassen kann. Versäumt der Auftraggeber die rechtzeitige Räumung der Gemeinschaftsflächen, so ist der Auftragnehmer berechtigt, die Räumung der Gemeinschaftsflächen des Gebäudes zu veranlassen und dem Auftraggeber eine Rechnung für die entstandenen Kosten auszustellen. Der Auftraggeber hat diese Rechnung unverzüglich, spätestens aber innerhalb von 20 Werktagen zu bezahlen.
   \item Während der Bauzeit hat der Auftragnehmer alle für die Durchführung der Maßnahmen erforderlichen Arbeitskräfte zur Verfügung zu stellen, einschließlich der erforderlichen Aufsicht, Werkzeuge, Materialien und Geräte geeigneter Art, Qualität und Menge.
   \item Während der Leistungsperiode hat der Auftragnehmer alle erforderlichen Arbeitskräfte für den Betrieb und die Wartung der Maßnahmen zur Verfügung zu stellen, einschließlich der erforderlichen Aufsicht, Werkzeuge, Materialien und Geräte geeigneter Art, Qualität und Menge.
   \item Während der Bauzeit hat der Auftragnehmer für die Stromversorgung mit getrennter Messung zu sorgen und den Stromverbrauch für die Durchführung und Installation der Maßnahmen zu bezahlen. Der Auftragnehmer hat Recht auf Zugang zum zentralen Stromversorgungssystem des Gebäudes.
   \item Der Auftragnehmer hat die Baustelle (die Gemeinschaftsflächen des Gebäudes, Fenster, Eingänge und Umgebung) nach Abschluss der Bau- und Installationsarbeiten der Maßnahmen vor dem Datum der Inbetriebnahme ordnungsgemäß zu reinigen.
   \item Der Auftragnehmer hat den Auftraggeber zur Inbetriebnahme der im Gebäude durchgeführten Maßnahmen einzuladen. Der Auftragnehmer hat dem Auftraggeber am Ende der Bauzeit das Liefer- und Abnahmeprotokoll des Gebäudes zur Verfügung zu stellen.
   \item Der Auftragnehmer hat den Auftraggeber während der Leistungszeit schriftlich zu benachrichtigen, wenn Abfälle und/oder Gegenstände von Wohnungseigentümern oder anderen Dritten, die nicht mit dem Auftragnehmer verbunden sind, im Gemeinschaftsbereich des Gebäudes gelagert und zurückgelassen werden, was dem Auftragnehmer Probleme bei Betriebs- und Wartungsarbeiten im Rahmen des Vertrages bereiten könnte. Wenn der Auftraggeber den Bereich nicht gemäß diesem Vertrag reinigen lässt, hat der Auftragnehmer das Recht, die Reinigung der Räumlichkeiten zu veranlassen und dem Auftraggeber eine Rechnung für die Durchführung dieser Reinigung auszustellen. Der Auftraggeber hat diese Rechnung unverzüglich, spätestens aber innerhalb von 20 Werktagen zu bezahlen.
   \item Der Auftragnehmer hat den Auftraggeber während der Leistungsperiode schriftlich über alle festgestellten Defekte, Diebstähle, Vandalismus oder Sabotagen an den Maßnahmen zu informieren.
   \item Der Auftragnehmer hat für eine ausreichende Wärmeenergieversorgung des Gebäudes während der Heizsaison in einer Weise zu sorgen, die den Komfortstandards dieses Vertrages entspricht. Der Auftragnehmer haftet im Rahmen dieses Vertrages nicht für Störungen oder Mängel der Wärmeenergieversorgung des Gebäudes in Fällen, die außerhalb des Einflussbereichs des Auftragnehmers liegen, einschließlich des Ausfalls der Wärmeversorgung im Bereich des Wärmeversorgungsunternehmens oder Ereignisse  höherer Gewalt.
   \item Der Auftragnehmer hat dem Auftraggeber vor Beginn der Bauzeit das Projekt für die Sanierung des Gebäudes gemäß der Verordnung des Ministerkabinetts 10/21/2014 Nr. 655 "Vorschriften über die lettische Baunorm LBN 310-‘Arbeitsprojekt‘“ vorzulegen und mit dem Auftraggeber und der Bauleitung abzustimmen.
   \item Der Auftragnehmer hat den Auftraggeber während der Bauzeit zu informieren und einzuladen, an den wöchentlichen Monitoringmeetings über den Stand der Bauarbeiten teilzunehmen. Der Auftragnehmer hat dem Auftraggeber dann 2 (zwei) Fortschrittsberichte pro Monat über den Stand und Fortschritt der Bauarbeiten vorzulegen. Diese Berichte können elektronisch über die Online-EPC-Plattform sunshineplatform.eu eingereicht werden.
   \item In den Fällen, in denen der Auftragnehmer Mittlerfunktion zwischen dem Auftraggeber und dem Wärmeenergieversorger hat, hat der Auftragnehmer im Namen des Auftraggebers die der Wärmegesellschaft zustehenden Rechnungen zu bezahlen nach Erhalt der entsprechenden, nach diesem Vertrag  dem Auftragnehmer zustehenden Wärmeenergiegebühr. Ungeachtet dessen gehen die Rechnungen für die Heizung während der Bauzeit zu Lasten des Auftraggebers.
   \item Der Auftragnehmer ist berechtigt, die im Vertrag festgelegten Arbeiten und Leistungen an Dritte (Subunternehmer) zu übertragen oder an Dritte weiterzugeben. Der Auftragnehmer haftet für Subunternehmer in vollem Umfang im Hinblick auf die Verpflichtungen aus diesem Vertrag.
   \item Der Auftragnehmer hat das Recht, die Energieeinsparungsgarantie im Falle einer Nutzungsänderung des Gebäudes zu ändern (Artikel 17). Jede Änderung ist zwischen den Parteien mit einer schriftlichen Änderung dieses Vertrages zu vereinbaren.
   \item Der Auftragnehmer ist berechtigt, auf eigene Kosten einen ordnungsgemäß qualifizierten und erfahrenen unabhängigen Sachverständigen, der vom Auftraggeber vorab zugelassen wurde (diese Genehmigung darf nicht unangemessen verweigert werden), einzuladen, um die Entsprechung der vorgeschlagenen Maßnahmen mit den in der Republik Lettland geltenden Gesetzen und Vorschriften oder mit den für den Auftraggeber verbindlichen Entscheidungen der örtlichen Behörden zu bewerten, falls der Auftraggeber sein Vetorecht ausübt. Die Stellungnahme dieses Sachverständigen ist für die Parteien verbindlich.
   \item Der Auftragnehmer hat den Auftraggeber spätestens innerhalb von 5 (fünf) Werktagen über eine Änderung der im Vertrag genannten Anschrift oder andere Änderungen seiner Rechtsstellung, Leitung und Rechtslage zu informieren, insbesondere wenn der Auftragnehmer eine Fusion oder Übernahme durchführt bzw. ein Liquidations- oder Konkursverfahren einleitet.
  \end{enumerate}

  \subsection{RECHTE UND PFLICHTEN DES AUFTRAGGEBERS}
  \begin{enumerate}
   \item Der Auftraggeber hat eine gültige und durchsetzbare Entscheidung getroffen, die diesen Vertrag für jeden einzelnen Wohnungseigentümer verbindlich macht, die jeden Einzelnen dazu verpflichtet, deren Bestimmungen zu erfüllen, unabhängig davon, ob die betreffende Wohnung verpachtet oder vermietet wurde, und unabhängig davon, ob die Wohnungen von den jeweiligen Wohnungseigentümern genutzt werden oder nicht.
   \item Der Auftraggeber (jeder Wohnungseigentümer) hat Mieter, Pächter und alle anderen Wohnungsnutzer oder gewöhnlichen Bewohner über die relevanten Verpflichtungen aus diesem Vertrag zu informieren.
   \item Der Auftraggeber ist verpflichtet, vom Auftragnehmer angeforderte Informationen und Unterlagen für die Durchführung der Sanierungsarbeiten und die Erbringung der im Vertrag genannten Leistungen so schnell wie möglich nach Eingang der Aufforderung des Auftragnehmers zur Verfügung zu stellen. Der Auftraggeber haftet nicht für die Nichtvorlage von Dokumenten, die relevant, aber vom Auftragnehmer nicht eindeutig und ausreichend detailliert beschrieben sind.
   \item Der Auftraggeber ist verpflichtet, dem Auftragnehmer rechtzeitig bei der Beschaffung notwendiger Zulassungen, Genehmigungen oder sonstiger Dokumente von der Regierung, kommunalen Einrichtungen und Agenturen im Zusammenhang mit der erfolgreichen Durchführung des Vertrages zu unterstützen, einschließlich, aber nicht beschränkt auf, Zertifizierung und/oder Bereitstellung notwendiger Dokumente, Erteilung der erforderlichen Vollmacht und Zurverfügungstellung von Informationen für den Auftragnehmer. Der Auftraggeber hat den Auftragnehmer in der erforderlichen Form ordnungsgemäß zu ermächtigen, alle sachlichen oder rechtlichen Schritte vor den zuständigen Behörden zum Zwecke der erfolgreichen Durchführung des Vertrages vorzunehmen. Der Auftraggeber haftet jedoch nicht für die Nichtbereitstellung aller dieser Informationen, es sei denn, der Auftragnehmer hat dies im Einzelnen klar festgelegt und diese Informationen stehen dem Auftraggeber nach vernünftigem Ermessen zur Verfügung.
   \item Der Auftraggeber darf die Zustimmung zu den Sanierungsarbeiten, der Durchführung der Maßnahmen während der Bauphase und deren Wartung während der Leistungszeit nicht behindern oder verweigern; im Gegenteil, der Auftraggeber hat in gutem Glauben zu handeln, um deren Durchführung und Wartung sowie die Erreichung der Energieeinsparungsgarantie zu erleichtern.
   \item Der Auftraggeber hat das Recht, sich innerhalb von 10 (zehn) Werktagen nach Unterzeichnung des Liefer- und Abnahmeprotokolls über die Qualität oder Ausführungsart der durchgeführten Maßnahme zu beschweren. Nach Ablauf dieser Frist gelten die vom Auftragnehmer durchgeführten Maßnahmen als akzeptiert und die Bauzeit als beendet.
   \item Der Auftraggeber ist berechtigt, das Vetorecht auszuüben oder die Durchführung einer im Rahmen der Sanierungsarbeiten geplanten Maßnahme zu verweigern, wenn der Auftraggeber zweifelsfrei nachweist, dass die betreffende(n) Maßnahme(n) gegen die in der Republik Lettland geltenden Gesetze und Vorschriften oder gegen für den Auftraggeber verbindliche Entscheidungen der örtlichen Behörden verstößt (verstoßen).
   \item Der Auftraggeber hat dem Auftragnehmer oder einer anderen vom Auftragnehmer beauftragten Person während der Bauzeit und während der Leistungsperiode Zugang zum Gebäude zu gewähren, einschließlich jeder Wohnung des Gebäudes, zum Zwecke der Erbringung der Leistungen des Vertrages. Der Auftraggeber hat den Zutritt zum Gebäude an Werktagen (zwischen 8:00 und 20:00 Uhr), in Notfällen außerhalb der Arbeitszeit und auch an Wochenenden und Feiertagen jederzeit sicherzustellen.
   \item Der Auftraggeber hat vor dem Anfangsdatum sicherzustellen, dass Gemeinschaftsflächen (einschließlich Treppenhäuser, Kellerräume, Dachgeschosse, Dach, Kohle-/Holz- und Gaslagerflächen, Strom- und Telefonverteilerkästen sowie Kesselräume) frei von Abfall, zurückgelassenen Objekten und allen dort befindlichen Gegenständen sind, indem er entweder mit dem Auftragnehmer eine Vereinbarung zur Beseitigung trifft, sie an ein Entsorgungsunternehmen liefert oder sie an den bekannten Eigentümer liefert.
   \item Der Auftraggeber hat während der Leistungsperiode dafür zu sorgen, dass die Gemeinschaftsflächen des Gebäudes wie Treppenhäuser, Kellerräume und Dachgeschossflächen durch regelmäßige Reinigung sauber und in gutem Betriebszustand gehalten werden.
   \item Der Auftraggeber darf nicht ohne schriftliche Zustimmung und Genehmigung des Auftragnehmers oder im Widerspruch zu den vom Auftragnehmer zur Verfügung gestellten Betriebsanweisungen in die vom Auftragnehmer eingeführten Maßnahmen eingreifen, insbesondere wenn sich der Eingriff negativ auf die Höhe der Energieeinsparung auswirkt. Der Eingriff des Auftraggebers in die Einstellungen der Heizungsanlage, des Warmwassersystems und der Lüftungsanlage gilt als wesentlicher Verstoß des Auftraggebers gegen seine Verpflichtungen aus diesem Vertrag und ist ein guter und ausreichender Grund für die Beendigung des Vertrages durch den Auftragnehmer.
   \item Der Auftraggeber hat alle angemessenen Maßnahmen zu ergreifen, um sicherzustellen, dass niemand im Gebäude die Einstellungen der Heizungsanlage, des Warmwassersystems und der Lüftungsanlage stört oder manipuliert oder die Maßnahmen in irgendeiner Weise beschädigt und sabotiert.
   \item Der Auftraggeber hat den Auftragnehmer unverzüglich nach Entdeckung (innerhalb eines Werktages) über Schäden, Änderungen oder Eingriffe in die vom Auftragnehmer installierten Maßnahmen zu informieren.
   \item Der Auftraggeber hat jeden Umstand zu melden, der negative Auswirkungen auf die Energieeinsparung hat und/oder haben könnte. Unterlässt der Auftraggeber diese Mitteilung, erlischt die Haftung des Auftragnehmers für die Energieeinsparungsgarantie nicht, es sei denn, es wird festgestellt, dass der Auftraggeber die Höhe der Energieeinsparungen zu verringern beabsichtigte.
   \item Der Auftraggeber hat den Auftragnehmer spätestens innerhalb von 20 (zwanzig) Werktagen zu benachrichtigen und gegebenenfalls mit dem Auftragnehmer die Durchführung von Bau-, Installations- und Wartungsarbeiten abzustimmen, die nicht Bestandteil dieses Vertrages sind und einen potenziellen Einfluss auf den Energieverbrauch des Gebäudes haben, einschließlich, aber nicht beschränkt auf: (i) die Erweiterung des Gebäudebereichs, (ii) die weitere Modernisierung des Gebäudes, (iii) der Austausch oder die Installation von neuen/anderen Heizkörpern und/oder Wärmekonvektoren und/oder Heizelementen, (iv) die Installation einer neuen Wärmeerzeugungsanlage.
   \item Der Auftraggeber (Wohnungseigentümer) hat den Auftragnehmer im Falle einer Sanierung der Wohnung und über den Zeitplan zu informieren, wenn eine solche Sanierung den Energieverbrauch des Gebäudes beeinträchtigen könnte, einschließlich, aber nicht beschränkt auf: (i) Austausch der Heizkörper und/oder Wärmekonvektoren und/oder Heizelemente; (ii) Austausch von Fenstern; (iii) Erweiterung der beheizten Fläche der Wohnung einschließlich der Balkon-/Loggiafläche; (iv) Installation von mechanischen Lüftungssystemen. In diesen Fällen ist der Auftragnehmer berechtigt, die im Vertrag festgelegte Energieeinsparungsgarantie zu ändern.
   \item Der Auftraggeber hat das Recht, während der Heizperiode die Fenster in den Wohnungen des Gebäudes für maximal 10 (zehn) Minuten pro Tag zu öffnen, zur Zirkulation von Frischluft und Beseitigung von: (i) Staub oder starkem Geruch von Reinigungsmitteln, der nach der Reinigung in der Wohnung verbleibt, und (ii) starkem Geruch nach dem Kochen, der in der Wohnung verbleibt.
   \item Der Auftraggeber hat während der Heizperiode das Recht, die Fenster in den Wohnungen des Gebäudes aus gesundheitlichen Gründen der darin lebenden Personen jederzeit zu öffnen.
   \item Der Auftraggeber hat dafür zu sorgen, dass alle Fenster des Gemeinschaftsbereichs während der Heizperiode geschlossen bleiben.
   \item Der Auftraggeber hat dafür zu sorgen, dass alle Eingangstüren des Gebäudes während der Heizperiode nicht offen gelassen werden.
   \item Findet im Gebäude ein Wohnungseigentümerwechsel statt, hat der Auftraggeber unverzüglich, spätestens jedoch innerhalb von 5 (fünf) Werktagen nach dieser Änderung, den Auftragnehmer über die Änderungen zu informieren.
   \item Der Auftraggeber (jeder Wohnungseigentümer) hat sicherzustellen, dass im Falle einer Übertragung von Eigentumsrechten an seiner Wohnung und unabhängig davon, aus welchen Gründen oder Rechtsgründen diese Übertragung durchgeführt wird, der neue Wohnungseigentümer (Übernehmer) eine Einhaltungserklärung oder ein anderes Rechtsdokument unterzeichnet, das die Übernahme von Rechten und die Übertragung von Verpflichtungen aus diesem Vertrag vom ursprünglichen Wohnungseigentümer auf den neuen Wohnungseigentümer ermöglicht. Die Nichteinhaltung obiger Bestimmung führt dazu, dass der alte Wohnungseigentümer und der Auftraggeber gemeinsam für die Erfüllung der in diesem Vertrag festgelegten Verpflichtungen und für eventuelle Verstöße durch den Übernehmer haften.
   \item Der Auftraggeber hat den Auftragnehmer spätestens innerhalb von 5 (fünf) Werktagen über den Wechsel des Verwalters zu informieren. Der neue Verwalter wird ordnungsgemäß über die Bestimmungen dieses Vertrages informiert.
  \end{enumerate}

  \subsection{ABWICKLUNGSVERFAHREN}
  \begin{enumerate}
   \item Der Auftraggeber hat dem Auftragnehmer die in diesem Vertrag festgelegten monatlichen Gebühren zu zahlen.
   \item Der Zeitraum für die Begleichung wechselseitiger Forderungen zwischen den Parteien ist ein Kalendermonat. Die Abrechnungsperiode für die Berechnung des Saldos zwischen den Rechnungen auf Basis des pauschalen Heizenergieverbrauchs und des tatsächlich gemessenen Heizenergieverbrauchs sowie für die Messung und Verifizierung der Energieeinsparungsgarantie beträgt ein Jahr.
   \item Jeden Monat berechnet der Auftragnehmer oder ein beauftragter Dritter, der den Auftragnehmer vertritt, die vom Auftraggeber gemäß dem Vertrag zu leistende Zahlung. Die Gesamtsumme aller berechneten Gebühren ist vom Auftraggeber an den Auftragnehmer für die im Rahmen dieses Vertrages erbrachten Leistungen zur Begleichung wechselseitiger Forderungen fällig und zahlbar.
   \item Alle 12 (zwölf) Monate zum Jahrestag ab Beginn der Leistungsperiode hat der Auftragnehmer eine jährliche Abrechnung auf der Grundlage der Ergebnisse der Messung und Verifizierung der Energieeinsparungsgarantie durchzuführen.
   \item Der Auftragnehmer oder der von ihm beauftragte Dritte hat spätestens bis zum 10. (zehnten) Tag eines jeden Monats eine Rechnung auszustellen, in der alle Gebühren gemäß dem Vertrag aufgeführt sind, und sie an den Auftraggeber oder an den Auftraggeber vertretenden Verwalter zu schicken.
   \item Der Auftragnehmer hat sicherzustellen, dass die Informationen über die erbrachten Leistungen in den Rechnungen, die an jeden Wohnungseigentümer gehen, klar und umfassend dargestellt werden, wobei die Sanierungsgebühr und die Betriebs- und Wartungsgebühr des Auftragnehmers gesondert ausgewiesen werden.
   \item Die erste Zahlung der Gebühren wird 1 (ein) Monat nach Unterzeichnung des Liefer- und Abnahmeprotokolls fällig. Bis dahin ist der Auftraggeber verpflichtet, alle anfallenden Neben- und Gemeinkosten zu tragen.
   \item Der Auftraggeber (jeder Wohnungseigentümer) zahlt die Gebühren an den Auftragnehmer (oder an einen vom Auftragnehmer benannten Dritten) direkt oder über den Verwalter auf der Grundlage der vom Verwalter ausgestellten Rechnungen für alle Nebenkosten und sonstigen betrieblichen Aufwendungen für die Instandhaltung des Gebäudes, zu denen auch die dem Auftragnehmer zustehenden Gebühren gehören. Der Auftraggeber hat die Gebühren gemäß der üblichen Praxis des Verwalters zu zahlen, spätestens jedoch innerhalb von 15 (fünfzehn) Tagen nach Erhalt der Rechnung durch Überweisung des erforderlichen Betrages auf das vom Verwalter angegebene Bankkonto.
   \item Der Auftragnehmer oder sein beauftragter Verwalter administriert Informationen im Hinblick auf die Abrechnung im Zusammenhang mit diesem Vertrag unter anderem durch:
   \begin{enumerate}
   \item Erfassung aller Informationen über die Rechnungen, die an jeden der Wohnungseigentümer ausgestellt werden, einschließlich ihrer Beträge;
   \item Führung von Aufzeichnungen über die Rechnungszahlungen und ständige Aktualisierung der Verbindlichkeiten jedes Wohnungseigentümers, falls vorhanden;
   \end{enumerate}
   \item Der Auftraggeber hat dem Auftragnehmer oder einem seiner Abtretungsempfänger auf Verlangen des Auftragnehmers oder eines seiner Abtretungsempfänger (soweit zutreffend) den aktuellen Bericht über die Zahlungen der Wohnungseigentümer des Gebäudes und die Liste der Schuldner zur Verfügung zu stellen.
   \end{enumerate}

  \subsection{LAUFZEIT DES VERTRAGES}
  \begin{enumerate}
   \item Die Laufzeit dieses Vertrages beginnt mit dem Datum dieses Vertrages und bleibt bis zum Ende der Leistungsperiode in vollem Umfang in Kraft, vorbehaltlich einer vorzeitigen Beendigung, wie in diesem Vertrag vorgesehen.
   \item Die Laufzeit dieses Vertrages kann durch schriftliche Zustimmung der Parteien verlängert werden, insbesondere haben die Parteien das Recht, das Anfangsdatum und das Datum der Inbetriebnahme durch schriftliche Zustimmung der Parteien vorzuziehen oder zu verschieben.
   \item Der Auftragnehmer hat mit der Durchführung der Bau- und Installationsarbeiten der Maßnahmen zum Anfangsdatum zu beginnen und sie innerhalb der vorgesehenen Bauzeit durchzuführen. Am Ende der Bauzeit unterzeichnen Auftraggeber und Auftragnehmer das Liefer- und Abnahmeprotokoll.
   \item Die Leistungsperiode und die Zahlungsfrist beginnen mit der Unterzeichnung des Liefer- und Abnahmeprotokolls.
   \item Bei Versäumnis oder Fahrlässigkeit des Auftraggebers, wie z. B. wenn dem Auftragnehmer nicht alle fälligen Unterlagen und/oder der Zugang zum Gebäude zur Verfügung gestellt werden, und/oder bei Ereignissen höherer Gewalt gelten das Anfangsdatum, die Bauzeit und die Leistungsperiode automatisch um den Zeitraum der Verzögerung verschoben. Diese Änderungen sind von den Parteien schriftlich zu vereinbaren und der Vertrag ist zu ändern.
   \end{enumerate}

  \subsection{LATENTE BEDINGUNGEN}
  \begin{enumerate}
    \item Wenn der Auftragnehmer während der Bauzeit von latenten Bedingungen Kenntnis erlangt, die sich auf die Fertigstellung der Maßnahmen auswirken, muss der Auftragnehmer (als Voraussetzung für einen etwaigen Anspruch auf mehr Zeit oder Geld) dem Auftraggeber innerhalb von 5 (fünf) Werktagen eine schriftliche Mitteilung machen über:
    \begin{enumerate}
   \item die aufgetretenen latenten Bedingungen und darüber, inwiefern sie sich wesentlich vom Zustand des Gebäudes unterscheiden, den ein kompetenter und erfahrener Auftragnehmer zum Zeitpunkt des Vertrages unter Anwendung einer guten Branchenpraxis vernünftigerweise vorhersehen kann;
   \item die zusätzlichen Arbeiten und Ressourcen, die nach Ansicht des Auftragnehmers zur Bewältigung der latenten Bedingungen erforderlich sind;
   \item die Zeit, die nach Schätzung des Auftragnehmers zur Bewältigung der latenten Bedingungen erforderlich ist, und die von ihm erwartete Verzögerung bei der Fertigstellung;
   \item die angemessene Schätzung des Auftragnehmers über die Kosten der Maßnahmen, die zur Bewältigung der latenten Bedingungen erforderlich sind; und
   \item alle anderen Angaben, die der Auftraggeber vernünftigerweise benötigt.
   \end{enumerate}
    \item Die durch eine latente Bedingung verursachte Verzögerung kann eine Verlängerung der Bauzeit rechtfertigen, wenn diese:
    \begin{enumerate}
   \item den Auftragnehmer dazu veranlasst, zusätzliche Arbeiten durchzuführen;
   \item den Auftragnehmer dazu veranlasst, zusätzliche Materialien zu verwenden; oder
   \item zusätzliche Kosten (einschließlich, aber nicht beschränkt auf, die Kosten der Verzögerung oder Störung) verursacht, die der Auftragnehmer zum Zeitpunkt der Unterzeichnung des Vertrages unter Anwendung einer guten Branchenpraxis nicht vorausgesehen hat und vernünftigerweise nicht voraussehen konnte.
   \end{enumerate}
   \item Der Auftraggeber hat alle tatsächlichen Kosten zu tragen, die im Zusammenhang mit den latenten Bedingungen entstehen und zwischen den Parteien vereinbart werden. Wenn der Auftraggeber nicht wünscht, dass der Auftragnehmer wie angekündigt vorgeht, muss der Auftraggeber dem Auftragnehmer unverzüglich mitteilen, von einem weiteren Vorgehen abzusehen, und der Auftragnehmer muss einer solchen Mitteilung nachkommen. Der Auftraggeber und der Auftragnehmer können verhandeln und sich auf eine andere Weise zur Bewältigung der latenten Bedingung einigen einschließlich, aber nicht beschränkt auf das Erbringen zusätzlicher notwendiger Arbeiten durch andere Unternehmen und deren Bezahlung durch den Auftraggeber.
   \item Bei Vorliegen von latenten Bedingungen ist der Auftragnehmer berechtigt, eine Verlängerung der Bauzeit zu verlangen, die der zur Lösung der latenten Bedingungen erforderlichen Zeit entspricht. Der Auftragnehmer ist berechtigt, die Mehrkosten, die direkt oder indirekt aus latenten Bedingungen resultieren, zurückerstattet zu bekommen.
   \item Der Auftragnehmer ist nicht berechtigt, eine Anpassung der Energieeinsparungsgarantie zu verlangen, den Umfang der Sanierungsarbeiten zu reduzieren oder die in diesem Vertrag vereinbarten Gebühren aufgrund latenter Bedingungen zu ändern.
  \end{enumerate}

  \subsection{MESSUNG UND VERIFIZIERUNG UND DATENMANAGEMENT}
  \begin{enumerate}
   \item Der Auftragnehmer muss alle Tätigkeiten zur Messung und Verifizierung auf der Grundlage des auf der EPC-Plattform sunshineplatform.eu verfügbaren IPMVP (International Performance Measurement and Verification Protocol) durchführen.
   \item Alle Tätigkeiten zur Messung und Verifizierung sollen klar und vollständig offengelegt und für alle Parteien transparent sein.
   \item Der Auftragnehmer hat dem Auftraggeber während der Leistungsperiode einen Jahresbericht vorzulegen. Dieser Jahresbericht dokumentiert die Berechnung der Gebühren sowie die vom Auftragnehmer durchgeführten Betriebs- und Wartungsarbeiten und gibt an, ob die Energieeinsparungsgarantie auf der Grundlage der Messung und Verifizierung während der Abrechnungsperiode erfüllt wurde. Der Bericht hat ausreichende Informationen über die Energieeinsparungen, die sich aus den durchgeführten Maßnahmen ergeben, und über die Berechnung zur Ermittlung der Energieeinsparungen zu enthalten. Der Jahresbericht ist vom Auftragnehmer dem Auftraggeber jährlich spätestens 20 (zwanzig) Werktage nach Ablauf der Abrechnungsperiode zuzustellen. Dieser Prozess kann über die EPC-Plattform sunshineplatform.eu erfolgen.
   \item Hat der Auftraggeber Einwände gegen die im Jahresbericht gezogenen Schlussfolgerungen, so hat er dies dem Auftragnehmer innerhalb von 15 (fünfzehn) Werktagen nach Erhalt des Berichts oder Erhalt der Mitteilung von der EPC-Plattform sunshineplatform.eu mitzuteilen. Der Auftraggeber hat dem Auftragnehmer Gründe für seine Einwände mitzuteilen. Der Auftragnehmer hat innerhalb der folgenden 15 (fünfzehn) Werktage nach Eingang der Einwände des Auftraggebers die erforderlichen Änderungen vorzunehmen und den Auftraggeber darüber zu informieren.
   \item Jede ungerechtfertigte Beeinträchtigung oder Manipulation der im Gebäude durchgeführten Maßnahmen durch den Auftraggeber, die zu einer Verringerung der Höhe der Energieeinsparungen führt, ist bei der Messung und Verifizierung der Energieeinsparungsgarantie des Vertrages zu berücksichtigen und dient dazu, die Garantie anteilig neu anzupassen.
   \item Der Auftraggeber akzeptiert und  gestattet die Nutzung folgender Daten durch den Auftragnehmer oder einen anderen von ihm beauftragten Dritten mit den Rechten und Pflichten aus diesem Vertrag:
   \begin{enumerate}
   \item anonyme Daten und Informationen über den Energieverbrauch des Gebäudes, die vom Auftraggeber zur Verfügung gestellt oder vom Auftragnehmer eingeholt werden, zu Zwecken des Benchmarking und der Erstellung einer nationalen, regionalen oder internationalen Datenbank oder für die Verwendung durch den Auftragnehmer als Referenz oder für einen mit dem Auftraggeber vereinbarten internen Zweck;
   \item personenbezogene Daten, die vom Auftraggeber oder seinem Verwalter zum Zwecke der Erbringung der Dienstleistungen des Auftragnehmers zur Verfügung gestellt werden, und um diese an Dritte weiterzuleiten, an die  Rechte oder Pflichten aus diesem Vertrag abgetreten wurden, einschließlich an jede Partei, die die Forderungen aus diesem Vertrag forfaitiert oder die Entwicklung, Implementierung, Betrieb und Wartung der Online-EPC-Plattform sunshineplatform.eu zur Überwachung der Leistung der durchgeführten Maßnahmen verwaltet oder für diese verantwortlich ist.
   \end{enumerate}
   \item Während eines angemessenen Zeitraums ist der Auftragnehmer nach eigenem Ermessen, allein oder durch seine Beauftragten, berechtigt, ein Energiemanagementsystem oder allgemein eine Messtechnik zu installieren, zu betreiben, zu warten und einzuführen und zu jedem angemessenen Zeitpunkt in Übereinstimmung mit dem Vertrag auf diese installierten Geräte zuzugreifen.
   \item Der Auftragnehmer ist berechtigt, Temperaturlogger in Wohnungen zu installieren, wenn Beschwerden über die Nichteinhaltung der Komfortstandards eingehen. Wenn die Wohnungseigentümer des Gebäudes mit der Installation des genannten Loggers in ihren Wohnungen nicht einverstanden sind oder keinen angemessenen Zugang zu dieser Installation gewähren, haftet der Auftragnehmer nicht für die behauptete unsachgemäße Erfüllung des Vertrages in Bezug auf diese Wohnungen.
   \item Die vom Messgerät des Auftragnehmers erhobenen Daten haben informativen Charakter und können nicht als Grundlage für die Feststellung eines Vertragsbruchs oder der Einhaltung der Komfortstandards im Streitfall anerkannt werden.
   \end{enumerate}

\subsection{STREITBEILEGUNGSVERFAHREN}
\begin{enumerate}
   \item Etwaige Meinungsverschiedenheiten zwischen den Parteien sind zunächst auszuhandeln. Zu diesem Zweck haben die Parteien jedes Schreiben einer anderen Partei über eine Meinungsverschiedenheit schriftlich zur Kenntnis zu nehmen und angemessene Zeit zu investieren, um die Meinungsverschiedenheit persönlich oder durch einen hochrangigen Vertreter der betreffenden Parteien beizulegen.
   \item Hat der Auftraggeber (oder ein einzelner Wohnungseigentümer) Beschwerden über den Auftragnehmer (z. B. zu den Komfortstandards oder der Höhe der Energieeinsparungen, oder generell zu den durchgeführten Maßnahmen und erbrachten Energieeffizienzdienstleistungen), so hat dieser direkt oder über den Verwalter den Auftragnehmer darüber zu informieren. Der Auftragnehmer hat die Beschwerde zu überprüfen, eine Erklärung darüber abzugeben und die aufgetretenen Probleme zu beheben. Wenn das Problem mehr als 20 (zwanzig) Werktage nach der Benachrichtigung fortbesteht, muss der Auftraggeber veranlassen, dass ein Ausschuss, bestehend aus ordnungsgemäß bevollmächtigten Vertretern des Auftragnehmers, des Verwalters und des Auftraggebers, zusammenkommt und einen Erklärungsentwurf auf der Grundlage der Beschwerde und der vor Ort überprüften Fakten erstellt oder der ein anderes Ermittlungsverfahren gemäß den auf der EPC-Plattform sunshineplatform.eu verfügbaren Mediationsregeln durchführt.
   \item Wenn der Auftragnehmer Beschwerden über den Auftraggeber hat (z. B. wegen Beschädigung installierter Geräte), hat dieser den Auftraggeber und den Verwalter darüber zu benachrichtigen. Der Auftraggeber hat den Verursacher zu überprüfen und gegebenenfalls zu identifizieren und eine Erklärung über die Beschwerde zu erstellen und die aufgetretenen Probleme zu beheben. Wenn das Problem länger als 30 Tage nach der Benachrichtigung besteht, muss der Auftragnehmer veranlassen, dass ein Ausschuss, bestehend aus ordnungsgemäß bevollmächtigten Vertretern des Auftragnehmers, des  Verwalters und des Auftraggebers, zusammenkommt und einen Erklärungsentwurf auf der Grundlage der Beschwerde und der vor Ort überprüften Fakten erstellt oder der ein anderes Ermittlungsverfahren gemäß den auf der EPC-Plattform sunshineplatform.eu verfügbaren Mediationsregeln durchführt.
   \item Für das bei Tatsachenstreitigkeiten anzuwendende Ermittlungsverfahren gilt Folgendes:
   \begin{enumerate}
   \item Die aktuellen Komfortstandards (Umgebungstemperatur der einzelnen Wohnungen im Gebäude) gelten als ordnungsgemäß aufgezeichnet, wenn die Temperaturmessungen von einem unabhängigen zertifizierten Energieauditor (nach MK Nr. 382) und in Übereinstimmung mit der Norm LVS EN 12599 durchgeführt werden. Die Erklärung wird auf der Grundlage der vom unabhängigen zertifizierten Energieauditor gemessenen Ergebnisse erstellt.
   \item Allgemeine Probleme bei den durchgeführten Maßnahmen, wie z. B. fehlerhafte Geräte und/oder Mängel und Schäden an den Maßnahmen, oder bei der Berechnung der Energieeinsparungen gelten als ordnungsgemäß aufgezeichnet, wenn sie von einem unabhängigen Sachverständigen wie einem zertifizierten Energieauditor (gemäß MK Nr. 382) gemeldet werden.
   \item Alle Parteien werden mindestens 5 (fünf) Werktage vor jeder Messung durch einen Dritten informiert. Ein bevollmächtigter Vertreter der Parteien hat das Recht, sich an dem Messverfahren zur Erstellung der Erklärung zu beteiligen. Die Abwesenheit der bevollmächtigten Vertreter einer der Parteien stellt kein Hindernis für die Vorbereitung und Durchführung der Erklärung durch die Parteien dar.
   \item Die Unterzeichnung der Erklärung durch die Parteien gilt nicht als Anerkennung einer Verletzung dieses Vertrages und/oder als Verzicht auf die Rechte und Pflichten der Parteien aus diesem Vertrag. Die Kosten für die unabhängigen Dritten sind gleichmäßig zwischen den Parteien aufzuteilen.
   \item Eine Kopie jeder ausgeführten Erklärung ist dem Auftragnehmer, dem Verwalter und dem Wohnungseigentümer, der die Beschwerde eingereicht hat, zuzustellen.
   \end{enumerate}
   \item Gelingt es den Parteien nicht, eine Einigung zu erzielen, so haben die Parteien in ein formelles Mediationsverfahren gemäß den auf der EPC-Plattform sunshineplatform.eu verfügbaren Mediationsregeln einzutreten, die während der Laufzeit des Vertrages wirksam sind und zum Zeitpunkt der Streitigkeit bestehen. Bei Streitigkeiten zwischen den Parteien über technische Fragen kann jede Partei beantragen, dass die Streitigkeit über festgestellte Tatsachen gemäß den Verfahrensregeln des Untersuchungsausschusses, die auf der EPC-Plattform sunshineplatform.eu verfügbar sind, beigelegt wird.
   \item Gelingt es den Parteien nicht, nach dem Mediationsverfahren und/oder dem Ermittlungsverfahren eine einvernehmliche Einigung zu erzielen, so ist die Streitigkeit von einem Gericht der allgemeinen Gerichtsbarkeit Lettlands in Übereinstimmung mit den in Lettland geltenden Gesetzen und Vorschriften beizulegen. Ein Antrag ist bei dem für den Wohnort oder der Anschrift des Beklagten zuständigen Gericht zu stellen; ist der Wohnort jedoch nicht Lettland, ist der Antrag beim Bezirksgericht Riga City Centre oder beim Landesgericht Riga zu stellen.
\end{enumerate}

  \subsection{WARTUNG DER VOM AUFTRAGNEHMER DURCHGEFÜHRTEN MAẞNAHMEN}
   \begin{enumerate}
  \item Der Auftragnehmer hat die im Rahmen der Sanierungsarbeiten installierten Geräte (oder Teile davon) nach Ablauf der Nutzungsdauer (gemäß Betriebs- und Wartungshandbuch) während der Leistungsperiode des Vertrages zu ersetzen, zu reparieren oder zu überholen.
  \item Der Auftragnehmer hat Wartungsarbeiten an den Maßnahmen durchzuführen, die den Anforderungen und Empfehlungen des Herstellers für die jeweilige Instandhaltung entsprechen oder diese übertreffen, und zwar in Übereinstimmung mit den Besonderen Bedingungen dieses Vertrages.
  \end{enumerate}

   \subsection{VERSICHERUNG}
   \begin{enumerate}
     \item Zu Beginn der Bauzeit hat der Auftragnehmer das Gebäude zu einem Betrag zu versichern, der den Wiederherstellungswert des Gebäudes nicht unterschreitet, mit einer Mindestversicherung gegen Feuer, Erdbeben, Überschwemmungen, Wasserschäden, andere Naturkatastrophen mit Auswirkungen auf das Gebäude, Bauschäden durch Setzungen und umgestürzte Bäume. Für diese Versicherung gelten die folgenden Bestimmungen:
     \begin{enumerate}
   \item Der Auftragnehmer hat diese Versicherung mit einem Versicherer abzuschließen, der gemäß den für Lettland geltenden einschlägigen Ratings mindestens mit A+ eingestuft ist.
   \item Der Auftragnehmer hat dem Auftraggeber vor Beginn der Bauzeit eine Kopie dieser Versicherungspolice und die Zahlungsbestätigung der Versicherungsprämie vorzulegen.
   \item Der Auftraggeber ist als Begünstigter im Falle der Zahlung einer Versicherungsleistung von einem Betrag in der Höhe von mindestens dem Wiederherstellungswert des Gebäudes anzugeben.
   \item Die Bauarbeiten im Gebäude sind erst dann zu beginnen, wenn der Auftragnehmer eine rechtsgültig abgeschlossene Versicherungspolice vorlegt.
   \item Der Auftragnehmer hat die Versicherungspolice während der Laufzeit des Vertrages aufrechtzuerhalten und auf Verlangen des Auftraggebers dem Auftraggeber das Original der Versicherungspolice vorzulegen oder eine Kopie der Versicherungspolice zuzustellen oder über die EPC-Plattform sunshineplatform.eu den Zugang zu dem Dokument oder anderen Dokumenten, die die Währung und die Zahlung der Versicherungsprämie schlüssig bestätigen,  zu gewähren.
   \item Der Auftragnehmer schließt die Gebäudeversicherung für die gesamte Bauzeit auf eigene Kosten  ab. Nach Abschluss der Sanierungsarbeiten und nach Unterzeichnung des Liefer- und Abnahmeprotokolls werden die Versicherungskosten für die Restlaufzeit des Vertrages unter den Wohnungseigentümern im Verhältnis zur im Gebäude befindlichen Wohnfläche aufgeteilt und in die Rechnungen des Auftragnehmers für Betrieb und Wartung aufgenommen. Der von den Parteien bestellte Verwalter stellt sicher, dass in den Rechnungen des Auftraggebers an die Wohnungseigentümer diese Versicherungskosten enthalten sind.
   \end{enumerate}
   \item Darüber hinaus hat der Auftragnehmer während die Bauzeit über eine gültige Berufshaftpflichtversicherung in Höhe von mindestens 110 % der Gesamtinvestitionskosten der Sanierungsarbeiten zu verfügen.
   \end{enumerate}

   \subsection{ABTRETUNG VON FORDERUNGEN}
   \begin{enumerate}
   \item Dem Auftragnehmer steht das uneingeschränkte Recht zu, seine Rechte und Ansprüche auf Forderungen gegen den Auftraggeber im Rahmen dieses Vertrages an Dritte abzutreten. Insbesondere ist der Auftragnehmer berechtigt, die Forderungen aus der Sanierungsgebühr an jeden Abtretungsempfänger abzutreten, der eine Finanzierungs-, Forfaitierungs-, Abtretungs- oder jegliche andere Vereinbarung mit dem Auftragnehmer getroffen hat.
   \item Die Abtretung von Forderungen entbindet den Auftragnehmer nicht von seinen Verpflichtungen und Verbindlichkeiten aus diesem Vertrag. Der Abtretungsempfänger hat jedoch Eintrittsrechte in diesen Vertrag, falls der Auftragnehmer seinen Verpflichtungen aus diesem Vertrag nicht nachkommt. Eintrittsrechte zielen nur darauf ab, einen vertragsbrüchigen Auftragnehmer durch ein anderes Unternehmen zu ersetzen, das in der Lage ist, alle Verpflichtungen und Verbindlichkeiten aus diesem Vertrag zugunsten des Auftraggebers und des Abtretungsempfängers zu erfüllen.
   \item Im Falle einer solchen Abtretung hat der Auftragnehmer dem Auftraggeber innerhalb von 5 (fünf) Werktagen nach Abschluss des Abtretungsvertrages eine Bekanntmachung zu übermitteln.
   \item Dieser Vertrag ist persönlich an den Auftraggeber gebunden und darf vom Auftraggeber nicht ohne vorherige Benachrichtigung des Auftragnehmers abgetreten oder übertragen werden.
   \item Wenn der Auftragnehmer eine Fusion oder Übernahme durchführt, ein Liquidations- oder Konkursverfahren einleitet, bleibt der Vertrag gültig, und seine Bestimmungen sind für Rechtsnachfolger und Abtretungsempfänger des Auftragnehmers verbindlich.
   \end{enumerate}

   \subsection{EIGENTUMSTITEL AN DEN IM RAHMEN DER SANIERUNGSARBEITEN IM GEBÄUDE INSTALLIERTEN MAẞNAHMEN}
   \begin{enumerate}
   \item Der Eigentumstitel an den Maßnahmen, die vom Gebäude abgetrennt werden können, ohne Sachschäden zu verursachen, steht dem Auftragnehmer zu, wenn der Auftragnehmer im Rahmen der Sanierungsarbeiten einen Finanzbeitrag leistet. Wenn die Sanierungsarbeiten vollständig vom Auftraggeber finanziert werden, gehört der Eigentumstitel an den Maßnahmen dem Auftraggeber.
   \item Unabhängig davon, welche Partei den Eigentumstitel an der Maßnahme hält, darf der Auftraggeber die im Rahmen der Sanierungsarbeiten durchgeführten Maßnahmen nicht entfernen, belasten (leasen oder vermieten unter anderem), verpfänden oder zerstören, beschädigen oder manipulieren, und wird alle angemessenen Schritte ergreifen, um sicherzustellen, dass keiner der Wohnungseigentümer oder andere Besucher die im Rahmen der Sanierungsarbeiten durchgeführten Maßnahmen entfernt, belastet (vermietet oder unter anderem), verpfändet oder zerstört, beschädigt oder manipuliert.
   \item Der Auftragnehmer ist berechtigt, die Maßnahmen (oder Teile davon) ohne Zustimmung des Auftraggebers (insbesondere ohne Zustimmung des jeweiligen Wohnungseigentümers) zu verpfänden und zu seinem alleinigen oder zum Vorteil Dritter zu belasten, wenn:
   \begin{enumerate}
   \item der Auftragnehmer den Eigentumstitel an ihnen  hält;
   \item es technisch möglich ist, sie zu abzumontieren, ohne das Gebäude wesentlich zu beschädigen;
   \item die Verpfändung und/oder Belastung als Sicherheit für den Finanzbeitrag des Auftragnehmers im Rahmen dieses Vertrages erforderlich ist. Der Auftragnehmer ist nicht berechtigt, die Maßnahmen zur Beschaffung von Finanzmitteln für andere Zwecke als die Erfüllung dieses Vertrages zu verpfänden; und
   \item die Dauer der Verpfändung und/oder Belastung die Laufzeit des Vertrages nicht überschreitet.
   \end{enumerate}
     \item Wenn der Vertrag den Eigentumstitel an den Maßnahmen hält, gilt, nach Eingang aller aus diesem Vertrag an den Auftragnehmer fälligen Zahlungen, der Eigentumstitel an allen im Rahmen der Sanierungsarbeiten des Vertrages durchgeführten Maßnahmen als automatisch auf den Auftraggeber übertragen. Dies schließlich gegen einen nicht rückerstattungsfähigen Preis von 1 EUR (ein Euro), der bei Unterzeichnung dieses Vertrages im Voraus zu zahlen ist. Die Eigentumstitelübertragung auf den Auftraggeber wird durch eine vom Auftragnehmer und Auftraggeber unterzeichnete Übertragungserklärung bestätigt.
   \end{enumerate}

   \subsection{SOFTWARE UND GEISTIGE EIGENTUMSRECHTE}
   \begin{enumerate}
   \item Der Auftragnehmer hat sicherzustellen, dass alle Rechte am geistigen und gewerblichen Eigentum für die im Gebäude durchgeführten Maßnahmen, einschließlich Geräte, Materialien, Systeme, Software oder andere Dinge oder Dokumente, die der Auftragnehmer dem Auftraggeber im Rahmen dieses Vertrages zur Verfügung stellt, im Besitz des Auftragnehmers sind oder durch ihn lizenziert sind. Die Parteien vereinbaren, dass diese Rechte im Eigentum des Auftragnehmers bleiben und nicht auf den Auftraggeber übergehen. Der Auftragnehmer räumt dem Auftraggeber eine unbefristete, unwiderrufliche, nicht ausschließliche, gebührenfreie Lizenz (mit dem Recht zur Unterlizenzierung) ein, die genannten Rechte am geistigen und gewerblichen Eigentum im Zusammenhang mit der Nutzung des Gebäudes, und nicht für eine andere Nutzung, zu verwenden.
   \item Der Auftraggeber darf keine Software ändern, kopieren, zurückentwickeln oder mit einer anderen Software zusammenführen, die der Auftragnehmer im Rahmen der Sanierungsarbeiten zur Verfügung gestellt hat. Während der Laufzeit dieses Vertrages stellt der Auftragnehmer dem Auftraggeber Benutzerhandbücher, technische Informationen sowie alle Updates und Revisionen der bereitgestellten Software zur Verfügung.
   \item Der Auftragnehmer stellt den Auftraggeber von allen Ansprüchen frei, für die der Auftraggeber haftbar ist, in Bezug auf die Verletzung von Rechten an geistigem Eigentum Dritter im Zusammenhang mit jeglichen Teilen der vom Auftragnehmer gelieferten Rechte am geistigen und gewerblichen Eigentum. Die Verpflichtung des Auftragnehmers, den Auftraggeber von solchen Ansprüchen freizustellen, setzt voraus, dass der Auftraggeber:
   \begin{enumerate}
   \item den Auftragnehmer über die Forderung unverzüglich schriftlich benachrichtigt;
   \item die Forderung nicht anerkennt oder die Verteidigung des Auftragnehmers gegen die Forderung oder die Fähigkeit des Auftragnehmers, über einen zufriedenstellenden Vergleich zu verhandeln, beeinträchtigt;
   \item dem Auftragnehmer die Möglichkeit gibt, auf Kosten des Auftragnehmers die Gestaltung der Verteidigung und etwaiger Verhandlungen zur Beilegung der Forderung zu kontrollieren; und
   \item dem Auftragnehmer (auf Kosten des Auftragnehmers) die Hilfe und Information gewährt, die der Auftragnehmer vernünftigerweise verlangen kann, um den Auftragnehmer bei der Gestaltung der Verteidigung und den Verhandlungen zur Beilegung der Forderung zu unterstützen.
   \end{enumerate}
   \item Der Auftragnehmer hat nach seiner Wahl den die Rechte am geistigen und gewerblichen Eigentum verletzenden Teil durch einen nicht verletzenden Teil zu ersetzen oder zu ändern oder dem Auftraggeber das Recht zu verschaffen, diesen eigentumsverletzenden Teil zu verwenden. Die in dieser Klausel dargelegten Rechtsbehelfe sind das einzige und ausschließliche Rechtsmittel bei Verletzung von Rechten des geistigen Eigentums.
   \end{enumerate}

   \subsection{ÄNDERUNGEN IN DER NUTZUNG DES GEBÄUDES}
   \begin{enumerate}
   \item Das Gebäude ist in den Besonderen Bedingungen des Vertrages beschrieben, einschließlich seiner Nutzung, Fläche und Größe. Wenn sich Umstände, auf denen die Berechnungen des Auftragnehmers basieren, auf Initiative des Auftraggebers oder mit Zustimmung oder Genehmigung des Auftraggebers ändern, hat die Änderung keine Auswirkung auf den Auftragnehmer und die Erfüllung des Vertrages. Nutzungsänderungen des Gebäudes und Umbauten des Gebäudes sind unter Berücksichtigung wirtschaftlicher Gesichtspunkte (insbesondere Kostenänderungen) zu bewerten und der Vertrag entsprechend an die neuen Gegebenheiten anzupassen.
   \item Zu den Nutzungsänderungen des Gebäudes gehören:
   \begin{enumerate}
   \item Vergrößerung oder Verkleinerung der Oberfläche des Gebäudes;
   \item Zusammenfügen, Beschädigung oder Demontage von entsprechenden Geräten oder anderen Anlagen, wenn sie zu einer wesentlichen Erhöhung oder Verringerung des Energieverbrauchs oder anderer technischer Parameter des Gebäudes führen;
   \item Nutzungsänderungen des Gebäudes (z. B. Umwandlung der Wohnfläche in Läden, Geschäfte, Restaurants und Büros oder neue Nutzung von nicht verwendeten/unbewohnten Wohnungen), die den Energieverbrauch des Gebäudes beeinflussen.
   \end{enumerate}
   \end{enumerate}

   \subsection{VERTEILUNG VON ENTKOPPTEN UND/ODER VERTEILTEN AUSRÜSTUNGEN UND MATERIALIEN}
   \begin{enumerate}
   \item Der Auftragnehmer hat die Entsorgung der im Rahmen dieses Vertrages anfallenden Abfälle auf eigene Kosten in Übereinstimmung mit den einschlägigen Gesetzen und Vorschriften der Republik Lettland über die Abfallentsorgung zu veranlassen.
   \item Der Auftragnehmer hat den Auftraggeber spätestens innerhalb von 5 (fünf) Werktagen vor der ersten geplanten Abfallentsorgung schriftlich zu benachrichtigen. Diese Benachrichtigung umfasst alle im Gebäude installierten Geräte, Materialien und sonstigen Sachgüter, die für die Durchführung und Installation der Maßnahmen während der Bauzeit demontiert und ersetzt werden müssen.
   \item Wenn der Auftraggeber eines der vom Auftragnehmer während der Bauzeit entfernten oder demontierten Geräte, Materialien oder sonstigen Sachgüter verwenden möchte, soll er den Auftragnehmer benachrichtigen und auf eigene Kosten Abholung und Transport organisieren.
   \end{enumerate}

   \subsection{VERBINDLICHKEITEN}
   \begin{enumerate}
   \item Der Auftragnehmer haftet für die rechtzeitige Umsetzung der Maßnahmen in der vereinbarten Bauzeit. Bei Nichteinhaltung dieser Haftung durch den Auftragnehmer hat der Auftraggeber Anspruch auf pauschalen Schadenersatz in Höhe von 0,02\% der gesamten geplanten Investitionskosten pro Tag. Der pauschalierte Schadenersatz darf 10\% (zehn Prozent) der geplanten Investitionskosten nicht überschreiten.
   \item Der Auftraggeber haftet für die rechtzeitige Zahlung der fälligen Kosten und Gebühren im Rahmen dieses Vertrages. Der Auftragnehmer hat Anspruch auf Schadenersatz bei Zahlungsverzug. Der Schadenersatz bei Zahlungsverzug entspricht 0,1\% pro Tag des überfälligen Betrages.
   \item Wenn der Auftraggeber mehr als 90 (neunzig) Tage lang keine nach dem Vertrag fälligen Zahlungen leistet, in deren Verlauf die Streitbeilegungsverfahren im Rahmen dieses Vertrages ordnungsgemäß und tatsächlich eingesetzt wurden, ist der Auftragnehmer berechtigt, den Vertrag wegen Verzugs des Auftraggebers und Vertragsverletzung zu kündigen.
   \item Der Auftragnehmer haftet dafür, dass das Gebäude den in diesem Vertrag festgelegten Komfortstandards entspricht. Wenn in einer der Wohnungen während der Heizperiode die Raumtemperatur durchschnittlich 2 (zwei) Grad auf der Celsius-Skala (unter Berücksichtigung der Genauigkeit der Instrumente) unter den in diesem Vertrag festgelegten Komfortstandards liegt, ist der Auftragnehmer dafür verantwortlich, den Verwalter anzuweisen, die Rechnung des Auftraggebers für den jeweiligen Wohnungseigentümer wie folgt zu reduzieren:
   \begin{enumerate}
   \item Rabatt von 5\% (fünf Prozent) der Energiegebühr für jedes Grad Celsius für jeden Monat der Heizsaison, wenn die Temperatur unter den vereinbarten Komfortstandards lag.
   \item Die Ermittlung der Raumtemperatur und Feststellung, ob das Temperaturniveau unter den Komfortstandards lag, haben gemäß den Streitbeilegungsverfahren dieses Vertrages zu erfolgen.
   \item Der Auftragnehmer hat den Rabatt nicht anzuwenden, wenn der Rückgang der Raumtemperatur in der Wohnung eingetreten ist: (i) infolge von Handlungen oder Unterlassungen der Bewohner oder der Wohnungseigentümer, die gegen diesen Vertrag verstoßen; (ii) infolge der Nichterfüllung der Verpflichtungen des Auftraggebers; oder (iii) aus anderen Gründen, die nicht auf das Verschulden des Auftragnehmers zurückzuführen sind.
   \end{enumerate}
   \item Der Auftraggeber haftet für Schäden, Manipulationen oder Pfusch, Vandalismus, Sabotage, Diebstahl (außer durch den Auftragnehmer oder durch diejenigen, für die der Auftragnehmer verantwortlich ist) an den Maßnahmen, insbesondere wenn diese die Höhe der Energieeinsparungen, die vereinbarten Komfortstandards oder die Sicherheit der Personen, die das Gebäude bewohnen und nutzen, beeinträchtigen. In diesem Fall ist der Auftraggeber verpflichtet:
   \begin{enumerate}
   \item dem Auftragnehmer die Kosten für die Wiederherstellung der betreffenden Maßnahme vollständig zu ersetzen;
   \item dem Auftragnehmer einen Schadenersatz in Höhe von 10\% (zehn Prozent) der Kosten der Wiederherstellung zur Deckung der Verwaltungskosten des Auftragnehmers zu zahlen.
   \item Die Kosten für die Wiederherstellung sind auf der Grundlage der zum Zeitpunkt der Berechnung geltenden Marktpreise zu berechnen.
   \item Die Ermittlung der Haftung des Auftraggebers für Schäden, Manipulationen oder Pfusch an den Maßnahmen ist nach den Streitbeilegungsverfahren dieses Vertrages festzustellen.
   \end{enumerate}
   \item Der Auftragnehmer hat den Auftraggeber von jeglicher Haftung, Kosten, Aufwendungen, Schäden, Gebühren freizustellen, die ihm aufgrund einer Klage oder Beschwerde, Verwaltungs- oder Gerichtsklage gegen den Auftraggeber durch den Staat oder Verwaltungsbehörden oder Dritte entstehen und die sich aus Handlungen des Auftragnehmers oder aus einer möglichen Verletzung von geistigen Eigentumsrechten im Zusammenhang mit den vom Auftragnehmer durchgeführten Maßnahmen ergeben. Der Auftraggeber hat vom Auftragnehmer eine Rückerstattung aller Kosten und Aufwendungen zu erhalten, die zur Behebung aller direkten Schäden vernünftigerweise erforderlich sind, die sich aus Handlungen des Auftragnehmers im Verstoß gegen geltendes Recht ergeben. Die Rückerstattung ist innerhalb von 30 Werktagen nach Eingang der jeweiligen Aufforderung des Auftraggebers an den Auftragnehmer unter ausdrücklicher Angabe des fälligen Betrages zu dokumentieren und zu zahlen.
   \item Entgelte für die Erbringung von Dienstleistungen von allgemeinem wirtschaftlichem Interesse (einschließlich der Wärmeversorgung) und auf jede Partei anwendbare Sanktionen, die in den in der Republik Lettland geltenden Rechts- und Verwaltungsvorschriften festgelegt sind, befreien die Parteien nicht von den Verpflichtungen dieses Vertrages, einschließlich der Haftung der Parteien und der Zahlung von anzuwendenden Vertragsstrafen, Entschädigungen und Pönalen im Rahmen dieses Vertrages.
   \item Die Zahlung von Vertragsstrafen, Entschädigungen und Pönalen entbindet die schuldige Partei nicht von der Erfüllung ihrer Verpflichtungen aus dem Vertrag.
   \end{enumerate}

   \subsection{KÜNDIGUNG DES VERTRAGES}
   \begin{enumerate}
   \item Für den Fall, dass eine der Parteien gegen eine wesentliche Bestimmung dieses Vertrages verstößt, kann die nicht vertragsbrüchige Partei diesen Vertrag unverzüglich kündigen und von der vertragsbrüchigen Partei verlangen, die nicht vertragsbrüchige Partei gemäß diesem Vertrag zu entschädigen.
   \item Kündigung des Vertrags vor dem Anfangsdatum und bevor  Investitionen für Bau- und Installationsarbeiten erfolgt sind:
   \begin{enumerate}
   \item Im Falle einer einseitigen Kündigung des Vertrages durch den Auftraggeber aufgrund eines wesentlichen Verzugs oder einer Vertragsverletzung durch den Auftragnehmer hat der Auftraggeber Anspruch auf eine Entschädigung in Höhe von 1\% der im Vertrag vorgesehenen Investitionskosten (ohne Umsatzsteuer).
   \item Im Falle einer einseitigen Kündigung des Vertrages durch den Auftragnehmer aufgrund von Verzug oder einer Vertragsverletzung durch den Auftraggeber hat der Auftragnehmer Anspruch auf eine Entschädigung in Höhe von 1\% der im Vertrag vorgesehenen Investitionskosten (ohne Umsatzsteuer).
   \item Im Falle einer einseitigen Kündigung des Vertrages durch den Auftraggeber aus anderen geschäftlichen oder kommerziellen Gründen, die nicht unbedingt mit diesem Vertrag verbunden sind, hat der Auftragnehmer Anspruch auf eine Entschädigung in Höhe von 1\% der im Vertrag vorgesehenen Investitionskosten (ohne Umsatzsteuer).
   \item Im Falle einer einseitigen Kündigung des Vertrages durch den Auftragnehmer aus anderen geschäftlichen oder kommerziellen Gründen, die nicht unbedingt mit diesem Vertrag verbunden sind, hat der Auftraggeber Anspruch auf eine Entschädigung in Höhe von 1\% der in dem Vertrag vorgesehenen Investitionskosten (ohne Umsatzsteuer).
   \end{enumerate}
   \item Kündigung des Vertrages, nachdem die Investitionskosten für Bau- und Installationsarbeiten der Maßnahmen erfolgt sind und durch den Finanzbeitrag des Auftragnehmers gedeckt wurden:
   \begin{enumerate}
   \item Im Falle einer einseitigen Kündigung des Vertrages durch den Auftraggeber aufgrund eines wesentlichen Verzugs oder einer Vertragsverletzung durch den Auftragnehmer hat der Auftraggeber nur den ausstehenden Betrag des vom Auftragnehmer geleisteten Finanzbeitrags unter Abzug von 3\% zu erstatten, der gemäß den im Vertrag festgelegten Leistungskriterien ordnungsgemäß funktioniert. Darüber hinaus hat der Auftraggeber das Recht, die gesamte Projektdokumentation zu erhalten, in der die bisher ausgeführten Arbeiten im Einzelnen aufgeführt sind, sowie alle Genehmigungen, Lizenzen oder sonstige Dokumente, die der Auftragnehmer gemäß diesem Vertrag und der Fertigstellung der dringendsten Arbeiten bekommen hat, alle Herstellergarantien, alle Unterlizenzen (und die Übertragung aller Lizenzen) für die Nutzung der erforderlichen Rechte an geistigem Eigentum und Software (einschließlich der installierten Software und gegebenenfalls alle Begleitdokumentationen, Informationen über Code, Quellcode, Dateien, Kalkulationen, elektronische Medien, Ausdrucke oder zugehörige Informationen), zusätzliche Schulungen für Dritte, die vom Auftraggeber ausdrücklich beauftragt wurden, falls die Durchführung der Sanierungsarbeiten abgeschlossen ist.
   \item Im Falle einer einseitigen Kündigung des Vertrages durch den Auftragnehmer aufgrund eines wesentlichen Verzugs oder einer Vertragsverletzung durch den Auftraggeber hat der Auftraggeber den ausstehenden Betrag des Finanzbeitrags zuzüglich einer Entschädigung in Höhe von 3\% des zurückzuzahlenden Betrages rückzuvergüten. Der Auftraggeber ist berechtigt, die gesamte Projektdokumentation zu erhalten, in der die bisher ausgeführten Arbeiten im Einzelnen aufgeführt sind, sowie alle Genehmigungen, Lizenzen oder sonstige Dokumente, die der Auftragnehmer gemäß diesem Vertrag erhalten hat, alle Herstellergarantien, alle Unterlizenzen (und die Übertragung aller Lizenzen) für die Nutzung der erforderlichen Rechte an geistigem Eigentum und Software (einschließlich der installierten Software und gegebenenfalls alle Begleitdokumentationen, Informationen über Code, Quellcode, Dateien, Kalkulationen, elektronische Medien, Ausdrucke oder zugehörige Informationen).
   \item Im Falle einer einseitigen Kündigung des Vertrages durch den Auftraggeber aus anderen geschäftlichen oder kommerziellen Gründen, die nicht unbedingt mit diesem Vertrag verbunden sind, hat der Auftragnehmer Anspruch auf eine Vergütung, die dem ausstehenden Betrag des vom Auftragnehmer geleisteten Finanzbeitrags entspricht, zuzüglich einer Entschädigung in Höhe von 3\% des zurückzuzahlenden Betrages.
   \item Im Falle einer einseitigen Kündigung des Vertrages durch den Auftragnehmer aus anderen geschäftlichen oder kommerziellen Gründen, die nicht unbedingt mit diesem Vertrag verbunden sind, hat der Auftragnehmer Anspruch auf eine Entschädigung, die dem ausstehenden Betrag des vom Auftragnehmer geleisteten Finanzbeitrags entspricht, abzüglich 3\% des berechneten Betrages.
   \end{enumerate}
   \item Der Auftragnehmer oder einer seiner Abtretungsempfänger hat dem Auftraggeber eine Rechnung über die berechnete Entschädigung auszustellen, aus der eindeutig hervorgeht, welche Informationen für die Kalkulation verwendet werden, und zwar auf der Grundlage der während der Leistungsperiode ausgestellten Zahlungspläne oder der Rechnungen, die während der Ausführung der Sanierungsarbeiten vor dem Datum der Kündigung des Vertrages bezahlt wurden . Der Auftraggeber hat innerhalb von 60 (sechzig) Tagen nach Ausstellung der Rechnung die dem Auftragnehmer oder einem seiner Vermögensverwalter, Abtretungsempfänger oder anderen juristischen Personen, die einseitig als rechtlich befugt identifiziert wurden, alle oder einen Teil der Rechte des Auftragnehmers aus diesem Vertrag zu erhalten, zustehende Entschädigung zu bezahlen.
   \item Die vorzeitige Kündigung des Vertrages ist von der kündigenden Partei schriftlich (Kündigung) mindestens 20 (zwanzig) Werktage im Voraus bekanntzugeben. Für den Fall, dass die Kündigung des Vertrages auf Verzug oder Vertragsbruch durch eine Partei zurückzuführen ist, muss eine gültige schriftliche Kündigung die im Rahmen der Streitbeilegungsverfahren dieses Vertrages unternommenen Schritte und die dazugehörige unterstützende Dokumentation enthalten.
   \item Der Auftraggeber ist jederzeit berechtigt, vom Auftragnehmer eine Kalkulation des dem Auftragnehmer zu erstattenden Betrages im Falle einer vorzeitigen Kündigung des Vertrages zu verlangen und zu erhalten.
   \item Im Allgemeinen entbindet die Kündigung des Vertrages die Parteien nicht von der Erfüllung der in dem Vertrag festgelegten Verpflichtungen, die vor dem Zeitpunkt der Kündigung des Vertrages eingetreten sind, es sei denn, die Parteien haben schriftlich andere Bestimmungen vereinbart oder der Vertrag sieht etwas anderes vor. Insbesondere entbindet die einseitige Kündigung des Vertrages durch den Auftraggeber im Falle eines wesentlichen Verzugs oder einer Vertragsverletzung durch den Auftragnehmer den Auftraggeber nicht von den Zahlungsverpflichtungen der Rechnungen, die für die Perioden vor dem Datum der Kündigung des Vertrages ausgestellt wurden.
   \item Reorganisation, Wechsel von Gesellschaftern und/oder Eigentümern, Wechsel in der Geschäftsführung der Parteien einschließlich Wechsel von Wohnungseigentümern des Gebäudes stellen keinen Grund für die Kündigung des Vertrages oder die Nichterfüllung der im Vertrag enthaltenen Verpflichtungen dar.
   \item Zusätzlich zu den Bestimmungen des Vertrages können die Parteien den Vertrag jederzeit nach gegenseitiger schriftlicher Vereinbarung über die Kündigungsbedingungen beenden.
   \item Die entschädigungsberechtigte Partei hat Schadenersatzansprüche entweder durch Ausübung ihrer Rechte aus diesem Vertrag geltend zu machen oder nach den in der Republik Lettland geltenden einschlägigen Gesetzen und Vorschriften. Die berechtigte Partei erhält jedoch keine doppelte Entschädigung für denselben Verzug oder Verstoß.
   \item Die Parteien können vereinbaren, die dem Auftragnehmer gänzlich oder teilweise gehörenden Maßnahmen aus dem Gebäude zu demontieren, wenn der Vertrag aufgrund von Umständen vorzeitig beendet wird und wenn der Wert der jeweiligen Maßnahmen von der betroffenen Partei akzeptiert und als Entschädigung anerkannt wird. Die Möglichkeit, die Maßnahmen unter den in diesem Absatz dargelegten Umständen zu demontieren, lässt alle Schadenersatz-, Kosten- und Aufwendungsansprüche unberührt, auf die die Parteien bei vorzeitiger Kündigung dieses Vertrages Anspruch haben.
   \end{enumerate}

   \subsection{EREIGNISSE HÖHERER GEWALT}
   \begin{enumerate}
     \item Als höhere Gewalt gilt jede Notfallsituation oder jedes unvorhersehbare Ereignis, das sich durch alle folgenden Merkmale auszeichnet:
     \begin{enumerate}
   \item die Parteien sind nicht in der Lage, diese vorherzusagen und zu beeinflussen;
   \item es behindert die Parteien bei der Erfüllung ihrer Verpflichtungen;
   \item es kann nicht als Fehler oder Fahrlässigkeit der Parteien angesehen werden;
   \item es kann als unüberwindbar nachgewiesen oder anerkannt werden, obwohl die Partei(en) angemessene Anstrengungen unternommen hat/haben, es zu verhindern.
   \end{enumerate}
   \item Ereignisse höherer Gewalt umfassen insbesondere, aber nicht ausschließlich: Krieg, Naturkatastrophen und Rechtsakte der staatlichen Verwaltung.
   \item Ereignisse höherer Gewalt sind NICHT: Mängel der Maßnahmen; Dienstleistungen, die nicht die vereinbarte Qualität oder Quantität aufweisen; vom Auftragnehmer verwendete, bereitgestellte oder installierte Geräte oder Materialien oder Verzögerungen bei deren Betrieb (sofern sie nicht durch Ereignisse höherer Gewalt verursacht werden); Streitigkeiten des Auftraggebers, Streiks, finanzielle Schwierigkeiten oder dergleichen, die die Partei betreffen, die sich auf ein Ereignis höherer Gewalt beruft.
   \item Die Parteien haften nicht für die vollständige oder teilweise Nichterfüllung der Verpflichtungen aus der Vereinbarung, wenn dies auf Ereignisse höherer Gewalt zurückzuführen ist. Die Partei, die sich auf ein Ereignis höherer Gewalt beruft, hat dies der anderen Partei nachzuweisen.
   \item Die Partei (die "betroffene Partei"), die an der Erfüllung ihrer Verpflichtungen aus diesem Vertrag gehindert wird, hat die andere Partei unverzüglich, spätestens innerhalb von 3 (drei) Werktagen über ein Ereignis höherer Gewalt zu informieren, sobald das Ereignis von der betroffenen Partei vorhergesehen wird oder ihr bekannt wird, und hat eine Erklärung der kommenden oder eingetretenen Situation und eine Beschreibung des Ereignisses, die mögliche Dauer, die geschätzten Folgen und die voraussichtliche Lösung derselben vorzulegen.
   \item Die Parteien haben alle erforderlichen Handlungen gemeinsam oder für sich durchzuführen, um die Auswirkungen des Ereignisses höherer Gewalt abzumildern, und haben angemessene Maßnahmen zur Beseitigung etwaiger entstandener Schäden zu ergreifen.
   \item Dauert das Ereignis höherer Gewalt länger als 6 (sechs) Monate ununterbrochen an und ist mit seiner Beendigung nicht innerhalb von weiteren 3 (drei) Monaten zu rechnen, so hat der Auftragnehmer oder der Auftraggeber das Recht, den Vertrag einseitig zu kündigen.
   \end{enumerate}

   \subsection{VERTRAULICHKEIT}
   \begin{enumerate}
   \item Informationen, die im Laufe des Abschlusses oder während der Erfüllung des Vertrages erhalten wurden und die Dritten nicht allgemein zugänglich sind und deren Offenlegung, von der die empfangende Partei Kenntnis hat oder hätte haben müssen, rechtmäßige Rechte oder Interessen der offenlegenden Partei beeinträchtigen können, gelten als vertraulich.
   \item Die Parteien verpflichten sich, vertrauliche Informationen der anderen Partei nicht an Dritte weiterzugeben sowie Daten der anderen Partei, die zu Wettbewerbszwecken, oder um rechtswidrige Handlungen zu begehen, verwendet werden könnten, sowohl während der Gültigkeit des Vertrages als auch für 3 (drei) Jahre nach Ablauf der Gültigkeit des Vertrages nicht weiterzugeben.
   \item Informationen, die von Dritten öffentlich bekannt gemacht werden, ohne dass die Parteien gegen die Bestimmungen des Vertrages verstoßen, gelten nicht als vertraulich.
   \item Die Parteien können vertrauliche Informationen an Dritte weitergeben, um die Verpflichtungen aus dem Vertrag zu erfüllen. Wenn die Parteien vertrauliche Informationen auf der Grundlage dieser Klausel weitergeben, haben sie sicherzustellen, dass der Dritte dieselben Vertraulichkeitsverpflichtungen einhält, die in diesem Vertrag festgelegt sind.
   \item Die Offenlegung vertraulicher Informationen, die in Übereinstimmung mit den in der Republik Lettland geltenden Gesetzen und Vorschriften erforderlich sind, stellt keinen Verstoß gegen den Vertrag dar.
   \item Zu Werbezwecken und zur Information der Öffentlichkeit sind der Auftragnehmer, alle seine Beauftragten und der Auftraggeber berechtigt, allgemeine Informationen über die Zusammenarbeit offenzulegen, unter anderem: Informationen über die Parteien, die bereits öffentlich zugänglich sind, die Art der Zusammenarbeit, die erzielten Energieeinsparungen und die Energieverbrauchsdaten. Dies, soweit die Offenlegung der Informationen nicht gegen die gesetzlichen Rechte und Interessen der anderen Partei in Bezug auf den Schutz vertraulicher Informationen verstößt. Hat eine Partei Zweifel an der Art der spezifischen Informationen, so wird diese Art von Informationen vor ihrer Offenlegung von der/den Partei(en) genehmigt, deren rechtmäßige Rechte und Interessen durch die Offenlegung dieser Informationen verletzt werden könnten, wenn diese Partei der Ansicht ist, dass diese Informationen nicht unter die Geheimhaltungspflicht des Vertrages fallen.
   \item Das vorstehend Genannte ist unbeschadet der ausdrücklichen Auflage für den Auftraggeber, einen Kunden, potenziellen Kunden oder Geschäftskontakt des Auftragnehmers und/oder andere im Aufbau befindliche Rechtsträger nicht aufzufordern oder zu beraten, ihre Geschäfte mit dem Auftragnehmer zu kürzen, zu stornieren, von ihm zurückzuziehen, zu begrenzen, zu reduzieren oder anderweitig einzuschränken.
   \item Die vorstehenden Bestimmungen berühren nicht das Recht des Auftragnehmers, Daten zu erheben, zu verarbeiten, zu speichern, umzuwandeln und die gesammelten Daten des Auftraggebers an seine beauftragten Finanzierungspartner zu übermitteln und zu verbreiten, um die Qualität seiner Dienstleistungen zu verbessern und um die Online-EPC-Plattform sunshineplatform.eu zu entwickeln, zu betreiben und zu pflegen, die alle Phasen und Teilnehmer bei einem typischen EPC-Projekt unterstützt.
   \end{enumerate}

   \subsection{ABSCHLUSS UND ÄNDERUNG DIESES VERTRAGES}
   \begin{enumerate}
   \item Der Vertrag tritt am Tag der Unterzeichnung durch alle Parteien unter den vorliegenden Allgemeinen Geschäftsbedingungen in Kraft und bleibt bis zur vollständigen Erfüllung aller seiner Verpflichtungen durch die Parteien in Kraft.
   \item Alle Änderungen, Aktualisierungen und Ergänzungen des Vertrages bedürfen der Schriftform im gegenseitigen Einvernehmen aller Parteien; sie werden nach der Unterzeichnung durch alle Parteien wirksam und sind dem vorliegenden Vertrag in Form von Anhängen beizufügen.
   \item Alle übrigen Bestimmungen der Allgemeinen Geschäftsbedingungen oder der jeweiligen Anhänge bleiben in vollem Umfang in Kraft. Vereinbarte Abweichungen gelten nur für den Teil des Vertrages, für den diese Abweichungen geregelt wurden.
   \item Der Vertrag gilt als gekündigt, wenn der Auftraggeber alle Zahlungen der Gebühren an den Auftragnehmer vollständig geleistet hat und der Auftragnehmer seine Verpflichtungen erfüllt hat.
   \item Treten während der Laufzeit des Vertrages Änderungen der lettischen Gesetze und Vorschriften in Kraft, die die Erfüllung der Verpflichtungen aus diesem  Vertrag ganz oder teilweise unmöglich machen, so berührt dies nicht die Gültigkeit der übrigen Verpflichtungen aus diesem Vertrag. In diesem Fall haben die Parteien geeignete Änderungen am Vertrag vorzunehmen, um die wirtschaftlichen Auswirkungen für die Parteien des Vertrages zu minimieren.
   \end{enumerate}

   \subsection{VERTRETUNG DER PARTEIEN}
   \begin{enumerate}
   \item Im Rahmen dieses Vertrages werden die Parteien durch ihre legitimen Vertreter (für juristische Personen) oder durch Personen, die in diesem Vertrag ausdrücklich genannt werden, vertreten. Nur die Personen, die unter den Besonderen Bedingungen dieses Vertrages genannt werden, sind berechtigt, den Auftraggeber bzw. den Auftragnehmer zu vertreten.
   \item Das Abkommen wird in 3 (drei) Originalen in lettischer Sprache mit gleicher Rechtswirkung erstellt und unterzeichnet. Die Parteien bestätigen mit ihrer Unterzeichnung, dass sie den Inhalt, die Bedeutung und die Folgen des Vertrages verstehen; sie erkennen an, dass dieser Vertrag korrekt und für beide Seiten vorteilhaft ist sowie alle Bestimmungen, Zusagen, Bedingungen und Absichtserklärungen zwischen den Parteien beinhaltet und dass sie den Vertrag freiwillig, ohne dass Zwang auf sie ausgeübt wurde, ausführen wollen.
   \end{enumerate}


   \subsection{BENACHRICHTIGUNGEN}
   \begin{enumerate}
   \item Form der Benachrichtigung: Alle unter diesem Vertrag erforderlichen und zulässigen Mitteilungen, Anfragen, Ansprüche, Aufforderungen, Forderungen und andere Kommunikationen sind an die Adressen der Parteien zuzustellen.
   \item Art der Benachrichtigung: Benachrichtigungen haben zu erfolgen (i) durch persönliche Zustellung, (ii) durch Botendienst am nächsten Tag, (iii) durch Einschreiben, (iv) per Fax, (v) durch elektronische Post (E-Mail) mit der Bitte um Empfangsbestätigung, an die in diesem Vertrag angegebene Adresse der Partei oder an eine andere von der Partei schriftlich angegebene Adresse, (vi) durch die von der EPC-Plattform sunshineplatform.eu angebotenen Benachrichtigungsdienste, vii) durch SMS-Nachrichten mit Empfangsbestätigung für Mitteilungen mit geringen Informationen  an die in diesem Vertrag angegebene Mobilfunknummer der Partei oder an andere von der Partei schriftlich angegebene Nummern.
   \item Erhalt der Benachrichtigung: Alle Mitteilungen werden wirksam, sobald (i) sie bei der Partei eingegangen sind, an die die Mitteilung erfolgt, oder (ii) am siebten (7.) Tag nach dem Versand, je nachdem, was zuerst eintritt.
   \end{enumerate}

\end{multicols}
