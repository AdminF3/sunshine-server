\begin{multicols}{2}
  [\section{TERMENI ŞI CONDIŢII GENERAL}]

\subsection{dEFINIŢII}
\begin{itemize}[label={}]
\item\textbf{Acord:} acest Contract de performanță energetică încheiat între Client și Contractant, inclusiv Condițiile specifice și anexele sale și Termenii și condițiile generale și dezvoltat și gestionat prin intermediul Sunshine platform - sunshineplatform.eu.
\item\textbf{Apartament:} o proprietate de apartament este o proprietate imobiliară autonomă separată legal într-o clădire rezidențială.
\item\textbf{Proprietar de apartament:} un proprietar de apartament este o persoană care a dobândit proprietatea apartamentului și a coroborat titlul cu cartea funciară.
\item\textbf{Ziua bancară:} este o zi în care într-un loc în care, în conformitate cu prevederile Contractului, trebuie efectuate transferuri bancare, băncile comerciale efectuează tranzacții bancare generale.
\item\textbf{Linie de bază:} înseamnă consumul de energie termică și apă caldă menajeră a clădirii, exprimat ca valoare medie anuală, care apare în perioada de bază.
\item\textbf{Perioada de bază:} este o perioadă de timp convenită reciproc, care reprezintă funcționarea clădirii înainte de punerea în aplicare a măsurilor.
\item\textbf{Clădire:} clădirea rezidențială cu mai multe apartamente în care Antreprenorul livrează lucrările de renovare și furnizează Serviciile conform prezentului Acord.
\item\textbf{Ziua lucrătoare:} o zi lucrătoare oficială care nu este o sărbătoare oficială sau o zi oficială de lucru în condițiile legii letone.
\item\textbf{Client:} Proprietarul (proprietarii) apartamentului Clădirii sau persoana autorizată care acționează în numele Proprietarului (apartamentelor).
\item\textbf{Standarde de confort:} ansamblul condițiilor și parametrilor climatului interior pe care Contractantul le garantează Clientului în baza prezentului Acord.
\item\textbf{Data începerii:} data la care începe perioada de construcție.
\item\textbf{Data punerii în funcțiune:} data la care părțile semnează Protocolul de livrare și acceptare pentru măsuri și data la care începe perioada de serviciu a acordului.
\item\textbf{Perioada de construcție:} perioada planificată de Antreprenor pentru punerea în aplicare a Măsurilor.
\item\textbf{Perioada de construcție:} începe de la data începerii și se încheie la data punerii în funcțiune.
\item\textbf{Antreprenor:} persoană juridică care se angajează în acest contract, Lucrează la renovare și furnizează Serviciile pe baza prevederilor Acordului.
\item\textbf{Protocolul de livrare și acceptare:} protocolul pregătit de antreprenor în conformitate cu reglementările letone și normele pentru punerea în funcțiune finală a măsurilor implementate în clădire de către contractant.
\item\textbf{Taxa de apă caldă menajeră:} taxa plătită de către client contractantului, care este datorată consumului propriu-zis de apă caldă menajeră, la tariful actual de energie termică.
\item\textbf{Energie:} un produs cu o anumită valoare - combustibil, energie termică, energie regenerabilă, energie electrică sau orice alt tip de energie.
\item\textbf{Auditul energetic:} acțiuni efectuate în scopul obținerii informațiilor despre modelul consumului de energie în clădiri sau grupuri de clădiri, proceduri sau echipamente, precum și pentru a măsura și verifica posibilitățile de economisire a energiei fezabile din punct de vedere economic și ale căror constatări sunt compilate în un raport.
\item\textbf{Garanția de consum de energie:} cantitatea de energie termică consumată în clădire pentru încălzirea spațiului și pierderile de circulație în perioada de serviciu obținută pe baza Garanției de economisire a energiei oferită de antreprenor și utilizată pentru determinarea consumului de energie termică plat.
\item\textbf{Serviciul de eficiență energetică (servicii):} un set de acțiuni efectuate de antreprenor, inclusiv punerea în aplicare a măsurilor în construirea, exploatarea și întreținerea măsurilor implementate, analize ale datelor privind consumul de energie, monitorizarea și evaluarea consumului de energie, în special, legate la îndeplinirea garanției de economisire a energiei.
\item\textbf{Economii de energie:} volumul de energie economisit, care se stabilește prin măsurarea și verificarea consumului înainte și după implementarea uneia sau mai multor măsuri de performanță energetică și realizate în clădire prin implementarea Măsurilor și furnizarea de performanță energetică.
\item\textbf{Garanția de economisire a energiei:} cantitatea minimă de economii de energie rezultată din implementarea măsurilor și din furnizarea de servicii de eficiență energetică, garantată în acord de către contractant și determinată în conformitate cu un plan de măsurare și verificare.
\item\textbf{Tarif energetic:} taxa pentru unitatea de energie unde se află clădirea.
\item\textbf{Sunshine platform - sunshineplatform.eu:} platforma online cu mai multe părți pentru mai multe părți pentru contractarea performanței energetice, disponibilă la sunshineplatform.eu, care sprijină dezvoltarea și gestionarea proiectelor de renovare a clădirilor bazate pe contractarea performanței energetice.
\item\textbf{Comision (uri):} Taxa recurentă lunară plătită de Client contractantului pentru prestarea Serviciilor prevăzute în Acord pe durata perioadei de serviciu, care include Taxa de energie termică, Taxa de apă caldă menajeră, Taxa de renovare și Taxa de exploatare și întreținere împreună cu orice taxe datorate (cum ar fi TVA).
\item\textbf{Consum termic de energie termică:} cantitatea de energie termică calculată de Antreprenor pentru încărcarea către Client a unei cantități fixe lunare de energie termică pentru fiecare perioadă de decontare din perioada de serviciu.
\item\textbf{Contribuție financiară:} ponderea costurilor de investiții pentru lucrările de renovare finanțate direct cu capitaluri proprii sau transferate indirect ca finanțare de către terți de către contractant și pentru care contractantul percepe taxa de renovare.
\item\textbf{Taxa de energie termică:} taxa plătită de client contractantului pentru energia consumată de clădire în perioada de serviciu supusă ajustărilor și soldului o dată pe an la sfârșitul perioadei de decontare pentru a ține cont de condițiile meteorologice efective din timpul decontării. Perioada și pentru măsurarea și verificarea garanției de economii de energie.
\item\textbf{Alimentarea cu căldură:} furnizarea de energie termică către clădire pentru nevoile de încălzire a spațiului și de preparare a apei calde menajere.
\item\textbf{Sezonul încălzirii:} perioada din anul în care contractantul respectă garanțiile privind standardele de confort prevăzute în prezentul acord începând cu 1 octombrie și se încheie la 30 aprilie a fiecărui an de decontare în perioada de serviciu.
\item\textbf{Factură:} factură emisă Clientului (Proprietarii de apartamente sau un reprezentant al proprietarilor de apartamente) pentru serviciile primite și pentru alte plăți datorate antreprenorului care rezultă din acord și emise în conformitate cu toate cerințele legale aplicabile în conformitate cu legislația Letoniei.
\item\textbf{IPMVP:} Protocolul internațional de măsurare și verificare a efectului conservării energiei pregătit de EVO (Organizarea de evaluare a eficienței), (1629 K Street NW, Suite 300, Washington, DC 20006, SUA), care se aplică în scopul măsurării și verificarea în cadrul acordului.
\item\textbf{LABEEF:} Lituania Baltică a Eficienței Energetice Jsc., care funcționează ca o societate pe acțiuni în mod corespunzător înregistrată în registrul comercial al Letoniei sub numărul corporativ 40103960646
Stare latentă: defecțiuni și defecte ale clădirii, sau adiacente clădirii, despre care clientul nu a avut cunoștințe și antreprenorul nu a putut identifica prin observații rezonabile și inspecții obișnuite atunci când a fost întocmit acordul.
\item\textbf{Manager:} persoană fizică sau juridică, care în conformitate cu prevederile aplicabile ale legii letone privind gestionarea locuințelor rezidențiale și bazată pe un acord de gestionare, desfășoară activități manageriale și de întreținere atribuite de client și stabilite prin acord.
\item\textbf{Măsurile  denumite, de asemenea, Măsuri de eficiență energetică:} astfel de acțiuni care au ca rezultat realizarea unei creșteri verificabile, măsurabile sau estimabile a performanței energetice și a altor lucrări de construcție și instalare care vizează reamenajarea și îmbunătățirea clădirii atât din motive structurale cât și din punct de vedere estetic.
\item\textbf{Măsurare și verificare:} procesul și activitățile desfășurate pentru a determina economiile de energie atribuite clădirii ca urmare a măsurilor implementate și a serviciilor furnizate.
\item\textbf{Taxa de exploatare și întreținere:} taxa plătită de client contractantului pentru serviciile aferente operațiunii și întreținerii măsurilor și supusă indexării anuale cu indicele de preț al consumatorului leton aplicabil pentru anul respectiv publicat de Biroul central de statistică.
\item\textbf{Manual de operare și întreținere:} un manual care indică programul de întreținere pentru măsurile implementate în temeiul prezentului acord și activitățile operaționale reglementate de acord.
\item\textbf{Părți:} clientul și antreprenorul în mod colectiv.
\item\textbf{Partea:} clientul și contractantul fiecare în parte.
\item\textbf{Programul de plată:} un document pregătit de antreprenor pentru client care arată taxa de renovare pentru rambursarea contribuției financiare, calculată pentru fiecare perioadă de calcul a dobânzii în conformitate cu prezentul acord.
\item\textbf{Funcționare corectă:} funcționarea măsurilor în așa fel încât să asigure îndeplinirea funcționalității și eficiența deplină a acestora și include toate acțiunile de întreținere necesare întreprinse de către contractant și în detrimentul contractantului.
\item\textbf{Regulator:} comisia pentru utilități publice sau o altă autoritate aplicabilă stabilită în legile și reglementările în vigoare în Letonia care aprobă tarifele pentru comercializarea energiei termice în administrația locală respectivă unde se află clădirea.
\item\textbf{Taxa de renovare:} tariful indexat de Euribor plătit de client contractantului pentru contribuția financiară a contractantului.
\item\textbf{Lucrări de renovare:} activități întreprinse de antreprenor necesare pentru punerea în aplicare a măsurilor din clădire, inclusiv inginerie, achiziții, furnizare, instalare, pornire, punere în funcțiune și finanțare a măsurilor.
\item\textbf{Perioada de serviciu:} perioada în care antreprenorul furnizează serviciile clientului. Perioada de serviciu începe la data punerii în funcțiune.
\item\textbf{Perioada de decontare:} perioada unui an calendaristic care reapare anual în perioada de serviciu.
\item\textbf{Declarație:} un document semnat de părți pentru a demonstra diverși parametri prezenți la clădire, așa cum este înregistrat la momentul executării unui astfel de document.
\item\textbf{TVA:} taxa pe valoarea adăugată plătită în conformitate cu legile și reglementările în vigoare în Letonia și dispozițiile acordului.
\end{itemize}


\subsection{ACCEPTAREA CONDIȚIILOR CONTRAC- TUALE}
\begin{enumerate}
\item Clientul este de părere că antreprenorul are calificările, experiența și abilitățile necesare pentru a efectua lucrările de renovare și a furniza serviciile clientului. Din acest motiv, clientul autorizează antreprenorul și îl va ajuta pe contractant să întreprindă, pe cheltuiala antreprenorului, toate acțiunile legale și de fapt pentru a executa acordul, fără a fi necesară o împuternicire expresă în beneficiul antreprenorului.
\item Antreprenorul va întreprinde lucrările de renovare și va furniza clientului serviciile în termenii și condițiile prevăzute în prezentul acord. Antreprenorul recunoaște că s-a mulțumit cu privire la natura, situația și amplasarea clădirii și toate celelalte aspecte care ar putea afecta în orice mod îndeplinirea obligațiilor sale în temeiul acordului. Orice neîndeplinire a contractantului de la cunoașterea clădirii sau a oricăror condiții ale șantierului în conformitate cu prezenta clauză nu-l va exonera de responsabilitatea îndeplinirii obligațiilor sale în conformitate cu acordul.
\item Contractul confirmă faptul că bugetul inclus în termenii specifici ai prezentului Acord include toate lucrările de construcție, materialele și echipamentele necesare pentru livrarea lucrărilor de renovare în conformitate cu specificațiile tehnice ale proiectului și condițiile prezentului Acord.
\item Definițiile pentru toate scopurile acordului, condițiile specifice, anexele sale și prezentele condiții și condiții generale au semnificațiile respective indicate la articolul 1 din termenii și condițiile generale ale prezentului acord.
\item În cazurile de discrepanță între termenii și condițiile generale și condițiile specifice și anexele sale, dispozițiile din acestea au prioritate.
\end{enumerate}


\subsection{SIGURANȚĂ, CALITATE ȘI CONFORT}
\begin{enumerate}
\item Serviciile prestate de contractant în temeiul prezentului acord:
  \begin{enumerate}
\item Să fie livrat cu cel mai înalt nivel de abilitate și îngrijire, așa cum este de așteptat contractanții cu experiență și profesioniști, care efectuează în mod regulat lucrări și servicii cu același domeniu de aplicare sau complexitate similară în prezentul acord;
\item Să fie dezvoltat folosind materiale și echipamente de calitate adecvată, noi, potrivite scopului;
\item Să respecte legislația în domeniul construcțiilor și orice alte norme, reglementări sau norme legale aplicabile în Republica Letonia în momentul prestării serviciilor;
\item Să fie executat pentru a provoca cât mai puține inconveniente în utilizarea clădirii de către client și alți ocupanți ai clădirii;
\end{enumerate}
\item Standardele de confort îndeplinesc sau depășesc nivelul prezentat în condițiile specifice ale prezentului acord în perioada de serviciu a contractului.
\item În perioada în care ferestrele dintr-un apartament al clădirii sunt deschise și timp de 2 (două) ore după închiderea geamurilor, contractantul nu garantează nivelurile de temperatură interioară convenite în condițiile specifice ale acordului pentru apartamentul specific unde ferestrele au fost deschise.
\item Antreprenorul va asigura un nivel adecvat de ventilație în apartamente conform reglementărilor și normelor letone relevante.
\item Antreprenorul va lua toate măsurile necesare pentru a asigura securitatea și protecția sănătății angajaților la locul de muncă, în conformitate cu legea protecției muncii și cu toate normele și normele letone relevante.
\item Antreprenorul va pune în aplicare măsuri de protecție adecvate pentru a proteja toate persoanele de moarte sau vătămare care pot fi cauzate de neplăceri sau neglijență gravă a antreprenorului, angajaților, agenților sau subcontractanților săi în perioada lucrărilor de construcție și a serviciului. Antreprenorul va proteja, de asemenea, întreaga clădire împotriva pagubelor legate de punerea în aplicare a măsurilor.
\item Antreprenorul se va asigura că toate serviciile de utilitate furnizate clădirii nu sunt deconectate sau perturbate din cauza unei neplăceri sau neglijențe a contractantului în orice moment fără o notificare prealabilă. Orice servicii de utilitate perturbate sau deconectate din cauza unei defecțiuni sau neglijențe a antreprenorului vor fi repede repuse de către contractant, cu costurile contractantului. Antreprenorul nu este responsabil pentru cazurile în care astfel de întreruperi sunt în afara controlului contractantului și / sau sunt datorate unor acte sau omisiuni ale întreprinderii de întreținere, a serviciilor de energie și apă sau a oricărei terțe părți care nu sunt legate de antreprenor.
\item Antreprenorul în perioada de construcție trebuie să asigure o protecție corespunzătoare a clădirii împotriva impactului vremii, prevenind infiltrarea apei pluviale și daunele aduse clădirii. Infiltrațiile de apă subterană și evenimentele de forţă  majoră sunt excluse.
\item Antreprenorul respectă codul european de conduită pentru contractarea performanței energetice (www. Http://transparense.eu/), care este un set de valori și principii care sunt considerate fundamentale pentru implementarea cu succes, profesională și transparentă a contractării de performanță energetică. în țările europene.
\end{enumerate}


\subsection{GARANŢII}
\begin{enumerate}
\item Antreprenorul în perioada de serviciu furnizează clientului o garanție de economisire a energiei ca parte a prezentului acord, care, anual, este supusă măsurării și verificării.
\item Antreprenorul în perioada de serviciu garantează standardele de confort convenite în conformitate cu prezentul acord.
\item Antreprenorul garantează în perioada de serviciu, pe cheltuieli proprii, funcționarea corectă a măsurilor instalate sau introduse de către contractant pentru sistemul de încălzire, sistemele de alimentare cu apă caldă menajeră, sistemele de ventilație și răcire cu aer, joncțiuni și conducte, în linie cu specificațiile și uzura lor normală, pe durata termenului prezentului acord, printre altele, prin repararea sau înlocuirea măsurilor, dacă este necesar.
\item Antreprenorul garantează în perioada de serviciu, pe cheltuieli proprii, efectul și eficiența materialelor izolante instalate sau introduse de contractant în conformitate cu specificațiile lor și uzura normală, pe întreaga valabilitate a acordului, printre altele prin repararea lor sau înlocuirea lor, dacă este necesar.
\item Antreprenorul va asigura la sfârșitul perioadei de funcționare funcționarea corectă a tuturor măsurilor implementate, în conformitate cu specificațiile lor și uzura normală și luând în considerare întreținerea corespunzătoare. Antreprenorul la sfârșitul perioadei de serviciu va furniza clientului toate manualele de utilizare, îngrijire și întreținere, înregistrări, instrucțiuni, alte documentații, software, licențe de proprietate intelectuală, instrumente speciale și protocoale și proceduri necesare sau convenabile pentru buna desfășurare continuă a măsurile de realizare a standardelor de confort în baza prezentului acord.
\item Antreprenorul înainte de începerea perioadei de construcție trebuie să prezinte clientului o garanție de performanță a unei instituții de credit sau a unei societăți de asigurare pentru îndeplinirea obligațiilor sale de 10% din costurile totale de investiții (fără TVA), după cum urmează:
 \begin{enumerate}
\item Atunci când antreprenorul este și compania generală de construcții, această garanție de performanță este furnizată de antreprenor în favoarea clientului împotriva prevederilor prezentului acord;
\item Atunci când antreprenorul achiziționează compania de construcții generale, această garanție de performanță este asigurată de firma de construcții generale în favoarea antreprenorului și se bazează pe furnizarea contractului de construcție între antreprenor și compania de construcții generale;
\item În cazul în care antreprenorul nu oferă această garanție de performanță inițială pentru perioada de construcție care asigură executarea activităților în perioada de construcție, contractantul nu are dreptul să inițieze lucrările de construcție.
\item Această garanție de performanță va fi valabilă pe toată perioada de construcție. În cazul în care perioada de construcție este prelungită, contractantul prelungește această garanție cu aceeași perioadă de timp.
\end{enumerate}
\item Antreprenorul în cel mult 10 (zece) zile de la semnarea protocolului de livrare și acceptare prezintă clientului o garanție de performanță a unei instituții de credit sau a unei societăți de asigurare pentru îndeplinirea obligațiilor sale de cel puțin 5% din costurile de investiție. (fără TVA) după cum urmează:
 \begin{enumerate}
\item Atunci când antreprenorul este și compania generală de construcții, această garanție de performanță este furnizată de antreprenor în favoarea clientului împotriva prevederilor prezentului acord;
\item Atunci când antreprenorul achiziționează compania de construcții generale, această garanție de performanță este asigurată de firma de construcții generale în favoarea antreprenorului și se bazează pe furnizarea contractului de construcție între antreprenor și compania de construcții generale;
\item Această garanție este valabilă timp de 36 (treizeci și șase) luni;
\end{enumerate}
\item Clientul are dreptul să apeleze la garanția de performanță menționată la articolul 4.6. și articolul 4.7. pentru lichidarea obligațiilor financiare ale antreprenorului sau a actelor normative.
\item Garanția de performanță menționată la prezentul articol este emisă de o instituție de credit sau o companie de asigurare înregistrată în Republica Letonia sau orice alt stat membru al Uniunii Europene sau Spațiului Economic European, care, conform procedurii stabilite prin actele juridice ale Republicii Letonia a început furnizarea de servicii pe teritoriul Republicii Letonia.
\end{enumerate}

\subsection{DREPTURI ȘI OBLIGAȚIILE CONTRAC-TANTULUI}
\begin{enumerate}
\item Contractantul are calificările, experiența și abilitățile profesionale necesare în furnizarea echipamentelor, materialelor și serviciilor în conformitate cu acordul.
\item Antreprenorul va obține toate avizele și aprobările necesare din partea guvernului, instituțiilor și autorităților municipale pentru implementarea lucrărilor de renovare și livrarea serviciilor fără implicarea clientului, cu excepția cazului în care este necesar în conformitate cu legile sau regulamentele aplicabile.
\item Antreprenorul începe implementarea lucrărilor de construcție și instalare a măsurilor la data începerii și le va completa în perioada de construcție prevăzută. Antreprenorul va informa clientul despre data provizorie de începere, cel târziu în termen de 20 de zile lucrătoare de la semnarea prezentului acord.
\item Antreprenorul va notifica în scris clientului cu cel puțin 10 (zece) zile lucrătoare în avans cu privire la data începerii lucrărilor de construcție și instalare a măsurilor, oferind o ocazie clientului de a șterge zonele comune ale clădirii ( inclusiv scările, spațiul subsolului, mansarda, acoperișul, spațiile de depozitare a cărbunelui / a lemnului și a gazelor, panourile de electricitate și telecomunicații și camerele centralei), din deșeuri, proprietăți abandonate și orice alt obiect acolo. În cazul în care clientul nu reușește să golească zonele comune la timp, Antreprenorul are dreptul să procure curățarea zonelor comune ale clădirii și să emită o factură către client pentru plata acestor lucrări în compensare pentru cheltuielile efectuate. Clientul va plăti prompt această factură în termen de cel mult 20 de zile lucrătoare.
\item În perioada de construcție, contractantul va asigura toată munca necesară pentru punerea în aplicare a măsurilor, inclusiv supravegherea necesară, instrumentele, materialele și echipamentele de natură, calitate și cantități adecvate.
\item În perioada de serviciu, contractantul va asigura toată munca necesară pentru operarea și întreținerea măsurilor, inclusiv supravegherea necesară, instrumentele, materialele și echipamentele de natură, calitate și cantități adecvate.
\item În perioada de construcție, contractantul va aranja furnizarea de energie electrică cu contorizare separată și va plăti energia electrică consumată pentru lucrările de implementare și instalare a măsurilor. Antreprenorul are dreptul să aibă acces la sistemul comun de alimentare cu energie electrică a clădirii.
\item Antreprenorul va curăța în mod corespunzător șantierul (zonele comune ale clădirii, ferestrele, intrările și împrejurimile) la finalizarea lucrărilor de construcție și instalare a măsurilor înainte de data punerii în funcțiune.
\item Contractantul invită clientul pentru punerea în funcțiune a măsurilor implementate în clădire. Antreprenorul va furniza clientului protocolul de livrare și acceptare a clădirii la sfârșitul perioadei de construcție.
\item Antreprenorul în perioada de serviciu va notifica în scris clientul în cazul în care deșeurile și / sau obiectele sunt depozitate și abandonate de către proprietarii de apartamente sau alte terțe părți care nu au legătură cu antreprenorul în zona comună a clădirii, ceea ce ar putea cauza probleme antreprenorului în activități de operare și întreținere în conformitate cu acordul. În cazul în care clientul nu reușește să curețe zona în conformitate cu prezentul contract, antreprenorul are dreptul de a procura curățarea spațiilor și de a emite o factură către client pentru plata în compensație pentru efectuarea acestei curățări. Clientul va plăti prompt această factură în termen de cel mult 20 de zile lucrătoare.
\item Antreprenorul în perioada de serviciu va notifica în scris clientul despre orice defecțiune identificată, furt, vandalism sau sabotaj făcut la măsuri.
\item Antreprenorul va asigura suficientă energie termică la clădire în timpul sezonului de încălzire, într-o manieră adecvată pentru îndeplinirea standardelor de confort din prezentul acord. Antreprenorul în temeiul prezentului acord nu este răspunzător pentru întreruperile sau lipsa furnizării de energie termică către clădire în cazurile care nu sunt sub controlul contractantului, inclusiv în lipsa furnizării de energie termică sau datorită unor evenimente de forță majoră a companiei furnizoare de încălzire.
\item Antreprenorul înainte de începerea perioadei de construcție va prezenta clientului proiectul de renovare a clădirii în conformitate cu cabinetul de miniștri regulamentul nr. 21/10/2014. 655 "Reglementări privind standardul de construcții din Letonia LBN 310 -„ Proiect de lucru "" și coordonează cu clientul și supraveghetorul de construcții.
\item Antreprenorul în perioada de construcție va notifica și va invita clientul să participe la reuniunile săptămânale de monitorizare a stării lucrărilor de construcție. Contractantul va transmite apoi clientului 2 (două) rapoarte de progres pe lună cu privire la starea și progresul lucrărilor de construcție. Aceste rapoarte pot fi transmise electronic utilizând online EPC-platform- sunshineplatform.eu.
\item În cazul în care antreprenorul este intermediarul între client și furnizorul de energie termică, antreprenorul va plăti în numele și în numele clientului facturile datorate companiei de încălzire la primirea componentei corespunzătoare taxa de energie termică datorată în baza prezentului acord  antreprenorului. În pofida celor de mai sus, facturile pentru încălzire în perioada de construcție vor fi în detrimentul clientului.
\item Antreprenorul are dreptul de a atribui sau subcontracta terților (subcontractanților) executarea lucrărilor și serviciilor prevăzute în acord. Antreprenorul răspunde integral pentru subcontractanți cu privire la obligația prezentului acord.
\item Antreprenorul are dreptul de a modifica garanția de economii energetice în cazul unei schimbări de utilizare a clădirii (articolul 17). Orice modificare este convenită între părți cu o modificare scrisă a prezentului acord.
\item Antreprenorul are dreptul de a invita, la propriile cheltuieli, un expert independent, calificat și cu experiență, care a fost aprobat în prealabil de către client (o astfel de aprobare care nu este nerezonabilă reținută) pentru a evalua conformitatea măsurilor propuse cu legile și regulamentele eficiente în Republica Letonia sau decizii ale guvernelor locale obligatorii pentru client în cazul în care clientul își va exercita drepturile de veto. Opinia unui astfel de expert este obligatorie pentru părți.
\item Contractantul nu poate, în termen de cel mult 5 (cinci) zile lucrătoare, să notifice clientul despre o schimbare a adresei specificată în acord sau alte modificări ale statutului său juridic, ale administrației și ale situației sale juridice; în special, dacă antreprenorul suferă vreo fuziune sau achiziție, intră în proceduri de lichidare sau faliment.
\end{enumerate}

\subsection{DREPTURI ȘI OBLIGAȚIILE CLIENTULUI}
\begin{enumerate}
\item Clientul a adoptat o decizie valabilă și executorie, care face ca acest contract să fie obligatoriu pentru toți proprietarii de apartamente, pe care fiecare în parte este obligat să-și execute dispozițiile sale indiferent dacă apartamentul în vedere a fost închiriat sau închiriat și indiferent dacă sunt utilizate apartamentele. de către proprietarii de apartamente sau nu.
\item Clientul (fiecare proprietar de apartamente) informează chiriașii, locatarii și orice alte persoane care ocupă apartamente sau locuitori obișnuiți despre obligațiile relevante din prezentul acord.
\item Clientul va furniza informații și documente solicitate de antreprenor pentru implementarea lucrărilor de renovare și furnizarea serviciilor prezentate în acord cât mai curând posibil la primirea cererii contractantului. Clientul nu va fi răspunzător pentru neprezentarea oricăror documente care pot fi relevante, dar nu sunt specificate în mod clar, cu detalii suficiente de către contractant.
\item Clientul va oferi antreprenorului asistența în timp util în obținerea permiselor, aprobărilor sau a altor documente legate de executarea cu succes a acordului de la guvern, instituții municipale și agenție, inclusiv, fără a se limita la, certificarea și / sau furnizarea documentelor necesare, acordarea. puterea necesară a avocaților și informațiile disponibile contractantului. Clientul autorizează în mod corespunzător antreprenorul sub forma necesară pentru a întreprinde orice acțiuni de fapt sau legale în fața autorităților competente în scopul executării cu succes a acordului. Cu toate acestea, clientul nu este răspunzător pentru neprezentarea oricărei informații, cu excepția cazului în care contractantul specifică în detaliu clar, iar aceste informații sunt în mod rezonabil disponibile pentru client.
\item Clientul nu va împiedica sau reține consimțământul pentru lucrările de renovare, punerea în aplicare a măsurilor în perioada de construcție și menținerea acestora în perioada de serviciu; dimpotrivă, clientul va acționa cu bună-credință pentru a facilita implementarea și menținerea acestora și realizarea garanției de economisire a energiei.
\item Clientul are dreptul să se plângă de calitatea sau modul de execuție al măsurii implementate până la 10 (zece) zile lucrătoare după semnarea protocolului de livrare și acceptare. După această perioadă, măsurile implementate de orice contractant sunt considerate acceptate și se încheie perioada de construcție.
\item Clientul are dreptul de a exercita dreptul de veto sau de a refuza punerea în aplicare a unei măsuri planificate ca parte a lucrărilor de renovare, în cazul în care clientul dovedește fără îndoială rezonabilă că măsura (măsurile) respectivă / încalcă legile și reglementările. eficiente în Republica Letonia sau decizii ale guvernelor locale obligatorii pentru client.
\item Clientul va furniza Antreprenorului sau oricărei alte persoane autorizate de Antreprenor acces la clădire, inclusiv fiecare apartament al clădirii, în perioada de construcție și în perioada de serviciu în scopul prestării serviciilor acordului. Clientul va asigura accesul la clădire în orice moment din zilele lucrătoare (între orele 8:00 și 20:00), în timp ce se află într-o situație de urgență în afara programului de lucru și în weekend și de sărbători.
\item Clientul, înainte de data începerii, trebuie să se asigure că zonele comune (inclusiv scările, spațiul subsolului, mansarda, acoperișul, spațiile de depozitare a cărbunelui / lemnului și gazului, panourile de electricitate și telecomunicații și camerele centralei), sunt libere de deșeuri, proprietăți abandonate și orice obiect aflat acolo, fie prin acordul cu antreprenorul pentru înlăturare, fie livrarea acestora către o companie de colectare a deșeurilor, fie livrarea acestora către proprietarul cunoscut.
\item Clientul trebuie să se asigure în perioada de serviciu că zonele comune ale clădirii, cum ar fi scările, spațiul subsolului și spațiul mansardelor sunt păstrate curate și în bune condiții operaționale, efectuând curățări periodice.
\item Clientul nu va interveni cu măsurile instalate de antreprenor, fără acordul și autorizarea scrisă a antreprenorului sau în conflict cu instrucțiunile operaționale furnizate de antreprenor; în special, dacă interferența are un impact negativ asupra nivelului de economii de energie. Interferența clientului cu setările sistemului de încălzire, a sistemului de apă caldă menajeră și a sistemului de ventilație este considerată o încălcare semnificativă de către client a obligațiilor sale în temeiul prezentului acord și va servi drept temei bun și suficient pentru rezilierea contractului de către contractant.
\item Clientul va lua toate măsurile rezonabile pentru a se asigura că nimeni din clădire nu va interfera sau altera setările sistemului de încălzire, ale sistemului de apă caldă menajeră și ale sistemului de ventilație sau deteriorarea și sabotarea măsurilor în vreun fel.
\item Clientul va notifica antreprenorul, imediat după descoperire (într-o zi lucrătoare), cu privire la orice daune sau modificări sau interferențe cu măsurile instalate de antreprenor.
\item Clientul va notifica orice circumstanță care are și / sau poate avea un impact negativ asupra economiilor energetice. Nerespectarea de către client a unei astfel de notificări nu renunță la răspunderea antreprenorului de a atinge garanția de economii energetice decât dacă se stabilește că clientul intenționează să reducă nivelul de economii energetice.
\item Clientul va notifica în termen de cel mult 20 de douăzeci de zile lucrătoare contractantul și, dacă este necesar, coordonează cu antreprenorul, execuția lucrărilor de construcție, instalare și întreținere, care nu fac parte din prezentul acord și au un efect potențial asupra energiei. consumul clădirii, inclusiv, dar fără a se limita la: (i) extinderea zonei clădirii, (ii) modernizarea în continuare a clădirii, (iii) înlocuirea sau instalarea de radiatoare noi / diferite și / sau convectoare termice, și / sau elemente de încălzire (iv) instalarea unei noi unități de producere a căldurii.
\item Clientul (proprietarul apartamentului) va notifica antreprenorul în cazul renovării apartamentului și a calendarului când această renovare ar putea afecta consumul de energie al clădirii, inclusiv, dar fără a se limita la: (i) înlocuirea caloriferelor și / sau termice convectoare și / sau elemente de încălzire; (ii) înlocuirea ferestrelor; (iii) extinderea zonei încălzite a apartamentului, inclusiv spațiul balconului / loggia; (iv) instalarea sistemelor de ventilație mecanică. În aceste cazuri, antreprenorul are dreptul de a revizui garanția de economii energetice stabilită în acord.
\item Clientul în timpul sezonului de încălzire are dreptul să deschidă geamurile din apartamentele clădirii pentru cel mult 10 (zece) minute pe zi pentru a asigura circulația aerului proaspăt pentru a scăpa de: (i) praf sau miros puternic de curățare produsele care persistă în apartament după curățare și (ii) miros puternic după gătit persistând în apartament.
\item Clientul în timpul sezonului de încălzire are dreptul de a deschide geamurile din apartamentele clădirii în orice moment, din motive de sănătate ale persoanelor cazate în apartament.
\item Clientul trebuie să se asigure că toate ferestrele zonei comune sunt ținute închise în timpul sezonului de încălzire.
\item Clientul trebuie să se asigure că toate ușile de intrare ale clădirii nu sunt lăsate deschise în timpul sezonului de încălzire.
\item În cazul în care proprietarii de apartamente ale clădirii se schimbă, clientul trebuie să informeze imediat contractantul cu privire la modificări și nu mai târziu în termen de 5 (cinci) zile lucrătoare după această modificare.
\item Clientul (fiecare proprietar de apartament) se va asigura că, în cazul oricăror transferuri de drepturi de proprietate asupra apartamentului său și indiferent de motivele sau temeiurile legale în care se realizează acest transfer, noul proprietar de apartament (cesionar) va semna o faptă de aderare, sau orice alt document legal, care oferă asumarea drepturilor și transferul obligațiilor care decurg din prezentul acord de la proprietarul apartamentului inițial către noul proprietar de apartament. Nerespectarea celor menționate anterior face ca proprietarul apartamentului vechi și, în mod comun, clientul să răspundă pentru executarea oricăreia dintre obligațiile prevăzute în prezentul acord și oricare dintre încălcările perspectivei sale de către cesionar.
\item În termen de cel mult 5 (cinci) zile lucrătoare, clientul nu poate anunța antreprenorul despre schimbarea managerului. Noul manager este informat în mod corespunzător cu privire la prevederile prezentului acord.
\end{enumerate}

\subsection{PROCEDURI DE SETARE}
\begin{enumerate}
\item Clientul va plăti taxele lunare către contractant, prevăzute în prezentul contract.
\item Perioada de decontare reciprocă a conturilor între părți este o lună calendaristică. Perioada de decontare pentru calculul plății soldului între facturi pe baza consumului de energie termică forfetar și a consumului curent de energie termică contorizat și pentru măsurarea și verificarea garanției de economii de energie este de un an.
\item În fiecare lună antreprenorul, sau un terț desemnat care îl reprezintă pe contractant, calculează plata care va fi efectuată de client în baza contractului. Suma totală a tuturor comisioanelor calculate va fi considerată datorată și plătibilă de către client contractantului pentru serviciile prestate în temeiul prezentului acord pentru decontarea reciprocă a conturilor.
\item La fiecare 12 (douăsprezece) luni de la aniversarea începerii perioadei de serviciu, contractantul va efectua o decontare anuală pe baza rezultatelor măsurării și verificării garanției de economii de energie.
\item Antreprenorul sau terțul membru desemnat va emite cel târziu în a zecea (a zecea) zi a fiecărei luni o factură care cuprinde toate componentele comisioanelor, așa cum este prezentată în mod expres în acord și o va livra clientului sau managerului reprezentant.
\item Antreprenorul se va asigura că informațiile incluse în facturile fiecărui proprietar de apartamente pentru serviciile prestate sunt prezentate într-un mod clar și cuprinzător, indicând separat, taxa de renovare și taxa de operare și întreținere a contractantului.
\item Prima plată a comisioanelor se calculează ca fiind datorată la 1 (o) lună după semnarea protocolului de livrare și acceptare. Până atunci, clientul rămâne obligat să-și acopere toate cheltuielile de utilitate și de comunitate la termen.
\item Clientul (fiecare proprietar de apartament) va plăti taxele către antreprenor (sau către o terță parte indicată de antreprenor) direct sau prin facilitarea managerului pe baza Facturilor emise de manager pentru toate utilitățile și alte cheltuieli operaționale pentru întreținerea clădirii, dintre care o parte include taxele datorate contractantului. Clientul va plăti comisioanele în conformitate cu practicile stabilite de către administrator, dar nu mai târziu în termen de 15 (cincisprezece) zile de la data primirii facturii prin transferul fondurilor necesare în contul bancar specificat de manager.
\item Antreprenorul sau managerul său desemnat administrează informațiile asociate cu decontarea conturilor în legătură cu prezentul acord, printre altele, prin:
\begin{enumerate}
\item Înregistrarea tuturor informațiilor despre facturile emise fiecărui proprietar de apartamente, inclusiv sumele acestora;
\item Păstrarea evidenței cu privire la plata facturilor și actualizarea constantă a datoriei fiecărui proprietar de apartament, dacă există;
\end{enumerate}
\item Clientul la cererea antreprenorului sau a oricăruia dintre cesionarii acestuia trebuie să furnizeze antreprenorului sau cesionatul (după caz) raportul curent privind plățile efectuate de proprietarii de apartamente ale clădirii și lista debitorilor.
\end{enumerate}


\subsection{TERMENUL ACORDULUI}
\begin{enumerate}
\item Termenul prezentului acord începe la data prezentului acord și va rămâne în vigoare deplină și în vigoare până la finalizarea perioadei de furnizare a serviciului, sub rezerva rezilierii anticipate, astfel cum este prevăzut în prezentul acord.
\item Prezentul acord poate fi prelungit prin acordul scris al părților, în special, părțile au dreptul să anticipeze sau să amâne data de începere și data punerii în funcțiune prin acordul scris al părților.
\item Antreprenorul începe implementarea lucrărilor de construcție și instalare a măsurilor la data începerii și le va completa în perioada de construcție prevăzută. La încheierea perioadei de construcție, clientul și antreprenorul vor semna protocolul de livrare și acceptare.
\item Perioada de serviciu și termenul de plată încep la semnarea protocolului de livrare și acceptare.
\item În caz de neîndeplinire sau neglijență a clientului, cum ar fi dacă antreprenorul nu este furnizat cu toate documentele cuvenite și / sau acces la  clădire și / sau în caz de evenimente de forţă majoră, data începerii, perioada de construcție și perioada de servire. se consideră automat amânate de perioada de întârziere. Aceste modificări sunt convenite în scris de către părți și se modifică acordul.
\end{enumerate}


\subsection{CONDIȚII LATENTE}
\begin{enumerate}
\item Dacă în perioada de construcție, Antreprenorul ia cunoștință de orice condiții latente care vor afecta finalizarea măsurilor, antreprenorul trebuie (ca o condiție precedentă oricărui drept la timp sau bani suplimentari) să dea o notificare scrisă clientului în termen de 5 ( cinci) Zile lucrătoare care specifică:
  \begin{enumerate}
\item Condițiile latente întâlnite și în ce privință diferă substanțial de starea clădirii, ceea ce ar fi trebuit să fie anticipat în mod rezonabil de un contractant competent și cu experiență care exercită bune practici industriale la data acordului;
\item Lucrările suplimentare și resursele suplimentare pe care antreprenorul estimează că sunt necesare pentru a face față condițiilor latente;
\item Timpul estimat de antreprenor pentru a face față condițiilor latente și a întârzierii preconizate pentru finalizare;
\item Estimarea rezonabilă a contrac- tantului asupra costului măsurilor necesare pentru a face față condițiilor latente; și
\item Orice alte detalii care pot fi solicitate în mod rezonabil de către client.
\end{enumerate}
\item Întârzierea cauzată de o stare latentă poate justifica o prelungire a perioadei de construcție dacă acesta din urmă determină:
  \begin{enumerate}
\item Efectuează lucrări suplimentare;
\item Utilizați materiale suplimentare; sau
\item Suportă costuri suplimentare (inclusiv, dar fără a se limita la, costul întârzierii sau perturbării); pe care antreprenorul nu și nu ar fi putut-o anticipa în mod rezonabil la momentul semnării acordului care exercita o bună practică industrială.
\end{enumerate}
\item Clientul va acoperi toate cheltuielile reale care apar în legătură cu condițiile latente și convenite între părți. În cazul în care clientul nu dorește ca contractantul să procedeze ca notificat, clientul trebuie să îl anunțe imediat pe contractant să nu continue, iar contractantul trebuie să respecte această notificare. Clientul și antreprenorul pot negocia și acorda un alt mod de a depăși condiția latentă, inclusiv, fără a se limita la, lucrările suplimentare necesare efectuate de alte entități și plata acestora de către client.
\item În cazul condițiilor latente, contractantul are dreptul să solicite o prelungire a timpului pentru perioada de construcție corespunzătoare timpului necesar rezolvării condițiilor latente. Antreprenorul are dreptul de a recupera costurile suplimentare suportate direct sau indirect provenind din orice condiții latente.
\item Antreprenorul nu are dreptul să solicite nicio ajustare la garanția de economii energetice, să reducă domeniul de aplicare al lucrărilor de renovare sau să modifice tarifele convenite în prezentul acord din cauza condițiilor latente.
\end{enumerate}


\subsection{MĂSURARE, VERIFICARE ȘI GESTIONARE DATE}
\begin{enumerate}
\item Antreprenorul trebuie să efectueze toate activitățile de măsurare și verificare bazate pe planul de măsurare și verificare aderent IPMVP (Protocolul internațional de măsurare și verificare a performanței), disponibil pe Sunshine platform - sunshineplatform.eu.
\item Toate activitățile de măsurare și verificare ar trebui să fie dezvăluite în mod clar și complet și transparent pentru toate părțile.
\item Antreprenorul în perioada de serviciu va transmite clientului un raport anual. Prezentul raport anual documentează calculul tarifelor, activităților de operare și întreținere efectuate de antreprenor și dacă garanția de economisire a energiei a fost îndeplinită pe baza măsurării și verificării în perioada de decontare. Raportul furnizează suficiente informații despre economiile de energie rezultate din măsurile implementate și despre calculul făcut pentru determinarea economiilor de energie. Raportul anual este trimis de către contractant clientului în fiecare an în termen de cel mult 20 (douăzeci) zile lucrătoare după încheierea perioadei de decontare. Acest proces poate fi livrat prin intermediul Sunshine platform - sunshineplatform.eu.
\item Dacă clientul are obiecții cu privire la concluziile făcute în raportul anual, clientul va informa cntreprenorul în consecință în termen de 15 (cincisprezece) zile lucrătoare după primirea raportului sau primirea notificării de la Sunshine platform - sunshineplatform.eu. Clientul va furniza antreprenorului motivele obiecțiilor sale; antreprenorul va face, în următoarele 15 (cincisprezece) zile lucrătoare, de la primirea obiecțiunilor cesionarului, face modificările necesare și raportează și informează clientul în consecință.
\item Orice imixtiune sau alterare nejustificată de către client cu măsurile implementate în clădire, care au ca rezultat o scădere a nivelului de economii de energie, va fi luată în considerare în timpul măsurării și verificării garanției de economisire a energiei din acord și va servi la reajustarea garanției la baza pro rata.
\item Clientul recunoaște și acordă utilizarea, de către contractant sau de orice altă terță, atribuită de acesta cu drepturile și obligațiile prevăzute în prezentul acord:
\begin{enumerate}
\item A oricăror date și informații anonime referitoare la consumul de energie al clădirii, furnizate de către client sau obținute de antreprenor, în scopul evaluării comparative și al compilării bazelor de date largi naționale, regionale sau internaționale sau în scopul utilizării de către antreprenor ca referință sau pentru orice scop intern convenit cu alientul;
\item A datelor cu caracter personal furnizate de client sau de către managerul său, în scopul prestării serviciilor sale și transferului acestuia către oricare terță parte care poate fi atribuită cu drepturi sau obligații care decurg din prezentul acord, inclusiv oricărei părți care au pierdut creanțele care decurg din prezentul acord sau gestionarea sau responsabilul pentru dezvoltarea, implementarea, operarea și întreținerea unui EPC online -platform-sunshineplatform.eu urmărind performanțele măsurilor implementate.
\end{enumerate}
\item Pentru orice perioadă rezonabilă de timp, contractantul, la discreția sa, singur sau prin intermediul destinatarilor, are dreptul să instaleze, să funcționeze, să funcționeze și să introducă un sistem de gestionare a energiei sau în general instrumente de măsurare și să acceseze echipamentele instalate în orice moment rezonabil, în conformitate cu cu Acordul.
\item Antreprenorul are dreptul de a instala jurnale de temperatură în apartamente dacă sunt primite reclamații privind neconformitatea cu standardele de confort. În cazul în care proprietarii de apartamente ale clădirii nu sunt de acord cu instalarea menționatului registru în apartamentele lor sau nu furnizează acces rezonabil la o astfel de instalație, contractantul nu va fi răspunzător pentru presupusa îndeplinire necorespunzătoare a acordului cu privire la astfel de apartamente.
\item Datele colectate de instrumentul de măsurare a contractantului au caracter informativ și nu pot fi recunoscute ca bază pentru stabilirea unei încălcări a acordului sau a conformității cu standardele de confort în caz de litigii.
\end{enumerate}

\subsection{PROCEDURĂ DE REZOLUȚIE A DISPUTELOR}
\begin{enumerate}
\item Eventualele dezacorduri dintre părți sunt mai întâi negociate. În acest scop, părțile vor aduce o confirmare scrisă oricărei scrisori din partea altei părți cu privire la un dezacord, precum și vor acorda un timp rezonabil pentru a rezolva dezacordul personal sau prin intermediul unui reprezentant superior al părților în dezacord.
\item Dacă clientul (sau un singur proprietar de apartament) are plângeri cu privire la antreprenor (de exemplu, privind standardele de confort sau cuantumul de economii energetice, sau, în general, cu privire la măsurile implementate și serviciile de eficiență energetică furnizate), acesta din urmă trebuie direct sau prin managerul notifică contractorul cu privire la aceasta. Antreprenorul va verifica și pregăti o declarație a reclamației și va remedia problemele apărute. În cazul în care problema persistă mai mult de 20 (douăzeci) de zile lucrătoare de la notificare, clientul va organiza un comitet format din reprezentanți autorizați în mod corespunzător al antreprenorului, managerului și clientului să se întâlnească și să pregătească un proiect de declarație bazat pe plângere și fapte verificat pe site sau întreprinde orice altă procedură de constatare conform normelor de mediere disponibile pe Sunshine platform - sunshineplatform.eu.
\item Dacă antreprenorul are plângeri cu privire la client (de exemplu, pentru deteriorarea echipamentului instalat), acesta din urmă va notifica clientul și managerul cu privire la aceasta. Clientul va verifica și, eventual, identifica făptuitorul și va pregăti o declarație a plângerii și va remedia problemele apărute. Dacă problema persistă mai mult de 30 de zile de la notificare, antreprenorul va organiza un comitet format din reprezentanți autorizați în mod corespunzător al antreprenorului, managerului și clientului pentru a se întâlni și pentru a pregăti un proiect de declarație pe baza reclamației și a faptelor verificate pe site sau să întreprindă orice altă procedură de constatare în conformitate cu regulile de mediere disponibile pe Sunshine platform - sunshineplatform.eu.
\item Pentru procedura de constatare aplicabilă în cazul litigiilor referitoare la fapte, se aplică următoarele:
\begin{enumerate}
\item Standardele de confort reale (temperatura ambientală a apartamentelor individuale din clădire) sunt considerate în mod corespunzător înregistrate dacă măsurările de temperatură sunt efectuate de un auditor energetic independent certificat (conform MK Nr. 382) și în conformitate cu standardul LVS EN 12599. Declarația se execută pe baza constatărilor măsurate de auditorul energetic independent certificat;.
\item Problemele generale cu măsurile puse în aplicare, cum ar fi, de exemplu, funcționarea defectuoasă a echipamentului și / sau defectele și daunele aduse măsurilor sau calcularea economiilor de energie, sunt considerate în mod corespunzător înregistrate dacă sunt raportate de un expert independent, precum un auditor energetic certificat (conform la MK Nr. 382);
\item Toate părțile vor fi notificate cu cel puțin 5 (cinci) zile lucrătoare înainte de orice măsurare efectuată de o terță parte. Un reprezentant autorizat al părților are dreptul să participe la procesul de măsurare pentru pregătirea declarației. Lipsa reprezentanților autorizați ai oricăreia dintre părți nu constituie un obstacol în pregătirea și executarea declarației de către părți;
\item Semnarea declarației de către oricare dintre părți nu va fi considerată o recunoaștere a unei încălcări în temeiul prezentului acord și / sau nu va fi considerată o renunțare la niciunul dintre drepturile și obligațiile părților de mai jos. Costurile pentru terții independenți sunt împărțite uniform între părți;
\item O copie a fiecărei declarații executate va fi livrată contractantului, administratorului și proprietarului de apartamente care a depus plângerea.
\end{enumerate}
\item În cazul în care părțile nu ajung la un acord, părțile vor intra într-un proces formal de mediere conform regulilor de mediere disponibile pe Sunshine platform - sunshineplatform.eu, în vigoare în perioada de valabilitate a acordului și în funcție de cele existente la momentul respectiv. a litigiului. În cazul în care există o dispută între părți cu privire la probleme tehnice, orice parte poate solicita rezolvarea litigiului cu privire la faptele stabilite în conformitate cu normele procedurale ale comisiei de constatare disponibile pe Sunshine platform - sunshineplatform.eu.
\item Dacă părțile nu ajung la un acord reciproc după procesul de mediere și / sau procesul de constatare a faptelor, litigiul va fi soluționat de o instanță de jurisdicție generală a Letoniei în conformitate cu legile și reglementările aplicabile în Letonia. Cererea trebuie depusă la instanța de judecată, în funcție de jurisdicția locului de reședință sau a adresei înregistrate a inculpatului; cu toate acestea, dacă nu este Letonia, atunci la Judecătoria Riga sau la Tribunalul Riga.
\end{enumerate}

\subsection{ÎNTREȚINEREA MĂSURILOR IMPLEMEN-TATE DE CONTRACTOR}
\begin{enumerate}
\item Antreprenorul va înlocui sau repara sau revizui echipamentul (sau orice parte a acestuia) instalat ca parte a lucrărilor de renovare la expirarea perioadei de viață utilă (determinată de manualul de operare și întreținere) în timpul perioadei de service a acord.
\item Antreprenorul va implementa proceduri de întreținere pentru măsurile care corespund sau depășesc cerințele și recomandările producătorului pentru întreținerea respectivă și în conformitate cu condițiile specifice ale prezentului acord.
\end{enumerate}

\subsection{ASIGURARE}
\begin{enumerate}
\item La începerea perioadei de construcție, contractantul va asigura clădirea pentru o sumă nu mai mică decât valoarea de restaurare a clădirii, cu o acoperire minimă de asigurare împotriva incendiilor, cutremurului, inundațiilor, daunelor de apă, oricărei alte dezastre naturale care au impact asupra clădirii, daune structurale cauzate de subsidență și copaci căzuți. Pentru această asigurare se aplică următoarele dispoziții:
  \begin{enumerate}
\item Antreprenorul va încheia această asigurare cu un asigurator cu un nivel minim A +, conform calificărilor relevante aplicabile Letoniei.
\item Antreprenorul trebuie să prezinte clientului o copie a acestei polițe de asigurare și a documentelor care confirmă plata primei de asigurare înainte de data începerii perioadei de construcție.
\item Clientul este indicat ca beneficiar în cazul plății acoperirii asigurării într-o sumă cel puțin suficientă pentru a recupera valoarea de restaurare a clădirii.
\item Lucrările de construcție în clădire nu vor începe până când antreprenorul va asigura o poliță de asigurare încheiată în mod valabil.
\item Antreprenorul va menține polița de asigurare pe durata valabilității acordului și la cererea clientului trebuie să prezinte clientului originalul acestei polițe de asigurare sau să prezinte o copie a certificatului poliței sau să ofere acces la document prin intermediul CPE -platform-sunshineplatform.eu sau alte documente concludente care confirmă moneda și plata primei de asigurare.
\item Antreprenorul asigură asigurarea imobilului la costul propriu și la cheltuielile pentru întreaga perioadă de construcție. După finalizarea lucrărilor de renovare și după semnarea protocolului de livrare și acceptare, costurile de asigurare pentru perioada rămasă de valabilitate a acordului se împart între proprietarii de apartamente în proporție cu zona de apartament deținută în clădire și incluse în facturile contractantului pentru operare și întreținere. Administratorul numit de părți se va asigura că facturile clientului emise proprietarilor de apartamente includ aceste costuri de asigurare.
\end{enumerate}
\item În plus, pentru perioada de construcție, contractantul trebuie să mențină o poliță de asigurare valabilă pentru asigurarea de răspundere civilă și profesională într-o sumă de cel puțin 110% din costurile totale de investiții pentru lucrările de renovare.
  \end{enumerate}


\subsection{ASIGURAREA CERINȚELOR}
\begin{enumerate}
\item Antreprenorul are dreptul la dreptul nelimitat de a atribui terților drepturile sale și de a pretinde oricare dintre creanțele datorate de client în baza prezentului contract. În special, antreprenorul are dreptul de a atribui creanțele din taxa de renovare oricărui destinatar care a încheiat o finanțare, înființare, cedare sau orice alt aranjament cu contractantul.
\item Atribuirea creanțelor nu scutește contractantul de obligațiile și obligațiile sale în temeiul prezentului acord. Cu toate acestea, cesionarul are drepturi introduse în prezentul acord în cazul în care contractantul nu își îndeplinește obligațiile în temeiul prezentului acord. Drepturile de încadrare vor viza doar înlocuirea unui contractant care se află în incapacitate de executare cu o altă entitate capabilă să îndeplinească toate obligațiile și obligațiile din prezentul acord, în beneficiul clientului și al cesionarului.
\item În cazul unei astfel de atribuții, antreprenorul va transmite clientului o notificare în termen de 5 (cinci) zile lucrătoare de la încheierea acestui contract de cesiune.
\item Prezentul acord este personal pentru client și nu poate fi atribuit sau transferat de către client fără notificarea prealabilă către contractant.
\item Dacă aantreprenorul suferă vreo fuziune sau achiziție, încheie proceduri de lichidare sau faliment, acordul rămâne valabil, iar dispozițiile sale sunt obligatorii pentru succesorii legali și cedentul Contractantului.
\end{enumerate}

\subsection{TITLUL MĂSURILOR INSTALATE ÎN CONSTRUCȚIE, CA PARTE A LUCRĂRILOR DE RENOVARE}
\begin{enumerate}
\item Titlul măsurilor care pot fi separate de clădire fără a provoca daune materiale aparține antreprenorului, dacă antreprenorul furnizează o contribuție financiară ca parte a lucrărilor de renovare. Dacă lucrările de renovare sunt finanțate integral de către client, titlul măsurilor aparține clientului.
\item Clientul nu poate și va lua toate măsurile rezonabile pentru a se asigura că niciunul dintre proprietarii de apartamente sau alți vizitatori, nu va înlătura, încărca (închiria sau închiria printre altele), va găzdui sau distruge, deteriora sau altera măsurile implementate ca parte a renovării funcționează în mod independent la partid, cu titlu asupra măsurii.
\item Antreprenorul are dreptul, fără aprobarea clientului (și anume, fără a fi necesară obținerea consimțământului fiecăruia din proprietarii de apartamente), să promită și să încurce măsurile (sau părți din) în beneficiul său unic sau terț, dacă:
\begin{enumerate}
\item antreprenorul are un titlu asupra acestora;
\item din punct de vedere tehnic este posibil să le demontați fără a deteriora material clădirea;
\item gajul și / sau obligația este necesară ca garanție pentru contribuția financiară a contractantului în temeiul prezentului acord. Contractantul nu are dreptul să promoveze măsurile pentru a strânge resurse financiare în alte scopuri decât pentru executarea prezentului acord; și
\item durata gajului și / sau grevarea nu depășește termenul acordului.
\end{enumerate}
\item În cazul în care contractul are titlul asupra măsurilor, la primirea tuturor plăților datorate în temeiul prezentului contract către contractant, titlul tuturor măsurilor implementate ca parte a lucrărilor de renovare din acord este considerat automat transferat clientului. În cele din urmă, contra unui preț nerambursabil de 1 EUR (un euro) plătit în avans la semnarea prezentului acord. Transferul titlului către client se verifică printr-o declarație de transfer semnată de antreprenor și client.
\end{enumerate}

\subsection{DREPTURILE DE PROPRIETATE INTELECTUALĂ ȘI PROGRAME (SOFTWARE)}
\begin{enumerate}
\item Antreprenorul se va asigura că toate drepturile de proprietate intelectuală și industrială (DPI) pentru măsurile puse în aplicare în clădire, inclusiv echipamente, materiale, sisteme, software sau orice alt lucru sau document furnizat de către contractant clientului în temeiul prezentului acord, sunt deținute de sau licențiat Antreprenorului. Părțile sunt de acord că astfel de drepturi rămân în proprietatea contractantului și nu transmit clientului. antreprenorul acordă clientului o licență perpetuă, irevocabilă, neexclusivă de redevență (cu drept de sub-licență) pentru a utiliza IPR-ul menționat în legătură cu utilizarea clădirii și nu pentru altă utilizare.
\item Clientul nu trebuie să modifice, să copieze sau să reverseze niciun software sau să-l îmbine cu orice alt software pe care antreprenorul l-a furnizat ca parte a lucrărilor de renovare. Pe durata termenului prezentului acord, contractantul va furniza clientului manuale de utilizare, informații tehnice și toate actualizările și revizuirile software-ului furnizat.
\item Antreprenorul despăgubește clientul de orice pretenții de care clientul este răspunzător legal pentru orice încălcare a drepturilor de proprietate intelectuală terță parte referitoare la orice parte din DPI furnizate de antreprenor. Obligația antreprenorului de a indemniza clientul împotriva unor astfel de cereri este supusă clientului:
\begin{enumerate}
\item dând contractantului o notificare scrisă promptă a cererii;
\item să nu facă nicio admitere sau să aducă atingere apărării contractului de către contractant sau a capacității contractantului de a negocia o soluționare satisfăcătoare;
\item permițând antreprenorului posibilitatea de a controla pe cheltuiala antreprenorului desfășurarea apărării și orice negocieri pentru soluționarea creanței; și
\item oferirea antreprenorului (pe cheltuiala antreprenorului), asistența și informațiile care ar putea fi solicitate în mod rezonabil de antreprenor pentru a asista antreprenorul în desfășurarea apărării și orice negocieri pentru soluționarea creanței.
\end{enumerate}
\item Antreprenorul, la opțiunea sa, înlocuiește sau modifică partea care contravine dreptului de proprietate intelectuală cu o parte care nu încălcă, sau procură pentru client dreptul de a utiliza această parte care încalcă. Căile de atac prevăzute în prezenta clauză sunt singurul și exclusivul remediu pentru încălcarea drepturilor de proprietate intelectuală.
\end{enumerate}


\subsection{MODIFICĂRI DE UTILIZARE A CONSTRUCȚIEI}
\begin{enumerate}
\item Clădirea este descrisă în condițiile specifice ale acordului, inclusiv utilizarea, suprafața și dimensiunea acesteia. Dacă există circumstanțe pe care s-au bazat calculele antreprenorului, se modifică la inițiativa clientului sau cu acordul sau indemnizația clientului, modificarea nu va afecta contractantul și executarea contractului. Modificările de utilizare a clădirii și modificările clădirii se evaluează având în vedere considerente economice (în special modificări ale costurilor), iar acordul se adaptează la noile circumstanțe în consecință.
\item Modificările de utilizare a clădirii includ:
\begin{enumerate}
\item Extinderea sau reducerea suprafeței clădirii;
\item Asamblarea, deteriorarea sau demontarea echipamentelor respective sau a altor instalații dacă are ca rezultat o creștere sau o scădere materială a consumului de energie sau a altor parametri tehnici ai clădirii;
\item Modificările de utilizare a clădirii (de exemplu, zona de apartamente este transformată în magazine, magazine, restaurante și birouri sau apartamente neutilizate / nelocuite încep să fie utilizate) care afectează consumul de energie al clădirii.
\end{enumerate}
\end{enumerate}

\subsection{ELIMINAREA ECHIPAMENTELOR ȘI A MATERIALELOR DECONECTATE ȘI / SAU DEMONTATE}
\begin{enumerate}
\item Antreprenorul va aranja, la propriile costuri, eliminarea deșeurilor generate în temeiul prezentului acord, în conformitate cu legislația și reglementările relevante privind eliminarea deșeurilor din Republica Letonia.
\item Antreprenorul va notifica în scris clientul în termen de cel mult 5 (cinci) zile lucrătoare înainte de prima activitate de eliminare planificată. O astfel de notificare acoperă toate echipamentele, materialele și alte active instalate în clădire și care urmează să fie demontate și înlocuite pentru implementarea și instalarea măsurilor în perioada de construcție.
\item În cazul în care clientul dorește să utilizeze oricare dintre echipamentele, materialele sau alte active deconectate sau demontate de antreprenor în perioada de construcție, acesta trebuie să îl anunțe pe contractor și să își asigure costurile proprii pentru preluarea și transportul.
 \end{enumerate}

\subsection{DATORII}
\begin{enumerate}
\item Antreprenorul răspunde pentru punerea în aplicare la timp a măsurilor în perioada de construcție convenită. Nerespectarea de către contractant a acestei obligații dă clientului la daune lichidate în proporție de 0,02% pe zi din costurile totale de investiții planificate. Daunele lichidate nu pot depăși 10% (zece la sută) din costurile de investiții planificate.
\item Clientul este răspunzător pentru plata în timp util a costurilor și tarifelor datorate conform prezentului acord. Antreprenorul are dreptul să solicite despăgubiri pentru întârzierea plății. Compensarea pentru întârzierea plăților corespunde la 0,1% pe zi din suma întârziată.
\item În cazul în care clientul nu efectuează nicio plată cuvenită și care rezultă din acord pe o perioadă mai mare de 90 (nouăzeci) de zile, în cursul căreia procedurile de soluționare a litigiilor în temeiul prezentului acord au fost implementate în mod corespunzător și eficient, contractantul are dreptul să rezilieze contractul datorat la implicit și încălcarea contractului.
\item Antreprenorul este răspunzător pentru păstrarea clădirii la standardele de confort stabilite în prezentul acord. Dacă în sezonul de încălzire, temperatura interioară a fost în medie de 2 (două) grade pe scara Celsius (luând în considerare acuratețea instrumentației) sub standardele de confort stabilite în prezentul acord în oricare dintre apartamente, antreprenorul va fi responsabil să instruiască. Managerul pentru a reduce factura clientului pentru proprietarul apartamentului, după cum urmează:
\begin{enumerate}
\item Reducere de 5% (cinci procente) pentru fiecare grad Celsius din taxa de energie pentru fiecare lună a sezonului de încălzire, când temperatura era sub standardele de confort convenite;
\item Determinarea temperaturii interioare și dacă nivelul temperaturii a fost sub standardele de confort se determină în conformitate cu procedurile de soluționare a litigiilor din prezentul acord.
\item Antreprenorul nu va aplica reducerea dacă s-a produs scăderea temperaturii interioare în apartament: (i) ca urmare a acțiunilor sau omisiunilor ocupanților sau proprietarilor de apartamente cu încălcarea prezentului acord; (ii) scăderea este o consecință a neexecutării obligațiilor clientului; sau (iii) ca urmare a altor motive care nu sunt imputabile culpei contractantului.
\end{enumerate}
\item Clientul este răspunzător pentru daune, manipulare sau manipulare, vandalism, sabotaj, furt (cu excepția cazului în care contractantul sau de către cei pentru care este responsabil responsabilul contractantului) cu măsurile, în special dacă afectează nivelul de economii de energie, Standardele de confort convenite sau siguranța persoanelor care locuiesc și folosesc clădirea. În acest caz, clientul:
\begin{enumerate}
\item Despăgubiți complet contractorul pentru costurile restabilirii măsurii relevante;
\item plătește despăgubiri contractantului în proporție de 10% (zece la sută) din costurile de restaurare pentru a acoperi costurile de administrare ale contractantului;
\item Costurile pentru restaurare se calculează pe baza prețurilor de piață aplicabile, relevante la momentul calculării;
\item determinarea răspunderii clientului pentru daune, manipulare sau alterare a măsurilor se stabilește în conformitate cu procedura de soluționare a litigiilor din prezentul acord.
\end{enumerate}
\item Antreprenorul va deține clientul inofensiv de orice răspundere, cheltuieli, cheltuieli, daune, onorarii pe care le poate suporta ca urmare a unei cereri sau reclamații, acțiuni administrative sau judecătorești îndreptate împotriva clientului de către autoritățile statului sau administrativ sau terțe părți care decurg din acțiuni. a antreprenorului sau care rezultă din eventualele încălcări ale drepturilor de proprietate intelectuală referitoare la măsurile executate de antreprenor. Clientul primește de la contractant rambursarea tuturor costurilor și cheltuielilor necesare în mod rezonabil pentru repararea tuturor daunelor directe rezultate din acțiunile antreprenorului întreprinse cu încălcarea oricărei legislații aplicabile. Rambursarea trebuie documentată și plătibilă în termen de 30 de zile lucrătoare de la primirea cererii respective de la client către contractant, indicând în mod expres suma datorată.
\item Taxele pentru furnizarea de servicii de interes economic general (inclusiv furnizarea de căldură) și sancțiunile aplicabile oricărei părți prevăzute în legile și reglementările în vigoare în Republica Letonia nu limitează obligațiile părților în temeiul prezentului acord, inclusiv răspunderea pentru părțile și plățile pentru daune, compensații și penalități contractuale lichidate aplicabile în temeiul prezentului acord.
\item Plata daunelor, compensațiilor și penalităților lichidate nu eliberează partea care a suportat de la îndeplinirea obligațiilor care îi revin în baza acordului.
\end{enumerate}

\subsection{ÎNCETAREA ACORDULUI}
\begin{enumerate}
\item În cazul în care oricare dintre părți încalcă o dispoziție semnificativă a prezentului acord, partea care nu este implicată poate denunța imediat prezentul acord și trebuie să solicite părții care nu implică despăgubirea părții care nu este implicată în conformitate cu prezentul acord.
\item Încetarea acordului înainte de data începerii și investițiile pentru lucrările de construcție și instalare nu au avut loc:
\begin{enumerate}
\item În cazul rezilierii unilaterale a contractului de către client din cauza unei neplăceri materiale sau a încălcării contractului de către contractant, clientul are dreptul la o despăgubire în valoare de 1% din costurile de investiție (fără TVA) planificate în acord.
\item În cazul rezilierii unilaterale a contractului de către contractant datorită neîndeplinirii sau încălcării acordului de către client, contractantul are dreptul la o compensare în valoare de 1% din costurile de investiție (fără TVA) planificate în acord.
\item În cazul rezilierii unilaterale a contractului de către client din cauza altor motive comerciale sau comerciale care nu sunt neapărat legate de prezentul acord, antreprenorul are dreptul la o despăgubire în valoare de 1% din costurile de investiții (fără TVA) planificate în acord.
\item În cazul rezilierii unilaterale a contractului de către contractant din cauza altor motive comerciale sau comerciale care nu sunt neapărat legate de prezentul acord, clientul are dreptul la o compensare în cuantum de 1% costuri de investiții (fără TVA) planificate în acord.
\end{enumerate}
\item Încetarea acordului după aceea au avut loc costurile de investiții pentru lucrările de construcție și instalare ale măsurilor și au fost acoperite de contribuția financiară a contractantului:
\begin{enumerate}
\item În cazul rezilierii unilaterale a contractului de către client din cauza unei neplată materiale sau a unei încălcări a contractului de către antreprenor, clientul nu rambursează decât suma restantă a contribuției financiare realizată de antreprenor, actualizată cu 3%, care funcționează. în conformitate cu criteriile de performanță stabilite în acord. În plus față de cele de mai sus, clientul are dreptul să primească întreaga documentație a proiectului, furnizând în detaliu lucrările executate până în prezent, împreună cu toate permisele, licențele sau alte documente obținute de antreprenor în temeiul prezentului acord și finalizarea majorității lucrări urgente, toate garanțiile producătorilor, orice sub-licențe (și transferul oricăror licențe) pentru utilizarea drepturilor și software-ului de proprietate intelectuală necesare (inclusiv, software-ul instalat și după caz, orice documentație însoțitoare, informații legate de cod, orice sursă cod, fișiere de date, calcule, suporturi electronice, imprimări sau informații conexe), instruire suplimentară oricărei terțe părți care a fost numită în mod expres de către client în cazul în care s-a finalizat implementarea Lucrărilor de renovare.
\item În cazul rezilierii unilaterale a contractului de către contractant din cauza unei neplată materiale sau a unei încălcări a acordului de către client, clientul va rambursa suma restantă a contribuției financiare plus o compensație în proporție de 3% din suma fiind rambursat. Clientul are dreptul să primească întreaga documentație a proiectului, furnizând în detaliu lucrările executate până în prezent, împreună cu toate permisele, licențele sau alte documente obținute de antreprenor în conformitate cu prezentul acord, toate garanțiile producătorilor, orice sub-licențe ( și transferul oricăror licențe) pentru utilizarea drepturilor și software-ului de proprietate intelectuală necesare (inclusiv, software-ul instalat și după caz, orice documentație însoțitoare, informații legate de cod, orice cod sursă, fișiere de date, calcule, suporturi electronice, imprimări sau informații conexe).
\item În cazul rezilierii unilaterale a contractului de către client din cauza altor motive comerciale sau comerciale care nu sunt neapărat legate de prezentul contract, antreprenorul va avea dreptul la o compensație corespunzătoare principalului restant al contribuției financiare, făcut de contractant, plus o compensație la o rată de 3% din suma restituită.
\item În cazul rezilierii unilaterale a contractului de către contractant din cauza altor motive comerciale sau comerciale care nu sunt neapărat legate de prezentul acord, contractantul va avea dreptul la o compensație corespunzătoare principalului restant al contribuției financiare, făcut de contractant, actualizat cu 3% din suma calculată.
\end{enumerate}
\item Antreprenorul sau oricare dintre cesionarii săi emite o factură către client pentru compensația calculată indicând clar, informațiile utilizate pentru un astfel de calcul pe baza programelor de plată emise în perioada de serviciu sau facturile plătite în timpul executării lucrărilor de renovare înainte de data de încetare a acordului. Clientul plătește în termen de 60 (șaizeci) de zile de la data emiterii facturii pentru compensația datorată antreprenorului sau oricăruia dintre primitorii, cesionarii sau alte asemenea entități identificate unilateral ca având dreptul legal la toate sau la unele drepturi ale contractantul în temeiul prezentului acord.
\item Încetarea anticipată a acordului este notificată de către partea care încheie acordul în scris (aviz de reziliere) cu cel puțin 20 (douăzeci) zile lucrătoare înainte. În cazul în care rezilierea acordului este datorată unei neîndepliniri sau a unei încălcări a acordului de către o parte, o notificare scrisă valabilă de reziliere trebuie să includă etapele întreprinse în cadrul procedurilor de soluționare a litigiilor din prezentul acord și a documentației de susținere aferente.
\item Clientul are dreptul în orice moment să solicite și să primească de la antreprenor un calcul al sumei care trebuie compensată contractantului în cazul rezilierii anticipate a contractului.
\item În general, rezilierea acordului nu eliberează părțile de la îndeplinirea obligațiilor prevăzute în acord, care au avut loc înainte de momentul încetării acordului, cu excepția cazului în care părțile au convenit în scris despre alte dispoziții sau dacă acordul prevede altfel. În special, rezilierea unilaterală a contractului de către client în cazul unei neplată materiale sau a unei încălcări a contractului de către contractant nu eliberează de la sine de către obligațiile de plată a facturilor emise pentru perioadele anterioare datei de reziliere. din acord.
\item Reorganizarea, schimbarea acționarilor și / sau proprietatea, modificarea administrării părților, inclusiv modificările proprietarilor de apartamente ale clădirii, nu servesc drept temei pentru încetarea acordului sau pentru neexecutarea obligațiilor cuprinse în acord.
\item În plus față de dispozițiile acordului, părțile pot rezilia acordul în orice moment, după acordul reciproc scris în legătură cu condițiile de reziliere.
\item Partea îndreptățită la despăgubiri solicită compensații fie exercitându-și drepturile în temeiul prezentului acord, fie în conformitate cu legile și reglementările relevante în vigoare în Republica Letonia. Cu toate acestea, partea îndreptățită nu va primi dubla compensație pentru aceeași întârziere sau încălcare.
\item Părțile pot conveni să demonteze măsurile deținute sau deținute parțial de antreprenor din clădire dacă acordul este reziliat din timp, din cauza unor circumstanțe și dacă valoarea măsurilor respective este convenită și contabilizată ca o compensație de către partea afectată. Posibilitatea de a demonta măsurile în circumstanțele prevăzute în prezentul alineat, nu aduce atingere nici unei cereri de daune, cheltuieli și cheltuieli la care oricare dintre părți ar putea avea dreptul pentru rezilierea anticipată a prezentului acord.
\end{enumerate}

\subsection{EVENIMENTE  DE FORȚĂ MAJORĂ}
\begin{enumerate}
\item Orice situație de urgență sau eveniment neprevăzut în prealabil, caracterizat prin toate caracteristicile de mai jos, se definește ca forță majoră:
  \begin{enumerate}
\item părțile nu sunt în măsură să o prezică și să o influențeze;
\item intervine cu îndeplinirea de către părți a obligațiilor lor;
\item nu poate fi calificată ca o eroare sau neglijență făcută de niciuna dintre părți;
\item poate fi dovedit sau recunoscut ca insurmontabil, deși partea / părțile au depus / au depus eforturi rezonabile pentru prevenirea acesteia.
\end{enumerate}
\item Evenimentele de forță majoră includ, fără a se limita la, război, dezastre naturale și acte juridice ale administrației statului.
\item NU sunt evenimente de forță majorată: defecte ale măsurilor, servicii care nu au calitatea sau cantitatea convenite, echipamentele sau materialele utilizate, furnizate de către contractant sau instalate sau întârzierea funcționării acestora (dacă nu sunt cauzate de evenimente de forță majoră), litigii ale clienților, greve , dificultăți financiare sau altele asemenea specifice părții care se bazează pe un eveniment de forță majoră.
\item Părțile nu vor fi responsabile pentru neexecutarea completă sau parțială a obligațiilor în temeiul acordului, din cauza unor evenimente de forță majoră. Partea care se referă la evenimentul de forță majoră va face dovada celeilalte părți.
\item Partea („Partea afectată”) împiedicată să-și îndeplinească obligațiile prezentate în conformitate cu aceasta va notifica imediat și cel târziu în termen de 3 (trei) zile lucrătoare celeilalte părți ale unui eveniment de forță majoră, după ce a fost prevăzută sau de a deveni cunoscută către, partea afectată care descrie situația care va avea loc sau a apărut furnizând o descriere a evenimentului, durata posibilă, consecințele preconizate și soluția potențială a acestuia.
\item,6. Părțile vor desfășura toate activitățile necesare în comun sau în mai multe rânduri pentru a atenua impactul evenimentului de forță majoră și vor lua măsuri rezonabile pentru a atenua orice daune cauzate.
\item Dacă evenimentul de forță majoră continuă mai mult de 6 (șase) luni neîntrerupte și încetarea acestuia nu este așteptată pentru încă 3 (trei) luni, contractantul sau clientul are dreptul să rezilieze acordul unilateral.
\end{enumerate}

\subsection{CONFIDENȚIALITATE}
\begin{enumerate}
\item Informațiile obținute în cursul încheierii sau în timpul executării acordului care nu sunt în general disponibile pentru terțe părți și dezvăluirea căreia partea care primește este conștientă sau ar fi trebuit să o cunoască, poate dăuna drepturilor sau intereselor legale ale părții divulgatoare, se consideră confidențiale.
\item Părțile sunt de acord să nu dezvăluie informații confidențiale ale celeilalte părți către terțe părți, precum și să nu dezvăluie date ale celeilalte părți care ar putea fi utilizate în scopul concurenței sau al comiterii de activități ilegale atât în ​​timp cât acordul este valabil, cât și pentru 3 (trei) ani după ce acordul își pierde valabilitatea.
\item Informațiile care au intrat în domeniul public de către terți fără ca părțile să încalce dispozițiile acordului nu sunt considerate confidențiale.
\item Părțile pot dezvălui informațiile confidențiale terților pentru îndeplinirea obligațiilor acordului. Dacă părțile divulgă informații confidențiale bazate pe această clauză, acestea se asigură că terța parte va respecta aceleași obligații de confidențialitate stabilite în prezentul acord.
\item Dezvăluirea informațiilor confidențiale necesare în conformitate cu legile și reglementările în vigoare în Republica Letonia nu este considerată o încălcare a acordului.
\item În scop publicitar și în scopul informării publicului larg, contractantul, toți cesionarii săi și clientul au dreptul să dezvăluie informații generale despre cooperarea reciprocă, inclusiv, printre altele: divulgarea informațiilor care există deja în domeniul public despre părți, natura cooperare, a obținut date privind economiile de energie și consumul de energie. Aceasta în măsura în care divulgarea informațiilor nu încalcă drepturile și interesele legale ale celeilalte părți cu privire la protecția informațiilor confidențiale. Dacă o parte are îndoieli cu privire la natura informațiilor specifice, înainte de dezvăluirea acestora, această informație este aprobată de partea (părțile) ale căror drepturi și interese legale ar putea fi încălcate prin divulgarea acestor informații dacă această parte consideră că aceste informații sunt nu intră în obligația confidențialității din acord.
\item Cele de mai sus nu aduc atingere obligației exprese a clientului de a nu solicita sau sfătui niciun client, potențial client sau contact de afaceri al antreprenorului și / sau orice entități în curs de dezvoltare să reducă, să anuleze, să retragă, să limiteze, să reducă sau să restricționeze activitatea lor de la antreprenor .
\item Dispozițiile de mai sus nu aduc atingere dreptului contractantului de a colecta, prelucra, stoca, transforma, transfera destinatarilor săi parteneri de finanțare și de a difuza toate datele colectate de la client în scopul îmbunătățirii calității serviciilor sale și pentru dezvoltarea, operarea și întreținerea online Sunshine platform - sunshineplatform.eu care sprijină toate etapele și părțile participante la un proiect tipic EPC.
\end{enumerate}

\subsection{CONCLUZII ȘI MODIFICAREA ACESTUI ACORD}
\begin{enumerate}
\item Acordul intră în vigoare în ziua în care a fost semnat de către toate părțile în temeiul prezentelor termene și condiții generale și va rămâne în vigoare până la îndeplinirea completă a tuturor obligațiilor sale de către părți.
\item Toate modificările, completările și modificările la acord se fac în scris, cu acordul reciproc al tuturor părților și, vor intra în vigoare după semnarea tuturor părților și vor fi atașate la prezentul acord sub formă de anexe.
\item Toate dispozițiile rămase ale termenilor și condițiilor sau ale anexelor respective rămân în vigoare și efect. Orice abatere convenite se aplică numai părții din acord pentru care au fost convenite abaterile menționate.
\item Contractul se consideră reziliat după ce clientul și-a descărcat integral plata tuturor taxelor către contractant și contractantul și-a îndeplinit obligațiile.
\item Dacă în perioada de valabilitate a acordului intră în vigoare modificările legilor și reglementărilor din Letonia care fac îndeplinirea completă sau parțială a obligațiilor din prezentul acord, acesta nu va afecta validitatea obligațiilor rămase în temeiul prezentului acord. În acest caz, părțile introduc modificări adecvate acordului cu intenția și scopul de a reduce impactul economic al părților în acord.
\end{enumerate}

\subsection{REPREZENTAREA PĂRȚILOR}
\begin{enumerate}
\item Pentru prezentul acord, părțile sunt reprezentate de reprezentanții lor legitimi (pentru persoanele juridice) sau de persoanele specificate în prezentul acord. Doar persoanele menționate în condițiile specifice ale prezentului acord au dreptul să reprezinte clientul sau respectiv contractantul.
\item Acordul este întocmit și semnat în 3 (trei) copii originale în letonă cu efect juridic egal. Părțile certifică prin semnarea acestuia că înțeleg conținutul, sensul și consecințele acordului; aceștia recunosc că acest acord este corect, benefic reciproc, implicând toate dispozițiile, promisiunile, condițiile și reprezentările intențiilor dintre părți și că doresc în mod voluntar să-l execute fără exercitarea vreunui act asupra voinței oricăreia dintre ele.
\end{enumerate}

\subsection{NOTIFICĂRI}
\begin{enumerate}
\item Formular de notificare: toate notificările, cererile, revendicările, cerințele și alte comunicări solicitate sau permise de termenii prezentului acord se transmit și se transmit adreselor părților.
  \item Mod de notificare: toate notificările vor fi furnizate (i) prin livrare în persoană (ii) de către un serviciu de curierat de a doua zi, (iii) prin poștă înregistrată sau certificată, (iv) prin fax, (v) prin poștă electronică (e-mail ) cu cererea de primire a livrării la adresa părții specificată în prezentul acord sau orice altă adresă pe care oricare dintre părți o poate specifica în scris, (vi) de serviciile de notificare date de Sunshine platform - sunshineplatform.eu. (vii) pentru anunțurile cu o cantitate mică de mesaje SMS cu confirmare de primire trimise la numărul de telefon al părții specificat în prezentul acord sau alte numere pe care oricare dintre părți le poate specifica în scris.
  \item Primirea avizului: toate notificările sunt eficiente la data (i) primirii de către partea căreia i s-a transmis o notificare sau (ii) în a șaptea (a șaptea) zi de la expediere, după caz.
\end{enumerate}

\end{multicols}
