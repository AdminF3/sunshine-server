\begin{multicols}{2}
[\section{Appendix 2 {-} Your service Company’s commitment}]

\subsection{INTRODUCTION}
\textbf{The European Code of Conduct for Energy Performance Contracting} (the "EPC Code
of Conduct") \textbf{is a set of values and principles} that are considered fundamental
for the successful, professional and transparent implementation of Energy
Performance Contracting ("EPC") projects in European countries.
The EPC Code of Conduct defines the principles of the behaviour primarily of
the textbf{EPC} providers. At the same time, the EPC Code of Conduct is an EPC quality
indicator for textbf{Clients} on what they should expect and require from EPC providers
and which principles they themselves should adhere to in order to achieve
expected energy savings and related benefits.
The EPC Code of Conduct is a voluntary commitment and is not legally binding.
The key message of the Code of Conduct is that EPC represents a fair energy
service business model.
According to the Energy Efficiency Directive 2012/27/EU (EED), Energy
Performance Contracting "means a contractual arrangement between the
beneficiary and the provider of an energy efficiency improvement measure,
verified and monitored during the whole term of the contract, where investments
(work, supply or service) in that measure are paid for in relation to a
contractually agreed level of energy efficiency improvement or other agreed
energy performance criterion, such as financial savings." EPC projects may also
include additional services related to efficient energy supply.
Within this text, EPC provider means an energy service provider who delivers
energy service in the form of EPC. Client means any natural or legal person to
whom an EPC provider delivers energy service in the form of EPC.

\subsection{VALUES}
The EPC Code of Conduct reflects the values shared among European EPC
providers, which makes EPC a remarkable approach to energy efficiency. These
values illustrate the effective, professional and transparent approach to
managing EPC projects in terms of:
Efficiency:
\begin{itemize}
	\item	Energy savings
	\item	Economic efficiency
	\item	Sustainability in time
\end{itemize}
Professionalism:
\begin{itemize}
	\item	Expertise
	\item	High{-}quality service
	\item	Health and safety concerns
	\item	Good name in the sector and project
	\item	Reliability
	\item	Responsibility
	\item	Respect
	\item	Responsiveness
	\item	Objectivity
\end{itemize}
Transparency:
\begin{itemize}
	\item	Integrity
	\item	Openness
	\item	Long{-}term approach
	\item	Transparency of all steps and financing arrangements
	\item	Clear, regular and honest communication
\end{itemize}

\subsection{PRINCIPLES}
The EPC Code of Conduct consists of a set of nine guiding principles on EPC
projects implementation to support the high quality and transparency of
European EPC markets.
The principles use the term "savings", which means energy savings and/or
related financial savings.
\textbf{The EPC provider delivers economically efficient savings}
The EPC provider aims at an economically efficient combination of energy
efficiency improvement measures as part of the Renovation Works. This
combination maximises the net present value of an EPC project for the Client
defined as the sum of all the discounted costs and benefits (especially
operational cost savings) associated with implementing the project.
\textbf{The EPC provider takes over the performance risks}
The EPC provider assumes the contractually agreed performance risks of the
project during the whole duration of the EPC contract (the "contract"). These
include the risks of not achieving contractually agreed savings as described
below as well as design risks, implementation risks and risks related to the
operation of installed measures. 
\textbf{Savings are guaranteed by the EPC provider and determined by M\&V} 
The EPC provider guarantees that the contractually agreed level of savings will
be achieved. If an EPC project fails to achieve performance specified in the
contract, the EPC provider is obligated by the contract to compensate savings
shortfalls that occurred over the life of the contract. The excess savings
should be shared in a fair manner according to the methodology defined in the
contract. 
Contractually agreed savings as well as achieved savings are determined in a
fair and transparent manner by Measurement and Verification (M\&V) using
appropriate methodology (such as IPMVP) as defined in the contract. The
contractually agreed savings are determined based on data provided by the
Client and realistic assumptions. The achieved savings are calculated as the
difference between energy consumption and/or related costs before and after the
implementation of the Renovation Works.
\textbf{The EPC provider supports long{‐}term use of energy management} 
The EPC provider actively supports the Client in the implementation of an
energy management system during the contract period and eventually after the
contract period by agreement. This helps sustain the benefits from the project
even after the contract period. 
\textbf{The relationship between the EPC provider and the Client is long{‐}term, fair and
transparent} 
The EPC provider works closely with the Client as partners with the common
objective of achieving the contractually agreed level of savings. The EPC
provider strives to keep its relationship long{‐}term, fair and transparent. 
Both the EPC provider and the Client provide access to their project{‐}relevant
information in a clear manner and both fulfil their obligations according to
the contract terms. For instance, the EPC provider is committed to informing
the Client about the results of Measurement and Verification of the savings,
while the Client is committed to informing the EPC provider about any changes
in the use and operation of its facilities during the contract duration that
could affect energy demand. 
\textbf{All steps in the process of the EPC project are conducted lawfully and with integrity} 
The EPC provider and the Client comply with all laws and regulations that apply
to the EPC project in the country in which it is implemented. The EPC provider
and the Client avoid conflicts of interest and have a zero{‐}tolerance policy to
corruption and self{‐}dealing. 
\textbf{The EPC provider supports the Client in financing of EPC project}
The EPC provider supports the Client in finding the most suitable solution
providing for project financing taking into account the relevant conditions of
both parties. The capital to finance the EPC project can either be supplied out
of the Client's own funds, by the EPC provider or by a third party. Provision
of financing by the EPC provider is an option, not a necessary part of the EPC
project. 
\textbf{The EPC provider ensures qualified staff for EPC project implementation} 
The EPC provider maintains a qualified staff in order to provide the right
technical, commercial, legal and financial know{‐}how and skills. It ensures that
its experts have the adequate qualifications and capacities related to the
preparation and implementation of the EPC project. Less experience on the
Client's side can be balanced by a specialised advisory company (such as an EPC
facilitator) that will steer it toward the correct implementation and
procurement of the EPC project. 
\textbf{The EPC provider focuses on high quality and care in all phases of project implementation} 
The EPC provider uses well{‐}designed procedures, high{‐}quality and reliable
equipment and products, and works with reliable sub{‐}suppliers. It adheres to
the principles of ethical business conduct, meets its obligations towards
sub{‐}suppliers, and conducts itself responsibly with respect to the Client and
its representatives. 

\end{multicols}
