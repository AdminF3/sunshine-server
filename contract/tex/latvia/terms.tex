\begin{multicols}{2}
[\section{VISPĀRĪGIE NOTEIKUMI UN NOSACĪJUMI}]

\subsection{DEFINĪCIJAS}
\begin{itemize}[label={}]
	\item\textbf{Līgums:} šis Energoefektivitātes pakalpojuma līgums, kas noslēgts starp Pasūtītāju un Izpildītāju, ieskaitot Īpašos noteikumus un tā Pielikumus, un Vispārīgos noteikumus un nosacījumus, kas izstrādāts un tiek pārvaldīts ar EPC Platformas sunshineplatform.eu starpniecību.
	\item\textbf{Dzīvoklis:} Dzīvokļa īpašums ir dzīvojamā mājā tiesiski nodalīts patstāvīgs nekustamais īpašums.
	\item\textbf{Dzīvokļa} īpašnieks: Dzīvokļa īpašnieks ir persona, kas ieguvusi dzīvokļa īpašumu un īpašuma tiesības nostiprinājusi zemesgrāmatā.
	\item\textbf{Banku} darba dienā: ir diena, kad, vietā, kur saskaņā ar Līguma noteikumiem jāizdara bezskaidras naudas pārskaitījumi (angl. “bank transfers”), komercbankas veic vispārējās banku operācijas.
	\item\textbf{Bāzlīnija:} nozīmē siltumenerģijas un mājsaimniecību karstā ūdens patēriņu ēkā, kas izteikts gada vidējā vērtībā, kura vērojams Bāzlīnijas novērtēšanas periodā.
	\item\textbf{Bāzlīnijas} novērtēšanas periods: ir savstarpēji atrunāts laika periods, kas atspoguļo Ēkas rādītājus pirms Pasākumu īstenošanas.
	\item\textbf{Ēka:} daudzdzīvokļu dzīvojamā māja, kur Izpildītājs veic Atjaunošanas darbus un sniedz Pakalpojumus saskaņā ar šo Līgumu.
	\item\textbf{Darba} diena:  oficiāla darba diena, kas saskaņā ar Latvijas tiesību aktiem nav valsts svētku diena vai oficiāla brīvdiena.
	\item\textbf{Pasūtītājs:} Ēkas Dzīvokļa īpašnieks (īpašnieki) vai to pilnvarotā persona.
	\item\textbf{Komforta} standarti: Iekštelpu klimata apstākļi un rādītāji, kurus Izpildītājs garantē Pasūtītājam saskaņā ar šo Līgumu.
	\item\textbf{Sākuma} datums: diena, kurā sākas Būvniecības periods.
	\item\textbf{Nodošanas} ekspluatācijā datums: diena, kurā Puses paraksta Pasākumu pieņemšanas-nodošanas aktu, un diena, kurā sākas Līgumā noteiktais Pakalpojumu sniegšanas periods.
	\item\textbf{Būvniecības} periods:  Izpildītāja plānots periods Pasākumu īstenošanai. Būvniecības periods sākas Sākuma datumā un noslēdzas Nodošanas ekspluatācijā datumā.
	\item\textbf{Izpildītājs:} juridiska persona, kas apņemas izpildīt šo līgumu, Atjaunošanas darbus un sniedz Pakalpojumus saskaņā ar Līguma noteikumiem.
	\item\textbf{Pieņemšanas-nodošanas} akts: Izpildītāja saskaņā ar Latvijas tiesisko regulējumu un Normatīviem sagatavots akts Izpildītāja Ēkā īstenoto Pasākumu galīgai nodošanai ekspluatācijā.
	\item\textbf{Maksa} par karsto ūdeni: Pasūtītāja Izpildītājam maksājama Maksa, kas ir maksājama par mājsaimniecību karstā ūdens faktisko patēriņu atbilstoši spēkā esošajam Siltumenerģijas tarifam.
	\item\textbf{Enerģija:} prece ar noteiktu vērtību — kurināmais, siltumenerģija, atjaunojamā enerģija, elektroenerģija vai jebkāds cits enerģijas veids.
	\item\textbf{Energoaudits:} darbības, kuras tiek veiktas, lai iegūtu informāciju par enerģijas patēriņa struktūru ēkās vai ēku grupās, procesos vai iekārtās, kā arī noteiktu un novērtētu ekonomiski pamatotas enerģijas ietaupījuma iespējas, un kuru rezultāti tiek apkopoti ziņojumā.
	\item\textbf{Garantētais} enerģijas patēriņš: Ēkā telpu apkurei un cirkulācijas zudumiem patērētais siltumenerģijas daudzums Pakalpojumu sniegšanas periodā, kuru Izpildītājs panācis atbilstoši Garantētajam enerģijas ietaupījumam un kas tiek izmantots Izlīdzinātā siltumenerģijas patēriņa noteikšanai.
	\item\textbf{Energoefektivitātes} pakalpojums (pakalpojumi): Izpildītāja veiktu darbību kopums, ieskaitot Pasākumu īstenošanu Ēkā, īstenoto Pasākumu ekspluatācija un apkope (uzturēšana), enerģijas patēriņa datu analīzes, enerģijas patēriņa kontrole un novērtēšana jo īpaši saistībā ar Garantētā enerģijas ietaupījuma izpildi.
	\item\textbf{Enerģijas} ietaupījums: ietaupītās enerģijas apjoms, ko nosaka, izmērot vai novērtējot patēriņu pirms un pēc viena vai vairāku energoefektivitātes uzlabošanas pasākumu īstenošanas, un kas Ēkā panākts, īstenojot Pasākumus un sniedzot Energoefektivitātes pakalpojumus.
	\item\textbf{Garantētais} enerģijas ietaupījums: minimālais Enerģijas ietaupījuma apjoms, kas panākts Pasākumu īstenošanas un Energoefektivitātes pakalpojumu sniegšanas rezultātā, kuru Līgumā garantē Izpildītājs un ko nosaka saskaņā ar Mērījumu un kvalitātes pārbaudes plānu.
	\item\textbf{Enerģijas} tarifs: maksa par enerģijas vienību vietā, kur atrodas Ēka.
	\item\textbf{EPC} platforma sunshineplatform.eu: daudzpusīga tiešsaistes platforma ar daudzām iesaistītajām personām Energoefektivitātes pakalpojuma līgumu īstenošanai, kas pieejama tīmekļvietnē sunshineplatform.eu, kas atbalsta ēku atjaunošanas projektu izstrādi un vadību, izmantojot Energoefektivitātes pakalpojuma līgumu.
	\item\textbf{Maksa} (maksas): ikmēneša pastāvīga Maksa (maksas), ko Pasūtītājs maksā Izpildītājam par Līgumā noteikto Pakalpojumu sniegšanu Pakalpojumu sniegšanas periodā, kas ietver Maksu par siltumenerģiju, Maksu par karsto ūdeni, Maksu par atjaunošanu un Ekspluatācijas un apkopes maksu līdz ar jebkādiem piemērojamajiem nodokļiem (piemēram, PVN).
	\item\textbf{Izlīdzinātais} siltumenerģijas patēriņš:  siltumenerģijas apjoms, kuru Izpildītājs aprēķinājis nolūkā katrā Norēķinu periodā iekasēt no Pasūtītāja fiksētu mēneša maksu par siltumenerģiju visā Pakalpojumu sniegšanas periodā..
	\item\textbf{Finanšu} ieguldījums: Atjaunošanas darbos veikto ieguldījumu izmaksu daļa, kuru finansē tieši, no pašu kapitāla, vai netieši, Izpildītājam piesaistot trešo personu finansējumu, un saistībā ar kuru Izpildītājs iekasē Maksu par atjaunošanu.
	\item\textbf{Maksa} par siltumenerģiju: Pasūtītāja Izpildītājam maksātā Maksa, kas tam pienākas par Ēkā Pakalopjumu sniegšanas periodā patērēto enerģiju, ievērojot korekcijas un starpības (atlikuma) aprēķinu reizi gadā Norēķinu periodā, lai ņemtu vērā faktiskos laika apstākļus Norēķinu periodā un Garantētā enerģijas ietaupījuma Mērījumu un kvalitātes pārbaudes nolūkā.
	\item\textbf{Siltumapgāde:} siltumenerģijas piegāde Ēkai telpu apkures un mājsaimniecību karstā ūdens sagatavošanas vajadzībām.
	\item\textbf{Apkures} sezona: gada periods, kad Izpildītājam ir jāizpilda šajā Līgumā noteiktās Komforta standartu garantijas, sākot no katra norēķinu gada 1.\ oktobra līdz 30.\ aprīlim visā Pakalpojumu sniegšanas periodā.
	\item\textbf{Rēķins:} Pasūtītāja (Dzīvokļu īpašnieku vai Dzīvokļu īpašnieku pārstāvja) izrakstīts rēķins par saņemtajiem Pakalpojumiem un citiem maksājumiem, kuri Izpildītājam pienākas saistībās ar Līgumu, kas tiek izrakstīts, pilnībā ievērojot piemērojamajās normatīvās prasības saskaņā ar Latvijas tiesību aktiem.
	\item\textbf{IPMVP:} Starptautiskais izpildes rādītāju mērījumu un pārbaudes protokols attiecībā uz enerģijas taupīšanu, kuru sagatavojusi Efektivitātes vērtēšanas organizācija EVO (1629 K Street NW, Suite 300, Vašingtona, DC 20006, ASV) un kuru piemēro nolūkā Līguma ietvaros veikt mērījumus un kvalitātes pārbaudi.
	\item\textbf{LABEEF:} Latvijas-Baltijas  Energoefektivitātes fonds AS, kas darbojas kā Latvijas Komercreģistrā pienācīgi reģistrēta akciju sabiedrība ar uzņēmuma reģistrācijas numuru 40103960646
	\item\textbf{Slēptais} stāvoklis: Ēkas vai tās pieguļošajā teritorijā esoši trūkumi un defekti, par kuriem Pasūtītājs nezināja un kurus Izpildītājs nevarēja konstatēt saprātīgu novērojumu un parastas apsekošanas laikā Līguma sagatavošanas posmā.
	\item\textbf{Pārvaldnieks:} fiziska vai juridiska persona, kas saskaņā ar Latvijas Dzīvojamo ēku pārvaldīšanas likuma piemērojamajām normām un uz pārvaldīšanas līguma pamata veic Pasūtītāja uzdotas un Līgumā noteiktas pārvaldīšanas un apsaimniekošanas darbības.
	\item\textbf{Pasākums} (pasākumi), saukti arī – Energoefektivitātes pasākumi: tādas darbības, kuru rezultātā tiek panākts pārbaudāms, izmērāms vai aprēķināms energoefektivitātes pieaugums, un citi būvniecības un uzstādīšanas darbi ar mērķi atjaunot un uzlabot Ēku gan strukturālu, gan estētisku apsvērumu dēļ.
	\item\textbf{Mērījumi} un kvalitātes pārbaude: process un darbības, kas tiek veiktas nolūkā noteikt ar Ēku saistīto īstenoto Pasākumu un sniegto Pakalpojumu rezultātā panākto Enerģijas ietaupījumu.
	\item\textbf{Ekspluatācijas} un apkopes maksa: Pasūtītāja Izpildītājam maksāta Maksa par Pakalpojumiem, kas saistīti ar Pasākumu Ekspluatāciju un apkopi un kurai piemērojama ikgadēja indeksācija atbilstoši Latvijas patēriņa cenu indeksam attiecīgajā gadā, kuru publicē Centrālā statistikas pārvalde.
	\item\textbf{Ekspluatācijas} un apkopes rokasgrāmata:  rokasgrāmata, kurā norādīts saskaņā ar šo Līgumu īstenoto Pasākumu apkopes (uzturēšanas) grafiks un šajā Līgumā ietvertās ekspluatācijas darbības.
	\item\textbf{Puses:} Pasūtītājs un Izpildītājs kopā.
	\item\textbf{Puse:} Pasūtītājs un Izpildītājs katrs atsevišķi.
	\item\textbf{Maksājumu} grafiks: Izpildītāja Pasūtītājam sagatavots dokuments, kurā norādīta Maksa par atjaunošanu Finanšu ieguldījuma atmaksai, kas aprēķināta katram atsevišķam procentu aprēķina periodam saskaņā ar šo Līgumu.
	\item\textbf{Pienācīga} darbība: Pasākumu darbība tādā veidā, lai nodrošinātu visu funkciju pilnvērtīgu darbību un efektivitāti, un ietver visas nepieciešamās apkopes (uzturēšanas) darbības, kuras uzņēmies un par saviem līdzekļiem veic Izpildītājs.
	\item\textbf{Regulators:} Sabiedrisko pakalpojumu regulēšanas komisija vai cita kompetentā iestāde, kas izveidota saskaņā ar Latvijā spēkā esošajiem normatīvajiem aktiem, kas apstiprina siltumenerģijas tirdzniecības tarifus attiecīgajā vietējā pašvaldībā, kur atrodas Ēka.
	\item\textbf{Maksa} par atjaunošanu: atbilstoši EURIBOR likmei indeksēta Maksa, kuru Pasūtītājs maksā Izpildītājam saistībā ar Izpildītāja Finanšu ieguldījumu.
	\item\textbf{Atjaunošanas} darbi: darbības, kuras apņēmies veikt Izpildītājs un kuras nepieciešamas Pasākumu īstenošanai Ēkā, ieskaitot Pasākumu inženierpakalpojumus, iepirkumus, piegādes, uzstādīšanu, palaišanu, nodošanu ekspluatācijā un finansējumu.
	\item\textbf{Pakalpojumu} sniegšanas periods: periods, kurā Izpildītājs sniedz Pasūtītājam Pakalpojumus. Pakalpojumu sniegšanas periods sākas Nodošanas ekspluatācijā datumā.
	\item\textbf{Norēķinu} periods: viena kalendāra gada periods, kas Pakalpojumu sniegšanas periodā turpinās katru gadu.
	\item\textbf{Akts:} Pušu parakstīts dokuments, kas apliecina dažādus Ēkai raksturīgus rādītājus, kuri fiksēti šāda dokumenta sagatavošanas laikā.
	\item\textbf{PVN:} pievienotās vērtības nodoklis, kas maksājams saskaņā ar Latvijā spēkā esošajiem normatīvajiem aktiem un Līguma noteikumiem.
\end{itemize}

\subsection{LĪGUMA NOTEIKUMU PIEŅEMŠANA}
\begin{enumerate}
	\item Pasūtītājs uzskata, ka Izpildītājam ir nepieciešamās kvalifikācijas, pieredze un spējas veikt Atjaunošanas darbus un sniegt Pasūtītājam Pakalpojumus. Uz šā pamata Pasūtītājs pilnvaro Izpildītāju un palīdzēs Izpildītājam par Izpildītāja līdzekļiem veikt visas tiesiskās un faktiskās darbības nolūkā izpildīt Līgumu, bez nepieciešamības izsniegt Izpildītājam nepārprotamu pilnvaru.
	\item Izpildītājs apņemsies veikt Atjaunošanas darbus un sniegt Pasūtītājam Pakalpojumus saskaņā ar šajā Līgumā ietvertajiem noteikumiem un nosacījumiem. Izpildītājs apliecina, ka ir iepazinies ar Ēkas raksturu, stāvokli un atrašanās vietu, un visiem citiem aspektiem, kas jebkādā veidā varētu ietekmēt tā Līgumā noteiktās saistības. Jebkāda Izpildītāja neiepazīšanās ar Ēku vai jebkādiem Ēkas teritorijas apstākļiem saskaņā ar šo Punktu neatbrīvos to no atbildības par Līgumā noteikto saistību izpildi.
	\item Izpildītājs apstiprina, ka šā Līguma Īpašajos noteikumos norādītais budžets ietver visus būvdarbus, materiālus un aprīkojumu, kas ir nepieciešams Atjaunošanas darbu izpildei saskaņā ar projekta tehniskajām specifikācijām un šā Līguma nosacījumiem.
	\item Šā Līguma, tā Īpašo noteikumu, tā Pielikumu un tā Vispārīgo noteikumu un nosacījumu izpratnē definīcijām ir šā Līguma Vispārīgo noteikumu un nosacījumu 1.\ punktā norādītā nozīme.
	\item Gadījumā, ja rodas pretrunas starp Vispārīgajiem noteikumiem un nosacījumiem un Īpašajiem noteikumiem, un to Pielikumiem, noteicošie ir Īpašie noteikumi.
\end{enumerate}

\subsection{DROŠĪBA, KVALITĀTE UN KOMFORTS}
\begin{enumerate}
	\item Pakalpojumi, kurus Izpildītājs sniedz saskaņā ar šo Līgumu, ir:
	\begin{enumerate}
		\item jāveic augstākajā prasmju un rūpības līmenī, kāds sagaidāms no pieredzējušiem un profesionāliem izpildītājiem, kuri regulāri veic tāda paša vai līdzīga apjoma un sarežģītības pakāpes darbus un pakalpojumus kā šajā Līgumā;
		\item jāveic, izmantojot atbilstošas kvalitātes, jaunus, nolūkam atbilstošus materiālus un aprīkojumu;
		\item tiem ir jāatbilst būvniecības tiesiskajam regulējumam un citām piemērojamajām tiesību normām, noteikumiem vai normatīviem, kas Pakalpojumu sniegšanas laikā ir spēkā Latvijas Republikā;
		\item jāizpilda  tā, lai radītu pēc iepsējas mazākas neērtības, Pasūtītājam un citiem Ēkas iemītniekiem izmantojot Ēku;
	\end{enumerate}
	\item Komforta standarti Līgumā noteiktajā Pakalpojumu sniegšanas periodā atbilst vai pārsniedz šā Līguma Īpašajos noteikumos noteikto līmeni.
	\item Laikā, kad logi Ēkas Dzīvoklī ir atvērti un 2 (divas) stunda pēc logu aizvēršanas Izpildītājs negarantē Līguma Īpašajos noteikumos atrunāto telpu gaisa temperatūras līmeni konkrētajā Dzīvoklī, kur logi ir bijuši atvērti.
	\item Izpildītājam ir jānodrošina Dzīvokļos atbilstošs gaisa apmaiņas līmenis saskaņā ar Latvijas normatīvajiem aktiem.
	\item Izpildītājs veic visas nepieciešamās darbības, lai nodrošinātu darbinieku drošības un veselības aizsardzību darbā saskaņā ar Darba aizsardzības likumu un visiem atbilstošajiem Latvijas normatīvajiem aktiem.
	\item Izpildītājam ir jāīsteno atbilstoši aizsardzības pasākumu, lai visas personas pasargātu no Izpildītāja, tā darbinieku, pārstāvju vai apakšuzņēmēju neizpildes vai rupjas nolaidības izraisītas nāves vai traumas Būvniecības periodā vai Pakalpojumu sniegšanas periodā. Izpildītājam tāpat ir jāpasargā visa Ēka no Pasākumu īstenošanas rezultātā nodarītiem bojājumiem.
	\item Izpildītājam ir jānodrošina, lai visi Ēkai piegādātie komunālie pakalpojumi netiktu jebkurā laikā atslēgti vai pārtraukti Izpildītāja neizpildes vai nolaidības dēļ, iepriekš neinformējot par to. Izpildītājam ir nekavējoties par Izpildītāja līdzekļiem jāatjauno jebkādi komunālie pakalpojumi, kuru sniegšana ir pārtraukta vai kuri ir atslēgti Izpildītāja neizpildes vai nolaidības dēļ. Izpildītājs nav atbildīgs par gadījumiem, kad šādi pārtraukumi nav Izpildītāja kontrolē un/vai rodas apsaimniekošanas uzņēmuma, enerģētikas un ūdens apgādes uzņēmumu vai jebkādu ar Izpildītāju nesaistītu trešo personu darbības vai bezdarbības rezultātā.
	\item Izpildītājam Būvniecības periodā ir jānodrošina Ēkas pienācīga aizsardzība no laika apstākļu iedarbības, nepieļaujot lietus ūdens iesūkšanos un Ēkas bojājumus. Izņemot gruntsūdeņu iesūkšanos un nepārvaramas varas apstākļus.
	\item Izpildītājs ievēro Eiropas Rīcības kodeksu energoefektivitātes pakalpojuma līgumiem (http://www.transparense.eu/), kurā apkopotas par būtiskām uzskatītas vērtības un principi, kas ir pamatā veiksmīgiem, profesionāliem un pārskatāmiem Energoefektivitātes pakalpojuma līgumiem Eiropas valstīs.
\end{enumerate}

\subsection{GARANTIJAS}
\begin{enumerate}
	\item Izpildītājs Pakalpojumu sniegšanas periodā šā Līguma ietvaros nodrošina Garantēto enerģijas ietaupījumu, attiecībā uz kuru katru gadu tiek veikti Mērījumi un kvalitātes pārbaude.
	\item Izpildītājs Pakalpojumu sniegšanas periodā garantē saskaņā ar šo Līgumu nolīgtos Komforta standartus.
	\item Izpildītājs Pakalpojumu sniegšanas periodā par saviem līdzekļiem garantē apkures sistēmā, mājsaimniecību karstā ūdens apgādes sistēmās, Apkures, ventilācijas un gaisa dzesēšanas sistēmās, mezglos un cauruļvados Izpildītāja ierīkoto vai ieviesto Pasākumu Pienācīgu darbību saskaņā ar to specifikācijām un ievērojot normālu nolietojumu Līguma darbība laikā, tai skaitā nepieciešamības gadījumā veicot Pasākumu remontu vai nomaiņu.
	\item Izpildītājs Pakalpojumu sniegšanas periodā par saviem līdzekļiem garantē Izpildītāja uzstādīto vai ierīkoto siltumizolācijas materiālu iedarbīgumu un lietderīgumu atbilstoši to specifikācijām un normālam nolietojumam visā Līguma darbības laikā, tai skaitā nepieciešamības gadījumā novēršot to bojājumus vai nomainot tos.
	\item Izpildītājam Pakalpojumu sniegšanas perioda beigās ir jānodrošina visu īstenoto Pasākumu Pienācīga darbība atbilstoši to specifikācijām un normālam nolietojumam, un ņemot vērā pienācīgi apkopi. Izpildītājam Pakalpojumu sniegšanas perioda beigās ir jāiesniedz Pasūtītājam visas lietošanas, kopšanas un apkopes rokasgrāmatas, uzskaites dokumenti, cita dokumentācija, programmatūra, intelektuāla īpašuma licences, īpašie rīki un protokoli, un procedūras, kas nepieciešamas vai lietderīgas pastāvīgai Pasākumu labai veiktspējai, lai nodrošinātu šajā Līgumā noteiktos Komforta standartus.
	\item Izpildītājs pirms Būvniecības perioda sākuma iesniedz Pasūtītājam kredītiestādes vai apdrošināšanas sabiedrības izsniegtu saistību izpildes garantiju attiecībā uz tā saistību izpildi 10\% apmērā no kopējām Ieguldījumu izmaksām (bez PVN) atbilstoši zemāk norādītajam:
	\begin{enumerate}
		\item ja Izpildītājs ir arī ģenerāluzņēmēja sabiedrība, šo saistību izpildes garantiju Pasūtītāja labā sniedz Izpildītājs saskaņā ar šā Līguma noteikumiem;
		\item ja Izpildītājs piesaista ģenerāluzņēmēja sabiedrību, šo saistību izpildes garantiju Izpildītāja labā sniedz ģenerāluzņēmēja sabiedrība, balstoties uz starp Izpildītāju un ģenerāluzņēmēja sabiedrību noslēgta būvniecības līguma noteikumiem;
		\item gadījumā, ja Izpildītājs neiesniedz šādu saistību izpildes garantijas oriģinālu attiecībā uz Būvniecības periodu, nodrošinot darbību izpildi Būvniecības periodā, Izpildītājam nav tiesību uzsākt būvdarbus;
		\item šai saistību izpildes garantijai ir jābūt spēkā visu Būvniecības periodu. Gadījumā, ja Būvniecības periods tiek pagarināts, Izpildītājam par tādu pašu laika periodu ir jāpagarina šī garantija.
	\end{enumerate}
	\item Izpildītājam ne vēlāk kā 10 (desmit) dienu laikā pēc Pieņemšanas-nodošanas akta parakstīšanas ir jāiesniedz Pasūtītājam kredītiestādes vai apdrošināšanas sabiedrības izsniegta saistību izpildes garantija attiecībā uz tā saistībām vismaz 5\% apmērā no Ieguldījumu izmaksām (bez PVN) atbilstoši turpmāk norādītajam:

	\begin{enumerate}
		\item ja Izpildītājs ir arī ģenerāluzņēmēja sabiedrība, šo saistību izpildes garantiju Pasūtītāja labā sniedz Izpildītājs saskaņā ar šā Līguma noteikumiem;
		\item ja Izpildītājs piesaista ģenerāluzņēmēja sabiedrību, šo saistību izpildes garantiju Izpildītāja labā sniedz ģenerāluzņēmēja sabiedrība, balstoties uz starp Izpildītāju un ģenerāluzņēmēja sabiedrību noslēgta būvniecības līguma noteikumiem;
		\item šai garantijai ir jābūt spēkā 36 (trīsdesmit sešus) mēnešus;
	\end{enumerate}
	\item Pasūtītājam ir tiesības realizēt 4.6.\ un 4.7.\ punktā minētās saistību izpildes garantijas Izpildītāja finanšu saistību segšanai vai saskaņā ar normatīvajiem aktiem.
	\item Šajā punktā minētajai saistību izpildes garantijai ir jābūt izsniegtai Latvijas Republikā vai jebkurā citā Eiropas Savienības vai Eiropas Ekonomikas zonas dalībvalstī reģistrētā kredītiestādē vai apdrošināšanas sabiedrībā, kas saskaņā ar Latvijas Republikas tiesību aktos noteikto kārtību ir uzsākusi pakalpojumu sniegšanu Latvijas Republikas teritorijā.
\end{enumerate}

\subsection{IZPILDĪTĀJA TIESĪBAS UN PIENĀKUMI}
\begin{enumerate}
	\item Izpildītājam ir nepieciešamā profesionālā kvalifikācija, pieredze un spējas piegādāt aprīkojumu, materiālus un pakalpojumus saskaņā ar Līgumu.
	\item Izpildītājam ir jāiegūst visas Atjaunošanas darbu veikšanai un Pakalpojumu sniegšanai nepieciešamās atļaujas un saskaņojumi no valsts, pašvaldību iestādēm un pārvaldēm, neiesaistot Pasūtītāju, ja vien tas nav nepieciešams saskaņā ar piemērojamajiem normatīvajiem aktiem.
	\item Izpildītājam Pasākumu būvniecības un uzstādīšanas darbi ir jāuzsāk Sākuma datumā un jāpabeidz noteiktajā Būvniecības periodā. Izpildītājs informē Pasūtītāju par provizorisko Sākuma datumu ne vēlāk kā 20 Darba dienu laikā pēc šā Līguma parakstīšanas.
	\item Izpildītājs rakstveidā informē Pasūtītāju vismaz 10 (desmit) Darba dienas iepriekš par Pasākumu būvniecības un uzstādīšanas darbiem, dodot iespēju Pasūtītājam atbrīvot Ēkas koplietošanas platības (tai skaitā kāpņutelpas, pagrabtelpas, bēniņus, jumtu, ogļu/malkas un gāzes glabātavas, elektroapgādes un telekomunikāciju paneļus un apkures katlu telpas) no atkritumiem, bezsaimnieka mantas vai jebkāda cita tur esoša priekšmeta. Ja Pasūtītājs savlaicīgi neatbrīvo koplietošanas platības, Izpildītājam ir tiesības nodrošināt Ēkas koplietošanas platību atbrīvošanu un izrakstīt Pasūtītājam rēķinu par šādu darbu apmaksu radušos izdevumu atlīdzināšanai. Pasūtītājam ir nekavējoties jāapmaksā šāds rēķins ne vēlāk kā 20 Darba dienu laikā.
	\item Būvniecības periodā Izpildītājam ir jānodrošina viss Pasākumu īstenošanai nepieciešamais darbaspēks, ieskaitot nepieciešamo uzraudzību, atbilstoša rakstura, kvalitātes un daudzuma rīkus, materiālus un aprīkojumu.
	\item Pakalpojumu sniegšanas periodā Izpildītājam ir jānodrošina viss Pasākumu ekspluatācijai un apkopei nepieciešamais darbaspēks, ieskaitot nepieciešamo uzraudzību, atbilstoša rakstura, kvalitātes un daudzuma rīkus, materiālus un aprīkojumu.
	\item Būvniecības periodā Izpildītājam ir jānoorganizē elektroapgāde ar atsevišķu uzskaites ierīci un jāapmaksā Pasākumu īstenošanas un uzstādīšanas darbos patērētā elektroenerģija. Izpildītājam ir tiesības  pieslēgties Ēkas kopējai elektroapgādes sistēmai.
	\item Izpildītājam pēc Pasākumu būvniecības un uzstādīšanas darbu pabeigšanas, pirms Nodošanas ekspluatācijā datuma ir pienācīgi jāuzkopj būvobjekts (Ēkas koplietošanas platības, logi, ieejas un apkārtne).
	\item Izpildītājs uzaicina Pasūtītāju uz Ēkā īstenoto Pasākumu nodošanu ekspluatācijā. Izpildītājs Būvniecības perioda beigās iesniedz Pasūtītājam Ēkas Pieņemšanas-nodošanas aktu.
	\item Izpildītājs Pakalpojumu sniegšanas periodā rakstveidā informē Pasūtītāju gadījumā, ja Ēkas koplietošanas telpās Dzīvokļu īpašnieki vai citas ar Izpildītāju nesaistītas trešās personas glabā vai ir atstājuši atkritumus un/vai priekšmetus, kas var apgrūtināt saskaņā  ar Līgumu Izpildītāja veiktās ekspluatācijas un apkopes darbības. Ja Pasūtītājs neuzkopj platības saskaņā ar Līgumu, Izpildītājam ir tiesības nodrošināt telpu uzkopšanu un izrakstīt Pasūtītājam rēķinu kompensācijas par šādas uzkopšanas veikšanu samaksai. Pasūtītājam ir nekavējoties, bet ne vēlāk kā 20 Darba dienu laikā jāapmaksā šāds rēķins.
	\item Izpildītājs Pakalpojumu sniegšanas periodā rakstveidā informē Pasūtītāju par jebkādu atklātu Pasākumiem nodarītu bojājumu, zādzību, vandālismu vai sabotāžu.
	\item Izpildītājam Apkures sezonā ir jānodrošina pietiekama Siltumenerģijas apgāde Ēkā tādā veidā, lai nodrošinātu šajā Līgumā noteiktos Komforta standartus. Izpildītājs saskaņā ar šo Līgumu nav atbildīgs par traucējumiem vai Siltumenerģijas apgādes neesamību Ēkā gadījumos, kas nav Izpildītāja ziņā, tai skaitā siltumapgādes uzņēmuma nespējas piegādāt Siltumenerģiju vai Nepārvaramas varas apstākļu rezultātā..
	\item Izpildītājs pirms Būvniecības perioda uzsākšanas iesniedz Pasūtītājam Ēkas atjaunošanas projektu saskaņā ar Ministru kabineta 2014.\ gada 21.\ oktobra noteikumiem Nr.\ 655 ``Noteikumi par Latvijas būvnormatīvu LBN 310--14 ``Darbu veikšanas projekts'' un saskaņo to ar Pasūtītāju un būvuzraugu.
	\item Izpildītājs Būvniecības periodā informē un uzaicina Pasūtītāju piedalīties iknedēļas projekta progresa sanāksmēs attiecībā uz būvdarbiem. Izpildītājam tad ir jāiesniedz Pasūtītājam 2 (divi) progresa ziņojumi mēnesī par būvdarbu statusu un progresu. Šos ziņojumus var iesniegt elektroniski, izmantojot EPC tiešsaistes platformu sunshineplatform.eu.
	\item Gadījumos, kad Izpildītājs ir starpnieks starp Pasūtītāju un siltumenerģijas piegādātāju, Izpildītājs Pasūtītāja vārdā un uzdevumā apmaksā Siltumapgādes sabiedrības izrakstītos rēķinus, ieturot atbilstošo Maksas par siltumenerģiju komponenti, kas Izpildītājam pienākas saskaņā ar šo Līgumu. Neņemot vērā iepriekšminēto, rēķinus par apkuri Būvniecības periodā sedz Pasūtītājs.
	\item Izpildītājam ir tiesības uzdot vai nodot Līgumā noteikto darbu un pakalpojumu izpildi apakšuzņēmumā trešajām personām (apakšuzņēmējiem). Izpildītājs ir pilnībā atbildīgs par apakšuzņēmējiem attiecībā uz šajā Līgumā noteiktajām saistībām.
	\item Izpildītājam ir tiesības mainīt Garantēto enerģijas ietaupījumu gadījumā, ja mainās Ēkas izmantošana (17.\ punkts). Pusēm ir jāvienojas par jebkādām izmaiņām, noslēdzot rakstiskus grozījumus šajā Līgumā.
	\item Izpildītājam ir tiesības par saviem līdzekļiem ar Pasūtītāja iepriekšēju piekrišanu (un šādu piekrišanu nedrīkst nepamatoti liegt) pieaicināt pienācīgi kvalificētu un pieredzējušu neatkarīgu ekspertu piedāvāto Pasākumu atbilstības Latvijas Republikā spēkā esošajiem normatīvajiem aktiem vai Pasūtītājam saistošiem pašvaldības lēmumiem novērtēšanai gadījumā, ja Pasūtītājs izmanto savas veto tiesības. Šāda eksperta slēdziens ir Pusēm saistošs.
	\item Izpildītājs ne vēlāk kā 5 (piecu) Darba dienu laikā informē Pasūtītāju par izmaiņām Līgumā norādītajā adresē vai citām izmaiņām savā tiesiskajā statusā, administratīvajā un juridiskajā stāvoklī; jo īpaši, ja Izpildītājs ir iesaistīts jebkādā apvienošanās vai iegādes darījumā, uzsākts likvidācijas vai bankrota process.
\end{enumerate}

\subsection{PASŪTĪTĀJA TIESĪBAS UN PIENĀKUMI}
\begin{enumerate}
	\item Pasūtītājs ir pieņēmis spēkā esošu un izpildāmu lēmumu, ar kuru šis Līgums kļūst saistošs ikvienam Dzīvokļa īpašniekam, kuram katram atsevišķi ir pienākums izpildīt tā noteikumus, neatkarīgi no tā, vai konkrētais Dzīvoklis ir iznomāts vai izīrēts, un neatkarīgi no tā, vai konkrētos Dzīvokļus izmanto to attiecīgie Dzīvokļu īpašnieki vai ne.
	\item Pasūtītājam (katram Dzīvokļa īpašniekam) ir jāinformē nomnieki, īrnieki un jebkādi citi dzīvokļa iemītnieki vai pastāvīgie iedzīvotāji par attiecīgajām šajā Līgumā noteiktajām saistībām.
	\item Pasūtītājam ir jāiesniedz Līgumā norādīto Atjaunošanas darbu īstenošanai un Pakalpojumu sniegšanai Izpildītāja pieprasītā informācija un dokumenti, cik vien ātri iespējams pēc  Izpildītāja pieprasījuma saņemšanas. Pasūtītājs nav atbildīgs par jebkādu atbilstošo dokumentu nesniegšanu, kurus Izpildītājs nav pietiekami detalizēti norādījis.
	\item Pasūtītājam ir jāsniedz Izpildītājam savlaicīga palīdzība nepieciešamo atļauju, saskaņojumu vai citu ar veiksmīgu Līguma izpildi saistītu dokumentu saņemšanā no valsts, pašvaldības iestādēm un pārvaldēm, tai skaitā, bet ne tikai jāapliecina un/vai jāiesniedz nepieciešamie dokumenti, jāpiešķir vajadzīgās pilnvaras un jāsniedz Izpildītājam pieejamā informācija. Pasūtītājam ir pienācīgi, nepieciešamajā formā jāpilnvaro Izpildītājs veikt jebkādas faktiskas vai tiesiskas darbības kompetentajās iestādēs, lai veiksmīgi īstenotu Līgumu. Tomēr Pasūtītājs nav atbildīgs par jebkādas un visas šādas informācijas nesniegšanu, ja vien Izpildītājs to nav skaidri un detalizēti norādījis un šāda informācija nav pamatoti pieejama Pasūtītājam.
	\item Pasūtītājs nedrīkst radīt šķēršļus vai liegt piekrišanu Atjaunošanas darbiem, Pasākumu īstenošanai Būvniecības periodā un to apkopei (uzturēšanai) Pakalpojumu sniegšanas periodā; gluži pretēji – Pasūtītājam ir labā ticībā jārīkojas, lai veicinātu to īstenošanu un apkopi un Garantētā enerģijas ietaupījuma panākšanu.
	\item Pasūtītājam ir tiesības sūdzēties par īstenotā Pasākuma kvalitāti vai izpildes veidu 10 (desmit) Darba dienu laikā pēc Pieņemšanas-nodošanas akta parakstīšanas. Pēc šā termiņa jebkāds Izpildītāja īstenotais Pasākums uzskatāms par pieņemtu un Būvniecības periods – par noslēgušos.
	\item Pasūtītājs ir tiesīgs izmantot veto tiesības vai atteikties no Atjaunošanas darbu ietvaros plānotu Pasākumu, ja Pasūtītājs nepārprotami pierāda, ka attiecīgais Pasākums (Pasākumi) rada  Latvijas Republikā spēkā esošo normatīvo aktu vai Pasūtītājam saistošu pašvaldības lēmumu pārkāpumu.
	\item Pasūtītājam Būvniecības periodā un Līgumā noteiktajā Pakalpojumu sniegšanas periodā ir jānodrošina Izpildītājam vai jebkurai citai Izpildītāja pilnvarotai personai piekļuve Ēkai, tai skaitā katram Ēkas Dzīvoklim. Pasūtītājam ir jānodrošina piekļuve Ēkai jebkura laikā Darba dienās (no pulksten 8:00 līdz 20:00), savukārt ārkārtas situācijās – arī ārpus darba laika un nedēļas nogalēs un svētku dienās.
	\item Pasūtītājam pirms Sākuma datuma ir jānodrošina, lai koplietošanas platības (tai skaitā kāpņutelpas, pagrabtelpas, bēniņi, jumts, ogļu/malkas un gāzes glabātavas, elektroapgādes un telekomunikāciju paneļi un apkures katlu telpas) būtu atbrīvotas no atkritumiem, bezsaimnieka mantas vai jebkāda cita tur esoša priekšmeta, vienojoties ar Izpildītāju par to aizvākšanu vai nogādājot tos atkritumu savākšanas uzņēmumam vai arī nododot tos zināmam īpašniekam.
	\item Pasūtītājam Pakalpojumu sniegšanas periodā ir jānodrošina, lai Ēkas koplietošanas platības, piemēram, kāpņutelpas, pagrabtelpas un bēniņu telpas tiktu uzturētas tīras un labā darba kārtībā, veicot regulāru uzkopšanu.
	\item Pasūtītājs nedrīkst iejaukties Izpildītāja uzstādītajos Pasākumos bez Izpildītāja rakstiskas piekrišanas un atļaujas vai pretēji Izpildītāja ekspluatācijas instrukcijās norādītajam; jo īpaši, ja iejaukšanās nelabvēlīgi ietekmē Enerģijas ietaupījuma apjomu. Pasūtītāja iejaukšanās apkures sistēmas, mājsaimniecību karstā ūdens sistēmas un ventilācijas sistēmas iestatījumos uzskatāma par Pasūtītāja pieļautu savu šajā Līgumā noteikto saistību būtisku pārkāpumu un uzskatāma par pamatotu un pietiekamu iemeslu Līguma Izbeigšanai no Izpildītāja puses.
	\item Pasūtītājam ir jāveic visas saprātīgās darbības, lai nodrošinātu, ka Ēkā neviens nejaucas un negroza apkures sistēmas, mājsaimniecību karstā ūdens sistēmas un ventilācijas sistēmas iestatījumus un nekādā veidā nebojā un nesabotē Pasākumus.
	\item Pasūtītājs nekavējoties pēc atklāšanas (vienas Darba dienas laikā) paziņo Izpildītājam par jebkādu bojājumu vai izmaiņām, vai iejaukšanos Izpildītāja uzstādītajos Pasākumos.
	\item Pasūtītājam ir jāinformē par jebkādu apstākli, kam ir un/vai varētu būt negatīva ietekme uz Enerģijas ietaupījumu. Pasūtītāja neziņošana neatbrīvo Izpildītāju no atbildības par Garantētā enerģijas ietaupījuma nodrošināšanu, ja vien netiek konstatēts, ka Pasūtītājs ir plānojis samazināt Enerģijas ietaupījuma līmeni.
	\item Pasūtītājam ne vēlāk kā 20 (divdesmit) Darba dienu laikā ir jāinformē Izpildītājs un nepieciešamības gadījumā jāsaskaņo ar Izpildītāju būvniecības, uzstādīšanas un apkopes darbi, kas nav iekļauti šajā Līgumā un varētu ietekmēt Ēkas enerģijas patēriņu, tai skaitā, bet ne tikai: (i) Ēkas platības paplašināšana, (ii) Ēkas turpmāka modernizācija, (iii) jaunu/citādāku radiatoru un/vai siltuma konvektoru, un/vai apkures elementu uzstādīšana vai nomaiņa, (iv) jauna siltuma ģeneratora uzstādīšana.
	\item Pasūtītājs (Dzīvokļa īpašnieks) informē Izpildītāju par Dzīvokļa atjaunošanu un termiņu, kad šāda atjaunošana var ietekmēt enerģijas patēriņu Ēkā, tai skaitā, bet ne tikai: (i) radiatoru un/vai siltuma konvektoru, un/vai apkures elementu nomaiņu; (ii) logu nomaiņu; (iii) dzīvokļa apkurināmās platības paplašināšanu, ietverot balkonu/lodžiju; (iv) mehānisku ventilācijas sistēmu uzstādīšanu. Šādos gadījumos Izpildītājs ir tiesīgs pārskatīt Līgumā noteikto Garantēto enerģijas ietaupījumu.
	\item Pasūtītājam Apkures sezonas laikā ir tiesības logus Ēkas dzīvokļos atvērt ne ilgāk kā uz 10 (desmit) minūtēm dienā, lai nodrošinātu svaiga gaisa apmaiņu un atbrīvotos no: (i) putekļiem vai spēcīgiem tīrīšanas līdzekļu aromātiem, kas Dzīvoklī saglabājušies pēc uzkopšanas, un (ii) spēcīga aromāta pēc ēdiena gatavošanas, kas saglabājies Dzīvoklī.
	\item Pasūtītājam Apkures sezonas laikā ir tiesības jebkurā laikā atvērt Ēkas Dzīvokļa logus Dzīvoklī mītošo personu veselības apsvērumu dēļ.
	\item Pasūtītājam ir jānodrošina, lai visi logi koplietošanas platībā Apkures sezonas laikā būtu pastāvīgi aizvērti.
	\item Pasūtītājam ir jānodrošina, lai visas Ēkas ieejas durvis Apkures sezonas laikā netiktu atstātas atvērtas.
	\item Ja Ēkas Dzīvokļu īpašnieki mainās, Pasūtītājam ir nekavējoties, bet ne vēlāk kā 5 (piecu) Darba dienu laikā pēc šādām izmaiņām par izmaiņām ir jāinformē Izpildītājs.
	\item Pasūtītājam (katram Dzīvokļa īpašniekam) ir jānodrošina, lai jebkādas īpašumtiesību uz tā Dzīvokli atsavināšanas gadījumā un neatkarīgi no tā, uz kāda pamata vai juridiska pamatojuma šāda atsavināšana tiek īstenota, jaunais Dzīvokļa īpašnieks (ieguvējs) parakstītu Pievienošanās aktu vai jebkādu citu juridisku dokumentu, saskaņā ar kuru jaunais Dzīvokļa īpašnieks pārņem no šā Līguma izrietošās tiesības un pienākumus no sākotnējā Dzīvokļa īpašnieka. Iepriekšminētā neievērošanas gadījumā iepriekšējais Dzīvokļa īpašnieks un jaunais Dzīvokļa īpašnieks ir solidāri atbildīgi par jebkādu šajā Līgumā noteiktu saistību izpildi un jebkādu to attiecīgu pārkāpumu no ieguvēja puses.
	\item Pasūtītājam ne vēlāk kā 5 (piecu) Darba dienu laikā ir jāinformē Izpildītājs par Pārvaldnieka maiņu. Jaunais Pārvaldnieks ir pienācīgi jāinformē par šā Līguma noteikumiem.
\end{enumerate}

\subsection{NORĒĶINU KĀRTĪBA}
\begin{enumerate}
	\item Pasūtītājs maksā Izpildītājam šajā Līgumā noteiktās ikmēneša Maksas.
	\item Pušu savstarpējo norēķinu periods ir kalendārais mēnesis. Norēķinu periods Starpības (atlikuma) maksājuma aprēķināšanai starp rēķiniem par izlīdzināto Siltumenerģijas patēriņu un faktisko izmērīto Siltumenerģijas patēriņu, kā arī Garantētā enerģijas ietaupījuma Mērījumiem un kvalitātes pārbaudei ir viens gads.
	\item Katru mēnesi Izpildītājs vai tā iecelta trešā persona, kas pārstāv Izpildītāju, aprēķina maksājumu, kas Pasūtītājam ir jāmaksā saskaņā ar Līgumu. Aprēķinātā visu Maksu kopsumma uzskatāma par maksājamu, kas Pasūtītāja jāveic Izpildītājam par saskaņā ar šo Līgumu sniegtajiem Pakalpojumiem savstarpējo norēķinu nolūkā.
	\item Ik pēc 12 (divpadsmit) mēnešiem Pakalpojumu sniegšanas perioda sākuma datumā Izpildītājam ir jāveic ikgadējie norēķini, balstoties uz Garantētā enerģijas ietaupījuma Mērījumiem un kvalitātes pārbaudi.
	\item Izpildītājs vai tā iecelta trešā persona ne vēlāk kā katra mēneša 10.  (desmitajā) datumā izraksta Rēķinu, kurā norādītas visu Maksu komponenšu pozīcijas atbilstoši Līgumā skaidri norādītajam, un iesniedz to Pasūtītājam vai Pasūtītāju pārstāvošajam Pārvaldniekam.
	\item Izpildītājam ir jānodrošina, lai katram Dzīvokļa īpašniekam adresētajos Rēķinos par sniegtajiem Pakalpojumiem iekļautā informācija būtu atspoguļota skaidri un saprotami, atsevišķi norādot Maksu par atjaunošanu un Izpildītāja Ekspluatācijas un apkopes maksu.
	\item Pirmais veicamais Maksu maksājums aprēķināms 1 (vienu) mēnesi pēc Pieņemšanas-nodošanas akta parakstīšanas. Līdz tam Pasūtītājam ir pienākums segt savus komunālo pakalpojumu izdevumus, kuru termiņš ir iestājies.
	\item Pasūtītājam (katram Dzīvokļa īpašniekam) Maksas ir jāmaksā Izpildītājam (vai Izpildītāja norādītai trešajai personai) tieši vai ar Pārvaldnieka starpniecību, balstoties uz Pārvaldnieka par komunālajiem pakalpojumiem un citiem Ēkas ekspluatācijas izdevumiem izrakstītajiem rēķiniem, kuru daļa ietver Maksas, kas pienākas Izpildītājam. Pasūtītājs maksā Maksas atbilstoši Pārvaldnieka pieņemtajai praksei, bet ne vēlāk kā 15 (piecpadsmit) dienu laikā no Rēķina saņemšanas dienas, nepieciešamos naudas līdzekļus pārskaitot uz Pārvaldnieka norādīto bankas kontu.
	\item Izpildītājs vai tā nozīmētais Pārvaldnieks pārvalda ar norēķiniem saistīto informāciju saistībā ar šo Līgumu, tai skaitā:
	\item reģistrējot visu informāciju par katram no Dzīvokļu īpašniekiem izrakstītajiem rēķiniem, ieskaitot to summas;
	\item uzturot Rēķinu apmaksas uzskaites reģistrus un pastāvīgi aktualizējot katra Dzīvokļa īpašnieka parādsaistības, ja tādas rodas;
	\item Pasūtītājs pēc Izpildītāja vai jebkuras tā nozīmētās personas pieprasījuma iesniedz Izpildītājam vai tā nozīmētajai personai (atbilstošos gadījumos) pārskatu par Ēkas Dzīvokļu īpašnieku veiktajiem maksājumiem un parādnieku sarakstu.
\end{enumerate}

\subsection{LĪGUMA TERMIŅŠ}
\begin{enumerate}
	\item Šā Līguma termiņš sākas šā Līguma datumā, un tas paliks pilnībā spēkā un izpildāms līdz Pakalpojumu sniegšanas perioda beigām, ja netiks izbeigts pirms termiņa atbilstoši šajā Līgumā noteiktajam.
	\item Šā Līguma termiņu var pagarināt, Pusēm rakstveidā vienojoties, jo īpaši Pusēm ir tiesības uz Pušu rakstiskas piekrišanas pamata paredzēt vai pārcelt Sākuma datumu un Nodošanas ekspluatācijā datumu.
	\item Izpildītājs Pasākumu būvniecības un uzstādīšanas darbus uzsāk Sākuma datumā un pabeidz tos norādītajā Būvniecības periodā. Būvniecības perioda noslēgumā Pasūtītājs un Izpildītājs paraksta Pieņemšanas-nodošanas aktu.
	\item Pakalpojumu sniegšanas periods un norēķinu periods sākas pēc Pieņemšanas-nodošanas akta parakstīšanas.
	\item Pasūtītāja saistību neizpildes vai nolaidības gadījumā, piemēram, ja Izpildītājam nav iesniegti visi nepieciešamie dokumenti un/vai nodrošināta piekļuve Ēkai, un/vai Nepārvaramas varas apstākļu gadījumā uzskatāms, ka Sākuma datums, Būvniecības periods un Pakalpojumu sniegšanas periods ir automātiski pārcelti par kavējuma periodu. Pusēm ir rakstveidā jāsaskaņo šīs izmaiņas un jāgroza Līgums.
\end{enumerate}

\subsection{SLĒPTAIS STĀVOKLIS}
\begin{enumerate}
	\item Ja Būvniecības periodā Izpildītājs uzzina par jebkādu Slēptu stāvokli, kas ietekmēs Pasākumu pabeigšanu, Izpildītājam ir 5 (piecu) Darba dienu laikā jānosūta Pasūtītājam rakstisks Paziņojums (kā priekšnoteikums jebkādām tiesībām uz papildu laiku vai naudu), kurā norādīts:
	\begin{enumerate}
		\item konstatētais Slēptais stāvoklis, un kādā ziņā tas būtiski atšķiras no Ēkas stāvokļa, kāds būtu pamatoti sagaidāms, Līguma datumā kompetentam un pieredzējušam Izpildītājam ievērojot vispārpieņemto nozares praksi;
		\item papildu darbs un papildu resursi, kuri pēc Izpildītāja aplēsēm ir nepieciešami Slēptā stāvokļa novēršanai;
		\item laiks, kas pēc Izpildītāja aplēsēm būs nepieciešams Slēptā stāvokļa novēršanai, un sagaidāmais pabeigšanas kavējums;
		\item Izpildītāja pamatotas aplēses attiecībā uz to pasākumu izmaksām, kas nepieciešama Slēptā stāvokļa novēršanai; un
		\item jebkāda cita informācija, kas Pasūtītājam būtu pamatoti nepieciešama.
	\end{enumerate}
	\item Slēptā stāvokļa izraisītais kavējums var būt par pamatu Būvniecības perioda pagarinājumam, ja tā dēļ Izpildītājam:
	\begin{enumerate}
		\item ir jāveic papildu darbs;
		\item ir jāizmanto papildu materiāli; vai
		\item rodas papildu izmaksas (tai skaitā, bet ne tikai ar kavējumu vai darba pārtraukumu saistītas izmaksas), kuras Izpildītājs neparedzēja un nevarēja saprātīgi paredzēt Līguma noslēgšanas brīdī, ievērojot vispārpieņemto nozares praksi.
	\end{enumerate}
	\item Pasūtītājam ir jāsedz visi faktiskie izdevumi, kas rodas saistībā ar Slēpto stāvokli un par kuriem Puses ir vienojušās. Ja Pasūtītājs nevēlas, lai Izpildītājs turpinātu darbu atbilstoši sniegtajai informācijai, Pasūtītājam ir nekavējošies jādod norādījumi Izpildītājam neturpināt darbu, un Izpildītājam ir jāievēro šādi norādījumi. Pasūtītājs un Izpildītājs var saskaņot un vienoties par kādu citu veidu, kā novērst Slēpto stāvokli, tai skaitā, bet ne tikai papildu darba veikšanu uzdot citām struktūrām, un šo darbu apmaksā Pasūtītājs.
	\item Slēptā stāvokļa gadījumā Izpildītājam ir tiesības pieprasīt Būvniecības perioda pagarinājumu atbilstoši laikam, kas bija nepieciešams Slēptā stāvokļa novēršanai. Izpildītājs ir tiesīgs uz papildu izmaksu, kas tieši vai netieši izriet no jebkāda Slēptā stāvokļa, atlīdzinājumu.
	\item Izpildītājs ir tiesīgs pieprasīt jebkādu Garantētā enerģijas ietaupījuma korekciju, Atjaunošanas darbu apjoma samazinājumu vai šajā Līgumā atrunāto Maksu grozījumus Slēptā stāvokļa rezultātā.
\end{enumerate}

\subsection{MĒRĪJUMI UN KVALITĀTES PĀRBAUDE, UN DATU PĀRVALDĪBA}
\begin{enumerate}
	\item Izpildītājam visas Mērījumu un kvalitātes pārbaudes darbības ir jāveic saskaņā ar Starptautiskajam izpildes rādītāju mērījumu un pārbaudes protokolam atbilstošu Mērījumu un kvalitātes pārbaudes plānu, kas pieejams EPC platformā sunshineplatform.eu.
	\item Visas Mērījumu un kvalitātes pārbaudes darbības ir skaidri un pilnībā jāatklāj visām Pusēm, un tām ir jābūt pārskatāmām.
	\item Izpildītājs Pakalpojumu sniegšanas periodā iesniedz Pasūtītājam Gada pārskatu. Šajā Gada pārskatā ir jābūt dokumentētam Maksu aprēķinam, Izpildītāja veiktajām Ekspluatācijas un apkopes darbībām un norādei, vai Garantētais enerģijas ietaupījums ir ticis panākts, balstoties uz Mērījumiem un kvalitātes pārbaudi Norēķinu periodā. Šim pārskatam ir jāsniedz pietiekama informācija par īstenoto Pasākumu rezultātā panākto Enerģijas ietaupījumu un par aprēķiniem, kas veikti Enerģijas ietaupījuma noteikšanai. Izpildītājs katru gadu, ne vēlāk kā 20 (divdesmit) Darba dienu laikā pēc Norēķinu perioda beigām nosūta Gada pārskatu Pasūtītājam. Šis process var tikt nodrošināts ar EPC platformas sunshineplatform.eu starpniecību.
	\item Ja Pasūtītājam ir iebildumi pret Gada pārskatā ietvertajiem slēdzieniem, Pasūtītājs atbilstoši informē Izpildītāju 15 (piecpadsmit) Darba dienu laikā pēc pārskata saņemšanas vai Paziņojuma no EPC platformas sunshineplatform.eu saņemšanas. Pasūtītājam ir jānorāda Izpildītājam savu iebildumu iemesli; Izpildītājam nākamo 15 (piecpadsmit) dienu laikā pēc Pasūtītāja iebildumu saņemšanas jāveic nepieciešamie grozījumi un attiecīgi jāinformē Pasūtītājs.
	\item Līgumā noteiktā Garantētā enerģijas ietaupījuma Mērījumos un kvalitātes pārbaudē ir jāņem vērā jebkāda nepamatota Pasūtītāja iejaukšanās vai veiktas izmaiņas Ēkā īstenotajos Pasākumos, kā rezultātā samazinās Enerģijas ietaupījums, un tas kalpo par pamatu Garantijas proporcionālai korekcijai.
	\item Pasūtītājs atzīst un piekrīt, ka Izpildītājs vai jebkāda cita trešā persona, kurai Izpildītājs nodevis šajā Līgumā noteiktās tiesības un pienākumus, izmanto:
	\begin{enumerate}
		\item jebkādus anonīmus datus un informāciju saistībā ar Ēkas enerģijas patēriņu neatkarīgi no tā, vai tos sniedzis Pasūtītājs vai ieguvis Izpildītājs, nolūkā standartizēt un apkopot valsts, reģionāla un starptautiska mēroga datubāzi vai Izpildītāja atsauces nolūkā, vai jebkādā citā ar Pasūtītāju saskaņotā nolūkā;
		\item Pasūtītāja vai tā Pārvaldnieka sniegtus personas datus nolūkā sniegt savus Pakalpojumus un nodot tos jebkurai trešajai personai, kurai var tikt nodotas no šā Līguma izrietošās tiesības vai pienākumi, tai skaitā jebkurai personai, kas uz forfaitinga pamata iegūst no šā Līguma izrietošos debitoru parādus vai kas pārvalda vai atbild par tiešsaistes EPC platformas sunshineplatform.eu izstrādi, ieviešanu, darbību un uzturēšanu, uzraugot īstenoto Pasākumu veiktspēju.
	\end{enumerate}
	\item Jebkādā saprātīgā laika periodā Izpildītājs pēc saviem ieskatiem, pats vai ar savu nozīmēto personu starpniecību ir tiesīgs uzstādīt, ekspluatēt, apkalpot un ieviest energopārvaldības sistēmu vai vispārīgus mērinstrumentus un saprātīgā laikā saskaņā ar Līgumu piekļūt šādam uzstādītajam aprīkojumam.
	\item Izpildītājs ir tiesīgs Dzīvokļos uzstādīt temperatūras datu reģistrētājus, ja tiek saņemtas sūdzības par neatbilstību Komforta standartiem. Ja Ēkas Dzīvokļu īpašnieki nepiekrīt minētā datu reģistrētāja uzstādīšanai savos Dzīvokļos vai nenodrošina saprātīgu piekļuvi nolūkā veikt šādu uzstādīšanu, Izpildītājs nav atbildīgs par apgalvoto nepienācīgo Līguma izpildi attiecībā uz šādiem Dzīvokļiem.
	\item Izpildītāja mērinstrumentu ievāktie dati ir informatīva rakstura un strīdu gadījumā nevar tikt atzīti par pamatu Līguma pārkāpuma vai neatbilstības Komforta standartiem konstatēšanai.
\end{enumerate}

\subsection{STRĪDU RISINĀŠANAS KĀRTĪBA}
\begin{enumerate}
	\item Jebkādi strīdi starp Pusēm vispirms risināmi sarunu ceļā. Šim nolūkam Puses nodrošina savlaicīgu rakstveida atbildes sniegšanu uz otras Puses vēstuli saistībā ar domstarpībām, kā arī iespēju robežās velta savu laiku, lai attiecīgās domstarpības atrisinātu klātienē vai arī ar strīdus Pušu augstākā līmeņa pārstāvja starpniecību.
	\item Ja Pasūtītājam (vai atsevišķam Dzīvokļa īpašniekam) ir sūdzības par Izpildītāju (piemēram, par Komforta standartiem vai Enerģijas ietaupījumu apjomu, vai vispārīgi par īstenotajiem Pasākumiem un sniegtajiem Energoefektivitātes pakalpojumiem), tam ir tieši vai ar Pārvaldnieka starpniecību jāinformē par to Izpildītājs. Izpildītājs pārbauda un sagatavo Aktu par sūdzību un radušos problēmu novēršanu. Ja problēma netiek novērsta vairāk nekā 20 (divdesmit) Darba dienu laikā no paziņojuma brīža, Pasūtītājam ir jāizveido komiteja, kuras sastāvā ir pienācīgi pilnvaroti Izpildītāja, Pārvaldnieka un Pasūtītāja pārstāvji, kas notur sanāksmes un sagatavo Akta projektu, balstoties uz sūdzību un uz vietas pārbaudītajiem faktiem, vai veic jebkādu apstākļu noskaidrošanas (faktu fiksācijas) procedūru saskaņā ar mediācijas noteikumiem, kas pieejami EPC platformā sunshineplatform.eu.
	\item Ja Izpildītājam ir sūdzības par Pasūtītāju (piemēram, uzstādītā aprīkojuma bojāšanu), tas informē par to Pasūtītāju un Pārvaldnieku. Pasūtītājam ir jāpārbauda un, iespējams, jāatklāj pārkāpējs un jāsagatavo Akts par sūdzību un radušos problēmu novēršanas līdzekļiem. Ja problēma netiek atrisināta vairāk nekā 30 dienu laikā pēc paziņojuma, Izpildītājam ir jāizveido komiteja, kuras sastāvā ir pienācīgi pilnvaroti Izpildītāja, Pārvaldnieka un Pasūtītāja pārstāvji, kas notur sanāksmes un sagatavo Akta projektu, balstoties uz sūdzību un uz vietas pārbaudītajiem faktiem, vai veic jebkādu apstākļu noskaidrošanas (faktu fiksācijas) procedūru saskaņā ar mediācijas noteikumiem, kas pieejami EPC platformā sunshineplatform.eu.
	\item Attiecībā uz apstākļu noskaidrošanas procedūru, kas ir piemērojama strīdu gadījumā par faktiem, ir piemērojams turpmāk norādītais:
	\begin{enumerate}
		\item Faktiskie Komforta standarti (gaisa temperatūra telpās atsevišķos Ēkas Dzīvokļos) uzskatāmi par pienācīgi fiksētiem, ja temperatūras mērījumus veic neatkarīgs sertificēts energoauditors (saskaņā ar Ministru kabineta noteikumiem Nr. 382) un saskaņā ar standartu LVS EN 12599. Akts noformējams, balstoties uz neatkarīgā sertificētā energoauditora veikto mērījumu rezultātiem;
		\item Vispārīgas problēmas ar īstenotajiem Pasākumiem, piemēram, iekārtu darbības traucējumi un/vai defekti vai Pasākumiem nodarītie bojājumi, vai saistībā ar Enerģijas ietaupījuma aprēķinu, uzskatāmas par pienācīgi fiksētām, ja par tām ziņo neatkarīgs eksperts, piemēram, sertificēts energoauditors (saskaņā ar MK noteikumiem Nr. 382);
		\item Visas Puses ir jāinformē vismaz 5 (piecas) Darba dienas, pirms trešā persona veic jebkādus mērījumus. Pušu pilnvarotam pārstāvim ir tiesības piedalīties mērījumu procesā nolūkā sagatavot Aktu. Jebkuras Puses pilnvaroto pārstāvju nepiedalīšanās nav šķērslis Pušu īstenotajai Akta sagatavošanai un noformēšanai;
		\item Jebkuras no Pusēm veikta Akta parakstīšana nav uzskatāma par šā Līguma pārkāpuma atzīšanu un/vai nav uzskatāma par atteikšanos no jebkurām Pušu tiesībām un saistībām saskaņā ar šo Līgumu. Neatkarīgo trešo personu izmaksas sadalāmas starp Pusēm vienādās daļās;
		\item Pa vienam eksemplāram no noformētā Akta ir jānodod Izpildītājam, Pārvaldniekam un sūdzību iesniegušajam Dzīvokļa īpašniekam.
	\end{enumerate}
	\item Ja Puses nepanāk vienošanos, Puses uzsāk formālu mediācijas procesu saskaņā ar mediācijas noteikumiem, kas pieejami EPC platformā sunshineplatform.eu un kuri ir spēkā Līguma darbības laikā, un pastāv strīda brīdī. Ja starp Pusēm rodas strīds par tehniskiem jautājumiem, jebkura Puse var pieprasīt, lai strīds par konstatētajiem faktiem tiktu risināts saskaņā ar Apstākļu noskaidrošanas komisijas procesuālajiem noteikumiem, kas pieejami EPC platformā sunshineplatform.eu.
	\item Ja Puses nevar savstarpēji vienoties arī pēc mediācijas procesa un/vai apstākļu noskaidrošanas procesa, strīds izšķirams Latvijas vispārējās jurisdikcijas tiesā saskaņā ar Latvijā spēkā esošajiem piemērojamajiem normatīvajiem aktiem. Pieteikums iesniedzams tiesā pēc atbildētāja dzīves vietas vai juridiskās adreses piekritības, bet, ja tā nav Latvijā, tad Rīgas pilsētas Centra rajona tiesā vai Rīgas apgabaltiesā.
\end{enumerate}

\subsection{IZPILDĪTĀJA ĪSTENOTO PASĀKUMU UZTURĒŠANA (APKOPE)}
\begin{enumerate}
	\item Izpildītājs veic Atjaunošanas darbu ietvaros uzstādītā aprīkojuma (vai jebkuras tā daļas) nomaiņu vai remontu, vai kapitālo remontu pēc to lietderīgās lietošanas laika beigām (atbilstoši Ekspluatācijas un apkopes rokasgrāmatā noteiktajam) Līgumā noteiktajā Pakalpojumu sniegšanas periodā.
	\item Izpildītājam ir jāievieš Pasākumu apkopes kārtība, kas atbilst vai pārsniedz ražotāja prasības attiecībā uz to attiecīgo apkopi, un ir saskaņā ar šā Līguma Īpašajiem noteikumiem.
\end{enumerate}

\subsection{APDROŠINĀŠANA}
\begin{enumerate}
	\item Būvniecības perioda sākumā Izpildītājam ir jāapdrošina Ēka par summu, kas nav mazāka par Ēkas atjaunošanas vērtību, ar minimālo apdrošināšanas segums pret ugunsgrēku, zemestrīci, plūdiem, ūdens radītiem bojājumiem, jebkādām citām dabas stihijām, kas ietekmē Ēku, nosēšanās un krītošu koku radītiem strukturāliem bojājumiem. Šādai apdrošināšanai ir piemērojami turpmāk norādītie nosacījumi:
	\begin{enumerate}
		\item Izpildītājam šāds apdrošināšanas līgums ir jāslēdz ar apdrošinātāju, kura novērtējums ir vismaz A+ saskaņā ar attiecīgajiem Latvijai piemērojamajiem reitingiem.
		\item Izpildītājam ir jāiesniedz Pasūtītājam šīs apdrošināšanas polises kopija un apdrošināšanas prēmijas samaksu apstiprinoši dokumenti līdz Būvniecības perioda sākuma datumam.
		\item Apdrošināšanas seguma izmaksas gadījumā tādā apmērā, kas ir vismaz pietiekams, lai atgūtu Ēkas atjaunošanas vērtību, Pasūtītājam ir jābūt norādītam kā saņēmējam.
		\item Būvdarbus Ēkā nedrīkst uzsākt, kamēr Izpildītājs nav iesniedzis tiesiski noslēgtu apdrošināšanas polisi.
		\item Izpildītājam ir jāuztur apdrošinašanas polise Līguma spēkā esamības laikā un pēc Pasūtītāja pieprasījuma jāuzrāda šādas apdrošināšanas polises oriģināls vai jāiesniedz kopija vai polises sertifikāts, vai jānodrošina piekļuve dokumentam caur EPC platformu-sunshineplatform.eu, vai citi pārliecinoši dokumenti, kas apstiprina apdrošināšanas prēmijas valūtu un samaksu.
		\item Izpildītājs nodrošina Ēkas apdrošināšanu par saviem līdzekļiem visā Būvniecības periodā. Pēc Atjaunošanas darbu pabeigšanas un Pieņemšanas-nodošanas akta parakstīšanas, apdrošināšanas izmaksas par atlikušo Līguma spēkā esamības periodu sadalāmas starp Dzīvokļu īpašniekiem proporcionāli tiem piederošā Dzīvokļa platībai Ēkā un pieskaitāmas Izpildītāja rēķiniem par ekspluatāciju un apkopi. Pušu ieceltajam Pārvaldniekam ir jānodrošina, lai Dzīvokļu īpašniekiem izrakstītajos Pasūtītāja rēķinos būtu iekļautas šādas apdrošināšanas izmaksas.
	\end{enumerate}
	\item Papildus tam Būvniecības periodā Izpildītājam ir jāuztur spēkā esoša civiltiesiskās un profesionālās atbildības apdrošināšanas polise par summu, kas nav mazāka par 110\% no kopējām ieguldījumu izmaksām saistībā ar  Atjaunošanas darbiem.
\end{enumerate}

\subsection{PRASĪJUMU CESIJA}
\begin{enumerate}
	\item Izpildītājam ir neierobežotas tiesības cedēt trešajām personām savas no Līguma izrietošās tiesības un prasījumus saistībā ar jebkādiem debitoru parādiem, kas pienākas no Pasūtītāja saskaņā ar šo Līgumu. Jo īpaši Izpildītājs ir tiesīgs cedēt Maksas par atjaunošanu debitoru parādus jebkuram Cesionāram, kas noslēdzis finansēšanas, forfaitinga, cesijas vai jebkuru citu vienošanos ar Izpildītāju.
	\item Prasījumu cesija neatbrīvo Izpildītāju no šajā Līgumā noteiktajām saistībām un pienākumiem. Tomēr Cesionāram ir tiesības iestāties šajā Līgumā gadījumā, ja Izpildītājs nepilda savas šajā Līgumā noteiktās saistības. Iejaukšanās/iestāšanās tiesību mērķis ir vienīgi nomainīt izpildītāju, kurš nepilda savas saistības, ar citu struktūru, kas spēj izpildīt visas šajā Līgumā noteiktās saistības un pienākumus Pasūtītāja un Cesionāra labā.
	\item Šādas cesijas gadījumā Izpildītājs nosūta Paziņojumu Pasūtītājam 5 (piecu) Darba dienu laikā pēc šāda cesijas līguma noslēgšanas.
	\item Šis Līgums Pasūtītājam ir personīgs, un Pasūtītājs nevar to cedēt vai nodot bez iepriekšējas Izpildītāja rakstveida piekrišanas.
	\item Ja Izpildītājs ir iesaistīts uzņēmumu apvienošanās vai iegādes procesā, uzsāk likvidācijas vai bankrota procesu, Līgums paliek spēkā, un tā noteikumi ir saistoši Izpildītāja tiesību pēctečiem un cesionāriem.
\end{enumerate}

\subsection{ĪPAŠUMTIESĪBAS UZ ATJAUNOŠANAS DARBU IETVAROS ĒKĀ UZSTĀDĪTAJIEM PASĀKUMIEM}
\begin{enumerate}
	\item Īpašumtiesības uz pasākumiem, kas ir nodalāmi no Ēkas, neradot būtisku kaitējumu, pieder Izpildītājam, ja Izpildītājs Atjaunošanas darbu ietvaros nodrošina Finanšu ieguldījumu. Ja Atjaunošanas darbus pilnībā finansē Pasūtītājs, īpašumtiesības uz Pasākumiem pieder Pasūtītājam.
	\item Pasūtītājs nedrīkst, un tam ir jāveic visi saprātīgie pasākumi, lai nodrošinātu, ka neviens no Dzīvokļu īpašniekiem vai citiem viesiem nenoņem, neapgrūtina (cita starpā neizīrē vai neiznomā), neieķīlā vai neiznīcina, nesabojā vai neiejaucas Atjaunošanas darbu ietvaros īstenoto Pasākumu darbībā neatkarīgi no Puses, kurai ir īpašumtiesības uz Pasākumu.
	\item Izpildītājs ir tiesīgs bez Pasūtītāja piekrišanas vai vienošanās ar to (proti, bez nepieciešamības saņemt katra Dzīvokļa īpašnieka piekrišanu) ieķīlāt un apgrūtināt Pasākumus (vai to daļas), ja:
	\begin{enumerate}
		\item Izpildītājam ir īpašumtiesības uz tiem;
		\item tehniski ir iespējams demontēt tās, būtiski nesabojājot Ēku;
		\item ķīla un/vai apgrūtinājums ir nepieciešami kā nodrošinājums Izpildītāja Finanšu ieguldījumam saskaņā ar šo Līgumu. Tomēr Izpildītājs nav tiesīgs ieķīlāt Pasākumus, lai iegūtu finanšu līdzekļus citiem mērķiem, kas nav saistīti ar Līguma izpildi; un
		\item ķīlas un/vai apgrūtinājuma termiņš nepārsniedz Līguma spēkā esamības termiņu.
	\end{enumerate}
	\item Gadījumā, ja Izpildītājam ir īpašumtiesības uz Pasākumu, pēc maksājumu, kas saskaņā ar šo Līgumu pienākas Izpildītājam, saņemšanas uzskatāms, ka īpašumtiesības uz visiem Līgumā noteikto Atjaunošanas darbu ietvaros īstenotajiem Pasākumiem ir automātiski pārgājušas Pasūtītājam. Tas noslēgumā tiek īstenots par neatmaksājamas maksas 1 EUR (viena eiro) apmērā samaksu, kas maksājama avansā šā Līguma parakstīšanas brīdī. Īpašumtiesību nodošanu Pasūtītājam apliecina Nodošanas akts, ko paraksta Izpildītājs un Pasūtītājs.
\end{enumerate}

\subsection{PROGRAMMATŪRA UN INTELEKTUĀLĀ ĪPAŠUMA TIESĪBAS}
\begin{enumerate}
	\item Izpildītājam ir jānodrošina, lai visas intelektuālā un rūpnieciskā īpašuma tiesības uz Ēkā īstenotajiem Pasākumiem, ieskatot aprīkojumu, materiālus, sistēmas, programmatūru un jebkādu citu Izpildītāja Pasūtītājam saskaņā ar šo Līgumu nodrošinātu lietu vai dokumentu, piederētu Izpildītājam vai būtu tam licencētas. Puses vienojas, ka šādas tiesības paliek Izpildītāja īpašumā un nepāriet Pasūtītājam. Izpildītājs piešķir Pasūtītājam beztermiņa, neatsaucamu, neekskluzīvu bezmaksas licenci (ar tiesībām nodot apakšlicencē) uz minēto intelektuālo un rūpniecisko īpašuma tiesību izmantošanu saistībā ar Ēkas lietošanu, bet nekādai citai izmantošanai.
	\item Pasūtītājs nedrīkst modificēt, kopēt vai dekompilēt jebkādu programmatūru vai apvienot to ar jebkādu citu programmatūru, kuru Izpildītājs ir nodrošinājis Atjaunošanas darbu ietvaros. Šā Līguma spēkā esamības laikā Izpildītājs nodrošinās Pasūtītājam lietotāju rokasgrāmatas, tehnisko informāciju un visus nodrošinātās programmatūras atjauninājumus un labojumus.
	\item Izpildītājs pasargā Pasūtītāju no jebkādiem prasījumiem, par kuriem Pasūtītājs saskaņā ar tiesību aktiem ir atbildīgs, saistībā ar jebkādu trešo personu intelektuālā īpašuma tiesību pārkāpumu attiecībā uz jebkuru Izpildītāja nodrošinātā intelektuālā un rūpnieciskā īpašuma daļu. Izpildītāja pienākuma pasargāt Pasūtītāju no šādiem prasījumiem nosacījums ir tas, ka Pasūtītājs:
	\begin{enumerate}
		\item nekavējoties rakstveidā paziņo Izpildītājam par prasījumu;
		\item neatzīstas un neierobežo Izpildītāja aizstāvību pret prasījumu vai Izpildītāja iespējas vienoties par pieņemamu izlīgumu;
		\item dod Izpildītājam iespēju par Izpildītāja līdzekļiem vadīt aizstāvību un jebkādas izlīguma sarunas saistībā ar prasījumu; un
		\item nodrošina Izpildītājam (par Izpildītāja līdzekļiem) tādu palīdzību un informāciju, kāda Izpildītājam var būt pamatoti nepieciešama, lai palīdzētu Izpildītājam vadīt aizstāvību un jebkādas sarunas par izlīgumu saistībā ar prasījumu.
	\end{enumerate}
	\item Izpildītājs pēc saviem ieskaitiem nomaina vai pārveido intelektuālā un rūpnieciskā īpašuma tiesības aizskarošo daļu ar neaizskarošu vai nodrošina Pasūtītājam tiesības izmantot šādu aizskarošo daļu. Šajā Punktā noteiktie tiesību aizsardzības līdzekļi ir vienīgie tiesību aizsardzības līdzekļi intelektuālā īpašuma tiesību aizskāruma gadījumā.
\end{enumerate}

\subsection{IZMAIŅAS ĒKAS IZMANTOŠANĀ}
\begin{enumerate}
	\item Ēkas apraksts ir sniegts Līguma Īpašajos noteikumos, ieskaitot tās izmatošanu, platību un parametrus. Ja pēc Pasūtītāja iniciatīvas vai ar Pasūtītāja piekrišanu vai atļauju mainās jebkādi apstākļi, uz kuriem bija balstīti Izpildītāja aprēķini, izmaiņas neietekmē Izpildītāju un Līguma izpildi, un Ēkas izmaiņas ir jāizvērtē no ekonomisko apsvērumu viedokļu (jo īpaši izmaksu izmaiņas), un Līgums attiecīgi jāpielāgo jaunajiem apstākļiem.
	\item Izmaiņas ēkas izmantošanā ietver:
	\begin{enumerate}
		\item Ēkas virsmas laukuma palielināšanos vai samazināšanos;
		\item attiecīgo iekārtu vai citu instalāciju montāžu, bojājumus vai demontāžu, ja tā rezultātā būtiski palielinās vai samazinās enerģijas patēriņš vai citi Ēkas tehniskie parametri;
		\item izmaiņas Ēkas izmantošanā (piemēram, dzīvokļu platība tiek pārveidota par veikaliem, restorāniem un birojiem, sāk izmantot neizmantotus/neapdzīvotus dzīvokļus), kas ietekmē Ēkas enerģijas patēriņu.
	\end{enumerate}
\end{enumerate}

\subsection{ATSLĒGTĀ UN/VAI DEMONTĒTĀ APRĪKOJUMA apsaimniekošana}
\begin{enumerate}
	\item Izpildītājam par saviem līdzekļiem ir jānoorganizē šā Līguma ietvaros radīto atkritumu apsaimniekošana saskaņā ar piemērojamajiem Latvijas Republikas normatīvajiem aktiem attiecībā uz atkritumu apsaimniekošanu.
	\item Izpildītājam ne vēlāk kā 5 (piecas) Darba dienas pirms pirmās plānotās apsaimniekošanas darbības ir rakstveidā jāinformē Pasūtītājs. Šādā paziņojumā ir jāiever viss Ēkā uzstādītais aprīkojums, materiāli un citi aktīvi, kurus plānots demontēt un nomainīt, lai Būvniecības periodā īstenotu un uzstādītu Pasākumus.
	\item Ja Pasūtītājs vēlas izmantot jebkādu Izpildītāja Būvniecības periodā atslēgto vai demontēto aprīkojumu, materiālu vai citus aktīvus, tam ir jāinformē Izpildītājs un par saviem līdzekļiem jānoorganizē savākšana un pārvešana.
\end{enumerate}

\subsection{ATBILDĪBA}
\begin{enumerate}
	\item Izpildītājs ir atbildīgs par Pasākumu savlaicīgu īstenošanu saskaņotajā Būvniecības periodā. Ja Izpildītājs neizpilda šo saistību, Pasūtītājam rodas tiesības uz līgumsodu 0,02\% apmērā no kopējām plānotajām ieguldījumu izmaksām par katru dienu. Līgumsods nedrīkst pārsniegt 10% (desmit procentus) no plānotajām ieguldījumu izmaksām.
	\item Pasūtītājs ir atbildīgs par attiecīgo izmaksu un Maksu savlaicīgu samaksu atbilstoši šim Līgumam. Izpildītājs ir tiesīgs pieprasīt kavējuma naudas samaksu. Kavējuma nauda maksājama 0,1% apmērā no nesamaksātās summas par katru kavējuma dienu.
	\item Ja Pasūtītājs neveic jebkādu maksājumu, kuram iestājies termiņš un kas izriet no Līguma, vairāk nekā 90 (deviņdesmit) dienas, kuru laikā šajā Līgumā noteiktā Strīdu risināšanas kārtība tika pienācīgi un efektīvi pielietota, Izpildītājs ir tiesīgs izbeigt Līgumu uz Pasūtītāja saistību neizpildes un Līguma pārkāpuma pamata.
	\item Izpildītājs ir atbildīgs par Komforta standartu nodrošināšanu Ēkā atbilstoši šajā Līgumā noteiktajam. Ja Apkures sezonā gaisa temperatūra telpās jebkurā no Dzīvokļiem ir bijusi vidēji par 2 (diviem) grādiem pēc Celsija skalas (ņemot vērā instrumentu precizitāti) mazāka par šajā Līgumā noteiktajiem Komforta standartiem, Izpildītāja pienākums būs dot norādījumus Pārvaldniekam samazināt Pasūtītāja rēķinu attiecīgajam Dzīvokļa īpašniekam sekojošā veidā:
	\begin{enumerate}
		\item atlaide 5\% (piecu procentu) apmērā par katru grādu pēc Celsija skalas no Maksas par enerģiju katrā Apkures sezonas mēnesī, kad temperatūra bijusi zemāka par pielīgtajiem Komforta standartiem;
		\item Gaisa temperatūras telpās nosakāma, un tas, vai temperatūras līmenis bija zemāk par Komforta standartiem, nosakāms saskaņā ar šajā Līgumā noteikto Strīdu risināšanas kārtību.
		\item Izpildītājs nepiemēro atlaidi, ja gaisa temperatūras telpās samazinājums Dzīvoklī ir noticis: (i) iedzīvotāju vai Dzīvokļa īpašnieku darbības vai bezdarbības rezultātā pretēji šajā Līgumā noteiktajam; (ii) samazinājums ir Pasūtītāja saistību neizpildes sekas; vai (iii) citu, ar Izpildītāja vainu nesaistītu iemeslu dēļ.
	\end{enumerate}
	\item Pasūtītājs ir atbildīgs par Pasākumiem nodarītiem bojājumiem, manipulācijām ar tiem vai iejaukšanos to darbībā, vandālismu, sabotāžu, zādzību (ja vien to nav pieļāvis Izpildītājs vai personas, par kurām atbildīgs ir Izpildītājs), jo īpaši, ja tas ietekmē Enerģijas ietaupījuma līmeni, pielīgots Komforta standartus vai Ēkā dzīvojošo un to izmantojošo personu drošību. Šādā gadījumā Pasūtītājam ir pienākums:
	\begin{enumerate}
	\item pilnībā atlīdzināt Izpildītājam attiecīgā Pasākuma atjaunošanas izmaksas;
	\item samaksāt Izpildītājam kompensāciju 10\% (desmit procentu) apmērā no atjaunošanas izmaksām, lai segtu Izpildītāja administratīvās izmaksas;
	\item atjaunošanas izmaksas aprēķināmas saskaņā ar piemērojamajām tirgus cenām, kas ir spēkā aprēķinu brīdī;
	\item Pasūtītāja atbildība par Pasākumu bojāšanu, manipulācijām ar tiem vai iejaukšanos to darbībā nosakāma saskaņā ar šajā Līgumā ietverot Strīdu risināšanas kārtību.
	\end{enumerate}
	\item Izpildītājs apņemas pasargāt Pasūtītāju no jebkādas atbildības, izmaksām, izdevumiem, zaudējumu atlīdzības, maksājumiem, kas tam var rasties valsts vai pārvaldes iestāžu, vai trešo personu pret Pasūtītāju ierosinātas prasības vai sūdzības, administratīvā vai tiesa procesa, kas izriet no Izpildītāja darbībām vai no jebkāda potenciālā intelektuālā īpašuma tiesību pārkāpuma saistībā ar Izpildītāja īstenotajiem Pasākumiem. Pasūtītājs saņem no Izpildītāja  visu to izmaksu un izdevumu kompensāciju, kas pamatoti ir nepieciešami visu tiešo zaudējumu atlīdzināšanai saistībā ar Izpildītāja darbībām, kas pieļautas, pārkāpjot piemērojamos tiesību aktus. Atlīdzinājums ir pienācīgi jādokumentē un jāizmaksā 30 Darba dienu laikā pēc attiecīgā Pasūtītāja Izpildītājam adresētā pieprasījuma saņemšanas, kurā skaidri norādīta maksājamā summa.
	\item Maksājumi par vispārējas tautsaimnieciskas nozīmes pakalpojumiem (tai skaitā siltumapgādi) un jebkurai Pusei piemērojamās Latvijas Republikā spēkā esošajos normatīvajos aktos noteiktās sankcijas neierobežo šajā Līgumā noteiktās Pušu saistības, tai skaitā Pušu atbildību un saskaņā ar šo Līgumu piemērojamo līgumsodu, kompensāciju un naudassodu samaksu.
	\item Līgumsodu, kompensāciju un sodanaudu samaksa neatbrīvo vainīgo Pusi no savu Līgumā noteikto saistību izpildes.
\end{enumerate}

\subsection{LĪGUMA IZBEIGŠANA}
\begin{enumerate}
	\item Gadījumā, ja jebkura Puse pārkāpj būtisku šā Līguma noteikumu, otra Puse var nekavējoties izbeigt šo Līgumu un pieprasīt vainīgajai Pusei atlīdzināt otras Puses zaudējumus saskaņā ar šo Līgumu.
	\item Līguma izbeigšana pirms Sākuma datuma un ieguldījumu veikšanas būvniecības un uzstādīšanas darbos:
	\begin{enumerate}
		\item Gadījumā, ja Pasūtītājs vienpusēji izbeidz Līgumu Izpildītāja pieļautas būtiskas saistību neizpildes vai Līguma pārkāpuma rezultātā, Pasūtītājs ir tiesīgs uz kompensāciju 1\% (viena procenta) apmērā Līgumā plānoto Ieguldījumu izmaksām (bez PVN);
		\item Gadījumā, ja Izpildītājs vienpusēji izbeidz Līgumu Pasūtītāja pieļautas būtiskas saistību neizpildes vai Līguma pārkāpuma rezultātā, Izpildītājs ir tiesīgs uz kompensāciju 1\% (viena procenta) apmērā Līgumā plānoto Ieguldījumu izmaksām (bez PVN).
		\item Gadījumā, ja Pasūtītājs vienpusēji izbeidz Līgumu citu ar komercdarbību saistītu vai komerciālu iemeslu dēļ, kas var nebūt saistīti ar šo Līgumu, Izpildītājs ir tiesīgs uz kompensāciju 1\% (viena procenta) apmērā Līgumā plānoto Ieguldījumu izmaksām (bez PVN).
		\item Gadījumā, ja Izpildītājs vienpusēji izbeidz Līgumu citu ar komercdarbību saistītu vai komerciālu iemeslu dēļ, kas var nebūt saistīti ar šo Līgumu, Pasūtītājs ir tiesīgs uz kompensāciju 1\% (viena procenta) apmērā Līgumā plānoto Ieguldījumu izmaksām (bez PVN).
	\end{enumerate}
	\item Līguma izbeigšana pēc ieguldījuma izmaksu saistībā ar Pasākumu būvniecības un uzstādīšanas darbiem rašanās un segšanas Izpildītāja Finanšu ieguldījuma veidā:
	\begin{enumerate}
		\item Gadījumā, ja Pasūtītājs vienpusēji izbeidz Līgumu Izpildītāja pieļautas būtiskas saistību neizpildes vai Līguma pārkāpuma rezultātā, Pasūtītājam ir jāatlīdzina tikai Izpildītāja veiktā Finanšu ieguldījuma neapmaksātā summa ar atlaidi 3% apmērā, ja tas darbojas pienācīgi, saskaņā ar Līgumā noteiktajiem veiktspējas kritērijiem. Papildus iepriekš minētajam, Pasūtītājs ir tiesīgs saņemt visu projekta dokumentāciju, kurā detalizēti aprakstīti līdz tam brīdim izpildītie darbi, līdz ar visām atļaujām, licencēm vai citiem dokumentiem, kurus Izpildītājs ieguvis saskaņā ar šo Līgumu, un visneatliekamāko darbu pabeigšanu, visas ražotāju garantijas, jebkādas apakšlicences (un jebkādu licenču nodošanu) nepieciešamo intelektuālā īpašuma tiesību un programmatūru (ieskaitot uzstādīto programmatūru un atbilstošos gadījumos jebkādus pavaddokumentus, ar kodu saistītu informāciju, jebkādu pirmkodu, datnes, aprēķinus, elektroniskos datu nesējus, izdrukas vai saistīto informāciju) izmantošanai, papildu apmācību jebkādai trešajai personai, kuru Pasūtītājs ir nepārprotami nozīmējis gadījumā, ja Atjaunošanas darbi ir pabeigti.
		\item Gadījumā, ja Izpildītājs vienpusēji izbeidz Līgumu Pasūtītāja pieļautas būtiskas saistību neizpildes vai Līguma pārkāpuma rezultātā, Pasūtītājam ir jāatlīdzina Finanšu ieguldījuma neapmaksātā summa, plus kompensācija 3% apmēra no maksājamās summas. Pasūtītājs ir tiesīgs saņemt visu projekta dokumentāciju, kurā detalizēti aprakstīti līdz tam brīdim izpildītie darbi, līdz ar visām atļaujām, licencēm vai citiem dokumentiem, kurus Izpildītājs ieguvis saskaņā ar šo Līgumu, visas ražotāju garantijas, jebkādas apakšlicences (un jebkādu licenču nodošanu) nepieciešamo intelektuālā īpašuma tiesību un programmatūru (ieskaitot uzstādīto programmatūru un atbilstošos gadījumos jebkādus pavaddokumentus, ar kodu saistītu informāciju, jebkādu pirmkodu, datnes, aprēķinus, elektroniskos datu nesējus, izdrukas vai saistīto informāciju).
		\item Gadījumā, ja Pasūtītājs vienpusēji izbeidz Līgumu citu ar komercdarbību saistītu vai komerciālu iemeslu dēļ, kas var nebūt saistīti ar šo Līgumu, Izpildītājs ir tiesīgs uz kompensāciju Izpildītāja veiktā Finanšu ieguldījuma neatmaksātās pamatsummas apmērā, plus kompensāciju 3% apmērā no maksājamās summas.
		\item Gadījumā, ja Izpildītājs vienpusēji izbeidz Līgumu citu ar komercdarbību saistītu vai komerciālu iemeslu dēļ, kas var nebūt saistīti ar šo Līgumu, Pasūtītājs ir tiesīgs uz kompensāciju Izpildītāja veiktā Finanšu ieguldījuma neatmaksātās pamatsummas apmērā, atskaitot 3% no aprēķinātās summas.
	\end{enumerate}
	\item Izpildītājs vai jebkuri tā tiesību pēcteči izraksta Pasūtītājam rēķinu par aprēķināto kompensāciju, skaidri norādot šādam aprēķinam izmantoto informāciju, balstoties uz Pakalpojumu sniegšanas periodā izdotajiem Maksājumu grafikiem vai rēķiniem, kas apmaksāti Atjaunošana darbu veikšanas laikā pirms Līguma izbeigšanas datuma. Pasūtītājs 60 (sešdesmit) dienu laika no rēķina izrakstīšanas dienas pienākošos kompensāciju samaksā Izpildītājam vai jebkuriem tā mantas (aktīvu) pārvaldītājiem, tiesību pēctečiem vai tādiem citiem subjektiem, kuri norādīti kā likumīgi tiesīgi uz visām vai dažām no Izpildītājam šajā Līgumā noteiktajām tiesībām.
	\item Par Līguma pirmstermiņa izbeigšanu Pusei, kas izbeidz Līgumu, jāinformē rakstveidā (jāiesniedz paziņojums par Līguma izbeigšanu) vismaz 20 (divdesmit) Darba dienas iepriekš. Gadījumā, ja Līgums tiek izbeigts Puses saistību neizpildes vai Līguma pārkāpuma rezultātā, spēkā esošā paziņojumā par Līguma izbeigšanu ir jābūt norādītām saskaņā ar šajā Līgumā noteikto Strīdu risināšanas kārtību veiktām darbībām un pievienotiem saistītiem pamatojošajiem dokumentiem.
	\item Pasūtītājs ir tiesīgs jebkurā brīdī pieprasīt un saņemt no Izpildītāja Līguma pirmstermiņa izbeigšanas gadījumā Izpildītājam kompensējamās summas aprēķinu.
	\item Parasti Līguma izbeigšana neatbrīvo Puses no Līgumā noteikto saistību izpildes, kuru termiņš ir iestājies pirms Līguma izbeigšanas, ja vien Puses nav rakstveidā vienojušās par citiem nosacījumiem vai Līgumā nav norādīts citādi. Jo īpaši Līguma vienpusēja izbeigšana no Pasūtītāja puses Izpildītāja pieļautas būtiskas saistību neizpildes vai Līguma pārkāpuma rezultātā pati par sevi neatbrīvo Pasūtītāju no Rēķinu, kas izrakstīti par periodiem pirms Līguma izbeigšanas, apmaksas saistībām.
	\item Pušu reorganizācija, dalībnieku (akcionāru) un/vai īpašnieku struktūras maiņa, vadības maiņa, tai skaitā Ēkas Dzīvokļu īpašnieku maiņa nevar būt par pamatu Līguma izbeigšanai vai Līgumā ietverto saistību neizpildei.
	\item Papildus Līguma noteikumiem Puses var jebkurā laikā izbeigt Līgumu uz savstarpējas rakstiskas vienošanās par izbeigšanas nosacījumiem pamata.
	\item Pusei, kura ir tiesīga uz kompensāciju, ir jāpieprasa kompensācija, izmantojot savas šajā Līgumā vai Latvijas Republikā spēkā esošajos piemērojamajos tiesību aktos noteiktās tiesības. Tomēr tiesīgā Puse nedrīkst saņemt divkāršu kompensāciju par vienu un to pašu saistību neizpildi vai pārkāpumu.
	\item Puses var vienoties par Izpildītājam piederošo vai daļēji piederošo Pasākumu demontāžu Ēkā, ja Līgums jebkādu apstākļu dēļ tiek izbeigts pirms termiņa vai ja attiecīgo Pasākumu vērtība tiek atrunāta un ieskaitīta cietušajai Pusei maksājamajā kompensācijā. Iespēja demontēt Pasākumus šajā punktā minētajos apstākļos neierobežo nekādas zaudējumu, izmaksu un izdevumu atlīdzināšanas prasības, uz ko jebkurai no Pusēm var būt tiesības šā Līguma pirmstermiņa izbeigšanas gadījumā.
\end{enumerate}

\subsection{NEPĀRVARAMAS VARAS APSTĀKĻI}
\begin{enumerate}
	\item Par nepārvaramas varas apstākļiem ir uzskatāma jebkura iepriekš neparedzama ārkārtas situācija vai notikums, kam raksturīgas visas norādītās pazīmes:
	\begin{enumerate}
		\item Puses nespēj to paredzēt un ietekmēt;
		\item tas traucē Pusēm pildīt savas saistības;
		\item nevar kvalificēt kā jebkuras Puses pieļautu kļūdu vai nolaidību;
		\item to var pierādīt vai atzīt kā nepārvaramu, lai gan Puse (Puses) ir veikusi saprātīgas darbības, mēģinot to novērst.
	\end{enumerate}
	\item Nepārvaramas varas apstākļi ietver, bet neaprobežojas ar karadarbību, dabas katastrofām un valsts pārvaldes tiesību aktiem.
	\item Par nepārvaramas varas apstākļiem NAV uzskatāmi: Pasākumu defekti, saskaņotajai kvalitātei vai daudzumam neatbilstoši pakalpojumi, Izpildītāja nodrošinātās vai uzstādītās iekārtas vai materiāli vai to nodošanas ekspluatācijā kavējums (ja to nav izraisījuši nepārvaramas varas apstākļi), Pasūtītāja strīdi, streiki, finanšu grūtības un tādi citi apstākļi, kas raksturīgi Pusei, kura paļaujas uz nepārvaramas varas apstākļiem.
	\item Puses nav atbildīgas par pilnīgu vai daļēju šajā Līgumā noteikto saistību neizpildi, ja cēlonis ir nepārvaramas varas apstākļi. Pusei, kura atsaucas uz nepārvaramas varas apstākļiem, ir jāsniedz otrai Pusei pierādījumi par tiem.
	\item Puse (turpmāk – “Cietusī puse”), kurai liegts izpildīt savas šajā Līgumā noteiktās saistības, ir nekavējoties un ne vēlāk kā 3 (trīs) Darba dienu laikā jāpaziņo otrai Pusei par Nepārvaramas varas apstākļiem, līdzko Cietusī puse paredzējusi vai uzzinājusi par tiem, aprakstot paredzamo vai iestājušos situāciju, sniedzot apstākļu aprakstu, iespējamo ilgumu, sagaidāmās sekas un iespējamo to risinājumu.
	\item Pusēm ir solidāri jāveic jebkādas nepieciešamās darbības, lai mazinātu nepārvaramas varas apstākļu ietekmi, un jāīsteno saprātīgi pasākumi, lai mazinātu jebkādu radušos kaitējumu.
	\item Ja nepārvaramas varas apstākļi turpinās ilgāk nekā 6 (sešus) mēnešus pēc kārtas un to izbeigšanās nav gaidāma vēl 3 (trīs) mēnešus, Izpildītājam vai Pasūtītājam ir tiesības vienpusēji izbeigt Līgumu.
\end{enumerate}

\subsection{KONFIDENCIALITĀTE}
\begin{enumerate}
	\item Līguma noslēgšanas vai izpildes gaitā iegūtā informācija, kas nav vispārpieejama trešajām personām un kuras izpaušana, kā zināms vai vajadzētu būt zināmam saņēmējai Pusei, var kaitēt atklājējas Puses likumīgajām tiesībām vai interesēm, uzskatāma par konfidenciālu.
	\item Puses vienojas trešajām personām neizpaust otras Puses Konfidenciālo informāciju, kā arī neizpaust otras Puses datus, kurus varētu izmantot konkurences nolūkā vai nelikumīgu darbību veikšanai gan Līguma spēkā esamības laikā, gan 3 (trīs gadus) pēc Līguma darbības izbeigšanās.
	\item Informācija, kuru ir publiskojušas trešās personas, Pusēm nepārkāpjot Līguma noteikumus, nav uzskatāma par konfidenciālu.
	\item Lai nodrošinātu Līgumā noteikto pienākumu izpildi, Puses var trešajām personām izpaust Konfidenciālo informāciju. Ja Puses izpauž Konfidenciālo informāciju, pamatojoties uz šā Punkta noteikumiem, tām ir jānodrošina, lai trešās personas ievērotu tādu pašu konfidencialitātes pienākumu, kāds ir noteikts šajā Līgumā.
	\item Konfidenciālās informācijas izpaušana saskaņā ar Latvijas Republikā spēkā esošo normatīvo aktu prasībām nav uzskatāma par Līguma pārkāpumu.
	\item Reklāmas nolūkā un sabiedrības informēšanas nolūkā Izpildītājs, visi tā tiesību pēcteči un Pasūtītājs ir tiesīgi izpaust vispārīgu informāciju par savstarpējo sadarbību, tai skaitā: izpaust informāciju, kas jau ir vispārpieejama par Pusēm, sadarbības raksturojumu, panākto Enerģijas ietaupījumu un Enerģijas patēriņa datus. Ciktāl tas neaizskar otras Puses likumīgās tiesības un intereses attiecībā uz konfidenciālas informācijas aizsardzību. Ja Pusei rodas šaubas par konkrētās informācijas raksturu, pirms tās izpaušanas šādas darbības un informācijas raksturs saskaņojami ar Pusi (Pusēm), kuras likumīgās tiesības un intereses var aizskart šīs informācijas izpaušana, ja šī Puse uzskata, ka šai informācijai nav piemērojams Līgumā noteiktais konfidencialitātes pienākums.
	\item Iepriekšminētais noteikums neierobežo Pasūtītāja nepārprotamu pienākumu nepieprasīt un neieteikt jebkuram Pasūtītājam, potenciālam Pasūtītājam vai Izpildītāja sadarbības partnerim un/vai jebkādam jaunizveidotam uzņēmumam pārtraukt, atsaukt, anulēt, limitēt, samazināt vai citādi ierobežot savas darījuma attiecības ar Izpildītāju.
	\item Iepriekšminētais noteikums neierobežo Izpildītāja tiesības ievākt, apstrādāt, glabāt, pārveidot, nodot tā ieceltajām personām finansējumu nodrošinātājos partneros un izplatīt visus šādus no Pasūtītāja ievāktus datus nolūkā uzlabot savu Pakalpojumu kvalitāti un attīstīt, pārvaldīt un uzturēt tiešsaistes EPC platformu sunshineplatform.eu, kas sniedz atbalstu visos posmos un tipiskos Energoefektivitātes pakalpojuma līguma projektos iesaistītajām personām.
\end{enumerate}

\subsection{ŠĀ LĪGUMA NOSLĒGŠANA UN GROZĪŠANA}
\begin{enumerate}
	\item Līgums stājas spēkā dienā, kad to saskaņā ar šiem Vispārīgajiem noteikumiem un nosacījumiem paraksta visas Puses, un paliek spēkā līdz pilnīgai visu Pušu saistību izpildei.
	\item Visas izmaiņas, papildinājumi un grozījumi Līgumā ir jāveic rakstveidā, visām Pusēm savstarpēji vienojoties, un tie stājas spēkā pēc visu Pušu parakstīšanas un ir pievienojami šim Līgumam  pielikumu formā.
	\item Visi pārējie noteikumi un nosacījumi vai attiecīgie Pielikumi paliek pilnībā spēkā un izpildāmi. Jebkādas saskaņotās novirzes ir piemērojamas vienīgi tai Līguma daļai, par kuras novirzēm ir panākta vienošanās.
	\item Līgums uzskatāms par izbeigtu pēc tam, kad Pasūtītājs ir pilnā apmērā veicis Izpildītājam visu Maksu maksājumus un Izpildītājs ir izpildījis savas saistības.
	\item Ja Līguma spēkā esamības laikā spēkā stājas grozījumi Latvijas normatīvajos aktos, kuru dēļ jebkuru šajā Līgumā noteiktu saistību izpilde kļūst pilnībā vai daļēji neiespējama, tas neietekmē pārējo šajā Līgumā noteikto saistību spēkā esamību. Šādā gadījumā Puses ievieš Līgumā atbilstošus grozījumus, kuru nolūks un mērķis ir līdz minimumam samazināt ekonomisko ietekmi uz Līguma Pusēm.
\end{enumerate}

\subsection{PUŠU APLIECINĀJUMI}
\begin{enumerate}
	\item Visos ar šo Līgumu saistītajos jautājumos Puses pārstāv to likumīgie pārstāvji (juridiskajām personām) vai šajā Līgumā īpaši norādītas personas. Tikai šā Līguma Īpašajos noteikumos norādītās personas ir tiesīgas pārstāvēt attiecīgi Pasūtītāju vai Izpildītāju.
	\item Līgums ir sagatavots un parakstīts 3 (trīs) eksemplāros latviešu valodā ar vienādu juridisko spēku. Parakstot Līgumu, Puses apliecina, ka tām ir saprotams Līguma saturs, jēga un sekas; tās atzīst šo Līgumu par pareizu, savstarpēji izdevīgu, visus Pušu nosacījumus, apsolījumus, noteikumus un nodoma apliecinājumus ietverošu, un ka tās brīvprātīgi vēlas noslēgt šo Līgumu bez jebkādiem spaidiem attiecībā uz jebkuras Puses gribu.
\end{enumerate}


\subsection{PAZIŅOJUMI}
\begin{enumerate}
	\item Paziņojumu forma: visi paziņojumi, pieprasījumi, prasījumi, prasības un cita saziņa atbilstoši šajā Līgumā noteiktajām prasībām un atļautajam ir jānodod un jānosūta uz Pušu adresēm.
	\item Paziņošanas veids: visi paziņojumi ir jānodod (i) nogādājot personīgi, (ii) ar kurjerdienestu ar piegādi nākamajā dienā, (iii) ierakstītā pasta sūtījumā un ar piegādes apliecinājumu, (iv) pa faksu, (v) elektroniskajā pastā (e-pastā) ar Piegādes apstiprinājumu pieprasījumu, uz šajā Līgumā norādīto Puses adresi vai tādu citu adresi, kuru jebkura Puse var rakstveidā norādīt, (vi) izmantojot EPC platformā sunshineplatform.eu piedāvātos informēšanas pakalpojumus, (vii) Paziņojumiem, kas satur nelielu informācijas daudzumu – īsziņas ar nosūtīšanas uz šajā Līgumā norādīto Puses mobilā tālruņa numuru vai citiem jebkuras Puses rakstveidā norādītajiem numuriem apstiprinājumu.
	\item Paziņojuma saņemšana: visi paziņojumi ir spēkā, kad (i) tos saņem Puse, kurai paziņojums ir adresēts, vai (ii) 7. (septītajā) dienā pēc nosūtīšanas, atkarībā no tā, kas iestājas ātrāk.
\end{enumerate}

\end{multicols}
