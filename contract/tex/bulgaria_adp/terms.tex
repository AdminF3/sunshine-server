\begin{multicols}{2} [\section{GENERAL TERMS AND CONDITIONS}]

  \subsection{DEFINITIONS}
  As used in this agreement and the relations between the parties,
  concerning its conclusion and implementation, the following terms
  have the following meanings:
  \begin{itemize}[label={}]

  \item\textbf{Energy Performance Contract (EPC)} means this contract,
    concluded between the Client and the Contractor; it includes these
    general terms and conditions, as well as specific terms and their
    annexes. The contract shall be signed between the parties and can
    also be generated via the EPC platform.
  \item\textbf{An apartment} is a set of premises, covered or open
    spaces, combined functionally and spatially in one unit to meet
    housing needs and representing an individual apartment in a
    multifamily residential building in condominium mode.
  \item\textbf{Multifamily residential building in condominium mode}
    is a permanent residence building, consisting of apartments,
    occupying at least 60 per cent of its total built-up area. The
    Forfaiting Assignee apartments are more than three and belong to
    more than one owner, in them are carried out energy-saving
    measures, subject to this contract, and other activities, if
    applicable /the building/.
  \item\textbf{Individual apartments in a multifamily residential
      building in condominium mode} are apartments, as well as other
    separate parts of the Building with an independent functional
    purpose.
  \item\textbf{Owners of individual apartments} are persons, having
    acquired the property right on individual apartments in the
    Building and registered as owners in the Property Register with
    the Registry Agency.
  \item\textbf{A business day} is a day on which the commercial banks
    in the Republic of Bulgaria perform general banking transactions.
  \item\textbf{The baseline of energy consumption} means the energy
    consumption required to ensure the necessary statutory temperature
    for the existing condition in the residential Building in
    condominium mode. It is also a basis for comparing the energy
    performance of a structure and determination of the potential for
    reducing energy consumption. The baseline of energy consumption is
    set out in the Energy Efficiency Audit, Annex 1 – Report and
    Summary on Building EE Audit (see 1.2.7) to the specific terms of
    the contract.
  \item\textbf{Baseline (normalised) Annual Energy Consumption (BAEC)}
    means energy consumption based on the energy audit and the
    baseline indicated in it before the implementation of
    energy-saving measures.
  \item\textbf{Guaranteed Annual Energy Consumption (GAEC)} means the
    Contractor’s guaranteed annual energy consumption for a
    multifamily residential building after the implementation of
    energy-saving measures.
  \item\textbf{Achieved Annual Energy Consumption (AAEC)} is the
    amount of the actual energy consumed in the Building for one
    calendar year after the implementation of energy-saving measures.
  \item\textbf{Energy-Saving/Saving of Energy} is the amount of energy
    saved, determined by measuring and estimating energy consumption
    as a difference in the energy consumed before and after
    implementation of energy-saving measures, by settling and
    normalising given the conditions affecting the energy consumption.
  \item\textbf{Guaranteed Annual Energy-Saving (GAES)} is the total
    quantity of final energy saved in the Building for the reporting
    period of one year (kWh/year), guaranteed by the Contractor to be
    achieved, as a result from implementation of the energy-saving
    measures and meaning the difference between BAEC and AAEC.
  \item\textbf{Achieved Annual Energy-Saving (AAES)} means the
    difference between BAEC and AAEC.
  \item\textbf{Energy} are energy products, combustible fuels, thermal
    energy, energy from renewable sources, electric power or any other
    form of energy as defined in art. 2 (d) of Regulation (EC)
    No. 1099/2008 of the European Parliament and of the Council of 22
    October 2008 on statistics for the energy sector (OJ, L 304 (1 of
    14 November 2008);
  \item\textbf{Energy Efficiency Audit} is a process, based on a
    systematic method for the determination and valuation of energy
    flows and consumption in the Building, determining the scope of
    technical and economic parameters of measures for energy
    efficiency increases. The Audit Report and Summary are Annex 1 –
    Report and Summary on Building EE Audit (see 1.2.7), to the
    specific terms of this contract.
  \item\textbf{Energy-Efficient Service} is the material benefit,
    benefit or commodity, obtained by combining energy with energy
    efficiency technology or action that may cover the operation,
    maintenance and control required to provide service under this
    contract and is proven that under normal circumstances it results
    in verifiable and measurable improvements of the energy efficiency
    and primary energy savings;
  \item\textbf{Energy-Saving Measures (ESM)} are the specific measures
    recommended in the energy efficiency audit, as well as the
    measures additionally proposed by the Contractor, and they mean
    planned, technical, technological, or other Contractor’s
    activities, entailing achievement of the guaranteed result on this
    contract.
  \item\textbf{Efficiency Coefficient (EC)} is the ratio between AAES
    and GAES for the Building.
  \item\textbf{Value of Guaranteed Annual Energy Consumption (VGAEC)}
    means the sum of the products of the guaranteed annual consumption
    of each energy-carrier and their price, determined in the specific
    terms of the contract.
  \item\textbf{Value of Achieved Annual Energy Consumption (VAAEC)}
    means the sum of the products of achieved annual consumption of
    each energy-carrier and their price, determined in the specific
    terms of the contract.
  \item\textbf{Method of guaranteed result evaluation} is a method
    developed by the Contractor and is Annex 4 – Method of guaranteed
    result evaluation (see 1.2.10); to the specific terms of the
    contract, governing the conditions for measuring energy
    consumption, monitoring rules and how to calculate and account
    contract performance and achieve the guaranteed result during the
    monitoring period;
  \item\textbf{Parties} are the Client and the Contractor
    collectively, and every one of them the respective Party.
  \item\textbf{Clients} are all owners of individual apartments in the
    Building.
  \item\textbf{A contractor} is a company within the meaning of the
    Commerce Act or the purpose of the legislation of another member
    state of the European Union, or another State – Party to the
    Agreement on the European Economic Area, or of the Swiss
    Confederation, having an activity involving the performance of
    contractual services with guaranteed result. The Contractor is a
    party to this contract and shall agree to implement the
    energy-saving measures, subject of the contract.
  \item\textbf{Comfort Standards} are set of parameters in the
    Building and in the separate individual apartments of it, which
    the Contractor shall guarantee under this contract, described in
    detail in Annex 3 – Comfort Standards (see 1.2.9); to the specific
    terms of the Agreement;
  \item\textbf{The commencement date of contract implementation} is
    the date of signing this contract; The performance of stage 1 of
    the contract starts on the commencement date.
  \item\textbf{The commencement date of construction implementation}
    is the date on which the period of construction starts.
  \item\textbf{Commissioning date is the date} on which the
    implemented measures are officially accepted and commissioned
    according to the applicable laws and regulations in the Republic
    of Bulgaria. The monitoring period starts on the commissioning
    date. If the implemented measures are not subject to official
    acceptance and commissioning, the monitoring period begins from
    signing the Protocol for Delivery and Acceptance.
  \item\textbf{The construction period} is the time for implementation
    of construction and assembly works for measures
    implementation. The construction period finishes with the signing
    of the Protocol for Delivery and Acceptance.
  \item\textbf{Protocol for Delivery and acceptance of implemented
      measures} is a document that marks the completion of the
    construction period, and the Building is delivered by the
    Contractor /builder/ to the Client, drawn up under art. 176 to the
    Spatial Development Act and Regulation No. 3 on drawing up of
    records and statements during the construction. If under the
    applicable legislation for the specific Building Act template 15
    is not required, the Delivery and acceptance protocol with which
    the Building is delivered by the Contractor to the Client shall be
    deemed the final document proving that the implemented works are
    duly completed under the conditions of this contract;
  \item\textbf{The monitoring period} is the period of measurement,
    reporting on, and comparing of the effectively achieved energy
    savings compared to the guaranteed result, which starts from the
    date of commissioning and continues to the end of the term of the
    contract.
  \item\textbf{Result measuring and evaluation} are activities carried
    out with a purpose to determine achievement of the guaranteed
    outcome, implemented during the monitoring period, according to
    the rules of the Method of guaranteed result evaluation.
  \item\textbf{EPC platform} {-} a multilateral online platform for
    interested parties for the conclusion of energy performance
    contracts, accessible at [●], which supports the development and
    management of projects for energy renovation of buildings, based
    on negotiation with guaranteed result.
  \item\textbf{Fund} [●]
  \item\textbf{Latent works} are defaults and deficiencies of the
    Building, for which the Client was not aware, and the Contractor
    could not identify by reasonable monitoring and routine checks at
    signing of this contract.
  \item\textbf{The manager} is a managing authority of a condominium
    within the meaning of the Condominium Ownership Management Act,
    including a physical person under art. 19, par. 8 of the
    Condominium Ownership Management Act, to whom can be assigned
    under a contract and by a decision of the General meeting specific
    rights and obligations, related to the subject of this contract;
  \item\textbf{Service and Maintenance Price} represents the monthly
    remuneration, due by the Client to the Contractor for services,
    related to servicing and maintenance of the measures, subject of
    annual indexation according to this contract.
  \item\textbf{Monitoring Price} represents the monthly remuneration,
    due by the Client to the Contractor for services, related to
    result measuring and accounting evaluation, including measuring of
    energy consumption, installation of measuring devices and other
    activities, subject of annual indexation according to this
    contract.
  \item\textbf{Price of ESM implementation} is the sum, which amount
    is determined in the specific terms of the contract and includes
    all costs of the Contractor for the design and implementation of
    ESM. Annual interest shall be accrued on the price for ESM
    implementation. The annual interest rate shall be agreed upon
    between the Client and the Contractor according to the specific
    terms of the contract.
  \item\textbf{Operation and Maintenance Manual} is a manual, where
    the schedule for maintenance of measures, implemented under this
    Agreement is set out, and operation activities related to the
    upkeep are outlined; The Manual is provided as Annex 5 {-}
    Operation and Maintenance Manual (see 1.2.11); to the specific
    terms of the contract;
  \item\textbf{Indicative Q \& V Account} {–} the quantity and value
    account, representing Annex 2 {-} Scope of the measures and BOQ
    (see 1.2.8) by the moment of signing this contract and including a
    description of the measures and unit prices, in line with the
    Energy efficiency audit, and pre-project studies carried out by
    the Contractor.
  \end{itemize}

  \subsection{ACCEPTANCE OF THE CONTRACTUAL TERMS AND CONDITIONS}
  \begin{enumerate}
  \item The Client shall accept that the Contractor has the necessary
    qualification, expertise, and abilities to implement energy-saving
    measures, and other activities and services, assigned by the
    present contract. By signing the Agreement, the Client shall
    authorise the Contractor to carry out all legal and factual
    actions necessary for the implementation of the contract
    subject. If necessary and upon request, the Client is obliged to
    submit an explicit and separate power of attorney to the
    Contractor.
  \item The Contractor shall implement the energy-saving measures in
    the multifamily residential Building under the conditions set out
    in this contract. At signing of this agreement, the Contractor has
    to be acquainted with the building status before implementation of
    the measures.
  \item The Contractor confirms that the price for the implementation
    of the energy-saving measures, under the specific terms of the
    contract, includes all necessary costs of the Contractor for
    implementation of ESM, including these for design, preparation,
    and implementation of the construction, workforce, disposal of
    building waste, and others.
  \item The terms used in the general terms and conditions and the
    specific terms, and the Annexes to them, have the respective
    meanings as specified in art. 1.1.1 above.
  \item In case of discrepancy between the general terms and
    conditions and the specific terms, and the Annexes to them, the
    specific terms shall take precedence.
  \end{enumerate}

  \subsection{SAFETY, QUALITY, AND CONTROL}
  \begin{enumerate}
  \item The Contractor shall render the services, subject of this
    contract:
    \begin{enumerate}
    \item With the highest standard of skills and care, as expected
      from experienced and professional Contractors, who regularly
      carry out work and services with the same or similar scope and
      complexity, as in this Agreement.
    \item Use materials and equipment with appropriate quality and
      corresponding to the applicable rules and norms in the field of
      Building construction, and all other applicable legal standards.
    \item Cause the least inconvenience when the Building and the
      individual apartments in it are used by the Client, inhabitants,
      and visitors.
    \end{enumerate}
  \item The comfort standards achieved in the monitoring period comply
    with or exceed the level set out in Annex 3 {-} Comfort Standards
    (see 1.2.9), to the specific terms of the contract.
  \item When the windows in an apartment or another individual unit in
    the Building are open, and 2 (two) hours after closing the
    windows, the Contractor does not guarantee the temperature levels
    in the premises, as agreed in Annex 3 {-} Comfort Standards (see
    1.2.9), to the specific terms.
  \item The Contractor shall provide adequate ventilation level in the
    individual apartments, in compliance with the applicable norms.
  \item The Contractor shall take all necessary actions to guarantee
    the safety and health of workers on the construction site.
  \item The Contractor shall apply appropriate measures for protection
    against death or injuries, which may be caused by faulty
    implementation or gross negligence of the Contractor, his
    employees, or subcontractors, during the term of this
    contract. The Contractor shall take care of the protection of the
    Building against damages, caused during the implementation of the
    measures.
  \item The Contractor shall guarantee that all utility services,
    provided in the Building, shall not be turned off or interrupted
    due to faulty implementation without prior notice. All utility
    services, interrupted by the fault of the Contractor, shall
    forthwith be reinstated by the Contractor at his expense. The
    Contractor is not liable of these interruptions out of his control
    and/or are due to Supplier’s actions or inactions of utility
    services and/or of acts of third persons.
  \item Over the construction period, the Contractor is obliged to
    ensure appropriate protection of the Building against the
    influence of meteorological conditions, prevent the penetration of
    rainwater and damages on the Building. The Contractor’s obligation
    shall be waived in case of infiltration of underground waters and
    force majeure.
  \item The Contractor shall comply with the European Professional
    Code on Energy Performance Contracts ([●]), which represents a set
    of values and principles, considered as fundamental for the
    successful, professional, and transparent implementation of the
    Energy Performance Contracts in the European countries.
  \end{enumerate}

  \subsection{GUARANTEES}
  \begin{enumerate}

  \item The Contractor shall provide the Client with an Energy-Saving
    Guarantee, which is subject of measurement and verification for
    the entire term of the Agreement. The guaranteed energy-saving,
    the technical indicators which are to be followed during
    implementation of the activities, subject of the Agreement, and
    recording of the guaranteed result, shall be determined in the
    specific terms to the Contract and Annex 4 – Method of guaranteed
    result evaluation (see 1.2.10), to them.
  \item During the monitoring period, the Contractor shall guarantee
    compliance with the comfort standards, according to Annex 3 –
    Comfort Standards (see 1.2.9).
  \item In the monitoring period, the Contractor shall guarantee the
    proper functioning of the measures, subject of this contract,
    except for the cases of normal wear and tear.
  \item In the monitoring period, the Contractor shall guarantee the
    suitability of the insulation materials used, in line with their
    specifications and normal wear and tear.
  \item The Contractor shall ensure, at the end of the contract term,
    the proper functioning of all implemented measures, in line with
    their specifications and normal wear and tear and taking into
    consideration the appropriate maintenance. The Contractor, at the
    end of the contract term, provides the Client with all guides and
    manuals for use, maintenance, records, instructions, other
    documentation, software, licenses for intellectual property,
    necessary for the maintenance of the implemented measures in a
    good state, and compliance with the comfort standards under this
    Agreement.
  \item The above guarantees are valid only when the Client complies
    with all instructions and rules of operation.
  \item The Contractor, before the commencement of the construction
    period, provides the Client with a bank guarantee for good
    performance to the amount of 10\% on the ESM implementation cost
    (VAT exclusive), determined, according to the specific terms of
    the contract. The bank guarantee is irrevocable, unconditional,
    and is valid until the completion of the construction period.
    \begin{enumerate}
    \item If the Contractor does not provide a performance guarantee,
      the Contractor does not have the right to start construction
      activities.
    \item The guarantee under art. 21 is valid over the entire
      construction period. In the case that the construction period is
      extended, the Contractor shall extend the guarantee for the new
      period.
    \item The Client releases the guarantee within thirty (30)
      calendar days from the date of signing the Acceptance and
      delivery protocol for implementation of the measures.
    \end{enumerate}
  \item The Contractor, not later than 10 (ten) days after signing of
    the Acceptance and delivery protocol for implementation of the
    measures provides the Client with a bank guarantee for good
    performance, to the amount of 5\% on the price of ESM
    implementation, VAT excluded, determined according to the specific
    terms of the contract. This bank guarantee is
    irrevocable. unconditional, and is valid for 36 (thirty-six)
    months.
  \item The performance guarantees, set out in art. 21 and 22 will be
    issued by a bank, registered in the Republic of Bulgaria or
    another member state of the European Union or the European
    Economic Space, which can render services on the territory of the
    Republic of Bulgaria.
  \end{enumerate}

  \subsection{RIGHTS AND OBLIGATIONS OF THE CONTRACOR}
  \begin{enumerate}
  \item The Contractor confirms the availability of all the necessary
    professional qualifications, experience, and abilities needed for
    rendering services under this contract.
  \item The Contractor shall obtain all necessary coordinating
    documents, permits and approvals from the competent state and
    municipal authorities for the implementation of the construction
    works and the delivery of services according to this contract. If
    the Contractor needs assistance from the Client, the Contractor
    shall notify the Client. The costs for fees and coordination are
    at the expense of the Client. If an institution rejects
    coordination and/or approval of the project/projects due to
    Contractor’s fault, the latter is obliged to supplement forthwith,
    amend or revise at his expense the project/projects, according to
    the instructions given by the respective institution.
  \item The Contractor is obliged to provide the Client with
    investment projects for the construction implementation following
    the requirements of Regulation No. 4 of 2001 y. on the scope and
    content of investment projects, as well as all under the
    applicable technical regulations for design, implementation and
    control in the construction on the respective parts of the
    investment project. The Client is not entitled to refuse
    coordination and approval if the projects are following the
    regulatory requirements and the energy efficiency audit.
  \item After the implementation of the construction design, the
    Contractor provides the Client with precise bill of quantities,
    which is to be enclosed to this contract as Annex 2 {-} Scope of
    the measures and BOQ (see 1.2.8), to the specific terms. The total
    amount on the bill of quantities could not exceed the amount on
    the indicative BOQ, Annex 2 – Scope of the measures and BOQ (see
    1.2.8), to this contract by the moment of its conclusion. If the
    indicative and the final BOQ are the same, as an enclosure to the
    Agreement shall remain the indicative BOQ.
  \item The Contractor shall carry out the construction and
    installation works within the prescribed period of
    construction. The Contractor shall notify the Client about the
    estimated commencement date of the construction implementation
    within 20 (twenty) business days after the signing of this
    contract.
  \item The Contractor shall notify the Client at least 10 (ten)
    business days in advance about the commencement date of the
    construction implementation, which shall enable the Client to
    clean the common parts of the Building (including stairways,
    basements, attic, roof, storage areas for coal/wood and gas,
    electric and telecommunication panels and boiler rooms), from
    waste, abandoned items and any other objects there. The Client is
    obliged to provide the Contractor with the necessary assistance to
    start the construction, including to provide access to the
    Building and its common parts and sign all required records and
    statements for beginning of construction, including but not only a
    protocol for opening a construction site and determination of the
    construction line and level, and all other necessary documents. If
    the Client is not able to clean the common parts in due time, the
    Contractor is entitled to clean the common parts of the Building
    and to issue an invoice to the Client for payment of this work,
    payable by the Client within 20 business days. The remuneration of
    the Contractor for cleaning the common parts shall be paid
    independently and separately from the price under art. 30 of the
    current terms and conditions.
  \item In the construction period, the Contractor provides all
    necessary work for implementation of the measures, including
    materials and equipment with good quality and quantities,
    according to the requirements of the Law on technical requirements
    for products, BDS, the project documentation.
  \item During the construction period, the Contractor organises the
    supply of electric energy with separate metering and pays for
    energy consumed for implementation of the construction
    activities. The Contractor is entitled to access to the available
    power supply system of the Building.
  \item The Contractor shall clean the construction site (the common
    parts of the Building, windows, entrance halls, and others.) upon
    completion of the construction and installation works before the
    commissioning date.
  \item The Contractor, during the construction period, shall notify
    and invite the Client to participate in weekly meetings for
    monitoring of the construction works progress. The Contractor
    shall provide to the Client 2 (two) reports on the monthly
    progress on the condition and the course of the construction
    works. These reports can be sent via the Energy Performance
    Contract platform.
  \item The Contractor shall carry out the construction and
    installation works in compliance with the technical parameters,
    set out in the contract and its Annexes, without defaults and
    deficiencies.
  \item The Contractor shall identify and designate his qualified
    representative, who shall monitor the construction implementation
    and be responsible for the contacts with the Client.
  \item The Contractor shall invite the Client to accept the
    construction, carried out in the Building, the parties shall sign
    the Acceptance and delivery protocol for implementation of the
    measures. Within ten days from the date of Client`s receipt of the
    invitation, the Parties shall draw up a bilateral Protocol, in
    case there are deficiencies identified or other remarks made by
    the Client on the quality of construction and installation works
    carried out by the Contractor. This protocol shall be signed by
    duly authorized representatives of the Parties. After elimination
    of the omissions or remarks, the Contractor shall send a new
    invitation to the Client for signing of the Delivery Acceptance
    Protocol. If there are no omissions found, or such omissions have
    been duly removed after receipt of the invitation, the Acceptance
    and delivery protocol shall be signed within ten business days
    from the date of the respective invitation. In case the Client
    does not appoint representatives for the signing of the Protocol
    within ten days from receipt of the invitation for signing, the
    site shall be deemed accepted without any remarks by the Client.
  \item During the monitoring period, the Contractor shall notify the
    Client if waste and/or objects which could impede the Contractor
    in his activities of servicing and maintenance under this
    contract, are stored in the common parts of the Building.  If the
    Client cannot clean the common parts in due time, the Contractor
    shall be entitled to clean the common parts of the Building and to
    issue an invoice to the Client for payment of these works, payable
    by the Client within 20 business days. The Contractor’s
    remuneration for cleaning of the common parts shall be paid
    independently and separately from the price under art. 30from the
    present general terms and conditions.
  \item The Contractor, during the monitoring period, shall notify the
    Client of each default and/or theft, affecting the measures
    carried out.
  \item In the Service period, the Contractor shall provide the
    necessary work for maintenance of the measures, including tools,
    materials and equipment with acceptable quality and quantity.
  \item The Contractor shall ensure compliance with the comfort
    standards under this contract. The Contractor shall not be liable
    of any break-downs or lack of heat-supply or supply of electricity
    to the Building, beyond his control, including such if the
    heat-supply company and/or the power company cannot supply thermal
    energy and/or electricity and/or due to force majeure
    circumstances.
  \item The Contractor has the right to assign to third persons
    (subcontractors) the implementation of the construction works and
    services, set out in this contract. In such case, identical
    conditions and requirements to the respective subcontractor shall
    be included in the subcontractor agreement.
  \item The Contractor within 5 (five) business days shall notify the
    Client about change in the address for correspondence and/or the
    contact persons or other changes in his legal status, including
    transformation, liquidation, or insolvency.
  \item The Parties agree on that to achieve the guaranteed
    energy-savings, the Contractor shall take the financial,
    commercial, and technical risk. The risk for the Contractor means
    that in case of non-achieving the guaranteed energy-savings, the
    Contractor shall not receive the total amount of the respective
    annual price for the implementation of ESM under art.  1.1.7
    subart. 76 (xv), and if it is already accepted, he owes the Client
    a compensation. The annual price for the implementation of ESM
    under art. 1.1.7 subart. 76 (xv) is the sum of the payments due,
    according to the amortisation table at the price of ESM for the
    period of one calendar year.
  \item The Contractor shall notify in written the Client of every
    prospective delay in the operation schedule and of each
    circumstance, which could impact on the performance of any
    contractual obligation.
  \item The Contractor shall follow to fulfil his other obligations
    under this contract, including documentation of all changes made
    during the project.
  \item The Contractor shall be entitled to require from the Client
    all the necessary assistance for implementation of the contract
    subject, demand from the Client acceptance of the implemented
    measures, in case of correct, timely and quality implementation.
  \item The Contractor is entitled to receive the agreed-on price
    under the terms and conditions of this contract.
  \item The Contractor is entitled to access the Building, installed
    equipment, appliances with implemented ESM and tools for
    measurement over the entire contract period.
  \item The Contractor is entitled to obtain from the Client any other
    necessary assistance for implementation of the contract subject.
  \end{enumerate}

  \subsection{RIGHTS AND OBLIGATIONS OF THE CLIENT}
  \begin{enumerate}

  \item The Client (every apartment owner) confirms that, prior to
    signing of this agreement, have accepted a valid decision of the
    Condominium general meeting , which renders the present Energy
    Performance contract and all Annexes thereto legally binding to
    all owners of apartments and units within the Building. The Client
    (every owner of an apartment or individual apartment in the
    Building) shall notify the tenants, users, and all other
    inhabitants of the respective obligations under this Agreement.
  \item The Client shall provide information and documents, required
    by the Contractor for implementation of the measures as soon as
    possible upon receipt of the Contractor’s request.
  \item The Client shall provide the Contractor with the necessary
    assistance during the implementation of coordination procedures,
    obtaining of permits, approvals or other documents related to the
    successful performance of this contract, including support of the
    investment projects, issue of a construction permit, commissioning
    of the Building, and others. In case of need, the Client shall
    duly authorise the Contractor to take any legal and factual
    actions before the competent authorities for successful
    implementation of the contract.
  \item The Client shall not interfere with the Contractor
    construction activities, and the overall implementation of the
    measures during the construction period, their servicing and
    maintenance during the monitoring period. The Client undertakes to
    act in good faith to facilitate their implementation, maintenance,
    and achievement of the guaranteed energy-saving.
  \item The Client shall grant the Contractor or any other person,
    authorised by the Contractor, access to the Building, including to
    every separate apartment in the term of this Agreement, with the
    purpose to be provided with the agreed-on services. The Client
    must ensure access to the Building at any time on the business
    days (between 8:00 and 20:00 H), and to any individual apartment –
    in case of need and upon preliminary coordination of the time. The
    Client cannot without any grounds deny access to an individual
    apartment, as well as delay to allow the access.
  \item The Client, before the commencement date of the construction
    period, must clean the common parts (including stairways,
    basements, attic, roof, storages for coal/wood and gas, electric
    and telecommunication panels, and boiler rooms).
  \item The Client guarantees that during the monitoring period, the
    common parts of the Building shall be maintained clean and in good
    condition.
  \item The Client is not entitled to interfere with the
    implementation and/or the servicing of measures already
    implemented, without the written consent and authorisation of the
    Contractor or in contradiction with the operation instructions,
    provided by the Contractor, particularly if the intervention harms
    the energy-saving level and achievement of the guaranteed
    result. The interference of the Client with the adjustments of the
    heating system, electrical system, domestic hot water system, and
    the ventilation system, as well as all other systems, where ESM
    are implemented, is considered as a serious breach by the Client
    of his obligations on this Agreement and serves as a sufficient
    reason for the Contractor to terminate the contract, or to accept
    that the guaranteed energy-saving is achieved with all
    consequences, resulting from it. The choice to terminate the
    Agreement or to deem the energy-saving as performed in this case
    belongs to the Contractor.
  \item The Client shall take all necessary actions to guarantee that
    no one in the Building will interfere with or tamper the
    adjustments of the heating system, electrical system, domestic hot
    water system, and the ventilation system, as well as all other
    systems, where ESM are implemented.
  \item The Client shall forthwith notify the Contractor after finding
    (not later than one working day), of any default or change, or
    violation on the measures, installed by the Contractor.
  \item The Client shall notify of any circumstances which has or
    could harm the energy-saving guarantee. In the case the Client
    does not make this notification, it shall be considered that the
    energy-saving is achieved with all consequences stemming under
    this contract.
  \item The Client shall notify the Contractor and coordinate with the
    Contractor in advance within 20 (twenty) business days before
    implementation of construction, repair or other activities which
    are not part of this Agreement but have a future effect on the
    energy consumption in the Building, including such activities in
    the separate apartments. In the case of non-coordination with the
    Contractor, the guaranteed energy-saving shall be considered as
    achieved with all consequences stemming under this contract.
  \item The Client (owner of an individual apartment) shall notify and
    coordinate in advance with the Contractor the implementation of
    any activities in the individual apartment and/or the common parts
    of the Building which could affect the energy consumption of the
    Building, including but not limited to: i) replacement of
    radiators and/or convectors and/or heating elements, installation
    of other energy consumers; (ii) replacement of windows and
    casings; (iii) expansion of the heated area of the apartment,
    including space of the balcony/loggia; iv) installation of
    mechanic ventilation systems, etc. In these cases, the Contractor
    is entitled to reconsider the energy-saving guarantee, set out in
    the contract. In case of non-coordination with the Contractor, the
    guaranteed energy-saving shall be considered as achieved with all
    consequences stemming under this contract.
  \item The Client, during the heating period, has the right to open
    the windows in the apartments of the Building within 10 (ten)
    minutes per day, to provide circulation of fresh air, to get rid
    of: (i) dust or strong smell of cleaning products, and (ii) strong
    smell after cooking.
  \item The Client, during the heating season, has the right to open
    the windows in his apartment at any time for health reasons of the
    persons occupying it.
  \item The Client shall ensure that all windows in the common parts
    are kept closed in the heating season.
  \item The Client shall ensure that all entrance doors of the
    Building are not left open during the heating season.
  \item If the ownership in apartments or other individual units in
    the Building change, the grantor of rights forthwith and not later
    than 5 (five) business days after such a change shall notify the
    Contractor.
  \item The Client (each Apartment owner) shall ensure that in case of
    any transfer of property rights over its apartment, the new
    Apartment Owner (transferee) shall sign an additional agreement to
    take on all rights and obligations, arising under the
    Agreement. In all cases, the Contractor is entitled to assess
    whether the former owner remains jointly and severally liable with
    the transferee for implementation of the monetary obligations to
    the Contractor under this contract. The former owner is jointly
    and severally liable also in the cases when is not fulfilled the
    requirement for notification and signing of an additional
    agreement above.
  \item The Client, within 5 (five) business days, shall notify the
    Contractor about the change of the Manager of the Condominium. The
    Client shall inform the new Manager about the provisions of this
    Agreement.
  \item The Client also has the following obligations related to the
    implementation of the contract:
    \begin{enumerate}
    \item Provide, according to the contractual conditions of this
      Agreement, administrative, organisational, and other necessary
      assistance for implementation of the agreement.
    \item Provide the Contractor with the necessary input data, and
      consequently additional data, for implementation of his
      contractual obligations, including all available data for the
      architectural and construction parameters of the Building, the
      existing installations, equipment and energy consumption, as
      well as all other data, necessary for exact and correct
      implementation of the investment projects.
    \item If necessary, support the Contractor, as well as sign the
      documents under Regulation No 3 of 31.07.2003 on drawing up acts
      and protocols during the construction.
    \item Coordinate every investment project within the terms
      established in the contract, as well as in the determining
      regulatory period, when the coordination procedure requires the
      coordination to be made by an external (supervisory) body.
    \item Notify the Contractor of signed contracts or agreements with
      other natural or legal persons, which implementation could
      unfavourably influence the results from assumed contract
      obligations by the Contractor under this contract.
    \item After the implementation of ESM, ensure the use of the
      apartments with the care of a good owner, and in compliance with
      the instructions and provisions for operation, issued by the
      Contractor.
    \item The Client is obliged during the construction implementation
      to provide construction supervision, according to the provisions
      of the Spatial Development Act.
    \item The Client shall coordinate with the Contractor, preliminary
      and in written form, every change in mode of operation of the
      apartments, which separately or along with other changes can
      deviate the calculation of the technical parameters, and
      respectively of the energy savings.
    \end{enumerate}
  \item If the Client changes the mode of use of the apartments and/or
    the Building and/or part of it, or installs new energy consumers,
    without the preliminary coordination with the Contractor, which
    results in a deviation from the agreed energy-savings, the
    guaranteed result for the whole remaining term of the contract
    shall be considered as achieved.
  \item The Client shall accept the works implemented by the
    Contractor if he does not have any remarks on the implementation.
  \item The Client is obliged to pay to the Contractor the price due
    under the terms and conditions of this contract.
  \item The Client shall also fulfil his other obligations under the
    contract.
  \item The Client is entitled to require from the Contractor to
    perform well, exactly, and timely his assigned activities.
  \end{enumerate}

  \subsection{PRICE AND PAYMENT. ANNUAL SETTLEMENT}
  \begin{enumerate}
  \item The Client shall pay the Contractor a price for the
    implementation of this contract, which includes (xv) recovering of
    the investment made by the Contractor for the performance of ESM,
    (xvi) price for servicing and maintenance of the measures, as well
    as (xvii) price for monitoring. The Contractor entirely finances
    with own means the investment for implementation of ESM in the
    Building to the amount, according to the Specific terms of this
    contract.  The price for the implementation of this contract shall
    include the following components:
    \begin{enumerate}
    \item Price for implementation of ESM.
    \item Price for servicing and maintenance.
    \item Price for monitoring.
    \end{enumerate}
  \item The amount under art. 76 subart. (xv) includes all necessary
    expenses of the Contractor for the design and implementation of
    ESM. When they sign this contract, the parties also sign Annex 2 {-}
    Scope of the measures and BOQ (see 1.2.8) to the specific terms
    -indicative BOQ, where are set out the unit prices and the total
    value of all construction works. Upon implementation of the
    design, the Contractor provides the Client with the final BOQ,
    which becomes an integral part of the contract and replaces the
    indicative BOQ an enclosure. The total price under art.  76
    subart. (xv) on the final BOQ cannot exceed the price of the
    indicative BOQ, save as the parties agree additionally on it.
  \item The Contractor shall bear the financial risk, as well as the
    technical and commercial risk for the implementation of the energy
    efficiency measures and achieving the guaranteed savings.
  \item The price under art. 76 subart. (xv) shall be paid, according
    to the amortisation table Annex 7 – Amortisation plan and rules of
    determination of the sums due by every owner of individual
    apartment (see 1.2.13), to the Specific terms. The sum due under
    art. 76 subart. (xv) is subject of interest-accrual, according to
    the Specific terms of the contract. The sums due under art. 76
    subart. (xvi) and (xvii) are subject of annual indexation pursuant
    to Annex 8 – Rules of indexation of the price for servicing and
    maintenance, and the price for monitoring (see 1.2.14), to the
    Specific terms of the contract.
    \begin{enumerate}
    \item The sum under art. 76 subart. (xv) shall be paid in full
      from energy-savings generated; for this purpose, the parties
      make an annual settlement. The payments on the amortization
      table {-} Annex 7 {-} Amortization plan and rules of determination of
      the sums due by every owner of individual apartment (see 1.2.13)
      – for the respective year, are based on the value of the
      achieved savings calculated at energy prices under the Specific
      terms of the contract for energy saving. Annually, until 31-st
      January of the respective year, the Parties record AAES,
      reflecting the achievement or underachievement of GAES by
      calculating the Efficiency Coefficient. The Efficiency
      Coefficient shall be calculated under the Method of guaranteed
      result evaluation in Annex 4 – Method of guaranteed result
      evaluation (see 1.2.10) to the contract.
    \item When the value of the Efficiency Coefficient is equal to
      one, the obligation of the Contractor for achieving the
      guaranteed energy saving shall be considered completed.
    \item When the value of the Efficiency Coefficient is less than
      one or more than one, the parties make balance payment under
      Annex 6- Rules of determination of the balance payment after
      measuring and evaluation of the result (see 1.2.12), to the
      Specific terms.
    \end{enumerate}
  \item The Contractor shall issue an invoice for the amount under
    art. 76 subart. (xv), as well as invoices for the first amounts
    due under art. 76 subart. (xvi) and (xvii) within 10 (ten)
    business days from the date of commissioning, while every
    following invoice for the sums due under art. 76 subart. (xvi) and
    (xvii) within ten days from the beginning of every month.
  \item The Client shall pay the sums due to the Contractor, according
    to the Specific terms of this contract.
  \item The Parties agree that the specific sum due by every owner of
    an individual apartment in the Building, which includes the
    components under art. 76, shall be determined, according to the
    rules of Annex 7 {-} Amortization plan and rules of determination of
    the sums due by every owner of individual apartment (see 1.2.13),
    to the Specific terms of the contract. In the same way shall be
    calculated the specific sum, of every owner of an apartment,
    representing penalty, indemnity, or other sums due to the
    Contractor under this contract or sums due by the Contractor to
    the Client.
  \item In case of decision of the Condominium general meeting, as
    well as in case of consent of the Contractor, specific authorities
    to the Manager can be assigned to collect the sums due to the
    Contractor, implement the communication with the individual owners
    of individual apartments, as well as other rights and obligations.
  \end{enumerate}
      \subsection{TERM OF THE AGREEMENT}
      \begin{enumerate}
      \item This contract comes into force from the date of it being
        signed by the parties. The term for implementation of the
        contract includes a term for implementation of Stage 1, and
        term for implementation of Stage 2 as described below.
      \item Stage 1 starts from the signing of this contract and
        includes the time, necessary for the design of the measures,
        and the time of the construction and installation period, and
        finishes with commissioning of the measures.
      \item Stage 2 starts from the date of commissioning and includes
        the monitoring period. During Stage 2 servicing of the
        measures, payment of the sums due to the Contractor, and
        accounting the achieved result shall also be implemented.
      \item The term of implementation of Stage 1 includes:
        \begin{enumerate}
        \item Design of the measures.
        \item Construction and installation works.
        \item Removing of all defects, deficiencies, and inadequacies,
          including on the documentation, identified and specified as
          remarks on the implementation in the process of supervision,
          testing, until the signing of a Act Form 15 of Regulation No
          3 from 2003 on drawing up acts and protocols during the
          construction.
        \item Commissioning pursuant to Regulation No. 2 from 2003 on
          authorising the use of constructions in the Republic of
          Bulgaria and minimum guarantee terms for implemented
          construction and installation works, equipment, and
          construction projects.
        \end{enumerate}
      \item The term for implementation of Stage 1 can be extended on
        mutual agreement of the parties.
      \item The term for implementation of Stage 1 shall be extended
        in the following cases:
        \begin{enumerate}
        \item Delay in the process of coordination and approval of the
          investment project with the administrative body, authorised
          to issue construction license. In such case, the period
          shall be extended with the days of delay of the competent
          body.
        \item Client`s delay in providing of the necessary
          information, documentation and access to the Building, or
          the individual apartments in it, or insufficient support by
          the Client for coordination of the project documentation or
          signing of records and statements and/or other support,
          necessary. In such case, the term shall be extended with
          the days of delay of the Client.
        \item Delay in the term of commissioning of the measures, in
          this case the delay shall be extended with the days of delay
          of the competent body.
        \end{enumerate}
      \item The terms for implementation shall also be extended in the
        cases of force majeure.
      \item If the Client does not appropriately support the
        Contractor in implementation of the measures, for each year of
        delay, the parties shall consider the guaranteed result
        achieved with all consequences arising from it.
      \end{enumerate}

      \subsection{LATENT WORKS}
      \begin{enumerate}
      \item If during the design of the measures and/or the
        construction period, the Contractor becomes aware that latent
        circumstances concerning the condition of the building, which
        could entail additional works and could influence the
        implementation of the measures are available, the Contractor
        shall notify the Client within five business days from
        becoming aware of the circumstances. The notification under
        the previous sentence contains at least the following:
        \begin{enumerate}
        \item List of the latent works identified and how they differ
          from the normal condition of the building, which could be
          reasonably foreseen by an experienced Contractor, at the
          date of signing of this contract.
        \item Additional works and additional resources, which the
          Contractor deems necessary for overcoming the latent works.
        \item Time estimated as necessary for the Contractor to
          overcome the Latent works and the expected delay in
          completion of the construction works.
        \item The estimated price of the measures, necessary for
          overcoming the Latent works, and
        \item Other information at the discretion of the Contractor.
        \end{enumerate}
      \item The delay of the Contractor, caused by such Latent works
        results in extension of the period of construction if:
        \begin{enumerate}
        \item The Latent works require the implementation of
          additional activities or input of supplemental materials, or
        \item The Latent works entail additional costs.
        \end{enumerate}
      \item The additional costs, arising from overcoming of Latent
        works, shall be at the expense of the Client.
      \end{enumerate}

      \subsection{STAGE 2 {-} MONITORING OF THE BUILDING ENERGY
        CONSUMPTION, MEASUREMENT AND EVALUATION OF THE RESULT}
      \begin{enumerate}
      \item The evaluation of the contract implementation is carried
        out during the monitoring period in line with the Method of
        guaranteed result evaluation, Annex 4 – Method of guaranteed
        result evaluation (see 1.2.10) to the Specific terms.
      \item During the monitoring period, the Client shall pay to the
        Contractor the amounts due under this contract.
      \item The monitoring shall be carried out by the Contractor and
        shall include verification of the energy-saving effect
        achieved as a result of the implementation of the measures, as
        well as verification of the compliance with the conditions,
        under which the guaranteed energy-savings are achieved. The
        rules of monitoring, including for carrying out of the current
        verifications as a part of its scope, are contained in Annex 4
        – Method of guaranteed result evaluation (see 1.2.10).
        \begin{enumerate}
        \item For the implementation of the monitoring, the Contractor
          shall be entitled to install measuring devices in the
          Building. Also, he shall be allowed to access all measuring
          instruments, included but not only for measuring the
          consumed quantities of electrical and thermal energy,
          including for water heating, located in the individual
          apartments / or common parts of the Building, as well as he
          shall be entitled to install, operate, service and introduce
          an energy management system and have access to the installed
          equipment at any time. The Contractor is entitled to getting
          information in real-time of measured quantities of consumed
          energy in the Building and its individual apartments.
        \item The Contractor is entitled at any time to install
          sensors for measuring the temperature in the common parts
          and/or the individual apartments in the Building. If the
          owners of individual apartments in the Building do not agree
          to be installed sensors or do not allow access to them, the
          Contractor is not liable of not compliance with the comfort
          standards and not the implementation of this contract.
        \item The data collected from the measuring tools are
          informative for the Contractor.
        \end{enumerate}
      \item If the Contractor finds that at a given moment the
        consumed quantities of energy considerably exceed the measured
        ones at a prior moment and there is a probable leak or another
        event which could negatively influence the guaranteed result,
        the Contractor gives instructions to the Client, which are
        binding on the latter. If the Client does not fulfil the
        instructions given and/or does not forthwith provide access of
        the Contractor to the individual apartment in the Building or
        its common parts, the guaranteed result for the respective
        year shall be considered as achieved with all stemming
        consequences from it.
      \item Until the 31-st of January each year, the Contractor shall
        prepare a monitoring report for the previous calendar year and
        send it to the Client. The report can also be submitted to the
        Client via EPC platform. The report contains all data,
        according to the Method of guaranteed result evaluation in
        Annex 4 – Method of guaranteed result evaluation (see 1.2.10)
        to the contract. The last monitoring report shall include the
        period from January of the last year to the end of the
        contract term.
      \item In case that the Client does not agree on the annual
        monitoring report, including on the reported on results, he
        shall notify the Contractor within 15 (fifteen) business days
        from receipt of the report, explaining the reasons for
        this. If he does not do it, the report shall be considered as
        accepted. In case of objections by the Client, the Contractor
        shall make changes (if any necessary), report and notify the
        Client of this within 15 business days from receipt of the
        objections.
      \item In case of dispute concerning the achieved results and
        respectively achievement or non-achievement of the guaranteed
        result under the contract, the Parties can assign an
        independent organisation to verify the results, entered in the
        annual monitoring report, according to the Method of
        guaranteed result evaluation. The costs for the verification
        by an independent organisation shall be at the expense of:
        \begin{enumerate}
        \item The Client {-} in case the results in the report are
          correctly calculated, and the verification is unjustified.
        \item Of the Contractor – in the case that the verification
          finds substantial errors and discrepancies in the data
          entered in the report.
        \end{enumerate}
      \item In the annual monitoring report, the Contractor also
        calculates the EC, according to the Method of guaranteed
        result evaluation.
      \item If in the first six months from commencement of the
        monitoring stage, the Contractor finds a tendency of
        non-achievement of the energy-savings guaranteed within the
        year, he shall do an expert evaluation of the risks and give
        instructions, including he shall determine a term and period
        for their implementations. The instructions of the Contractor
        are binding on the Client.
      \item All notifications between the Parties on the monitoring
        report, their sending, the Client’s objections, and others can
        also be delivered via the EPC platform.
      \item Every unregulated or uncoordinated action by the Client on
        the implemented measures, measuring devices, and any other act
        of the Client, which is not preliminary coordinated with the
        Contractor, which reduces the energy-saving level, shall not
        be taken into consideration. The guaranteed result shall be
        considered as achieved with all consequences stemming from it.
      \item The Client shall confirm and agree on the use by the
        Contractor or his Subcontractor, and by a person to whom the
        receivables of the Contractor under this contract can be ceded
        and/or assigned for collection, as well as to persons carrying
        out the implementation, operation and management of the EPC
        platform of:
        \begin{enumerate}
        \item Any data and information, connected with the energy
          consumption in the Building, notwithstanding whether they
          are provided by the Client or received by the Contractor,
          for comparison and drawing up of national, regional or
          international database or for use by the Contractor as a
          reference or any other purpose, agreed on with the Client.
        \item Personal data, provided by the Client or third person,
          authorized by the Client, for the purpose of provision of
          services under this Agreement.
        \end{enumerate}
      \end{enumerate}

      \subsection{DISPUTE RESOLUTION PROCEDURE}
      \begin{enumerate}
      \item Any disagreements between the Parties shall first be
        negotiated.
        \begin{enumerate}
        \item If a Client (owner of the apartment) has complaints
          about the Contractor (including on the comfort standards or
          energy-saving, or in general on the implemented measures),
          the latter shall directly or via the Manager of the
          condominium notify the Contractor thereof. The Contractor
          shall verify the admissibility of the objection and remedy
          the problem that has occurred (if any).
        \item If the Contractor has complaints about the Client
          (including for damaging the installed equipment), he shall
          notify the Client (owner of an apartment) directly or via
          the Manager of the condominium. The Client shall verify the
          admissibility of the objection and remedy the problem
          arisen.
        \end{enumerate}
      \item If the problem persists for more than 20 business days
        from the notifications under art. 107 subart. (xlii) and
        (xliii), any Party is entitled to arrange a committee,
        comprised of one representative of the Client, one
        representative of the Contractor, one representative of the
        Manager of the condominium, and one representative of an
        auditing company, licensed by the Sustainable Energy
        Development Agency (SEDA). The costs for the committee shall
        be equally divided between the parties. The committee shall
        draw up a Statement of findings.
      \item For the procedure of fact-finding, applicable to case of
        disputes, pertaining to facts, and their resolution under
        art.s 107 and 108, the following shall be applied:
        \begin{enumerate}
        \item The actual Comfort Standards shall be considered duly
          recorded if the temperature measurements are conducted by an
          independent energy auditor, licensed by SEDA.
        \item General problems with the implemented measures, for
          example, malfunctioning equipment and/or defect and damages
          made to the measures, and/or with the calculation of the
          Energy-Saving, shall be considered duly recorded, if
          conducted by an independent expert like a certified energy
          auditor, licensed by SEDA.
        \item All parties shall be notified at least 5 (five) working
          days before any measurement, made by a third party. An
          authorised representative of the Parties has the right to
          participate in the measurement for the preparation of the
          Statement of findings.
        \item The signing of the Statement of findings by any of the
          Parties shall not be considered as an acknowledgement of a
          breach under this Agreement and/or shall not be deemed a
          waiver of any of the Parties’ rights obligations hereunder,
          but shall have a binding action for the Parties in case of
          litigation.
        \end{enumerate}
      \item Notwithstanding the implementation or non-implementation
        of the procedure under art.  107, 108 and 109 and/or in case
        non-agreement, the disputes between the parties under this
        contract, including the disputes arisen or concerning the
        interpretation, invalidity, non-implementation or termination
        of the agreement, shall be decided by a competent Bulgarian
        court, under the Civil Code of Procedure, in compliance with
        the existing legislation of the Republic of Bulgaria.
      \end{enumerate}

      \subsection{MAINTENANCE OF MEASURES IMPLEMENTED BY THE
        CONTRACTOR. GUARANTEE}
      \begin{enumerate}
      \item The Contractor shall provide servicing and technical
        maintenance of ESM, including equipment and fittings mounted,
        resulting from the contract implementation for the contractual
        term.
      \item The Contractor shall implement maintenance of ESM with a
        view to their optimal functioning and achievement of the
        guaranteed result.
      \item If necessary, the Contractor shall give obligatory
        prescriptions to the Client on appropriate operation.
      \item The guarantee terms and the conditions for components and
        equipment are according to the term of the producers’
        commercial guarantee, and the implemented construction and
        installation works (CIW) {-} in line with Regulation No. 2/2003.
      \item Within the guarantee terms, the Contractor shall remove at
        his own expense any defect on any ESM implemented by him, save
        the defects are caused or arisen in consequence of a force
        majeure, actions and/or inactions of third non-authorised
        persons, as well as in the cases of wrong operation of the
        assembled and modernised equipment and installation, then the
        Contractor removes the defect at the Client’s expense. Out of
        the guarantee terms and until the end of the monitoring
        period, the defects shall be removed by the Contractor but on
        the Client’s expense.
      \item When carrying out the activity under art. 111 the
        Contractor shall comply with the manufacturer’s requirements
        and recommendations for the respective maintenance.
      \item The Client shall be bound within 2-days, in written form,
        to notify the Contractor about the defects arisen. The claim
        must contain a detailed and justified description of the case
        and the current requests of the Client. In emergency cases,
        the Client shall follow to note in his notification this
        circumstance explicitly.
      \item The claim shall be sent to the Contractor, along with an
        invitation for audit in the presence of his representative on
        one of the following two days from receipt of the
        invitation. In emergency cases, it is sent at the latest until
        13:00 H on the next day from receipt of the notification by
        the Contractor.
      \item After the audit, it is obligatory to draw up a statement
        of findings, signed by the Parties. In case of non-appearance
        of the Contractor and/or of the Client, the statement shall be
        signed without him, enclosing a written invitation and
        evidence of its receipt.
      \item The Contractor removes the defects in the minimum
        necessary and possible lead time.
      \item In case that Contractor does not remove the defect under
        art. 120, if the fault must be removed at the Contractor’s
        expense in line with the provisions of this Agreement, the
        Client is entitled to assign to a third person the removal of
        the defect at the Contractor’s cost.
      \end{enumerate}

      \subsection{INSURANCE}
      \begin{enumerate}
      \item The Contractor is obliged to insure the Building at his
        expense against natural disasters, fire, flood, destroying,
        damage or vandalism for the construction time, and the Client
        is obliged to approve under art. 402 of the Insurance
        Code. For this insurance shall be applied the following
        provisions:
        \begin{enumerate}
        \item The Contractor shall conclude the insurance with an
          insurer at minimum rating A+ in line with the relevant
          ratings, applicable for the Republic of Bulgaria.
        \item The Contractor shall provide a copy from the insurance
          policy and the documents, confirming the payment of the
          Client’s insurance premium before the commencement date of
          the construction period.
        \item The Client is cited as a beneficiary of the indemnity on
          the insurance policy.
        \item The construction works in the Building start only when
          the Contractor provides a validly concluded insurance
          policy.
        \item After the end of the construction period, all cost for
          maintenance of the insurance and for payment of the premiums
          shall be at the Client’s expense, the costs of the insurance
          premiums shall be divided between the owners of individual
          apartments, respectively to the possessed by them shares
          from the common parts of the Building (save as otherwise
          agreed on).
        \item The Contractor is also obliged to maintain professional
          liability insurance under art. 171 of the Spatial
          Development Act, which includes all activities from the
          contract subject and is valid at least until commissioning
          of the measures.
        \end{enumerate}
      \end{enumerate}

      \subsection{CESSION OF RIGHTS}
      \begin{enumerate}
      \item The Contractor is entitled to transfer (cede) his
        receivables from the Client under art. 99 from the Obligations
        and Contract Law, as well as to assign to a third person to
        collect the receivables of the Contractor from the Client. The
        transfer or the assignment of collection shall not release the
        Contractor of his obligations under this contract.
      \item In case of transfer, the Contractor, or a person empowered
        by him, shall notify the Client within 5 (five) business days
        from the conclusion of the cession contract.
      \item The Client can transfer his rights under this Agreement
        only with the prior consent of the Contractor.
      \item The transfer of rights is not allowed, save a preliminary
        agreement between the parties. Regardless of that, a third
        person can fulfil the obligations of a Party at its expense,
        and if the implementation is correct, the other Party must
        accept it.
      \end{enumerate}

      \subsection{RIGHT OF PROPERTY ON INPUT MATERIALS AND IMPLEMENTED
        WORKS}
      \begin{enumerate}
      \item The ownership on the input materials, installations and
        equipment in case of implementation of the packet of ESM,
        which are permanently attached to the Building, and cannot be
        removed from it without causing damages, shall pass to the
        Client from the moment of their input in the apartment, from
        this moment the Client shall bear the risk of their accidental
        loss or damage.
      \item The ownership on the input materials, installations and
        equipment in case of implementation of the packet of ESM,
        which are not permanently attached to the Building remains for
        the Contractor until the expiry of the contract term and
        performance of all obligations from the Client to the
        Contractor.
      \item The Client must not remove, destroy, damage, place
        constraints, including the creation of a pledge on the input
        materials, installations and equipment in case of
        implementation of ESM, regardless which party is a holder of
        the property right under art. 127 and 128.
      \item The Contractor is entitled, without the approval of the
        Client (It is not necessary to obtain the consent of every
        owner of the individual apartment), to place constraints on
        the input materials, installations, equipment in case of
        implementation of the packet of ESM, including through the
        creation of a pledge on them, if the Contractor has the right
        of ownership under art. 128. The term for which is created the
        deposit is no longer than the monitoring period.
      \item The right of property on the input materials,
        installations, and equipment in case of implementation of the
        packet of ESM under art. 128 shall pass on the Client upon
        expiry of the contract term and provided that all amounts due
        to the Contractor are paid.
      \end{enumerate}

      \subsection{RIGHTS ON INTELLECTUAL PROPERTY}
      \begin{enumerate}
      \item The Contractor shall be a holder of the copyrights and
        similar rights, and other rights of intellectual property on
        the project documents for ESM, prepared by the Contractor. The
        Contractor is a holder of the respective copyrights and other
        rights of intellectual property on patents, useful models,
        industrial designs, know-how, used or for the construction,
        installation, operation, maintenance, monitoring, and use of
        ESM.
      \item The Contractor, under this provision, shall entitle the
        Client to use the rights on the intellectual property, related
        to ESM, as well as support him to obtain equal rights from a
        third person, when he is the bearer of the rights on the
        intellectual property.
      \item The Contractor, under this provision, shall entitle the
        Client to reproduce, use, and submit for information project
        designs, instructions, use and maintenance manuals, drawings
        and installation data, licenses, software codes, and others.
      \item The above-described rights shall be submitted for the
        entire operation term of ESM, including upon the termination
        or expiry of this contract, only for monitoring, operation,
        and maintenance of implemented ESM.
      \item The Contractor is obliged to indemnify the Client, if the
        latter suffers damages, due to infringement of another’s
        patent, industrial design, useful model, copyrights or
        similar, or another protected object of intellectual property,
        related to the design, construction, installation, operation,
        maintenance, monitoring or use of ESM. The Contractor’s
        obligation to indemnify the Client depends on the following
        duties of the Client:
      \item To notify forthwith in written form the Contractor of the
        claim of the third person.
      \item To not impede the Contractor to take protective actions
        against the claim of the third person, as well as to reach an
        agreement with him.
      \item To allow the Contractor to negotiate for the settlement of
        the relations with the third person, and
      \item To provide the Contractor with (at the expense of the
        Contractor) any support and information, in view to help the
        Contractor in carrying out the defence, and all negotiations
        for settlement of the relations with the third person.  \ The
        Contractor, on his choice, shall either replace or change the
        part, which infringes the rights of the third person or
        provide the Client with the right to use this part.
      \end{enumerate}

    \subsection{CHANGES IN THE PURPOSE AND USE OF THE BUILDING AND/OR
      INDIVIDUAL APARTMENTS IN IT}
    \begin{enumerate}
    \item The purpose and the condition of the Building are described
      in Annex 1 – Report and Summary on Building EE Audit (see 1.2.7)
      to the specific terms of this contract, including its area and
      dimensions. Every change in the purpose or the area, including
      the heating area, of the Building and/or the individual
      apartments in it (for example change of purpose of an apartment
      into an office, a shop, an establishment, and others), as well
      as any other change of the circumstances, on which are based the
      calculations for energy consumption and energy-saving effect of
      the measures, installation of new energy consumers, living in
      individual apartments, which until this moment have not been
      occupied, and others, which change is initiated by the Client,
      shall not influence this Agreement. Every change shall be
      considered undone in reference to the guaranteed result, which
      shall be deemed as achieved, save there is a preliminary
      approval by the Contractor, objectivated in an additional
      agreement to this contract and, if deemed necessary, with an
      adjusted energy-saving guarantee.
    \end{enumerate}

    \subsection{DISPOSAL OF WASTE}
    \begin{enumerate}
    \item The Contractor shall clean the Building from construction
      waste and other waste, which are the consequence of the
      construction and installation works, as well as to remove all
      materials and equipment, which are unnecessary.
    \item The Contractor shall notify the Client before the planned
      activity for removing the materials and equipment.
    \end{enumerate}

    \subsection{SANCTIONS AND RESPONSIBILITIES}
    \begin{enumerate}
    \item The Contractor shall be liable for the timely implementation
      of the measures in the construction period. In case of delay in
      the performance, which can be imputed to the Contractor, the
      Client is entitled to obtain penalty to the amount of 0.02\% per
      day on the price for the implementation of ESM, VAT excluded,
      under art. 1.2.5, subart. 193 (cli) from the specific terms of
      this contract but no more than 10\% of it.
    \item If the Client does not provide the necessary support to the
      Contractor to start the work on Stage 1, and/or the project is
      not commissioned within the term for implementation of Stage 1,
      stemming from the Client’s acts /including but not only and if
      the Client refuses without any grounds to sign acts and
      protocols, drawn up during the construction/, the same shall owe
      payment of penalty to the amount of 0.02\% per day on the price
      for the implementation of ESM, VAT excluded, under art. 1.2.5,
      subart. 193 (cli) from the specific terms of this contract but
      no more than 10\% on it.
    \item The Client shall be liable for the timely payment of the
      amounts due under this contract. In case of delay, the Client
      shall owe a penalty to the amount of 0.1\% per day on the amount
      overdue.
    \item If the Client is unable to make the payment due for more
      than 90 (ninety) days, The Contractor is entitled to break this
      contract because of the Client’s default or declare the total
      price due under art. 1.2.5, subart. 193 (cli) as an early
      expected price.
    \item The Contractor shall comply with the comfort standards,
      determined in this contract. If during the heating season the
      temperature in an individual apartment has been on the average 2
      (two) degrees Celsius (taking into consideration the precision
      of the tool) under the comfort standards, established in Annex 3
      – Comfort Standards (see 1.2.9), specific terms, the Contractor
      shall owe the Client a sum, as follows:
      \begin{enumerate}
      \item [●] for each degree Celsius for every month from the
        heating season, when the temperature has been under the
        agreed-on comfort standards.
      \item The Contractor shall not apply the discount on the
        previous item if the reduction in the temperature: (a) results
        from the actions or inactions of the occupants or the owners
        of the apartment in breach of this agreement; (b) results from
        non-implementation of the Client’s obligations, or (c) results
        from other reasons, which are not Contractor’s fault.
      \end{enumerate}
    \item The Client shall be liable of damages, manipulations, thefts
      and any other illegal activities on the measures, including but
      not only if this affects the energy-saving level, the agreed-on
      comfort standards or safety of the persons, living and using the
      Building. In this case, the Client must:
      \begin{enumerate}
      \item Pay the Contractor the costs for the recovery of the
        respective measure.
      \item Pay an additional indemnity to the Contractor to the
        amount of 10\% (ten per cent) on the recovery costs.
      \item The recovery costs shall be calculated on the base of the
        applicable market prices, by the moment of calculation.
      \end{enumerate}
    \item The Client shall not be responsible when during the
      implementation of the measures, the Contractor infringes the
      intellectual property rights of third persons. The Contractor
      shall pay to the Client all costs for recovery of the direct
      damages, resulting from the Contractor’s actions, taken in
      infringement of the rights. The recovery shall be due within 30
      working days from receipt of the respective claim from the
      Client to the Contractor, explicitly mentioning the sum due.
    \item The payment of a penalty or other indemnity shall not
      release the Party from its obligation to implement this
      contract.
    \item Each no-adherence to the Client’s obligations, resulting in
      the impossibility of achievement and/or exact reading of the
      energy-saving achieved, according to the Method of guaranteed
      result evaluation, enclosed to the contract, the Parties shall
      accept that the guaranteed result is reached with all
      consequences stemming from it.
    \end{enumerate}

    \subsection{TERMINATION OF THE CONTRACT}
    \begin{enumerate}
    \item In case of non-implementation of this contract, the
      non-defaulting Party shall have the right to break the contract,
      sending a 30-days notice (save as another term is prescribed
      under the provisions of this contract).
    \item If the contract is broken before the start of the
      construction period:
      \begin{enumerate}
      \item In case of unilateral termination of the Contract by the
        Client, due to substantial default or breach of the Contract
        by the Contractor, the Client is entitled to indemnity to the
        amount of 1\% on the sum, representing the price for the
        implementation of ESM, VAT excluded, under art. 1.2.5,
        subart. 193 (cli) from the specific terms of the contract.
      \item In case of unilateral termination of the Contract by the
        Contractor due to default or breach of the Contract by the
        Client, the Contractor is entitled to indemnity to the amount
        of 1\% on the sum, representing the price for the
        implementation of ESM, VAT excluded, under art.  1.2.5,
        subart. 193 (cli) from the specific terms of the Contract.
      \end{enumerate}
    \item Termination of the Contract after full implementation of the
      measures and performance of the construction works:
      \begin{enumerate}
      \item In case of unilateral termination of the Contract by the
        Client, due to substantial default or breach of the
        Contractor, the Client recover to the Contractor the value of
        the sum, representing a price for the implementation of ESM
        under art. 1.2.5, subart. 193 (cli) from the specific terms of
        the Contract, reduced with 3\%. In addition to the
        above-mentioned, the Client is entitled to receive the
        complete documentation on the project, representing in detail
        the implemented work by this time, along with all permits,
        licenses or other documents, received by the Contractor in
        line with this Agreement and any right for the use of software
        (including installed software and when applicable, any
        accompanying documentation, information, related to code,
        every source code, files with data, calculations, electronic
        carriers, prints out or related to information).
      \item In case of unilateral termination of the Contract by the
        Contractor, due to substantial default or breach of the
        Contract by the Client, also including non-payment of due
        sums, the Client recovers to the Contractor the value of the
        sum, representing a price for the implementation of ESM under
        art. 1.2.5, subart. 193 (cli) from the specific terms of the
        Contract, increased with an allowance of 3\%. The Client is
        entitled to receive the full documentation on the project,
        representing in detail implemented work until this time, along
        with all permits, licenses or other documents, obtained by the
        Contractor in line with this Agreement and any rights for the
        use of software (including installed software and, when
        applicable, any accompanying documentation, information
        related to code, every source code, files with data,
        calculations, electronic carriers, prints out or related to
        the data).
      \item In case of termination of the contract before the
        completion of Stage 1, but if the Contractor has implemented a
        part of the construction works, the Client is obliged to pay
        to the Contractor a part of the sum, representing a price for
        the implementation of ESM under art.  1.2.5, subart. 193 (cli)
        from the specific terms of the Contract, corresponding to the
        implemented one, increased or reduced with 3\% under art. 152
        subart. (lxv) and (lxvi) – depending on that which part is
        faulty for the termination.
      \item In case of breach of the Client’s obligations, resulting
        in impossibility to achieve and/or strictly account the
        achieved saving, according to the enclosed to the Contract
        Method / Annex 4 – Method of guaranteed result evaluation (see
        1.2.10), the Parties shall accept, that the guaranteed result
        is achieved with all consequences stemming from it under the
        Contract.
      \end{enumerate}
    \item The termination of this Contract shall not release the
      Parties from the implementation of the obligations arisen before
      the moment of the ending, save as the contract foresees anything
      else. The unilateral termination of the Contract by the Client
      in case of a substantial breach of the Contract by the
      Contractor does not release the Client from the obligation to
      pay the invoices, issued for the period preceding the date of
      the contract termination.
    \item The Parties can terminate the Agreement at any time upon
      mutual written agreement.
    \item The share of indemnities and other penalties and sums, due
      by the Contractor to the Client for every owner of an individual
      apartment in the Building shall be determined by the rules,
      according to Annex 7 – Amortisation plan and rules of
      determination of the sums due by every owner of individual
      apartment (see 1.2.13) to the specific terms.
    \end{enumerate}

    \subsection{FORCE MAJEURE}
    \begin{enumerate}
    \item The parties shall not be liable of the complete or partial
      default under this Contract if they are due to force
      majeure. The term for implementation of the obligation shall be
      extended with the period in which the performance was suspended
      by a force majeure event. The provision shall not affect the
      rights or obligations of the Parties, which were arisen and due
      before occurring of the force majeure.
    \item The Party, affected by the force majeure, shall within 3
      working days after the establishment of the event, notify the
      other Party and provide it with evidence for the appearance,
      nature and dimension of the force majeure event and assessment
      of its possible consequences and length. To the notification
      shall be enclosed all relevant and/or regulatory established
      evidence for occurring and nature of the force majeure and the
      causal connection between this circumstance and the
      impossibility for implementation.
    \item During the force majeure, the implementation of obligations
      and connected with them counter obligations shall be suspended,
      and the contract term shall be extended with the time of the
      force majeure.
    \item When the force majeure continues more than six consecutive
      months and is not expected to terminate for three months more,
      each one of the Party can require termination of the Contract.
    \end{enumerate}

    \subsection{CONFIDENTIALITY}
    \begin{enumerate}
    \item The Contractor and the Client shall accept as confidential
      any information, obtained during or in connection with the
      implementation of this Contract.
    \item The information fallen into the public space by third
      Parties, when the Parties do not breach the provisions of the
      Agreement, shall not be considered as confidential.
    \item The Parties must not provide third Parties with documents
      and/or information on the Contract implementation without the
      explicit written consent of the other Party, as well as to not
      disclose information, of which they are aware at or in
      connection with the Contract implementation, for the period of
      validity of the Contract, as well as within 3 years after its
      termination. The Contractor is obliged to ensure compliance with
      this obligation also by his employees.
    \item The Parties can disclose the confidential information to
      third parties for the implementation of the obligations on this
      Contract. If the Parties provide confidential information, based
      on this provision, they guarantee that the Third Party shall
      observe the same commitments for confidentiality, as established
      in this agreement.
    \item The disclosure of confidential information, required in
      compliance with the existing laws and regulations in the
      Republic of Bulgaria, is not considered as a breach of the
      agreement.
    \item For publicity purpose and information of the public, the
      Contractor and the Client are entitled to disclose general
      information for cooperation, including data for the saving
      achieved and the energy consumption.
    \item The above provisions do not affect the right of the
      Contractor to collect, process, store and transfer to his
      assignees and financing partners/including to assignees/ and
      spread all collected data by the Client for the purposes of
      improving the quality of Contractor’s services and for
      development, operation and maintenance of the online EPC
      platform.
    \item The Client shall confirm and give his explicit consent that
      the Contractor can receive, store and process personal data of
      every owner of an individual apartment in the Building, as well
      as to provide a third person with such data, to whom can be
      assigned rights and obligations, stemming from this Agreement,
      including to a person to whom the receivables of the Contractor,
      arising from this Contract, can be transferred and/or assigned
      for collection, or to a third person who manages or is liable of
      development, application, operation and maintenance of the EPC
      platform and/or to the Manager of the Condominium.
    \end{enumerate}

    \subsection{CONCLUSION AND AMENDMENT OF THE PRESENT AGREEMENT}
    \begin{enumerate}
    \item The Contract shall come into force on the day when it is
      being signed by all Parties under the General terms and
      conditions.
    \item All changes and supplements to the agreement shall be
      fulfilled in written form on mutual agreement of the Parties.
    \item If during the term of the Contract occurs a legislative
      change, which makes the implementation of the obligations under
      this agreement fully or partially impossible, this does not
      affect the validity of the remaining obligations of the parties
      of this contract. The parties sign an additional agreement,
      covering the necessary changes.
    \item The nullity of separate parts does not involve the nullity
      of the entire contract.
    \item Save as otherwise mentioned, the references to the separate
      items and/or articles are to the respective part, where is the
      article and/or the item.
    \end{enumerate}

      \subsection{REPRESENTATION OF THE PARTIES}
      \begin{enumerate}
      \item At the signing of the present Contract, the Parties are
        represented by their legitimate representatives (for legal
        persons) or explicitly authorised persons. Only the persons
        mentioned in the specific terms of this agreement are entitled
        to represent the Client or the Contractor.
      \item With the signing of this Contract the Parties certify that
        they understand the content, meaning, and consequences of the
        agreement; acknowledge that this contract is to their mutual
        benefit, voluntarily want to implement it, without exercising
        any violence against the will of any of them.
      \end{enumerate}

    \subsection{ADDRESSES FOR CORRESPONDENCE}
    \begin{enumerate}
    \item All messages and notifications between the parties, related
      to the implementation of this contract shall be made up in
      written form, signed by their authorised representatives. Any
      notices and information can also be sent via email and via the
      EPC platform.

    \item The addresses for correspondence of the parties under this
      contract are as follows:
    \item In the case of receiving a notification: all notifications
      come into force
      \begin{enumerate}
      \item Upon receipt of the party to which have been the
        notification, or
      \item On the 7-th (seventh) day after its sending, which of both
        occurs first.
      \end{enumerate}

    \end{enumerate}
    The Parties on this contract agree on that in case of arising of
    circumstances which they could not foresee by the moment of
    conclusion of the contract or by the moment of preparing and
    coordination of the Annexes to it, which could negatively
    influence the final result, each one of the party could require
    respective adaptation of this contract. The parties are obliged to
    act in good faith and exclusively in the interest of reaching the
    purposes of the contract.


  \end{multicols}
