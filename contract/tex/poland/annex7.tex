\section{ANEKS 7: OPŁATA ZA RENOWACJĘ}

\begin{enumerate}[label=\arabic*.]
	\item Klient zgadza się na Wkład finansowy Wykonawcy, ponieważ wniesienie go jest niezbędne dla:
	\begin{enumerate}
		\item udanego wdrożenia Środków,
		\item wyegzekwowania Gwarancji oszczędności energii udzielanej przez Wykonawcę,
		\item wyświadczenia Usług poprawy efektywności energetycznej z należytą jakością.
	\end{enumerate}

	\item Wykonawca zgadza się wnieść na rzecz Klienta Wkład finansowy, który przeznaczony jest do wykorzystania wyłącznie na potrzeby Umowy.
	\item Klient zwróci Wykonawcy Wkład finansowy wniesiony na mocy niniejszej Umowy w ramach płatności Opłaty za renowację, na następujących warunkach:
	\begin{enumerate}
		\item Wkład finansowy Wykonawcy (z VAT): \iffalse input fields.contractor_fin_contribution value="{{.Contract.Fields.contractor_fin_contribution}}" \fi {{.Contract.Fields.contractor_fin_contribution}} EUR
		\item Zmienna stopa oprocentowania:
			\begin{itemize}
				\item obejmująca komponent stały: \iffalse input fields.interest_rate_percent value="{{.Contract.Fields.interest_rate_percent}}" \fi {{.Contract.Fields.interest_rate_percent}} \% {-} kredytowanie oferowane przez spółkę \iffalse input fields.interest_rate_offerter value="{{.Contract.Fields.interest_rate_offerter}}" \fi {{.Contract.Fields.interest_rate_offerter}}
				\item i komponent zmienny: {{.EUROBOR}} miesięczna stopa EURIBOR
			\end{itemize}
		\item Okres spłaty jest równy Okresowi świadczenia usług: {{mul 12 .Project.ContractTerm}} miesięcy.
	\end{enumerate}

	\item Stała stopa oprocentowania to stopa roczna uzgodniona przez Strony.
	\item Zmienna stopa oprocentowania obejmuje komponent stały i komponent zmienny:
	\begin{enumerate}
		\item Komponent stały jest częścią Stopy oprocentowania będącą roczną stopą oprocentowania uzgodnioną przez Strony, która może zostać zmieniona za pisemnym porozumieniem Stron.
		\item Komponent zmienny jest oparty na aktualnej stopie procentowej EURIBOR.\@ Stopa procentowa EURIBOR to uśredniona stopa oprocentowania przyjęta przez banki z Eurostrefy dla danego okresu, wyliczana przez „Thomson Reuters” z datą rozliczenia EUR, publikowana na stronie http://www.euribor-ebf.eu.
	\end{enumerate}
	\item Komponent zmienny Stopy oprocentowania za pierwszy okres zostanie określona przez Wykonawcę w Dniu zakończenia świadczenia usług poprzez zastosowanie stopy opublikowanej w poprzednim dniu bankowym.
	\item Komponent zmienny Stopy oprocentowania na kolejny okres zostanie określony przez Wykonawcę w jednym z następujących terminów (z uwzględnieniem tego, który jest najbardziej oddalony od Dnia zakończenia świadczenia usług, przy czym nie późniejszy niż 6 lub 12 miesięcy po upływie Dnia zakończenia świadczenia usług): 15 stycznia lub 15 kwietnia lub 15 lipca lub 15 października, poprzez zastosowanie stopy opublikowanej w poprzednim dniu bankowym.
	\item Komponent zmienny Stopy oprocentowania na kolejne okresy będzie określony przez Wykonawcę w dniu 15 stycznia i/lub 15 kwietnia i/lub 15 lipca i/lub 15 października, z uwzględnieniem Okresu oprocentowania zmiennego.
	\item Jeśli komponent zmienny Stopy oprocentowania jest wartością ujemną, to wówczas kwota Zmiennej stopy oprocentowania będzie równa komponentowi stałemu Stopy oprocentowania.
	\item Jeśli data zamiany (swap date) komponentu zmiennego Stopy oprocentowania przypadnie w dniu wolnym od pracy, to datą zamiany będzie następny dzień roboczy. Po ustaleniu nowego komponentu zmiennego Stopy oprocentowania ‘Wykonawca prześle Zarządcy Klienta wraz z aktualną fakturą zawiadomienie o zmianach w Harmonogramie płatności. Stopy oprocentowania na kolejny okres i Harmonogram płatności zostaną uznane za zmienione począwszy od pierwszego dnia nowego okresu bez zawierania umowy do Umowy.
	\item Naliczanie odsetek rozpocznie się na początku Okresu świadczenia usług, po podpisaniu oświadczenia o przeniesieniu / przekazaniu.
	\item Zgodnie z postanowieniami Umowy, Odsetki muszą zostać naliczone za każdy dzień kalendarzowy przy założeniu roku 360-dniowego.
	\item Harmonogram płatności Opłaty za renowację oparty o Wkład finansowy Wykonawcy i o	warunki obowiązujące w chwili podpisania niniejszej Umowy:

% table: project_measurements_table

\begin{center}
\begin{longtabu}{|X|X|X|X|X|} \tabucline{}
{{with $t := translate "pl" .Contract.Tables.project_measurements_table}}
	{{.Columns | column}} \\\tabucline{}
	{{range .Rows}} {{rowf $t .}} \\\tabucline{} {{end}} %chktex 26
{{end}}
\end{longtabu}
\end{center}

	\item Wykonawca zawiadomi Zarządcę ze strony Klienta i dostarczy zaktualizowany Harmonogram płatności po każdej zmianie komponentu zmiennego Zmiennej stopy oprocentowania.
	\item Wykonawca będzie co miesiąc wystawiał Zarządcy ze strony Klienta faktury za miesięczną Opłatę za renowację wyliczoną zgodnie z Harmonogramem płatności. Zarządca będzie wystawiał indywidualne faktury Właścicielom mieszkań w oparciu o zajmowaną powierzchnię.
\end{enumerate}
