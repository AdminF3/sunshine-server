\begin{multicols}{2}
[\section{WARUNKI OGÓLNE}]

\subsection{DEFINICJE}
\begin{itemize}[label={}]
	\item\textbf{Umowa:} Niniejsza Umowa o poprawę efektywności energetycznej zawarta pomiędzy Klientem i Wykonawcą, obejmująca Warunki szczególne, AneksyiWarunki ogólne, opracowana i zarządzana na Platformie umów o poprawę efektywności energetycznej (sunshineplatform.eu) sunshineplatform.eu
	\item\textbf{Mieszkanie:} Nieruchomość mieszkaniowa wydzielona prawne z budynku mieszkalnego
	\item\textbf{Właściciel mieszkania:} Osoba, która nabyła nieruchomość mieszkaniową i potwierdziła swe prawo własności wpisem do Rejestru Nieruchomości
	\item\textbf{Dzień bankowy:} Dzień, w banki komercyjne realizują ogólne transakcje bankowe w miejscowości, w której znajduje się bank mający dokonać przelewu zgodnie z Umową
	\item\textbf{Stan odniesienia:} Zużycie energii cieplnej i ciepłej wody użytkowej w budynku, wyrażone jako roczna wartość średnia w Okresie odniesienia
	\item\textbf{Okres odniesienia:} Uzgodniony okres reprezentujący funkcjonowanie Budynku przed wdrożeniem Środków.
	\item\textbf{Budynek:} Budynek mieszkalny wielorodzinny w którym Wykonawca realizuje Prace renowacyjne i świadczy Usługi określone w niniejszej Umowie.
	\item\textbf{Dzień roboczy:} Dzień niebędący oficjalnie dniem wolnym od pracy zgodnie z prawem łotewskim
	\item\textbf{Klient:} ĒWłaściciel mieszkania w Budynku lub jego upoważniony przedstawiciel
	\item\textbf{Standardy komfortu:} Zbiór warunków i parametrów klimatycznych w pomieszczeniach, których osiągnięcie Wykonawca gwarantuje Klientowi na mocy Umowy
	\item\textbf{Data rozpoczęcia:} Początek biegu Okresu prowadzenia Prac renowacyjnych
	\item\textbf{Data przekazania do użytkowania:} Data podpisania przez Strony Protokołu zdawczo-odbiorczego Środków i data początkowa Okresu świadczenia Usług
	\item\textbf{Okres prowadzenia Prac renowacyjnych:} Zaplanowany przez Wykonawcę okres potrzebny na wdrożenie Środków. Okres prowadzenia Prac renowacyjnych rozpoczyna się od Daty rozpoczęcia Prac renowacyjnych i kończy z Datą przekazania do użytkowania.
	\item\textbf{Wykonawca:}Osoba prawna podejmująca się wykonania Umowy i Prac renowacyjnych oraz świadczenia Usług w oparciu o postanowienia Umowy
	\item\textbf{Protokół zdawczo-odbiorczy:} Protokół sporządzony przez Wykonawcę zgodnie z uregulowaniami i normami łotewskimi, będący postawą ostatecznego przekazanie do użytkowania Środków wdrożonych w Budynku przez Wykonawcę
	\item\textbf{Opłata za ciepłą wodę użytkową:} Opłata płacona Wykonawcy przez Klienta, należna za rzeczywiste zużycie ciepłej wody użytkowej, naliczana według aktualnej Taryfy za energię cieplną
	\item\textbf{Energia:} Produkt o pewnej wartości – paliwo, energia cieplna, energia ze źródeł odnawialnych, energia elektryczna i każdy inny rodzaj energii
	\item\textbf{Audyt energetyczny:} Działania prowadzone w celu uzyskania informacji o charakterystyce zużycia energii w budynkach lub grupach budynków, procedurach lub wyposażeniu, a także w celu zmierzenia i zweryfikowania możliwości uzyskania w sposób ekonomiczny oszczędności energii, wnioski z których zostają udokumentowane w raporcie
	\item\textbf{Gwarancja zużycia energii:} Ilość energii cieplnej zużytej w Budynku na ogrzewanie pomieszczeń i na straty obiegowe w Okresie świadczenia Usług, uzyskana w oparciu o Gwarancję oszczędności energii od Wykonawcy, zastosowana do określenia Stałego zużycia energii cieplnej
	\item\textbf{Usługa poprawy efektywności energetycznej:} Zbiór działań realizowanych przez Wykonawcę obejmujący wdrożenie Środków w Budynku, prowadzenie i utrzymanie wdrożonych Środków, analiza danych o zużyciu energii, monitorowanie i ocena zużycia energii – szczególnie w kontekście dotrzymania Gwarancji oszczędności energii
	\item\textbf{Oszczędność energii:} Ilość zaoszczędzonej energii ustalona pomiarem i weryfikacją zużycia przed wdrożeniem i po wdrożeniu jednej lub większej liczby miar charakterystyki energetycznej, uzyskana w Budynku dzięki wdrożeniu Środków i świadczeniu Usług poprawy efektywności energetycznej
	\item\textbf{Gwarancja oszczędności energii:} Minimalna Oszczędność energii wynikająca z wdrożenia Środków i ze świadczenia Usług poprawy efektywności energetycznej, zagwarantowana w Umowie przez Wykonawcę, określona zgodnie z Planem pomiaru i weryfikacji
	\item\textbf{Taryfa za energię:} Stawka za jednostkę energii w miejscu, w którym znajduje się Budynek
	\item\textbf{Platforma umów o poprawę efektywności energetycznej (sunshineplatform.eu):} Uniwersalna, wielostronna platforma internetowa ułatwiająca zawieranie umów o poprawę efektywności energetycznej (sunshineplatform.eu), wspierająca podejmowanie inwestycji renowacyjnych i zarządzanie nimi w oparciu o takie umowy
	\item\textbf{Opłata:} (maksas): Comiesięczna opłata płacona przez Klienta Wykonawcy za świadczenie Usług przewidzianych w Umowie w Okresie świadczenia Usług, obejmująca Opłatę za energię cieplną, Opłatę za ciepłą wodę użytkową, Opłatę za renowację i Opłatę za eksploatację i utrzymanie ze wszystkimi należnymi podatkami (w tym VAT).
	\item\textbf{Stałe zużycie energii cieplnej:} Ilość energii cieplnej obliczona przez Wykonawcę, będąca podstawą obciążania Klienta stałą kwotą miesięczną za zużycie energii cieplnej w każdym Okresie rozliczeniowym Okresu świadczenia Usług.
	\item\textbf{Wkład finansowy:} Część kosztów Prac renowacyjnych sfinansowana bezpośrednio (kapitałem Wykonawcy) lub pośrednio (kapitałem pozyskanym przez Wykonawcę od stron trzecich), będąca podstawą naliczania przez Wykonawcę Opłaty za renowację
	\item\textbf{Opłatę za energię cieplną:} Opłata płacona Wykonawcy przez Klienta za energię zużytąw Budynku w Okresie świadczenia Usług, podlegająca korekcie i bilansowaniu co rok, na końcu Okresu rozliczeniowego w celu uwzględnienia rzeczywistych warunków pogodowych w Okresie rozliczeniowym oraz pomiaru i weryfikacji Gwarancji oszczędności energii
	\item\textbf{Dostawa ciepła:} Zaopatrzenie Budynku w energię cieplną na potrzeby ogrzewania pomieszczeń i przygotowania ciepłej wody użytkowej
	\item\textbf{Sezon grzewczy:} Miesiące, w których Wykonawca musi zapewniać Standardy komfortu zagwarantowane w Umowie (od 1 października do 30 kwietnia) w Okresie świadczenia Usług
	\item\textbf{Faktura:} Rachunek wystawiony Klientowi (Właścicielowi mieszkania lub jego przedstawicielowi) zgodnie z prawem łotewskim za wyświadczone Usługi, obejmujący również pozostałe należności Wykonawcy wynikające z Umowy
	\item\textbf{Międzynarodowy protokół pomiaru i weryfikacji efektywności (IPMVP):} Dokument określający oszczędność energii sporządzony przez Efficiency Valuation Organization (1629 K Street NW, Suite 300, Washington, DC 20006, USA), stosowany w pomiarze i weryfikacji wykonania Umowy
	\item\textbf{LABEEF:} Latvian Baltic Energy Efficiency Facility Jsc., spółka akcyjna zarejestrowana w łotewskim Rejestrze Handlowym pod numerem 40103960646
	\item\textbf{Stan niejawny:} Usterka lub wada Budynku lub jego otoczenia, o której Klient nie wie, a której Wykonawca nie mógł powziąć wiedzy na podstawie normalnych obserwacji i zwyczajnych inspekcji dokonywanych podczas przygotowywania Umowy
	\item\textbf{Zarządca:} Osoba fizyczna lub prawna realizująca zadania zarządcze i utrzymaniowe zlecone przez Klienta, określone w Umowie – zgodnie z łotewską ustawą o zarządzaniu budynkami mieszkalnymi i w oparciu o Umowę
	\item\textbf{Środek lub Środek efektywności energetycznej:}Działanie skutkujące uzyskaniem sprawdzalnej i wymiernej lub możliwej do oszacowania poprawy charakterystyki energetycznej oraz pozostałe prace budowlane i montażowe służące wyremontowaniu i ulepszeniu Budynku pod względem zarówno konstrukcyjnym, jak i estetycznym
	\item\textbf{Pomiar i weryfikacja:} postępowanie i działania prowadzone w celu ustalenia Oszczędności energii w Budynku wynikającej z wdrożenia Środków i świadczenia Usług
	\item\textbf{Opłata za eksploatację i utrzymanie:} Opłata płacona Wykonawcy przez Klienta za świadczenie Usług związanych z Eksploatacją i utrzymaniem Środków, podlegająca corocznej indeksacji wg łotewskiego indeksu cen dóbr konsumpcyjnych obowiązującego w danym roku, opublikowanego przez Centralne Biuro Statystyki
	\item\textbf{Podręcznik eksploatacji i utrzymania:} Dokument zawierający harmonogram utrzymania Środków wdrożonych na mocy Umowy oraz opis czynności eksploatacyjnych objętych Umową
	\item\textbf{Strony:} Klient i Wykonawca
	\item\textbf{Strona:} Klient lub Wykonawca
	\item\textbf{Harmonogram płatności:} Dokument sporządzony przez Wykonawcę dla Klienta, przedstawiający Opłatę za renowację na spłatę Wkładu finansowego wyliczoną dla każdego okresu odsetkowego zgodnie z postanowieniami Umowy
	\item\textbf{Prawidłowe funkcjonowanie:} Funkcjonowanie Środków w sposób zapewniający osiągnięcie pełnej ich funkcjonalności i wydajności z uwzględnieniem wszystkich czynności utrzymaniowych realizowanych przez Wykonawcę i na jego koszt
	\item\textbf{Regulator:} Komisja ds. Przedsiębiorstw Użyteczności Publicznej lub inny organ łotewski zatwierdzający taryfy za energię cieplną w lokalizacji budynku
	\item\textbf{Opłata za renowację:} Opłata indeksowana stopą Euribor płacona przez Klienta Wykonawcy za Wkład finansowy Wykonawcy
	\item\textbf{Prace renowacyjne:} Działania podejmowane przez Wykonawcę, niezbędne dla wdrożenia Środków w Budynku, obejmujące projektowanie, nabywanie, dostarczanie, montowanie, uruchamianie, przekazywanie do użytkowania i finansowanie Środków
	\item\textbf{Okres świadczenia Usług:} Okres świadczenia Usług Klientowi przez Wykonawcę. Bieg Okresu świadczenia Usług rozpoczyna się od Daty przekazania do użytkowania.
	\item\textbf{Okres rozliczeniowy:} Okres jednego roku kalendarzowego rozpoczynający się i kończący w określonych dniach w Okresie świadczenia Usług.
	\item\textbf{Oświadczenie:} Dokument podpisany przez Strony, potwierdzający różne parametry Budynku, zarejestrowane w czasie sporządzania tego dokumentu
	\item\textbf{VAT:} Podatek od wartości dodanej w rozumieniu uregulowań łotewskich i postanowień Umowy
\end{itemize}

\subsection{AKCEPTACJA WARUNKÓW UMOWY}
\begin{enumerate}
	\item Zdaniem Klienta Wykonawca posiada kwalifikacje, doświadczenie i możliwości niezbędne do wykonywania Prac renowacyjnych i świadczenia Usług na rzecz Klienta. Dlatego Klient upoważnia Wykonawcę do podjęcia na koszt Wykonawcy wszelkich działań prawnych i faktycznych służących wykonaniu Umowy bez potrzeby wystawienia Wykonawcy pełnomocnictwa. Klient będzie wspierał Wykonawcę w tych działaniach.
	\item Wykonawca będzie wykonywał Prace renowacyjne i świadczył Usługi na rzecz Klienta zgodnie z warunkami Umowy. Wykonawca poświadcza, że zapoznał się z charakterem, sytuacją i lokalizacją Budynku, a także ze wszystkimi innymi kwestiami mogącymi wpływać w jakikolwiek sposób na wykonanie Umowy. Żadne zaniechanie w tym względzie nie zwolni Wykonawcy z obowiązku wykonania Umowy.
	\item Wykonawca potwierdza, że budżet określony w Warunkach szczególnych Umowy obejmuje wszystkie prace budowlane, materiały i wyposażenie niezbędne dla zrealizowania Prac renowacyjnych zgodnie ze specyfikacjami technicznymi inwestycji i warunkami Umowy.
	\item Sformułowania zawarte w Umowie, w Warunkach szczególnych, w Aneksach i w niniejszych Warunkach ogólnych będą rozumiane w sposób określony w punkcie 1 Warunków ogólnych.
	\item W razie rozbieżności pomiędzy postanowieniami Warunków ogólnych, Warunków szczególnych i Aneksów, obowiązuje następująca hierarchia ważności: 1. Aneksy, 2. Warunki szczególne, 3. Warunki ogólne.
\end{enumerate}

\subsection{BEZPIECZEŃSTWO, JAKOŚĆIKOMFORT}
\begin{enumerate}
	\item Usługi  Wykonawcy  wynikającez Umowy będą:
	\begin{enumerate}
		\item świadczone  w  sposób  najbardziej kompetentny i dbały, jak można oczekiwać od  doświadczonych  i  profesjonalnych wykonawców  świadczących  regularnie takie  lub  podobne  usługi  o  zbliżonympoziomie złożoności,
		\item świadczone  z  wykorzystaniem materiałów i wyposażenia oodpowiedniej jakości,   nowych,   zdatnych   do przewidzianego zastosowania,
		\item zgodne   z  prawem  budowlanym  i pozostałymi  uregulowaniami  i  normami obowiązującymi  w  Łotwie  podczas świadczenia Usług,
		\item świadczone  w  sposób  najmniej uciążliwy  dla  Klienta  i  pozostałych użytkowników Budynku.
	\end{enumerate}
	\item Standardy    komfortu    w    Okresie świadczenia  Usług  będą  takie,  jak określono w Warunkach szczególnych lub wyższe.
	\item Wykonawca nie gwarantuje utrzymania    w    Mieszkaniu temperatury uzgodnionej w Warunkach szczególnych w czasie, gdy w Mieszkaniu otwarte jest okno i przez 2(dwie) godzin od chwili zamknięcia okna.
	\item Wykonawca  zapewni  wentylację Mieszkań  zgodną  z  odpowiednimi łotewskimi uregulowaniami i normami.
	\item Wykonawca zapewni bezpieczeństwo  i  ochronę  zdrowia pracowników zgodnie z ustawą o ochronie pracy  i  wszystkimi  obowiązującymi łotewskimi uregulowaniami i normami.
	\item Wykonawca zastosuje odpowiednie  środki  chroniące  wszystkie osoby  przed  obrażeniami  i  śmiercią mogącymi  wyniknąć  z  naruszenia    lub rażącego   zaniedbania   po   stronie Wykonawcy  lub  jego  pracowników, przedstawicieli i podwykonawców podczas realizacji  Prac  renowacyjnych  i  w  Okresie świadczenia   Usług.   Wykonawca zabezpieczy  również  cały  Budynek  przed szkodami  związanymi z wdrażaniem Środków.
	\item Wykonawca  upewni  się,  że dostawa mediów do Budynku nie zostanie w  jakikolwiek  sposób  przerwana  bez wcześniejszego  zgłoszenia,  powodu naruszenia    lub    zaniedbania    po    stronie Wykonawcy. Dostawa mediów przerwana z powodu   naruszenia lub   zaniedbania   po stronie   Wykonawcy   musi   zostać bezzwłocznie   przywrócona   przez Wykonawcę na jego koszt. Wykonawca nie odpowiada za przerwy w dostawie mediów niezawinione  przez  Wykonawcę  i/lub wynikające z działań lub zaniedbań stron trzecich nie związanych z Wykonawcą (np. dostawcy mediów).
	\item Wykonawca  zapewni  ochronę Budynku przed warunkami atmosferycznymi   w   Okresie   prowadzenia Prac   renowacyjnych,   zapobiegając przesiąkaniu wód opadowych. Nie dotyczy to infiltracji wody gruntowej i Okoliczności siły wyższej.
	\item Wykonawca jest zobowiązany do przestrzegania Europejskiego Kodeksu Postępowania w umowach o poprawę efektywności energetycznej (http://transparense.eu/), będącego zestawieniem wartości i zasad uznawanych za fundamentalne dla udanej, profesjonalnej i transparentnej realizacji takich umów w Europie.
\end{enumerate}


\subsection{GWARANCJE}
\begin{enumerate}
	\item W ramach Umowy Wykonawca udzieli Klientowi Gwarancji oszczędności energii na Okresie świadczenia Usług. Dotrzymanie Gwarancji będzie podlegało corocznemu Pomiarowi i weryfikacji.
	\item Wykonawca zagwarantuje dotrzymanie Standardów komfortu określonych w Umowie w Okresie świadczenia Usług.
	\item Wykonawca zagwarantuje na własny koszt Prawidłowe funkcjonowanie Środków w Okresie świadczenia Usług. Dotyczy to Środków zamontowanych lub wprowadzonych przez Wykonawcę w instalacjach centralnego ogrzewania, ciepłej wody użytkowej, klimatyzacji, w węzłach i na orurowaniu. Prawidłowe funkcjonowanie oznacza dotrzymanie specyfikacji (z uwzględnieniem normalnego zużycia) w okresie obowiązywania Umowy, między innymi poprzez dokonywanie napraw i wymian.
	\item Wykonawca zagwarantuje na własny koszt skuteczność izolacji w Okresie świadczenia Usług. Dotyczy to izolacji zamontowanych lub wprowadzonych przez Wykonawcę, które powinny pozostawać zgodne ze specyfikacjami (z uwzględnieniem normalnego zużycia) w okresie obowiązywania Umowy, między innymi poprzez dokonywanie napraw i wymian.
	\item Wykonawca zapewni Prawidłowe funkcjonowanie wszystkich Środków na końcu Okresu świadczenia Usług. Prawidłowe funkcjonowanie oznacza zgodność ze specyfikacjami (z uwzględnieniem normalnego zużycia) w warunkach normalnego utrzymania. Na końcu Okresu świadczenia Usług Wykonawca dostarczy Klientowi wszystkie podręczniki, instrukcje i ewidencje eksploatacji, konserwacji i utrzymania, pozostałe dokumenty, oprogramowanie, licencje na własność intelektualną, narzędzia specjalne oraz protokoły i procedury niezbędne lub przydatne w utrzymaniu Prawidłowego funkcjonowania Środków w celu dotrzymania Standardów komfortu wynikających z Umowy.
	\item Przed rozpoczęciem się Okresu prowadzenia Prac renowacyjnych Wykonawca przedłoży Klientowi gwarancję dobrego wykonania Umowy wydaną przez instytucję kredytową lub ubezpieczeniową na 10\% całkowitego kosztu inwestycji (bez VAT) na następujących zasadach:
	\begin{enumerate}
		\item jeśli Wykonawca jest również firmą ogólnobudowlaną, gwarancja ta zostanie udzielona przez Wykonawcę na rzecz Klienta w oparciu o postanowienia Umowy,
		\item jeśli Wykonawca zatrudnia firmę ogólnobudowlaną, gwarancja ta zostanie dostarczona przez firmę ogólnobudowlaną na rzecz Wykonawcy, w oparciu o postanowienia umowy o wykonawstwo zawartej pomiędzy Wykonawcą i firmą ogólnobudowlaną,
		\item jeśli Wykonawca nie dostarczył takiej oryginalnej gwarancji dobrego wykonania Umowy na Okres prowadzenia Prac renowacyjnych w celu zabezpieczenia wykonania działań w Okresie Prowadzenia prac renowacyjnych, Wykonawcy nie wolno przystąpić do realizacji prac budowlanych,
		\item gwarancja dobrego wykonania Umowy musi pozostawać w mocy w całym Okresie prowadzenia Prac renowacyjnych, a w razie przedłużenia Okresu prowadzenia rac renowacyjnych Wykonawca przedłuży dpowiednio gwarancję
	\end{enumerate}
	\item Wykonawca przedłoży Klientowi gwarancję dobrego wykonania Umowy wydaną przez instytucję kredytową lub ubezpieczeniową na co najmniej 5\% kosztu inwestycji (bez VAT) na następującyc zasadach:
	\begin{enumerate}
		\item jeśli Wykonawca jest również firmą ogólnobudowlaną, gwarancja ta zostanie udzielona przez Wykonawcę na rzecz Klienta w oparciu o postanowienia Umowy,
		\item jeśli Wykonawca zatrudnia firmę ogólnobudowlaną, gwarancja ta zostanie dostarczona przez firmę ogólnobudowlaną na rzecz Wykonawcy, w oparciu o postanowienia umowy o wykonawstwo zawartej pomiędzy Wykonawcą i firmą ogólnobudowlaną,item šai garantijai ir jābūt spēkā 36 (trīsdesmit sešus) mēnešus;
		\item gwarancja ta musi pozostawać w mocy przez okres 36 (trzydziestu sześciu) miesięcy.
	\end{enumerate}
	\item Klient może skorzystać z gwarancji wspomnianych w punktach 4.6-7 do pokrycia zobowiązań finansowych Wykonawcy lub kosztów wynikających ze zmian prawa.
	\item Gwarancja dobrego wykonania Umowy wspomniana w punktach 4.6-7 musi zostać wydana przez instytucję kredytową lub ubezpieczeniową zarejestrowaną w Łotwie lub w dowolnym innym państwie członkowskim UE lub EOG, która rozpoczęła świadczenie usług w Łotwie zgodnie z krajowymi uregulowaniami.
\end{enumerate}

\subsection{PRAWA I OBOWIĄZKI WYKONAWCY}
\begin{enumerate}
	\item Wykonawca posiada kwalifikacje profesjonalne, doświadczenie i możliwości niezbędne dla wyświadczenia Usług oraz zapewnienia wyposażenia i materiałów zgodnie z Umową.
	\item Wykonawca uzyska wszystkie pozwolenia i zgody organów władzy centralnej i lokalnej niezbędne dla wykonania Prac renowacyjnych i dla świadczenia Usług bez udziału Klienta, o ile udział taki nie jest wymagany zgodnie z obowiązującymi uregulowaniami.
	\item Wykonawca rozpocznie realizację Prac renowacyjnych i wdrażanie Środków w Dniu rozpoczęcia Prac renowacyjnych i zakończy te działania przed upływem Okresu prowadzenia Prac renowacyjnych. Wykonawca poinformuje Klienta o przewidywanym Dniu rozpoczęcia Prac renowacyjnych w terminie 20 dni roboczych od podpisania Umowy.
	\item Wykonawca zawiadomi Klienta pisemnie z wyprzedzeniem co najmniej 10 (dziesięciu) dni roboczych o Dniu rozpoczęcia Prac renowacyjnych i wdrażania Środków, by umożliwić Klientowi opróżnienie obszarów wspólnych Budynków (w tym klatek schodowych, piwnic, poddasza, dachu, pomieszczeń składowych, urządzeń elektroenergetycznych oraz telekomunikacyjnych oraz kotłowni) z odpadków, porzuconego mienia i innych przedmiotów. Jeśli Klient nie opróżnił powierzchni wspólnych na czas, Wykonawca może wykonać lub zlecić wykonanie tych czynności i obciążyć Klienta ich kosztem. Klient zapłaci za taką fakturę w terminie 20 dni roboczych.
	\item W Okresie prowadzenia Prac renowacyjnych Wykonawca będzie zapewniał całość robocizny niezbędnej dla wdrożenia Środków, a w tym nadzór, narzędzia, materiały i wyposażenie o odpowiedniej charakterystyce, jakości i ilości.
	\item W Okresie świadczenia Usług Wykonawca będzie zapewniał całość robocizny niezbędnej eksploatacji i utrzymania Środków, a w tym nadzór, narzędzia, materiały i wyposażenie o odpowiedniej charakterystyce, jakości i ilości.
	\item W Okresie prowadzenia Prac renowacyjnych Wykonawca będzie zapewniał zasilanie energią elektryczną opomiarowaną odrębnie, pokrywając koszty energii zużytej na potrzeby wykonania Prac renowacyjnych i wdrożenia Środków. Wykonawca ma prawo dostępu do wspólnego źródła zasilania elektrycznego Budynku.
	\item Wykonawca należycie uprzątnie miejsca prowadzenia prac (obszary wspólne, okna, wejścia i otoczenie) po wykonaniu Prac renowacyjnych i wdrożeniu Środków – przed Datą przekazania do użytkowania.
	\item Wykonawca zaprosi Klienta do udziału w przekazaniu do użytkowania Środków wdrażanych w Budynku. Wykonawca dostarczy Klientowi Protokół zdawczo-odbiorczy budynku na końcu Okresu prowadzenia Prac renowacyjnych.
	\item W Okresie świadczenia Usług Wykonawca będzie informował Klienta pisemnie o odpadkach i/lub przedmiotach składowanych i porzuconych przez Właścicieli mieszkań lub strony trzecie (nie związane z Wykonawcą) w częściach wspólnych Budynku, mogących utrudnić Wykonawczy czynności eksploatacyjne i utrzymaniowe wynikające z Umowy. Jeśli Klient nie opróżnił powierzchni wspólnych zgodnie z postanowieniami Umowy, Wykonawca może wykonać lub zlecić wykonanie tych czynności i obciążyć Klienta ich kosztem. Klient zapłaci za taką fakturę w terminie 20 dni roboczych.
	\item W Okresie świadczenia Usług Wykonawca będzie informował pisemnie klienta o wszystkich związanych ze Środkami usterkach i przypadkach kradzieży, wandalizmu lub sabotażu.
	\item Wykonawca zapewni w Sezonie grzewczym Dostawy energii cieplnej do Budynku wystarczające dla utrzymania Standardów komfortu. Wykonawca nie odpowiada za przerwy w Dostawie energii cieplnej do Budynku niezawinione przez Wykonawcę (tj. zawinione przez dostawcę ciepła lub wynikające z Okoliczności siły wyższej).
	\item Przed rozpoczęciem się Okresuprowadzenia Prac renowacyjnych Wykonawca przedstawi Klientowi projekt renowacji budynku zgodnie z postanowieniami rozporządzenia Rady Ministrów nr 655 z 21 października 2014 r. w sprawie łotewskiej normy budowlanej LBN 310 („Projekt wykonawczy”) oraz skoordynuje ten projekt z Klientem i z inspektorem nadzoru budowy.
	\item W Okresie prowadzenia Prac renowacyjnych Wykonawca będzie zawiadamiał Klienta o cotygodniowych spotkaniach poświęconych monitorowaniu statusu prac budowlanych i będzie go zapraszał na te spotkania. Wykonawca będzie przedstawiał Klientowi co miesiąc 2 (dwa) raporty o statusie i postępach prac budowlanych. Raporty te można przesyłać elektronicznie przez Platformę umów o poprawę efektywności energetycznej – sunshineplatform.eu.
	\item W przypadkach pośredniczenia Wykonawcy pomiędzy Klientem i dostawcą energii cieplnej, Wykonawca będzie płacił w imieniu Klienta faktury wystawiane przez tego dostawcę po otrzymaniu odpowiedniego komponentu Opłaty za energię cieplną należnego Wykonawcy na mocy Umowy. Bez względu na powyższe postanowienie, faktury za ogrzewanie w Okresie prowadzenia Prac renowacyjnych będą opłacane na koszt Klienta.
	\item Wykonawca może powierzać świadczenie prac i usług określonych w Umowie stronom trzecim (podwykonawcom). Wykonawca ponosi wówczas pełną odpowiedzialność za swych podwykonawców.
	\item Wykonawca może zmodyfikować Gwarancję oszczędności energii w przypadku zmiany sposobu użytkowaniu budynku (zgodnie z postanowieniami punktu 17). Każda taka zmiana wymaga pisemnej zmiany Umowy.
	\item Jeśli Klient sprzeciwi się Środkom zaproponowanym przez Wykonawcę, Wykonawca może zatrudnić na własny koszt odpowiednio wykwalifikowanego i doświadczonego niezależnego rzeczoznawcę – zaakceptowanego wcześniej przez Klienta (bez nieuzasadnionej odmowy) – do przeprowadzenia oceny zgodności zaproponowanych Środków z uregulowaniami łotewskimi lub z wiążącymi Klienta decyzjami lokalnych władz. Opinia rzeczoznawcy będzie wiążąca dla Stron.
	\item Wykonawca jest zobowiązany dozawiadomienia Klienta o każdej zmianie adresu Wykonawcy wskazanego w Umowie i o innych zmianach w statusie prawnym lub w zarządzie Wykonawcy (szczególnie w przypadku połączenia lub wykupu przedsiębiorstw, bądź wszczęcia postępowania likwidacyjnego lub upadłościowego) w terminie 5 (pięciu) dni roboczych od zaistnienia takiej zmiany.
\end{enumerate}

\subsection{PRAWA I OBOWIĄZKI KLIENTA}
\begin{enumerate}
	\item Klient podjął prawomocną i wykonalną decyzję powodującą, że Umowa jest wiążąca dla każdego z Właścicieli mieszkań, z których każdy indywidualnie jest zobowiązany do wypełniania postanowień Umowy – bez względu na to, czy Mieszkanie jest posiadane na własność, czy zostało wynajęte i bez względu na to, czy Mieszkanie jest użytkowane przez Właściciela mieszkania, czy przez inną osobę.
	\item Klient (i każdy z Właścicieli mieszkań) jest zobowiązany do poinformowania najemców, mieszkańców i innych użytkowników Mieszkania o obowiązkach wynikających z Umowy.
	\item Klient będzie udostępniałWykonawcy bezzwłocznie wnioskowane przez niego informacje i dokumenty potrzebne do wykonania Prac renowacyjnych i do świadczenia Usług. Klient nie będzie odpowiedzialny za niedostarczenie jakichkolwiek dokumentów, które mogą być istotne, lecz nie zostały opisane przez Wykonawcę wystarczająco precyzyjnie.
	\item Klient będzie terminowo udzielał Wykonawcy pomocy w pozyskiwaniu niezbędnych pozwoleń, zatwierdzeń i innych dokumentów warunkujących wykonanie Umowy od wszelkich władz i organów, co obejmie m.in. zapewnianie Wykonawcy świadectw, niezbędnych dokumentów, pełnomocnictw i informacji. Klient należycie upoważni Wykonawcę do podejmowania wszelkich czynności faktycznych i prawnych przed kompetentnymi organami na potrzeby realizacji Umowy. Jednak Klient nie będzie odpowiedzialny za niedostarczenie takich informacji, o ile nie zostaną one opisane szczegółowo przez Wykonawcę lub jeśli Klient nie będzie nimi dysponował.
	\item Klient nie utrudni rozpoczęcia ani nie wstrzyma wydania zgody na podjęcie Prac renowacyjnych i wdrożenie Środków w Okresie prowadzenia Prac renowacyjnych, ani na utrzymanie Środków w Okresie świadczenia Usług. Przeciwnie – Klient będzie działał w dobrej wierze, by ułatwić wykonanie tych czynności i dotrzymanie Gwarancji oszczędności energii.
	\item Klient może zareklamować jakość lub sposób wdrożenia jednego lub większej liczby Środków w terminie do 10 (dziesięciu) dni roboczych od podpisania Protokołu zdawczo-odbiorczego. Po upływie tego okresu Środki zostaną uznane za zaakceptowany a Okres prowadzenia Prac renowacyjnych zostanie uznany za zakończony.
	\item Klient może sprzeciwić się wdrożeniu Środka zaplanowanego w ramach Prac renowacyjnych jeśli Klient dowiedzie poza wszelką wątpliwość, że Środek narusza uregulowania łotewskie lub decyzję władz lokalnych wiążącą dla Klienta.
	\item Klient zapewni Wykonawcy lub jego przedstawicielowi dostęp do Budynku i do każdego z Mieszkań na Okres prowadzenia Prac renowacyjnych i na Okres świadczenia Usług na potrzeby świadczenia Usług. Klient zapewni dostęp do Budynku w dni robocze, w godzinach od 8:00 do 20:00, a w przypadkach nagłych również całodobowo w soboty, niedziele i dni wolne.
	\item Przed Dniem rozpoczęcia Prac renowacyjnych Klient upewni się, że wszystkie powierzchnie wspólne (w tym klatki schodowe, piwnice, poddasze, dach, pomieszczenia składowe, instalacje elektroenergetyczne i komunikacyjne oraz kotłownie) są wolne od odpadków, porzuconego mienia i innych przedmiotów, zlecając ich wywiezienie odbiorcy odpadków, zlecając ich usunięcie Wykonawcy lub przekazując je znanym właścicielom.
	\item W Okresie świadczenia Usług Klient będzie utrzymywał powierzchnie wspólne w czystości i w funkcjonalnym stanie.
	\item Klient nie będzie ingerował w Środki wdrożone przez Wykonawcę bez jego pisemnej zgody, ani niezgodnie z instrukcjami eksploatacji dostarczonymi przez Wykonawcę – szczególnie wówczas, gdy ingerencja mogłaby mieć negatywny wpływ na Oszczędność energii. Ewentualna ingerencja Klienta w nastawy instalacji centralnego ogrzewania, ciepłej wody użytkowej lub klimatyzacji zostanie uznana za istotne naruszenie Umowy i będzie podstawą do rozwiązania Umowy przez Wykonawcy z winy Klienta.
	\item Klient dołoży wszelkich racjonalnych starań, by zapobiec takim ingerencjom i sabotowaniu Środków.
	\item Klient zawiadomi Wykonawcę bezzwłocznie (w terminie jednego Dnia roboczego) o wykryciu każdego przypadku uszkodzenia lub zmodyfikowania Środków, bądź innej ingerencji w Środki.
	\item Klient będzie informował Wykonawcę o wszystkich okolicznościach wpływających lub mogących wpłynąć negatywnie na Oszczędność energii. Nieprzekazanie takich informacji przez Klienta nie zwolni Wykonawcy z obowiązku dotrzymania Gwarancji oszczędności energii, o ile nie zostanie stwierdzone, że Klient zamierzał obniżyć poziom Oszczędności energii.
	\item Klient będzie zgłaszał Wykonawcy w terminie 20 (dwudziestu) dni roboczych – a w razie potrzeby koordynował z Wykonawcą – wykonanie prac budowlanych, montażowych i utrzymaniowych nie objętych Umową, lecz mogących wpłynąć na zużycie energii w Budynku, takich jak (i) zwiększenie powierzchni Budynku, (ii) dodatkowa modernizacja Budynku, (iii) wymiana lub montaż nowych / innych grzejników, (iv) montaż nowego źródła ciepła.
	\item Klient (i każdy z Właścicieli mieszkań) zawiadomi Wykonawcę o renowacji Mieszkania i o okresie, w którym prace renowacyjne mogłyby wpływać na zużycie energii w Budynku, a w tym m.in. prace takie jak: (i) wymiana grzejników, (ii) wymiana okien, (iii) zwiększenie ogrzewanej powierzchni (np. o balkon), (iv) montaż wentylacji mechanicznej. W tych przypadkach Wykonawca może zrewidować Gwarancję oszczędności energii zapisaną w Umowie.
	\item Podczas Sezonu grzewczego Klient może otwierać okna w Mieszkaniach na co najwyżej 10 (dziesięć) minut dziennie w celu zapewnienia wymiany powietrza i wyeliminowania z Mieszkania dymu lub zapachu po sprzątaniu lub gotowaniu.
	\item Podczas Sezonu grzewczego Klient może otwierać okna w Mieszkaniach w dowolnym czasie jeśli wymaga tego stan zdrowia osoby przebywającej w Mieszkaniu.
	\item Podczas Sezonu grzewczego Klient będzie utrzymywał wszystkie okna w obszarach wspólnych w stanie zamkniętym.
	\item Podczas Sezonu grzewczego Klient będzie zapewniał zamykanie wszystkich drzwi wejściowych Budynku.
	\item W przypadku zmiany Właściciela mieszkania, Klient zawiadomi Wykonawcę o tej zmianie bezzwłocznie, w ciągu maksymalnie 5 (pięciu) dni roboczych.
	\item W przypadku zmiany Właściciela mieszkania (bez względu na podstawę faktyczną lub prawną) Klient (i Właściciel mieszkania) zapewni podpisanie przez nowego Właściciela mieszkania deklaracji przestrzegania Umowy lub innego instrumentu przeniesienia Umowy na nowego Właściciela mieszkania. W razie niespełnienia tego warunku poprzedni Właściciel mieszkania i Klient będą odpowiedzialni solidarnie za wypełnienie obowiązków wynikających z Umowy i za ewentualne naruszenia po stronie nowego Właściciela mieszkania.
	\item Klient zawiadomi Wykonawcę o zmianie Zarządcy w terminie 5 (pięciu) dni roboczych. Nowy Zarządca musi zostać należycie poinformowany o postanowieniach Umowy.
\end{enumerate}

\subsection{PROCEDURA ROZLICZEŃ}
\begin{enumerate}
	\item Klient będzie płacił Wykonawcy Opłaty miesięczne określone w Umowie.
	\item Okresem wzajemnych rozliczeń Stron jest miesiąc kalendarzowy. Okresem rozliczeniowym przyjętym do kalkulacji Płatności bilansującej Faktury w oparciu o stałe zużycie energii cieplnej, o rzeczywiste zmierzone Zużycie energii cieplnej oraz o pomiar i weryfikację dotrzymania Gwarancji oszczędności energii jest jeden rok.
	\item Wykonawca lub jego przedstawiciel będzie co miesiąc wyliczał kwotę płatności należnej od Klienta na mocy Umowy. Łączna suma wszystkich wyliczonych Opłat będzie uznawana w rozliczeniach wzajemnych za należną Wykonawcy od Klienta z tytułu Usług wyświadczonych na mocy Umowy.
	\item Co 12 (dwanaście) miesięcy począwszy od początku pierwszego Okresu świadczenia Usług Wykonawca będzie dokonywał rozliczenia rocznego w oparciu o wyniki pomiaru i weryfikacji Gwarancji oszczędności energii.
	\item Wykonawca lub jego przedstawiciel będzie wystawiał Fakturę wyszczególniającą wszystkie komponenty Opłaty określone jednoznacznie w Umowie i będzie dostarczał fakturę Klientowi lub Zarządcy reprezentującego Klienta do 10. (dziesiątego) dnia każdego miesiąca.
	\item Wykonawca upewni się, że informacje w Fakturze wystawionej każdemu z Właścicieli mieszkań za wyświadczone usługi będą czytelne i zrozumiałe, a kwota Opłaty za renowację i kwota Opłaty za eksploatację i utrzymanie będą wyodrębnione.
	\item Pierwsza płatność Opłaty musi zostać wyliczona jako należna po 1 (jednym) miesiącu od podpisania Protokołu zdawczo-odbiorczego. Do tego czasu Klient jest zobowiązany do pokrywania kosztów mediów i usług komunalnych.
	\item Klient (i każdy z Właścicieli mieszkań) będzie wnosił Opłaty na rzecz Wykonawcy (lub jego przedstawiciela) bezpośrednio lub za pośrednictwem Zarządcy, w oparciu o Faktury wystawiane przez Zarządcę za wszystkie media i inne koszty eksploatacyjne i utrzymaniowe, których część stanowić będą Opłaty należne Wykonawcy. Klient będzie wnosił Opłaty zgodnie z praktykami ustanowionymi przez Zarządcę, jednak nie później niż w terminie 15 (piętnastu) dni od otrzymania Faktury, przelewając odpowiednie środki na rachunek bankowy wskazany przez Zarządcę.
	\item Wykonawca lub powołany przez niego Zarządca będzie administrował informacjami związanymi z rozliczeniami dotyczącymi Umowy, wykonując m.in. następujące czynności:
	\begin{enumerate}
		\item dokumentowanie wszystkich informacji o Fakturach wystawionych poszczególnym Właścicielom mieszkań, a w tym kwot,
		\item ewidencjonowanie płatności za Faktury i bieżące aktualizowania ewentualnych zaległości Właścicieli mieszkań.
	\end{enumerate}
	\item Klient będzie dostarczał Wykonawcy lub jego cesjonariuszowi (na jego żądanie) aktualne raporty o płatnościach wniesionych przez Właścicieli mieszkań i listy dłużników.
\end{enumerate}

\subsection{OKRES OBOWIĄZYWANIA UMOWY}
\begin{enumerate}
	\item Bieg Okresu Umowy rozpoczyna się od daty zawarcia Umowy i kończy po upływie Okresu świadczenia Usług, przy czym Umowa może zostać rozwiązana wcześniej zgodnie z jej postanowieniami.
	\item Okres Umowy może zostać przedłużony za zgodą Stron. W szczególności, Strony mogą przewidzieć lub opóźnić Dzień rozpoczęcia Prac renowacyjnych i Datę przekazania do użytkowania za pisemną zgodą Stron.
	\item Wykonawca rozpocznie realizację Prac renowacyjnych i wdrażanie Środków w Dniu rozpoczęcia Prac renowacyjnych i zakończy te działania przed upływem Okresu prowadzenia Prac renowacyjnych. Po upływie Okresu prowadzenia Prac renowacyjnych Klient i Wykonawca podpiszą Protokół zdawczo-odbiorczy.
	\item Bieg Okresu świadczenia Usług i termin płatności liczone są od daty podpisania Protokołu zdawczo-odbiorczego.
	\item Jeśli Klient popełni naruszanie lub zaniedbanie skutkujące tym, że Wykonawca nie otrzyma wszystkich potrzebnych dokumentów i/lub dostępu budynku, bądź w razie wystąpienia Okoliczności siły wyższej Dzień rozpoczęcia Prac renowacyjnych, Okres prowadzenia Prac renowacyjnych i Okres świadczenia Usług zostaną automatycznie odroczone o czas opóźnienia. Zmiany takie zostaną uzgodnione pisemnie i wprowadzone doa Umowy.
\end{enumerate}

\subsection{STANY NIEJAWNE}
\begin{enumerate}
	\item Jeśli w Okresie prowadzenia Prac renowacyjnych Wykonawca poweźmie wiedzę o jakichkolwiek Stanach niejawnych wpływających na wdrożenie Środków, Wykonawca musi zawiadomić o tym Klienta pisemnie (by móc wnioskować o dodatkowy czas lub wynagrodzenie) w terminie 5 (pięciu) dni roboczych, przekazując następujące informacje:
	\begin{enumerate}
		\item opis Stanu niejawnego i istotnych różnic w stosunku do stanu budynku, który powinien był zostać racjonalnie przewidziany przez kompetentnego i doświadczonego wykonawcę postępującego zgodnie z dobrymi praktykami branżowymi przed zawarciem Umowy,
		\item opis dodatkowych prac i zasobów koniecznych według Wykonawcy do rozwiązania problemów wynikających ze Stanów niejawnych,
		\item oszacowany przez Wykonawcę czas potrzebny na rozwiązanie problemów wynikających ze Stanów niejawnych i spodziewane opóźnienie ukończenia prac,
		\item oszacowany przez Wykonawcę koszt środków potrzebnych do rozwiązania problemów wynikających ze Stanów niejawnych,
		\item pozostałe szczegółowe informacje, których Klient może faktycznie potrzebować.
	\end{enumerate}
	\item Opóźnienie wynikające ze Stanu niejawnego może uzasadnić przedłużenie Okresu prowadzenia Prac renowacyjnych jeśli Wykonawca musi:
	\begin{enumerate}
		\item wykonać dodatkowe prace,
		\item zastosować dodatkowe materiały,
		\item ponieść dodatkowe koszty (w tym opóźnienia lub przerwy), których Wykonawca nie przewidział i nie mógł racjonalnie przewidzieć w oparciu o dobre praktyki branżowe przed zawarciem Umowy.
	\end{enumerate}
	\item Klient pokryje wszystkie uzgodnione przez Strony faktycznie poniesione wydatki związane ze Stanami niejawnymi. Jeśli Klient nie będzie sobie życzył, by Wykonawca postępował w sposób opisany w zawiadomieniu, musi bezzwłocznie zażądać od Wykonawcy wstrzymania prac, a Wykonawca musi spełnić to żądanie. Klient i Wykonawca mogą wynegocjować i uzgodnić inny sposób rozwiązania problemów wynikających ze Stanu niejawnego, a w tym zlecić wykonanie dodatkowych niezbędnych prac stronom trzecim na koszt Klienta.
	\item W razie wystąpienia Stanów niejawnych Wykonawca może wnioskować o przedłużenie Okresu prowadzenia Prac renowacyjnych o czas potrzebny na rozwiązanie problemów związanych z tymi Stanami niejawnymi. Wykonawca jest uprawniony do uzyskania zwrotu dodatkowych kosztów związanych bezpośrednio lub pośrednio ze Stanami niejawnymi.
	\item Wykonawca nie może wnioskować o modyfikację Gwarancji oszczędności energii, o zmniejszenie zakresu Prac renowacyjnych, ani o zmianę uzgodnionej Opłaty z powodu Stanów niejawnych.
\end{enumerate}

\subsection{POMIAR I WERYFIKACJA ORAZ ZARZĄDZANIE DANYMI}
\begin{enumerate}
	\item Wykonawca musi przeprowadzić Pomiar i weryfikację w oparciu o Międzynarodowy protokół pomiaru i weryfikacji efektywności (IPMVP), zgodnie z Planem pomiaru i weryfikacji dostępnym na Platformie umów o poprawę efektywności energetycznej-sunshineplatform.eu.
	\item Wszystkie związane z tym działania zostaną opisane w pełni, jednoznacznie i będą jawne dla wszystkich Stron.
	\item W Okresie świadczenia Usług Wykonawca będzie przedstawiał Klientowi Coroczny raport. Coroczny raport udokumentuje kalkulację Opłat, realizowane przez Wykonawcę Działania eksploatacyjne i utrzymaniowe oraz stopień dotrzymania Gwarancji oszczędności energii w oparciu o Pomiar i weryfikację w Okresie rozliczeniowym. Raport będzie zawierał wystarczające informacje o Oszczędności energii wynikającej z wdrożenia Środków oraz o kalkulacji Oszczędności energii. Wykonawca wyśle Klientowi Coroczny raport każdego roku, najpóźniej w terminie 20 (dwudziestu) dni roboczych od końca Okresu rozliczeniowego. Raport może zostać złożony za pośrednictwem Platformy umów o poprawę efektywności energetycznej – sunshineplatform.eu.
	\item Jeśli Klient zakwestionuje wnioski zawarte w Corocznym raporcie, poinformuje o tym Wykonawcę w terminie 15 (piętnastu) dni roboczych od otrzymania Corocznego raportu lub zawiadomienia na Platformie umów o poprawę efektywności energetycznej-sunshineplatform.eu. Klient uzasadni Wykonawcy powody wniesienia zastrzeżeń. Wykonawca wprowadzi do Corocznego raportu niezbędne poprawki i poinformuje o nich Klienta w terminie kolejnych 15 (piętnastu) dni roboczych.
	\item Każda nieuzasadniona ingerencja Klienta w Środki wdrożone w Budynku skutkująca obniżeniem Oszczędności energii zostanie uwzględniona w Pomiarze i weryfikacji dotrzymania Gwarancji oszczędności energii, a Gwarancja zostanie na tej podstawie odpowiednio skorygowana.
	\item Klient przyjmuje do wiadomości i zgadza się, że Wykonawca i każdy z jego przedstawicieli realizujących Umowę może:
	\begin{enumerate}
		\item wykorzystywać wszystkie zanonimizowane dane i informacje dotyczące zużycia energii w Budynku (dostarczone przez Klienta lub uzyskane przez Wykonawcę) na potrzeby analizy porównawczej i tworzenia krajowych, regionalnych lub międzynarodowych baz danych, a także na potrzeby własne Wykonawcy uzgodnione z Klientem (w tym w charakterze referencji),
		\item wykorzystywać dane osobowe udostępnione przez Klienta lub jego Zarządcę na potrzeby świadczenia Usług, a także przekazywać te dane stronom trzecim uczestniczącym w realizacji Umowy, a w tym stronie wykupujących wierzytelności terminowe wynikające z Umowy i stronie zarządzającej lub odpowiedzialnej za rozwój, wdrażanie, użytkowani i utrzymanie Platformy umów o poprawę efektywności energetycznej (sunshineplatform.eu) wykorzystywanej do śledzenia efektywności wdrożonych Środków.
	\end{enumerate}
	\item Wykonawca (samodzielnie lub za pośrednictwem swych cesjonariuszy, wg własnego uznania) może wprowadzić, zainstalować, użytkować i serwisować system zarządzania energią lub – generalnie – oprzyrządowanie pomiarowe oraz korzystać z takiego wyposażenia o racjonalnych porach, przez racjonalny okres czasu, zgodnie z postanowieniami Umowy.
	\item Wykonawca może zainstalować w Mieszkaniach rejestratory temperatury w razie otrzymania reklamacji wynikających z zarzutu o niedotrzymanie Standardów komfortu. Jeśli Właściciel mieszkania nie zgodzi się na zainstalowanie rejestratora w swym Mieszkaniu lub nie zapewni wystarczającego dostępu do odpowiedniej instalacji, Wykonawca nie będzie odpowiedzialny za rzekome niedotrzymanie warunków Umowy w odniesieniu do tego Mieszkania.
	\item Dane zarejestrowane przez przyrządy pomiarowe Wykonawcy będą miały charakter informacyjny i nie będą mogły zostać uznane w jakimkolwiek sporze za podstawę stwierdzenia niedotrzymania Umowy lub Standardów komfortu.
\end{enumerate}

\subsection{ROZSTRZYGANIE SPORÓW}
\begin{enumerate}
	\item Wszelkie rozbieżności pomiędzy Stronami będą w pierwszej kolejności podlegały negocjacjom. W tym celu każda ze Stron potwierdzi pisemnie otrzymanie od drugiej Strony każdego zawiadomienia związanego ze sporem sporze i dołoży należytych starań w celu rozwiązania sporu, działając na odpowiednim szczeblu (z ewentualną eskalacją do poziomu zarządu).
	\item Jeśli Klient (lub Właściciel mieszkania) będzie miał zażalenie do Wykonawcy (np. dotyczące Standardów komfortu, Oszczędności energii, wdrożonych Środków lub wyświadczonych Usług), zawiadomi o nim Wykonawcę bezpośrednio lub za pośrednictwem Zarządcy. Wykonawca zapozna się z zażaleniem, sporządzi Oświadczenie opisujące przedmiot skargi i rozwiąże problemy będące jej źródłem. Jeśli problem nie zostanie rozwiązany w terminie 20 (dwudziestu) dni roboczych od zgłoszenia go, Klient zwoła posiedzenie komitetu, w skład którego wejdą upoważnieni przedstawiciele Wykonawcy, Zarządca i Klient w celu przygotowania projektu Oświadczenia w oparciu o zażalenie i fakty stwierdzone na miejscu lub w celu uruchomienia postępowania dochodzeniowego zgodnie z zasadami mediacji dostępnymi na Platformie umów o poprawę efektywności energetycznej - sunshineplatform.eu.
	\item Jeśli Wykonawca będzie miał zażalenie do Klienta (np. z powodu uszkodzenia zainstalowanego wyposażenia), zawiadomi o nim Klienta i Zarządcę. Klient zbada sprawę, zidentyfikuje sprawcę (jeśli to możliwe), sporządzi Oświadczenie opisujące przedmiot skargi i rozwiąże problemybędące jej źródłem. Jeśli problem nie zostanie rozwiązany w terminie 30 (trzydziestu) dni od zgłoszenia go, Wykonawca zwoła posiedzenie komitetu, w skład którego wejdą upoważnieni przedstawiciele Wykonawcy, Zarządca i Klient w celu przygotowania projektu Oświadczenia w oparciu o zażalenie i fakty stwierdzone na miejscu lub w celu uruchomienia postępowania dochodzeniowego zgodnie z zasadami mediacji dostępnymi na Platformie umów o poprawę efektywności energetycznej - sunshineplatform.eu.
	\item Postępowanie dochodzeniowe będzie prowadzone zgodnie z następującymi zasadami:
	\begin{enumerate}
		\item Faktyczne Standardy komfortu (temperatura powietrza w Mieszkaniach) zostaną uznane za należycie zarejestrowane jeśli pomiary temperatury zostały przeprowadzone przez niezależnego, certyfikowanego audytora energii zgodnie z normami MK 382 i LVS EN 12599. Oświadczenie zostanie sporządzone w oparciu o pomiary niezależnego, certyfikowanego audytora energii.
		\item Problemy ogólne z wdrożonymi Środkami (np. usterka sprzętu i/lub wada lub uszkodzenie Środka) lub z kalkulacją Oszczędności energii zostaną uznane za należycie odnotowane, jeśli udokumentował je niezależny rzeczoznawca, taki jak certyfikowany audytor energii (zgodnie z normą MK 382).
		\item Każda ze Stron zostanie powiadomiona z wyprzedzeniem co najmniej 5 (pięciu) dni roboczych o każdym pomiarze przeprowadzanym przez stronę trzecią. Upoważniony przedstawiciel Stron może uczestniczyć w pomiarze na potrzeby sporządzenia Oświadczenia. Nieobecność Upoważnionych przedstawicieli którejkolwiek ze Stron nie będzi przeszkodą w sporządzeniu Oświadczenia.
		\item Podpisanie Oświadczenia przez którąkolwiek ze Stron nie zostanie uznane za potwierdzenie naruszenia Umowy ani za zrzeczenie się praw lub obowiązków przez którąkolwiek ze Stron. Koszty za trudnienia niezależnych stron trzecich będą pokrywane przez Strony w równych częściach.
		\item Wykonawca, Zarządca i skarżący Właściciel mieszkania otrzymają po jednym egzemplarzu Oświadczenia.
	\end{enumerate}
	\item Jeśli Strony nie osiągną porozumienia, zainicjują formalne postępowanie mediacyjne zgodnie z regulaminem dostępnym na Platformie umów o poprawę efektywności energetycznej (sunshineplatform.eu), obowiązującym w okresie realizacji Umowy i w czasie prowadzenia sporu. W przypadku sporu dotyczącego spraw technicznych każda ze Stron może wnioskować o rozstrzygnięcie sporu opartego na ustalonych faktach zgodnie z regulaminami postępowania komitetu dochodzeniowego dostępnymi na Platformie umów o poprawę efektywności energetycznej (sunshineplatform.eu).
	\item Jeśli Strony nie osiągną porozumienia w trybie postępowania dochodzeniowego lub mediacyjnego, spór zostanie rozstrzygnięty przez właściwy sąd łotewski zgodnie z prawem łotewskim. Wniosek zostanie złożony do sądu właściwego dla miejsca zamieszkania lub siedziby pozwanego, jednak jeśli miejsce to będzie znajdować się poza terytorium Łotwy, spór zostanie rozstrzygnięty przez Sąd Rejonowy dla Rygi-Śródmieścia lub przez Sąd Okręgowy dla Rygi.
\end{enumerate}

\subsection{UTRZYMANIE ŚRODKÓW WDROŻONYCH PRZEZ WYKONAWCĘ}
\begin{enumerate}
	\item Wykonawca wymieni, naprawi lub wyremontuje wyposażenie (lub dowolną jego część) zainstalowane w ramach Prac renowacyjnych po upływie okresu użyteczności tego wyposażenia (ustalonego zgodnie z Podręcznikiem eksploatacji i utrzymania) w Okresie świadczenia Usług.
	\item Wykonawca wykona na Środkach czynności utrzymaniowe co najmniej takie, jakich wymaga lub jakie zalecają producenci, zgodnie z Warunkami szczególnymi Umowy.
\end{enumerate}

\subsection{UBEZPIECZENIE}
\begin{enumerate}
	\item Po rozpoczęciu Okresu prowadzenia Prac renowacyjnych Wykonawca ubezpieczy Budynek na sumę nie mniejszą niż wartość odtworzeniowa Budynku, z minimalnym ubezpieczeniem od pożaru, wstrząsów sejsmicznych, powodzi, uszkodzenia od wody i innych katastrof naturalnych, a także od uszkodzeń konstrukcyjnych spowodowanych osiadaniem gruntu lub upadkiem drzew. Ubezpieczenie spełni następujące warunki:
	\begin{enumerate}
		\item Wykonawca zawrze umowę ubezpieczenia z ubezpieczycielem o klasyfikacji nie niższej niż „A+” wg odpowiednich systemów klasyfikacyjnych obowiązujących w Łotwie.
		\item Wykonawca dostarczy Klientowi kopię polisy ubezpieczeniowej i dokumenty potwierdzające opłacenie składki ubezpieczeniowej przed Dniem rozpoczęcia Prac renowacyjnych.
		\item Klient zostanie wskazany jako beneficjent w przypadku wypłaty ubezpieczenia w kwocie co najmniej wystarczającej dla odzyskania wartości odtworzeniowej Budynku.
		\item Prace budowlane w Budynku nie zostaną rozpoczęte do chwili dostarczenia przez Wykonawcę prawomocnie zawartej umowy polisy ubezpieczeniowej.
		\item Wykonawca będzie utrzymywał polisę w okresie obowiązywania Umowy, a na żądanie Klienta będzie mu przedstawiał oryginał polisy, dostarczał kopię świadectwa wykupienia polisy lub udostępniał na Platformie umów o poprawę efektywności energetycznej (sunshineplatform.eu) równoważny dokument potwierdzający walutę i opłacenie składki.
		\item Wykonawca ubezpieczy Budynek na własny koszt na cały Okres prowadzenia Prac renowacyjnych. Po ukończeniu Prac renowacyjnych i po podpisaniu Protokołu zdawczo-odbiorczego koszty ubezpieczen ia na pozostały okres obowiązywania Umowy zostaną podzielone pomiędzy Właścicieli mieszkań proporcjonalnie do ich powierzchni i włączone do wystawianych przez Wykonawcę faktur za eksploatację i utrzymanie. Zarządca wyznaczony przez Strony upewni się, że faktury wystawiane Właścicielom mieszkań przez Klienta będą uwzględniać te koszty ubezpieczenia.
	\end{enumerate}
	\item Ponadto Wykonawca będzie utrzymywał w Okresie prowadzenia Prac renowacyjnych wiążącą polisę ubezpieczenia od odpowiedzialności cywilnej i profesjonalnej na sumę nie mniejszą niż 110\% całkowitego kosztu Prac renowacyjnych.
\end{enumerate}

\subsection{PRZENIESIENIE ROSZCZEŃ}
\begin{enumerate}
	\item Wykonawca może bez ograniczeń przenosić swe prawa i roszczenia do wszelkich należności od Klienta wynikających z Umowy na strony trzecie. Wykonawca może w szczególności przenieść należności z tytułu Opłaty za renowację na dowolnego Cesjonariusza, z którym zawarł umowę o finansowanie, o wykup wierzytelności terminowych, o cesję lub inną umowę.
	\item Przeniesienie Roszczeń nie zwalnia wykonawcy z obowiązków wynikających z Umowy. Jednak Cesjonariuszowi przysługują w niniejszej Umowie prawa subrogacyjne w przypadku niewywiązywania się Wykonawcy z obowiązków. Prawa subrogacyjne będą służyć w takim przypadku wyłącznie zastąpieniu Wykonawcy innym podmiotem zdolnym do wypełniania obowiązków wynikających z Umowy na rzecz Klienta i Cesjonariusza.
	\item W przypadku dokonania takiego przeniesienia Wykonawca zawiadomi o nim pisemnie w terminie 5 (pięciu) dni roboczych.
	\item Niniejsza Umowa została zawarta konkretnie z Klientem i nie może zostać przeniesiona ani scedowana przez Klienta bez wcześniejszego zawiadomienia Wykonawcy.
	\item W razie poddania Wykonawcy jakiemukolwiek połączeniu lub wykupowi, bądź postępowaniu likwidacyjnemu lub upadłościowemu, Umowa pozostanie w mocy, a jej postanowienia będą wiążące dla następców prawnych i dla cedenta Wykonawcy.
\end{enumerate}

\subsection{PRAWO WŁASNOŚCI DO ŚRODKÓW WDROŻONYCH W BUDYNKU W RAMACH PRAC RENOWACYJNYCH}
\begin{enumerate}
	\item Wykonawca pozostanie właścicielem Środków, których nie można oddzielić od Budynku bez powodowania istotnych uszkodzeń jeśli Wykonawca wniósł Wkład finansowy w Prace renowacyjne. Jeśli Prace renowacyjne zostały sfinansowane w pełni przez Klienta, właścicielem Środków jest Klient.
	\item Klient nie może usuwać, obciążać (dzierżawić, wynajmować, zastawiać etc.), niszczyć, uszkadzać lub modyfikować Środków wdrożonych w ramach Prac renowacyjnych niezależnie od tego, czyją własnością są Środki. Klient upewni się, że podobnych działań nie będą podejmować również Właściciele mieszkań.
	\item Wykonawca może obciążyć Środki (lub ich część) na własną rzecz lub na rzecz strony trzeciej bez zgody Klienta (tj. bez konieczności uzyskania zgody wszystkich Właścicieli mieszkań) w następującycha przypadkach:
	\begin{enumerate}
		\item Wykonawca jest ich właścicielem,
		\item Środki można technicznie zdemontować bez istotnego uszkadzania Budynku,
		\item obciążenie Środków jest niezbędne dla zabezpieczenia spłaty Wkładu finansowego Wykonawcy (Wykonawca nie może obciążyć Środków w celu zdobycia środków finansowych na cele inne niż realizacja Umowy), 15.3.4. okres obciążenia nie może wykroczyć poza okres obowiązywania Umowy.
		\item okres obciążenia nie może wykroczyć poza okres obowiązywania Umowy.
	\end{enumerate}
	\item Jeśli właścicielem Środków jest Wykonawca, po otrzymaniu przez niego wszystkich należności wynikających z Umowy prawo własności do wszystkich Środków przejdzie automatycznie na Klienta. W celu zabezpieczenia się przed ewentualnym stwierdzeniem nieważności tego przeniesienia Strony postanawiają, że formalną zapłatą za przeniesione Środki jest niepodlegająca zwrotowi kwota 1 (jeden) EUR zapłacona Wykonawcy z góry w chwili podpisania Umowy. Przeniesienie prawa własności na Klienta zostanie potwierdzone Oświadczeniem o przeniesieniu podpisanym przez Wykonawcę i przez Klienta.
\end{enumerate}

\subsection{OPROGRAMOWANIE I PRAWA WŁASNOŚCI INTELEKTUALNEJ}
\begin{enumerate}
	\item Wykonawca upewni się, że jest właścicielem wszystkich praw własności intelektualnej i przemysłowej związanych ze Środkami (a w tym wyposażeniem, materiałami, systemami, oprogramowaniem i innymi przedmiotami dostawy) lub że posiada licencje do korzystania z nich. Strony zgadzają się, że prawa takie pozostaną własnością Wykonawcy i nie przejdą na Klienta. Wykonawca udziela Klientowi wieczystej, nieodwołalnej, niewyłącznej, bezpłatnej licencji (z prawem do udzielania sublicencji) na korzystanie z tych praw wyłącznie w związku z użytkowaniem Budynku.
	\item Klientowi nie wolno modyfikować, kopiować lub dekonstruować żadnego oprogramowania, ani łączyć z innym oprogramowaniem dostarczonym przez Wykonawcę w ramach Prac renowacyjnych. W okresie obowiązywania Umowy Wykonawca będzie dostarczał klientowi podręczniki, informacje techniczne, aktualizacje i nowe wersje oprogramowania.
	\item Wykonawca zabezpieczy Klienta przed wszelkimi roszczeniami z tytułu jakiegokolwiek naruszenia praw własności intelektualnej stron trzecich związanych z jakimikolwiek prawami praw własności intelektualnej i przemysłowej pochodzącymi od Wykonawcy. Warunkiem tego zwolnienia jest spełnienie przez Klienta następujących warunków:
	\begin{enumerate}
		\item bezzwłoczne zawiadomienie Wykonawcy o roszczeniu,
		\item nieuznanie roszczenia, nieuprzedzenie obrony Wykonawcy przed roszczeniem i nieumniejszenie zdolności Wykonawcy do wynegocjowania zadowalającej ugody,
		\item umożliwienie Wykonawcy prowadzenie obrony i negocjacji na jego koszt,
		\item udzielanie Wykonawcy (na jego koszt) uzasadnionej pomocy i udostępnianie informacji na potrzeby obrony i negocjacji.
	\end{enumerate}
	\item W razie potrzeby Wykonawca zastąpi lub zmodyfikuje wg własnego uznania prawa zakwestionowane w celu wyeliminowania naruszenia lub nabędzie dla Klienta prawo do korzystania z tych zakwestionowanych praw. Środki prawne określone w tym punkcie będą jedynymi środkami z tytułu naruszenia praw własności intelektualnej.
\end{enumerate}

\subsection{ZMIANY W UŻYTKOWANIU BUDYNKU}
\begin{enumerate}
	\item Budynek, a w tym warunki jego użytkowania, powierzchnię i wymiary opisano w Warunkach szczególnych. W przypadku dokonania przez Klienta lub za jego zgodą zmiany uwarunkowań, na których oparte zostały kalkulacje Wykonawcy, zmiana ta nie wpłynie na Wykonawcę ani na warunki realizacji Umowy. Zmiany w użytkowaniu i modyfikacje Budynku zostaną ocenione w aspekcie ekonomicznym (szczególnie pod względem zmian w kosztach), a Umowa zostanie odpowiednio dostosowana do nowych uwarunkowań.

	\item Zmiany w użytkowaniu Budynku obejmują:
	\begin{enumerate}
		\item zwiększenie lub zmniejszenie powierzchni Budynku,
		\item zamontowanie, uszkodzenie lub zdemontowanie wyposażenia lub instalacji, jeśli działanie to skutkuje istotnym zwiększeniem lub zmniejszeniem zużycia energii lub zmianą innych parametrów technicznych Budynku,
		\item zmianę w użytkowaniu Budynku wpływającą na zużycie energii w Budynku (np. przekształcenie powierzchni mieszkalnej w komercyjną lub włączenie do eksploatacji Mieszkań wcześniej nie wykorzystywanych).
	\end{enumerate}
\end{enumerate}

\subsection{WYWÓZ ODŁĄCZONYCH I ZDEMONTOWANYCH URZĄDZEŃ I MATERIAŁÓW}
\begin{enumerate}
	\item Wykonawca zorganizuje na własny koszt wywóz odpadków powstałych w związku z realizacją Umowy w sposób zgodny z obowiązującymi uregulowaniami łotewskimi.
	\item Wykonawca zawiadomi Klienta pisemnie z wyprzedzeniem co najmniej 5 (pięciu) dni roboczych o pierwszym zaplanowanym wywozie. Zawiadomienie to dotyczy wszystkich urządzeń i materiałów zainstalowanych w Budynku, podlegających demontażowi i zastąpieniu w związku z wdrażaniem Środków w Okresie prowadzenia Prac renowacyjnych.
	\item Jeśli Klient zażyczy sobie wykorzystać którekolwiek z tych urządzeń lub materiałów, powinien zawiadomić o tym Wykonawcę i zorganizować na własny koszt ich odebranie.
\end{enumerate}

\subsection{ZOBOWIĄZANIA}
\begin{enumerate}
	\item Wykonawca odpowiada za terminowe wdrażanie Środków w Okresie prowadzenia Prac renowacyjnych. W razie niewypełnienia tego obowiązku, Klient może obciążyć Wykonawcę karą umowną w wysokości 0,02\% całkowitego budżetu inwestycji za każdy dzień opóźnienia. Łączna kwota kar umownych nie może przekroczyć 10\% (dziesięciu procent) budżetu inwestycji.
	\item Klient jest zobowiązany do terminowego pokrywania kosztów i wnoszenia Opłat wynikających z Umowy. W razie niewypełnienia tego obowiązku Wykonawca może obciążyć Klienta karą umowną w wysokości 0,1\% zaległej kwoty za każdy dzień opóźnienia.
	\item Jeśli Klienta zalegał z należną płatnością przez okres przekraczający 90 (dziewięćdziesiąt) dni, w którym zastosowano należycie i efektywnie Procedury rozstrzygania sporów przewidziany w Umowie, Wykonawca może rozwiązać Umowę z winy Klienta.

	\item Wykonawca jest zobowiązany do utrzymania w Budynku Standardów komfortu określonych w Umowie. Jeśli w Sezonie grzewczym temperatura wewnątrz pomieszczeń (zmierzona z uwzględnieniem dokładności przyrządów) była średnio o 2 (dwa) stopnie Celsjusza niższa niż wskazana w Standardach komfortu w którymkolwiek z Mieszkań, to Wykonawca będzie zobowiązany do polecenia Zarządcy obniżenia kwoty faktury dla Klienta (w odniesieniu do odpowiednich Mieszkań) w następujący sposób:
	\begin{enumerate}
		\item Obniżka o 5\% (pięć procent) kwoty Opłaty za energię za każdy stopień Celsjusza, za każdy miesiąc Sezonu grzewczego, w którym temperatura była niższa niż wskazana w Standardach komfortu.
		\item Ustalenie temperatury wewnątrz pomieszczeń i tego, czy była ona niższa niż wskazana w Standardach komfortu musi być zgodnie z Procedurami rozstrzygania sporów.
		\item Wykonawca nie zastosuje zniżki, jeśli spadek temperatury w Mieszkaniu był wynikiem (i) działań lub pominięć Właściciela lub użytkownika Mieszkania uznawanych za naruszenie Umowy, (ii) niewywiązania się Klienta z jego obowiązków lub (iii) okoliczności, na które Wykonawca nie miał wpływu.
	\end{enumerate}
	\item Klient ponosi odpowiedzialność za uszkodzenie, zmodyfikowanie lub kradzież Środków przez osoby inne niż Wykonawca lub osoby, za które on odpowiada – szczególnie wówczas, gdy działanie takie wpływa na Oszczędność energii, dotrzymanie Standardów komfortu lub bezpieczeństwo mieszkańców i użytkowników Budynku. W takim przypadku Klient:
	\begin{enumerate}
		\item zwróci Wykonawcy pełny koszt odtworzenia odpowiednich Środków;
		\item zapłaci Wykonawcy wynagrodzenie w wysokości 10\% (dziesięciu procent) kosztu odtworzenia w charakterze opłaty administracyjnej;
		\item Koszty odtworzenia muszą zostać wyliczone na podstawie cen rynkowych obowiązujących w czasie prowadzenia kalkulacji.
		\item Odpowiedzialność Klienta za takie zdarzenia musi zostać określona zgodnie z Procedurą rozstrzygania sporów.
	\end{enumerate}
	\item Wykonawca zabezpieczy Klienta przed wszelkimi stratami, kosztami, wydatkami, opłatami i odpowiedzialnością z tytułu roszczeń, skarg, postępowań administracyjnych lub sądowych ze strony organów państwowych lub stron trzecich wynikającą z działań Wykonawcy lub z naruszenia praw własności intelektualnej związanych ze Środkami wdrożonymi przez Wykonawcę. Wykonawca zwróci Klientowi wszystkie uzasadnione koszty i wydatki związane z naprawieniem wszystkich szkód bezpośrednich wynikających z działań Wykonawcy naruszających jakiekolwiek obowiązujące uregulowania. Zwrot zostanie należycie udokumentowany i dokonany przez Wykonawcę w terminie 30 (trzydziestu) dni roboczych od otrzymania przez niego od Klienta wezwania wskazującego jednoznacznie kwotę należności.
	\item Stawki opłat za świadczenie usług ogólnych (a w tym za dostawy ciepła) i sankcje przewidziane wobec Stron przez prawo łotewskie nie ograniczą obowiązków Stron wynikających z niniejszej Umowy, a w tym odpowiedzialności Stron, kar umownych i rekompensat.
	\item Zapłacenie kar umownych i rekompensat nie zwolni strony winnej z obowiązku wykonania Umowy.
\end{enumerate}

\subsection{ROZWIĄZANIE UMOWY}
\begin{enumerate}
	\item W razie naruszenia istotnego postanowienia niniejszej Umowy przez jedną ze Stron, druga Strona może rozwiązać Umowę bezzwłocznie i zażądać od Strony winnej zabezpieczenia przed odpowiedzialnością, szkodami i stratami zgodnie z postanowieniami Umowy.
	\item Rozwiązanie Umowy przed Dniem rozpoczęcia Prac renowacyjnych (gdy nie zostały wykonane żadne prace):
	\begin{enumerate}
		\item W przypadku jednostronnego rozwiązania Umowy przez Klienta z powodu istotnego naruszenia po stronie Wykonawcy Klient będzie uprawniony do otrzymania wynagrodzenia w wysokości 1\% Budżetu inwestycji (bez VAT).
		\item W przypadku jednostronnego rozwiązania Umowy przez Wykonawcę z powodu istotnego naruszenia po stronie Klienta, Wykonawca będzie uprawniony do otrzymania wynagrodzenia w wysokości 1\% Budżetu inwestycji (bez VAT).
		\item W przypadku jednostronnego rozwiązania Umowy przez Klienta z innych powodów (niekoniecznie związanych z Umową), Wykonawca będzie uprawniony do otrzymania wynagrodzenia w wysokości 1\% Budżetu inwestycji (bez VAT).
		\item W przypadku jednostronnego rozwiązania Umowy przez Wykonawcę z innych powodów (niekoniecznie związanych z Umową) Klient będzie  uprawniony do otrzymania wynagrodzenia w wysokości 1\% Budżetu inwestycji (bez VAT).
	\end{enumerate}
	\item Rozwiązanie Umowy, gdy koszty Prac renowacyjnych i wdrożenia Środków zostały już poniesione i pokryte z Wkładu finansowego Wykonawcy:
	\begin{enumerate}
		\item W przypadku jednostronnego rozwiązania Umowy przez Klienta z powodu istotnego naruszenia Umowy przez Wykonawcę Klient zwróci Wykonawcy tylko niespłacone saldo Wkładu finansowego Wykonawcy wniesionego w Środki funkcjonujące prawidłowo w rozumieniu Umowy, pomniejszone o 3\%, Ponadto Klient będzie uprawniony do uzyskania całej dokumentacji projektowej opisującej szczegółowo wykonane dotychczas prace (ze wszystkimi pozwoleniami, licencjami i innymi dokumentami uzyskanymi przez Wykonawcę), dokończenia najpilniejszych prac, wszystkich gwarancji od producentów, sublicencji (i przeniesień licencji) do korzystania z praw własności intelektualnej i oprogramowania (a w tym zainstalowanego oprogramowania, towarzyszących informacji, informacji związanych z kodem, kodu źródłowego, plików z danymi, kalkulacji, nośników elektronicznych, wydruków i powiązanych informacji) oraz dodatkowego szkolenia dla stron trzecich wyznaczonych przez Klienta w przypadku, gdy Prace renowacyjne zostały ukończone.
		\item W przypadku jednostronnego rozwiązania Umowy przez Wykonawcę z powodu istotnego naruszenia Umowy przez Klienta, Klient zwróci Wykonawcy niespłacone saldo Wkładu finansowego Wykonawcy powiększone o 3\%. Klient będzie uprawniony do uzyskania całej dokumentacji projektowej opisującej szczegółowo wykonane dotychczas prace (ze wszystkimi pozwoleniami, licencjami i innymi dokumentami uzyskanymi przez Wykonawcę), wszystkich gwarancji od producentów, sublicencji (i przeniesień licencji) do korzystania z praw własności intelektualnej i oprogramowania (a w tym zainstalowanego oprogramowania, towarzyszących informacji, informacji związanych z kodem, kodu źródłowego, plików z danymi, kalkulacji, nośników elektronicznych, wydruków i powiązanych informacji).
		\item W przypadku jednostronnego rozwiązania Umowy przez Klienta z innych powodów (niekoniecznie związanych z Umową) Wykonawca będzie uprawniony do otrzymania wynagrodzenia odpowiadającego niespłaconej kwocie kapitału Wkładu finansowego Wykonawcy, powiększonej o 3\%.
		\item W przypadku jednostronnego rozwiązania Umowy przez Wykonawcę z innych powodów (niekoniecznie związanych z Umową), Wykonawca będzie uprawniony do otrzymania wynagrodzenia odpowiadającego niespłaconej kwocie kapitału Wkładu finansowego Wykonawcy pomniejszonej o 3\%.
	\end{enumerate}
	\item Wykonawca lub dowolny z jegocesjonariuszy wystawi Klientowi fakturę na wyliczone wynagrodzenie z jednoznacznym wskazaniem informacji przyjętych jako podstawa tego obliczenia w oparciu o Harmonogramy płatności wydawane w Okresie świadczenia Usług lub o faktury płacone podczas realizacji Prac renowacyjnych przed rozwiązaniem Umowy. Klient zapłaci za fakturę na wynagrodzenie Wykonawcy, dowolnego z jego likwidatorów lub cesjonariuszy, bądź innego takiego podmiotu wskazanego jednostronnie jako posiadający tytuł prawny do wszystkich lub niektórych praw Wykonawcy wynikających z Umowy w terminie 60 (sześćdziesięciu) dni od daty wystawienia faktury.
	\item Strona przedterminowo rozwiązująca Umowę musi zawiadomić drugą Stronę o rozwiązaniu w formie pisemnej (wypowiedzeniem Umowy) z wyprzedzeniem co najmniej 20 (dwudziestu) dni roboczych. Jeśli rozwiązanie Umowy wynika z naruszenia Umowy przez drugą Stronę, pisemne zawiadomienie musi zawierać opis działań podjętych w ramach Procedur rozstrzygania sporów i związaną z nimi dokumentację.
	\item Klient może w dowolnym czasie zażądać od Wykonawcy kalkulacji kwoty wynagrodzenia należnego Wykonawcy w razie przedterminowego rozwiązania Umowy, a Wykonawca musi tę kalkulację przedstawić.
	\item Generalnie rozwiązanie Umowy nie zwolni Stron z wypełnienia obowiązków powstałych przed rozwiązaniem, o ile Strony nie uzgodnią odmiennie w formie pisemnej lub o ile Umowa nie przewiduje inaczej. Zwłaszcza jednostronne rozwiązanie Umowy przez Klienta w przypadku istotnego naruszenia po stronie Wykonawcy nie zwalnia Klienta z obowiązku zapłacenia za Faktury wystawione za okresy poprzedzające datę rozwiązania.
	\item Reorganizacja Strony, zmiana udziałowców i/lub właścicieli Strony, zmiana w zarządzie Strony, ani zmiana Właściciela mieszkania nie będzie stanowić podstawy do rozwiązania Umowy ani odstąpienia od wypełnienia wynikających z niej obowiązków.
	\item Niezależnie od pozostałych postanowień Umowy, Strony mogą ją rozwiązać w dowolnym czasie za obopólną pisemną zgodą, na uzgodnionych warunkach.
	\item Strona uprawniona do otrzymania wynagrodzenia może dochodzić swych praw na mocy Umowy lub zgodnie z odpowiednimi uregulowaniami łotewskimi, jednak Strona uprawniona nie otrzyma podwójnej rekompensaty za to samo naruszenie.
	\item Strony mogą zgodzić się na wymontowanie z Budynku Środków posiadanych w pełni lub częściowo przez Wykonawcę, jeśli Umowa została rozwiązana przedwcześnie z jakiegokolwiek powodu i jeśli Strona poszkodowana zgodziła się na wartość tych Środków i zaliczyła ją na poczet swego wynagrodzenia. Ta możliwość wymontowania środków nie ujmuje innym roszczeniom o odszkodowanie lub o zwrot kosztów przysługujących Stronom w razie przedterminowego rozwiązania Umowy.
\end{enumerate}

\subsection{OKOLICZNOŚCI SIŁY WYŻSZEJ}
\begin{enumerate}
	\item Za okoliczność siły wyższej rozumie się każdą nadzwyczajną sytuację lub każde nieprzewidziane zdarzenie spełniające wszystkie poniższe kryteria:
	\begin{enumerate}
		\item Strona nie mogła przewidzieć okoliczności ani wpłynąć na nią,
		\item okoliczność utrudnia Stronie wypełnienie swych obowiązków
		\item okoliczność nie wynika z błędu lub zaniedbania którejkolwiek ze Stron.
		\item można dowieść lub uznać, że okoliczności nie można przezwyciężyć mimo dołożenia przez Strony racjonalnych starań.
	\end{enumerate}
	\item Okoliczności siły wyższej obejmują m.in. działania wojenne, katastrofy naturalne i działania organów państwowych.
	\item Za okoliczności siły wyższej nie uznaje się wad Środków, wad jakościowych lub ilościowych Usług / wyposażenia / materiałów dostarczonych lub zainstalowanych przez Wykonawcę, opóźnień (innych niż wynikające z okoliczności siły wyższej), sporów po stronie Klienta, strajków, trudności finansowych lub innych okoliczności związanych szczególnie ze Stroną powołującą się na okoliczności siły wyższej.
	\item Strony nie odpowiadają za niewypełnienie swych obowiązków umownych w całości lub w części jeśli wynika to z okoliczności siły wyższej. Strona powołująca się na okoliczność siły wyższej jest zobowiązana do udowodnienia jej wystąpienia drugiej Stronie.
	\item Strona nie mogąca wypełnić obowiązku („Strona poszkodowana”) zawiadomi o tym drugą Stronę bezzwłocznie, w terminie do 3 (trzech) dni roboczych, opisując okoliczności siły wyższej, spodziewany czas ich utrzymywania się, prawdopodobne konsekwencje i potencjalne rozwiązanie.
	\item Strony – działając indywidualnie i łącznie – podejmą wszelkie niezbędne działania w celu złagodzenia skutków okoliczności siły wyższej, a w tym szkód.
	\item W razie utrzymywania się okoliczności siły wyższej przez okres dłuższy niż 6 (sześć) kolejnych miesięcy, gdy nie można liczyć na ich ustanie w okresie kolejnych 3 (trzech) miesięcy, każda ze Stron może jednostronnie rozwiązać Umowę
\end{enumerate}

\subsection{POUFNOŚĆ}
\begin{enumerate}
	\item Za poufne uznaje się informacje nabyte w związku z zawarciem lub realizacją niniejszej Umowy, niedostępne na zasadach ogólnych stronom trzecim, co do których Strona otrzymująca jest świadoma lub powinna była być świadoma, że ujawnienie ich może zaszkodzić prawom lub interesom Strony ujawniającej.
	\item Każda ze Stron zgadza się nie ujawniać Informacji poufnych należących do drugiej Strony jakimkolwiek stronom trzecim, a także nie ujawniać takich danych drugiej Strony, które mogłyby zostać wykorzystane w sposób konkurencyjny lub bezprawny – w okresie obowiązywania Umowy i przez okres 3 (trzech) lat od jej rozwiązania lub wygaśnięcia.
	\item Informacji upublicznionych przez strony trzecie bez winy którejkolwiek ze Stron Umowy nie uznaje się za poufne.
	\item Strony mogą ujawniać Informacje poufne stronom trzecim w celu zrealizowania Umowy pod warunkiem dochowania przez te strony trzecie warunków dochowania poufnościa określonych w Umowie.
	\item Ujawnienie Informacji poufnych w sytuacji, gdy wymagają tego uregulowania łotewskie nie zostanie uznane zaa naruszenie Umowy.
	\item Wykonawca, wszyscy jego cesjonariusze i Klient mogą ujawniać informacje ogólne o swej współpracy (a w tym znane już publicznie informacje o Stronach), o charakterze tej współpracy, o uzyskanej Oszczędności energii i dane o Zużyciu energii w celach reklamowych i w celu poinformowania społeczeństwa, o ile nie narusza to uzasadnionych praw i interesów drugiej Strony dotyczących ochrony Informacji poufnych. W razie zaistnienia wątpliwości, czy określone informacje przewidziane do ujawnienia są Informacjami Poufnymi, Strona zamierzająca ujawnić informacje musi zawnioskować o zgodę na ich ujawnienie do Strony, której interesy mogłyby zostać naruszone.
	\item Powyższe postanowienie nie ujmuje jednoznacznemu zobowiązaniu Klienta do nie wpływania na jakichkolwiek istniejących lub potencjalnych klientów bądź kontrahentów Wykonawcy (i/lub jakichkolwiek podmiotów o których Klient wie, że Wykonawca wykupu dąży do rozwijania z nimi współpracy) w sposób mogący prowadzić do ograniczenia, anulowania, wycofania, zmniejszenia lub innego zawężenia ich współpracy z Wykonawcą.
	\item Powyższe postanowienia nie wpływają na prawo Wykonawcy do gromadzenia, przetwarzania, przechowywania, przekształcania, przekazywania swym cesjonariuszom i źródłom finansowania oraz rozpowszechniania wszystkich danych uzyskanych od Klienta na potrzeby poprawy jakości Usług i na potrzeby rozwoju, eksploatacji i utrzymania Platformy umów o poprawę efektywności energetycznej (sunshineplatform.eu) wspierającej wszystkie etapy i uczestników typowych inwestycji w renowacje budynków opartych o umowy o poprawę efektywności energetycznej.
\end{enumerate}

\subsection{ZAWARCIE I ZMIANA UMOWY}
\begin{enumerate}
	\item Umowa wchodzi w życie w dniu podpisania przez Strony niniejszych Warunków ogólnych i pozostaje w mocy do chwili pełnego zrealizowania jej przez Strony.
	\item Wszystkie zmiany do niniejszej Umowy muszą zostać uzgodnione przez Strony, sporządzone w formie pisemnej i dołączone do Umowy w formie aneksów dla swej ważności.
	\item Wszystkie pozostałe postanowienia Warunków szczególnych i odpowiednich Aneksów pozostają w mocy. Wszelkie odstępstwa będą obowiązywać tylko  w stosunku do tej części Umowy, dla której odstępstwa te zostały uzgodnione.
	\item Umowa zostanie uznana za wygasłą po wypełnieniu swych obowiązków przez obydwie Strony.
	\item Jeśli w okresie obowiązywania Umowy prawo łotewskie zostanie zmienione ze skutkiem uniemożliwiającym wykonanie Umowy całkowicie lub częściowo, bądź zmieniającym warunki wykonania, zmiana ta nie wpłynie na ważność pozostałych obowiązków wynikających z Umowy, a Strony postarają się zmienić Umowę w sposób najlepiej odzwierciedlający jej pierwotne zamierzenia, cele i skutki ekonomiczne.
\end{enumerate}

\subsection{OŚWIADCZENIA STRON}
\begin{enumerate}
	\item Strony będą we wszystkich sprawach związanych z Umową reprezentowane przez należycie umocowanych przedstawicieli (w przypadku osób prawnych) lub przez osoby wskazane w Umowie. Do reprezentowania Klienta i Wykonawcy uprawnione są tylko osoby wskazane w Warunkach szczególnych Umowy.
	\item Umowa została sporządzona i podpisana w 3 (trzech) oryginalnych egzemplarzach w języku łotewskim, równoważnych pod względem skutków prawnych. Strony poświadczają swymi podpisami, że rozumieją treść, znaczenie i skutki niniejszej Umowy. Poświadczają też, że jest ona prawidłowa i wzajemnie korzystna, zawiera wszystkie niezbędne postanowienia, uwzględnia przyrzeczenia, warunki i oświadczenia Stron oraz została zawarta dobrowolnie.
\end{enumerate}


\subsection{KORESPONDENCJA}
\begin{enumerate}
	\item Wszystkie zawiadomienia, wnioski, roszczenia, żądania i pozostałe pozycje korespondencji wymagane lub dozwolone w Umowie będą dostarczane na adresy Stron.
	\item Każda pozycja korespondencji będzie (i) dostarczana osobiście, (ii) wysyłana pocztą kurierską z doręczeniem w następnym dniu, (iii) wysyłana pocztą poleconą, (iv) wysyłana telefaksem, (v) wysyłana pocztą elektroniczną z potwierdzeniem odbioru na adres Strony wskazany w Umowie lub zmieniony pisemnym zawiadomieniem, (vi) przekazywana w postaci komunikatu przez Platformę umów o poprawę efektywności energetycznej (sunshineplatform.eu), (vii) przesyłana wiadomością tekstową (SMS) za potwierdzeniem odbioru do telefonu komórkowego Strony wskazanego w Umowie lub zmieniony pisemnym zawiadomieniem (w przypadku krótkich wiadomości).
	\item Każda pozycja korespondencji zostanie uznana za skutecznie doręczoną w pierwszym z następujących terminów: (i) w chwili otrzymania jej przez Stronę będącą adresatem lub (ii) w 7 (siódmym) dniu od wysłania.
\end{enumerate}

\end{multicols}
