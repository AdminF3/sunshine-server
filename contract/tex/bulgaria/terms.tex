\begin{multicols}{2} [\section{ОБЩИ УСЛОВИЯ НА НАСТОЯЩОТО
    СПОРАЗУМЕНИЕ}]

  \subsection{ИЗИСКУЕМИ ДОКУМЕНТИ И ГАРАНЦИИ}
  \begin{enumerate}
  \item С подписване на настоящото споразумение, Прехвърлителят
    предава на Приобретателя оригинал на Договора с гарантиран
    резултат и всички документи, удостоверяващи прехвърлените и
    възложените за събиране вземания, както и не по-късно от 5 дни
    след получаване на искането на Приобретателя цялата информация,
    необходима за събиране на вземанията, както и информация за
    предявяване на искове и претенции, произтичащи от сключения
    Договор с гарантиран резултат. Прехвърлителят потвърждава, че
    документите и информацията, предоставени на Приобретателя по
    отношение на прехвърлените и възложените за събиране вземания,
    представляват всички документи и информация във връзка с тях, или,
    ако това не е така, допълнително се задължава да предостави всички
    допълнителни документи и информация. Прехвърлителят е длъжен
    незабавно да информира писмено Приобретателя за обстоятелства,
    свързани с влошаване на платежоспособността на Клиента (всеки
    собственик на самостоятелен обект в сградата).
  \item Прехвърлителят отделно предоставя на Приобретателя по всяко
    време информация за финансовото състояние на Прехвърлителя,
    представляваща счетоводна документация като извлечение от
    счетоводни баланси, баланси за приходи и разходи, отчети за
    собствен капитал и пр., както и всякаква информация, свързана с
    енергийното потребление на сградата и отделните самостоятелни
    обекти в нея, както и друга информация, поискана от Приобретателя
    и свързана с настоящия договор или договора с гарантиран
    резултат. В случай, че Прехвърлителят не предостави исканата
    информация в срок, Приобретателят има право да се снабди с нея
    като пълномощник на Прехвърлителя за сметка на Прехвърлителя и на
    разноски на Прехвърлителя.
  \item Прехвърлителят е длъжен незабавно да информира писмено
    Приобретателя за всеки спор, неизпълнение, обжалване, възражение,
    иск и/или рекламации от клиента на Прехвърлителя, както и
    незабавно да се съобрази с всички основателни претенции на клиента
    да замени, подобри или добави липсващо оборудване или услуга или
    да отстрани неизпълнението на договора. Горепосоченото включва без
    ограничение всички гаранционни претенции или претенции към
    Прехвърлителя от всякакъв друг характер, произтичащи от Договора с
    гарантиран резултат, без да се ограничава само до претенции,
    произтичащи от дефекти или недостатъци при изпълнение на ЕСМ в
    сградата.
  \item Отделно от горното, Прехвърлителят е задължен да предостави
    при поискване от страна на Приобретателя и допълнителни документи,
    установяващи вземанията, предмет на настоящия договор, и доказващи
    правни, валидни, обвързващи и изпълними задължения на
    клиента. Приобретателят разглежда документите, предадени от
    Прехвърлителя в рамките на 10 работни дни от получаването им, и
    може да оспори всеки документ, който не отговаря на критериите за
    доказване на валидно, обвързващо и изпълнимо задължение, и да
    поиска документи, отговарящи на тези изисквания, които
    Прехвърлителят се задължава да предостави надлежно и своевременно.
  \item Прехвърлителят гарантира на Приобретателя, че:
    \begin{enumerate}
    \item дружеството на Прехвърлителя е надлежно регистрирано,
      валидно съществуващо и в добро състояние съгласно законите на
      Република България и може да изпълни задълженията си по
      настоящото споразумение;
    \item изпълнението на настоящото споразумение не води до
      нарушаване на разпоредби на (а) вътрешни организационни
      документи, (б) приложимото законодателство, (в) съдебни или
      арбитражни решения;
    \item подписването и изпълнението на настоящото споразумение е
      надлежно и валидно разрешено съгласно вътрешните му корпоративни
      актове; няма друго уведомление до клиента, освен съобщение за
      прехвърляне и възлагане за събиране на вземания съгласно
      настоящото споразумение, не е необходимо съгласие или разрешение
      от държавен или друг орган за сключване или изпълнение на този
      договор;
    \item има право да прехвърли вземанията, предмет на прехвърляне;
    \item прехвърлените и възложените за събиране вземания и всяка
      част от тях представляват законни, валидни и обвързващи
      задължения на Клиента и са платими от последния в съответствие
      със сключения договор с гарантиран резултат;
    \item прехвърлените и възложените за събиране вземания не са били
      преотстъпвани /прехвърляни или възлагани за събиране по друг
      начин, продадени, заложени или обременени с каквото и да било
      тежести или права на трето лице и вземанията или каквато и да
      било част от тях не подлежат на прихващане срещу задължения на
      Прехвърлителя към клиента; и
    \item (а) не се води процедура срещу Прехвърлителя и (б) доколкото
      е известно на Прехвърлителя, няма основание за завеждане на
      такава процедура пред държавен орган, съд или арбитраж, която е
      вероятно да засегне неблагоприятно (*) всички действия,
      предприети от Прехвърлителя съгласно настоящото споразумение, и/
      или (**) прехвърлените вземания.
    \end{enumerate}
  \end{enumerate}

  \subsection{ПРАВА И ЗАДЪЛЖЕНИЯ НА ПРЕХВЪРЛИТЕЛЯ. ГАРАНЦИИ НА
    ПРЕХВЪРЛИТЕЛЯ}

  \begin{enumerate}
  \item Прехвърлителят е отговорен пред Приобретателя, ако не е
    оповестил за наличието на каквито и да било права и/или претенции,
    които могат да повлияят на размера или съществуването на
    Прехвърлените вземания съгласно този споразумение. Прехвърлителят
    не носи отговорност за платежоспособността на клиента, но носи
    отговорност за съществуването на прехвърлените вземания в
    прехвърления размер, както и че всички вземания, предмет на
    настоящия договор не са погасени по давност, не са залагани или
    обременявани по какъвто и да е начин с права на трети лица.
  \item Прехвърлителят уведомява клиента (всеки собственик на
    самостоятелен обект в сградата) за прехвърлянето на част от
    вземанията въз основа на настоящото Споразумение, като
    уведомлението трябва да бъде във формата, предвиден в ПРИЛОЖЕНИЕ 1
    Уведомление и потвърждение; В същото уведомление Прехвърлителят
    уведомява клиента (всеки собственик на самостоятелен обект в
    сградата) за възлагане на друга част от вземанията за
    събиране. Уведомлението до клиента (всеки собственик на
    самостоятелен обект в сградата) и получаване на потвърждението от
    него трябва да бъде направено в срок от 5 работни дни след влизане
    в сила на договора. По силата на настоящия договор Прехвърлителят
    упълномощава Приобретателя, а последният има право без да е
    задължен сам да направи уведомление до клиента (всеки собственик
    на самостоятелен обект в сградата) и получи потвърждението.
  \item Прехвърлителят гарантира постигане на икономията на енергия
    съгласно посоченото в договора с гарантиран резултат, и
    поддържането й за целия срок на договора.
  \item Прехвърлителят гарантира, че ЕСМ са изпълнени с надлежна
    професионална грижа, вложените материали и оборудване отговарят на
    законовите изисквания и договореното качество, нямат дефекти и
    недостатъци (с изключение на такива, дължащи се на нормалното
    износване).
  \item Прехвърлителят гарантира, че всички ремонтни дейности са
    изпълнени в съответствие с финансовите и техническите правила и
    насоките за мерките за енергийна ефективност на ФОНДА. В
    допълнение, прехвърлителят гарантира, че всички ЕСМ са изпълнени,
    като се вземат предвид най-добрите европейски и национални
    практики, отговарящи на проекта за доверие на инвеститорите в
    Европа (ICPEU {-}
    http://europe.eeperformance.org/). Прехвърлителят стриктно спазва
    и е обвързан с разпоредбите на Европейски професионален кодекс за
    договорите с гарантиран резултат ([●]) и гарантира неговото
    надлежно изпълнение.
  \item Прехвърлителят ще остане отговорен и се задължава да изпълнява
    всички свои задължения в качеството си на Изпълнител съгласно
    Договора с гарантиран резултат, включително в случай на дефекти
    или недостатъци на извършените работи, възникнали по време на
    срока на договора, като се задължава да отстранява същите съгласно
    договореното в договора с гарантиран резултат. Неспазването на
    Прехвърлителя на това задължение дава право на Приобретателя да
    отстрани дефектите за сметка на Прехвърлителя и да получи
    възстановяване на разходите в рамките на 20 дни от получаването от
    съответната фактура, изпратена на Прехвърлителя (която може да
    бъде изпратена по електронна поща).
  \item Прехвърлителят незабавно след като узнае, докладва на
    Приобретателя за всяка повреда или промяна, направена без знанието
    или разрешението на Прехвърлителя, на всяко оборудване,
    инсталирано от Прехвърлителя, или за наличието на обстоятелства,
    които имат и/или биха могли да имат отрицателно въздействие върху
    нивото на потреблението на енергия в сградата (като цяло или в
    отделните самостоятелни обекти) с пет на сто (5\%) или повече от
    потреблението на енергия, очаквано от Прехвърлителя в този период.
  \item Прехвърлителят се съгласява, че съгласно настоящото
    споразумение Приобретателят има право да събира, обработва,
    съхранява, използва и прехвърля личните данни, предоставени от
    клиента на Прехвърлителя съгласно Договора с гарантиран резултат и
    да ги предоставя на всяко трето лице, на което могат да бъдат
    възложени права или задължения на Приобретателя, които произтичат
    от настоящото споразумение, включително, без ограничение, всяко
    лице, отговорно за разработването, прилагането, експлоатацията и
    поддръжката на ДГР платформата. Използването на данните от
    Приобретателя за други цели трябва да става само след получаване
    на съгласието на клиента за това.
  \item Приобретателят потвърждава и се съгласява с използването от
    Прехвърлителя или посочено от него трето лице на всякакви анонимни
    данни и информация, свързани с консумацията на енергия на
    сградата, независимо дали са предоставени от клиента или получени
    от Прехвърлителя, за целите на сравнителното сравняване и
    съставянето на национална, регионална база данни или за целия ЕС
    или за целите на използване от Прехвърлителя като референция или
    за всяка друга цел, договорена с клиента.
  \end{enumerate}

  \subsection{ПРАВА И ЗАДЪЛЖЕНИЯ НА ПРИОБРЕТАТЕЛЯ}
  \begin{enumerate}
  \item След получаване на дължимите документи от Прехвърлителя,
    Приобретателят трябва да ги прегледа и незабавно, но във всеки
    случай в рамките на 10 (десет) работни дни след получаването им,
    да уведоми Прехвърлителя за всеки документ, за който има съмнение,
    че не установява валидно, правно обвързващо или изпълнимо
    задължение на клиента.
  \item Приобретателят няма да има претенции към Прехвърлителя при
    неплащане на задълженията от клиента по прехвърлените вземания,
    освен ако:
    \begin{enumerate}
    \item е налице нарушение/неизпълнение на задълженията на
      Прехвърлителя по Договора с гарантиран резултат, включително и
      свързани с изпълнението на ЕСМ и поддръжката им и постигане на
      гарантирания резултат. Горепосоченото не засяга правото на
      Приобретателя да търси принудително изпълнение и събиране на
      което и да е от прехвърлените с настоящия договор вземания чрез
      предприемане на съдебно или извънсъдебно изпълнение срещу
      клиента, включително без ограничение срещу всеки собственик на
      самостоятелен обект, за което Прехвърлителят оказва необходимото
      съдействие.
    \end{enumerate}
  \item Прехвърлителят обезщетява Приобретателя и отговоря спрямо него
    при наличие на претенция и/или каквото и да било възражение срещу
    Приобретателя, включително и при неплащане на суми от страна на
    клиента (всеки собственик на самостоятелен обект в сградата)
    поради неспазване или нарушение на задълженията на Прехвърлителя,
    произтичащи от това Споразумение, или произтичащи от договора с
    гарантиран резултат.
  \item Приобретателят обезщетява Прехвърлителя и отговоря спрямо него
    при наличие на претенция срещу Прехвърлителя поради неспазване или
    нарушение на Приобретателя на задълженията му по настоящия
    договор.
  \item Съгласно т. 3) и 4) в случай, че трето лице започне някакво
    действие, предяви претенция или направи искане срещу Прехвърлителя
    или Приобретателя („Страна“), за което такава страна
    („Обезщетената страна“) има право на обезщетение по настоящото
    споразумение, тази обезщетена страна незабавно писмено уведомява
    другата страна („обезщетяващата страна“) за действието,
    претенцията или искането. Нито една от страните не може да сключи
    споразумение с третото лице без съгласието на другата страна,
    което съгласие:
    \begin{enumerate}
    \item ако се дава от обезщетяващата страна не може да бъде
      неоснователно задържано, ако споразумението се ограничава до
      изплащане на парично обезщетение;
    \item ако се дава от обезщетената страна може да бъде задържано в
      случай, че споразумението налага изпълнение/действие от
      обезщетената страна.
    \end{enumerate}
  \item Задължението за плащане на обезщетение е отделно и независимо
    от другите договорни задължения на страните и съществува и след
    прекратяване на настоящото споразумение.
  \item Приобретателят има право да изиска и получи от Прехвърлителя
    дължимо и своевременно изпълнение на всички задължения по това
    споразумение, както и да изисква изпълнение на всички задължения
    на Прехвърлителя в качеството му на изпълнител по Договора с
    гарантиран резултат.
  \item Приобретателят има право да предприеме всякакви съдебни или
    извънсъдебни действия, необходими за събиране на вземанията, за
    което Прехвърлителят оказва необходимото съдействие на
    Приобретателя.
  \end{enumerate}

  \subsection{УПРАВЛЕНИЕ И ПОДДРЪЖКА НА СГРАДАТА}
  \begin{enumerate}
  \item Прехвърлителят се задължава да осигури сключване на:
    \begin{enumerate}
    \item договор между лицето, което действа като управител на ЕС (за
      събиране на парични суми от Клиента, за осъществяване на
      посредничество между прехвърлителя и Клиента и за информиране на
      Приобретателя за всички въпроси, свързани с вземанията)
      (\"Управителят\"), прехвърлителя и клиента (\"Договор за
      управление\"); или вземане на съответното решение на ОС на ЕС с
      единодушие, включващо единодушно решение на етажните собственици
      по тези въпроси;
    \item договор между прехвърлителя, приобретателя и Управителя
      (\"Тристранно споразумение\").
    \end{enumerate}
  \end{enumerate}

  \subsection{ВЛИЗАНЕ В СИЛА НА НАСТОЯЩИЯ ДОГОВОР. ПРЕКРАТЯВАНЕ}
  \begin{enumerate}
  \item Този договор влиза в сила при подписване на протокола за
    сбъдване на условията.
  \item Всички изменения и допълнения на настоящото споразумение се
    правят в писмена форма по взаимно съгласие на всички страни.
  \item Всяка страна има право да развали едностранно настоящото
    Споразумение, като уведоми другата за това писмено най-малко 30
    (тридесет) дни предварително, при прекратяване и/или разваляне на
    договора с гарантиран резултат, независимо от причината за това,
    както и ако договорът с гарантиран резултат бъде прогласен за
    недействителен.
  \item Приобретателят има право да развали настоящия договор и в
    случай, че Прехвърлителят не изпълнява своите задължения по него,
    както и ако клиентът не погасява задълженията си по прехвърлените
    вземания поради неизпълнение на задължение от страна на
    Прехвърлителя като изпълнител по договора с гарантиран резултат,
    включително и ако гарантираният резултат не бъде постигнат по вина
    на Прехвърлителя, както и ако прехвърлените вземания не
    съществуват, обременени са с тежести или с права на трети лица,
    които препятстват упражняването на правата на
    Приобретателя. Приобретателят има право да развали този договор и
    в случай, че Прехвърлителят и клиентът са изменили договора с
    гарантиран резултат без предварителното съгласие на Приобретателя.
  \item В случаите по т. 3) и 4), Приобретателят има право да получи
    от Прехвърлителя сума, представляваща сбора от:
    \begin{enumerate}
    \item Платената от Приобретателя цена за прехвърлените вземания,
      както и всички разноски по цесията и всички разноски по събиране
      на вземанията;
    \item Неустойка за разваляне в размер на [●]XXXXX\%, като
      неустойката не лишава правото на страната да търси обезщетение
      за действително претърпените вреди над този размер по общия ред.
    \end{enumerate}
  \item Извън случаите, описани в настоящото споразумение, този
    договор не може да бъде прекратен преди прекратяването на Договора
    с гарантиран резултат, освен ако Прехвърлителят и Приобретателят
    не са се споразумели за това писмено.

  \item Прекратяването на настоящото споразумение не освобождава
    страните от изпълнение на съответните им задължения, които са
    станали дължими преди момента на прекратяване, освен ако този
    договор не предвижда друго.
  \item В случай, че Приобретателят е получил от клиента суми,
    свързани с разваляне или друго предсрочно прекратяване на договора
    с гарантиран резултат, тези получени суми се считат за възложени
    за събиране по силата на настоящото споразумение и Приобретателят
    има право да задържи всички или част от тях до размера по т. 5),
    ако Приобретателят не е обезщетен вече по реда на т. 5).
  \item В случай, че по време на действие на настоящия договор
    Приобретателят е получил суми, надвишаващи размера на сумата по
    т. 5), при разваляне на споразумението, горницата се връща от
    Приобретателя на Прехвърлителя.
  \item Страните могат да прекратят настоящото споразумение по всяко
    време при взаимно съгласие, изразено писмено.
  \end{enumerate}

  \subsection{ЗАСТРАХОВАНЕ}
  \begin{enumerate}
  \item По време на действието на договора с гарантиран резултат
    Прехвърлителят поддържа непрекъснато застрахователна полица за
    сградата с минимално застрахователно покритие по реда на
    подт. 1.1.13 на т.  от ОБЩИ УСЛОВИЯ на ДОГОВОР С ГАРАНТИРАН
    РЕЗУЛТАТ. Прехвърлителят предоставя на приобретателя оригинал или
    заверено копие на полицата или други окончателни документи,
    потвърждаващи плащането на застрахователната премия. При
    възможност след сключване на настоящото споразумение правата на
    Прехвърлителя като бенефициент по застрахователна полица (ако има
    такива права), следва да бъдат прехвърлени на Приобретателя.
  \item Приобретателят има право за своя сметка допълнително да
    застрахова сградата, изпълнените мерки, събирането на вземанията,
    произтичащи от договора с гарантиран резултат, без да е необходимо
    да уведомява или получава одобрението на Прехвърлителя и/или
    клиента, а ако такова одобрение е необходимо според действащото
    законодателство, прехвърлителят се задължава да го даде и осигури
    съответните действия на клиента.
  \end{enumerate}

  \subsection{ПОСЛЕДВАЩИ ПРЕХВЪРЛЯНИЯ}
  \begin{enumerate}
  \item Приобретателят е свободен да преотстъпва, обременява или
    прехвърля на трети лица своите права и / или вземания съгласно
    настоящото споразумение.
  \item Прехвърлителят не може да прехвърля и/или да възлага за
    събиране на трето лице вземанията, предмет на настоящото
    споразумение.
  \end{enumerate}

  \subsection{ПРАВО НА СОБСТВЕНОСТ}
  \begin{enumerate}
  \item Настоящото споразумение не засяга никоя от разпоредбите на
    т. 1.1.15 от ОБЩИ УСЛОВИЯ на ДОГОВОР С ГАРАНТИРАН РЕЗУЛТАТ и
    правото на собственост на Прехвърлителя върху съоръженията,
    инсталирани в сградата за изпълнение на ремонтните дейности, които
    могат да бъдат отделени от Сградата, без да причинява материални
    щети.
  \item При спазване на горепосоченото и като гаранция за задълженията
    на Прехвърлителя, произтичащи от настоящото споразумение,
    Прехвърлителят учредява и осигурява вписването в ЦРОЗ на особен
    залог върху съоръженията, инсталирани в сградата, които са
    собственост на Прехвърлителя (по реда на т. 1.1.15 от ОБЩИ УСЛОВИЯ
    на ДОГОВОР С ГАРАНТИРАН РЕЗУЛТАТ) в полза на Приобретателя не
    по-късно от 10 работни дни, считано от подписване на протокола за
    изпълнение на условията.
  \end{enumerate}

  \subsection{ОТГОВОРНОСТИ}
  \begin{enumerate}
  \item Прехвърлителят носи отговорност за претенции, свързани с
    прехвърлените вземания при нарушаването и/или неизпълнението на
    задълженията на Прехвърлителя като изпълнител по Договора с
    гарантиран резултат, включително, но не само непостигане на
    гарантираната икономия на енергия, както и ако прехвърлените
    вземания са заложени, запорирани или върху тях права и претенции
    имат трети лица, ако са погасени по давност, ако договорът с
    гарантиран резултат бъде прогласен за нищожен, унищожен, развален
    поради неизпълнение на задължения на изпълнителя, както и в други
    случаи, предвидени в настоящия договор. Прехвърлителят обаче не
    отговаря за платежоспособността на клиента.
  \item Приобретателят има право да прекрати едностранно това
    споразумение освен в описаните по-горе случаи и по някоя от
    следните причини:
    \begin{enumerate}
    \item спрямо Прехвърлителя бъде открито производство по
      несъстоятелност и/или ликвидация;
    \item Прехвърлителят има изискуеми публични задължения;
    \item Прехвърлителят не предоставя информация, документи и
      съдействие, необходими за събиране от страна на Приобретателя на
      прехвърлените вземания в рамките на 14 дни от искането за това;
    \item Прехвърлителят не изпълнява договора с гарантиран резултат
      или настоящото споразумение, което причинява загуби или вреди на
      Приобретателя и/или невъзможност да бъдат събрани прехвърлените
      вземания от страна на Приобретателя или каквато и да било част
      от тях;
    \item Прехвърлителят не е разкрил на приобретателя факти, които са
      били известни и които възпрепятстват успешното събиране на
      прехвърлените вземания;
    \end{enumerate}
  \item Ако Приобретателят прекрати настоящото споразумение по
    отношение на всички или на която и да е част от прехвърлените
    вземания, Прехвърлителят трябва да върне всички получени от него
    суми, представляващи цена за прехвърлените вземания, да плати
    неустойката по т. 2.1.5, подт. 5) (ii) от настоящите общи условия,
    както и да заплати на Приобретателя всички разноски по цесията и
    всички разноски по събиране на вземанията.
  \end{enumerate}

  \subsection{НЕПРЕОДОЛИМА СИЛА}
  \begin{enumerate}
  \item Всяка аварийна ситуация или събитие, което не може да бъде
    предварително предвидено, и което отговаря на следните условия, се
    определя като непреодолима сила:
    \begin{enumerate}
    \item страните не са в състояние да го предвидят и да му повлияят;
    \item пречи на изпълнението на задълженията на страните;
    \item не може да бъде квалифицирано като грешка или небрежност,
      допуснато от която и да е от страните;
    \item непреодолимо е, въпреки че страната/ите са положили разумни
      усилия да го предотвратят.
    \end{enumerate}
  \item Събитията на непреодолима сила включват, но не се ограничават
    до военни действия, природни бедствия, актове на държавната и/или
    общинската администрация.
  \item Не представляват непреодолима сила следните събития: дефекти
    на съоръженията, и недостатъци по изпълнените ЕСМ поради
    отклонения от договореното качество или количество, финансови
    затруднения, както и други събития, които не попадат под
    дефиницията за непреодолима сила.
  \item Страните не носят отговорност за пълно или частично
    неизпълнение на задължения по Споразумението, ако неизпълнението
    се дължи на непреодолима сила.
  \item Страната, позоваваща се на непреодолима сила и възпрепятствана
    да изпълни задълженията си по настоящото споразумение, уведомява
    незабавно и не по-късно от 3 (три) работни дни на другата страна
    за непреодолима сила, като описва и възможна продължителност и
    очаквани последици.
  \item Страните извършват всички необходими действия съвместно или
    поотделно за смекчаване на последиците от непреодолимата сила.
  \item Ако събитието продължи повече от 6 (шест) месеца и не се
    очаква прекратяването му за още 3 (три) месеца всяка страна има
    право да прекрати едностранно настоящото споразумение, а
    Приобретателят има право да получи обратно цената за прехвърлените
    вземания, както и всички разноски по цесията и всички разноски по
    събиране на вземанията.
  \end{enumerate}

  \subsection{ПОВЕРИТЕЛНОСТ И ЗАЩИТА НА ЛИЧНИТЕ ДАННИ}
  \begin{enumerate}
  \item Страните по настоящото споразумение се съгласяват да не
    разкриват конфиденциална информация за срока на действие на
    настоящото споразумение и 3 години след прекратяването му.
  \item Конфиденциална информация по силата на настоящото споразумение
    е информация, получена в хода на сключване или по време на
    изпълнение на договора, която по принцип не е достъпна за трети
    страни и разкриването на която може да навреди на интересите на
    страната, за която се отнася информацията.
  \item В допълнение към разпоредбите на т. 2), поверителната
    информация е следната:
    \begin{enumerate}
    \item всяка информация с ограничен достъп, но не само, информация,
      която има статут на търговска тайна за която и да е от страните
      по настоящото споразумение;
    \item всяка информация за организацията на работа, инвентар,
      оборудване и технологии, използвани от която и да е от страните
      по настоящото споразумение.
    \end{enumerate}
  \item Списъкът на поверителната информация, предоставен т. 3), не е
    изчерпателен.
  \item Информацията, постъпила в публичното пространство и разкрита
    от трети лица без никоя от страните да е нарушила разпоредбите на
    настоящото споразумение, не се счита за конфиденциална информация.
  \item Страните по настоящото споразумение могат да разкрият
    конфиденциална информация на своите трети лица- консултанти и
    подизпълнители, но носят отговорност, ако третите лица не
    изпълняват задълженията за конфиденциалност, установени с
    настоящото споразумение.
  \item Ако страна по настоящото споразумение, получаваща поверителна
    информация, е длъжна да разкрие информацията в съответствие със
    законите и подзаконовите актове, действащи в Република България,
    тогава:
    \begin{enumerate}
    \item разкриването на информацията не се счита за нарушение на
      клаузите на настоящия договор; и
    \item разкриващата информация страна трябва да уведоми незабавно
      другата страна, освен ако не е предвидено друго от разпоредбите
      на приложимото право.
    \end{enumerate}
  \item Страните имат право да оповестяват за рекламни цели обща
    информация за взаимно сътрудничество, постигнатото
    енергоспестяване и др., включително, inter alia, разгласяване на
    информация, която вече е публично достояние.
  \item Горните разпоредби не засягат правото на Приобретателя да
    събира, обработва, съхранява, преобразува и разпространява всички
    събрани данни от Прехвърлителя с цел подобряване на качеството на
    неговите услуги и за развитие, експлоатация и поддръжка на ДГР
    платформата.
  \item Прехвърлителят се съгласява, че Приобретателят има право да
    събира, обработва, съхранява, използва и прехвърля предоставените
    от Прехвърлителя лични данни на собствениците на самостоятелни
    обекти в сградата при изпълнение на законовите му задължения за
    целите на предоставянето на услугите му и да ги прехвърля на всяка
    трета страна, на която могат да бъдат възложени права или
    задължения, произтичащи от настоящото споразумение, включително,
    без ограничение, която и да е страна, която отговаря за
    разработването, прилагането, функционирането и поддръжката на ДГР
    платформата. Използването на данните от Приобретателя за други
    цели трябва да се извършва след получаване на съгласието на
    Прехвърлителя или, когато е приложимо, съгласието на клиента.
  \item Прехвърлителят признава и се съгласява с използването от
    Приобретателя или от всяко лице, на което са предоставени права
    върху вземанията, на всякакви данни и информация, свързани с
    консумацията на енергия на Сградата, независимо дали са
    предоставени от Прехвърлителя или получени от Приобретателя, за
    целите на сравнителния анализ и съставянето на национална,
    регионална база данни или база данни за целия ЕС или за целите на
    използване като референция или за всяка вътрешна цел, договорена с
    клиента.
  \end{enumerate}

  \subsection{ПРЕДСТАВИТЕЛСТВО НА СТРАНИТЕ}
  \begin{enumerate}
  \item Във всички въпроси, свързани с настоящото споразумение, всяка
    от страните се представлява от своите законни представители (за
    юридически лица) или от надлежно и валидно упълномощени лица.
  \item Страните имат право да оттеглят пълномощията на своите
    представители по всяко време, като уведомяват писмено другата
    страна за това и едновременно упълномощават друг представител.
  \end{enumerate}

  \subsection{ПРОЦЕДУРА ЗА РАЗРЕШАВАНЕ НА СПОРОВЕ}
  \begin{enumerate}
  \item Страните разрешават разногласията по отношение на настоящото
    споразумение чрез взаимни преговори. За тази цел страните
    своевременно писмено отговарят на всяко възражение от другата
    страна.
  \item Ако страните не успеят да постигнат взаимно съгласие, спорът
    се решава от компетентния съд в Република България.
  \end{enumerate}

  \subsection{ДРУГИ РАЗПОРЕДБИ}
  \begin{enumerate}
  \item В случай на плащания от клиента, извършени погрешно в полза на
    Прехвърлителя, последният се задължава да преведе незабавно всички
    получени суми по сметката на Приобретателя.
  \item Всяко писмено уведомление относно настоящото споразумение се
    счита за получено на 7-ия (седмия) ден след изпращането му, ако е
    изпратено с препоръчана поща на адреса, определен за получаване на
    съобщения. Известията, изпратени по електронна поща до посочения
    имейл, се считат за писмено уведомление с всички произтичащи от
    него последствия и се считат за получени от момента на влизане в
    операционната система на получателя, независимо дали са прочетени
    или не. Известия и уведомления между страните могат да бъдат
    правени и чрез ДГР платформата.
  \item Настоящото споразумение се подчинява на приложимото
    законодателство на Република България.
  \item Заглавията на разделите от разпоредбите на настоящото
    споразумение са с информативна цел и не се използват за тълкуване
    на този договор. Ако не е посочено друго, позоваванията на отделни
    точки и/или членове са към съответния раздел, в който се намира
    членът и/или точката.
  \item Настоящото споразумение съдържа всички договорености между
    страните.
  \item Ако по време на срока на действие на настоящото споразумение
    настъпи законодателна промяна, която прави изпълнението на
    задължение напълно или частично невъзможно или в резултат на която
    се променят условията за изпълнение на задълженията от която и да
    е от страните, това не засяга валидността на останалите задължения
    по настоящото споразумение и страните въвеждат необходимите
    изменения, които най-добре ще отразяват първоначалните намерения и
    цели на Прехвърлителя и Приобретателя при сключване на настоящия
    договор.
  \item Преобразуването на страните, както и смяна на
    акционери/съдружници (собственици на капитала) или на членове на
    управителни органи и/или промяна в собственика на отделни обекти
    от сградата не може да доведе до прекратяване на настоящото
    споразумение или до неизпълнение на задълженията, съдържащи се в
    него. Настоящият договор остава валиден и неговите разпоредби са
    задължителни за правоприемниците на съответната страна. Всяка
    страна незабавно и не по-късно от 5 (пет) дни уведомява другата
    страна за промяна на адреса, посочен в настоящото споразумение,
    или за други промени в правния си статус.
  \item Споразумението е съставено и подписано в [●]XXXX еднообразни
    екземпляра на български език. Страните удостоверяват със своите
    подписи, че разбират съдържанието, значението и последиците от
    настоящото споразумение; признават това Споразумение за правилно,
    взаимно изгодно и желаят да го изпълнят доброволно
  \end{enumerate}

  \subsection{ПРИЛОЖЕНИЯ}
  \begin{enumerate}
  \item Приложения към настоящия договор са:
    \begin{enumerate}
    \item ПРИЛОЖЕНИЕ 1 Уведомление и потвърждение;
    \item ПРИЛОЖЕНИЕ 2 Списък на собствениците на самостоятелни
      обекти, представляващи заедно клиент по договора с гарантиран
      резултат;
    \item ПРИЛОЖЕНИЕ 3 Договор с гарантиран резултат, състоящ се от
      общи условия, специфични условия и приложения към тях;
    \item ПРИЛОЖЕНИЕ 4 Представители на страните и лица за контакт по
      настоящия договор;
    \item ПРИЛОЖЕНИЕ 5 Дефиниции
    \end{enumerate}
  \item Всички приложения се считат за неразделна част от настоящото
    споразумение и са задължителни за страните. В случай на
    несъответствие между условията на настоящото споразумение и някое
    от приложенията, разпоредбите на настоящото споразумение имат
    предимство с изключение на тези приложения, подписани след
    подписването на настоящото споразумение. Приложенията, подписани
    след подписването на настоящото споразумение, имат предимство пред
    разпоредбите на настоящото споразумение.
  \end{enumerate}

\end{multicols}
