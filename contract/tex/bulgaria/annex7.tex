\section{Приложение 7 {-} Погасителен план и правила за определяне размер на дължими суми от всеки собственик на самостоятелен обект.}

\subsection{Погасителен план}

% TODO make this as table
Дължима цена за изпълнение на ЕСМ без ДДС
……
Дължима цена за изпълнение на ЕСМ с ДДС
……
Годишен лихвен процент
7\% + 12 месечен ЮРИБОР
Брой вноски при изплащане на цената за ЕСМ
……………

% table: project_measurements_table

\begin{center}
\begin{longtabu}{|X|X|X|X|X|} \tabucline{}
{{with $t := translate "bg" .Contract.Tables.project_measurements_table}}
	{{.Columns | column}} \\\tabucline{}
	{{range .Rows}} {{rowf $t .}} \\\tabucline{} {{end}} %chktex 26
{{end}}
\end{longtabu}
\end{center}

\begin{enumerate}
\item Лихвата в частта 12-месечен Euribor се преизчислява и променя на
  една от следните дати на 15 януари или 15 април или 15 юли или 15
  октомври, като се прилага курсът, публикуван в предходния работен
  ден {-} \"Дата на лихвено преизчисляване\". Преизчисляването се прави
  съгласно стойността на 12-месечния EURIBOR, обявена на страниците на
  общодостъпни водещи финансови сайтове като www.euribor.org или
  www.euribor-rates.eu (или на друга страница, която заменя страниците
  на www.euribor.org или www.euribor-rates.eu) на деня, предхождащ
  деня на преизчисляване.
\item Плаващата част от лихвения процент за първия 12 месечен период
  се определя от Изпълнителя към деня на започване на периода на
  мониторинг, като се прилага лихвата, публикувана в предходния
  работен ден.
\item В случай, че 12 месечен ЮРИБОР е отрицателна величина, в този
  период, сумата се олихвява само с фиксирания лихвен процент.
\end{enumerate}
