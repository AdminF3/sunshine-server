\section{Приложение 6 - Правила за определяне на балансово плащане след измерване и отчитане на резултата}

\subsection{При стойност на КЕ по-малък от едно, ДГПЕ надвишава ГГПЕ и Изпълнителят заплаща на Клиента компенсационно плащане.}

\subsection{Компенсационното плащане се изчислява по следната формула: КП= СДГПЕ – СГГПЕ, където}

\begin{enumerate}
	\item КП е компенсационното плащане,
	\item СДГПЕ е сумата от произведенията на достигнатото годишно потребление на всеки от енергоносителите по цената им, определена в специфичните условия на договора;
	\item СГГПЕ е сумата от произведенията на гарантираното годишно потребление на всеки от енергоносителите по цената им, определена в специфичните условия на договора;
\end{enumerate}

\subsection{Компенсационното плащане представлява санкция за непостигане на гарантираната икономия на енергия. Страните могат да изберат компенсационното плащане да бъде прихванато със задължения на Клиента от последващ период.}

\subsection{При стойност на КЕ по-голям от едно, Клиентът следва да заплати на Изпълнителя премиално плащане.}

\subsection{Премиалното плащане се изчислява по следната формула: ПП= СГГПЕ – СДГПЕ, където}
\begin{enumerate}
	\item ПП е премиалното плащане
\end{enumerate}

\subsection{Премиалното плащане представлява бонус за превишаване на гарантираната икономия.}

\subsection{Фактура за премиалното плащане се издава от Изпълнителя в срок от 10 (десет) работни дни след изчисляване на КЕ и изпращане на годишния мониторингов доклад на Клиента.}
