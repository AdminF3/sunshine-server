{{- $tables := .Contract.Tables -}}

\section{Приложение 4 {-} Методика за отчитане на гарантирания резултат}

\subsection{БАЗОВО ГОДИШНО ПОТРЕБЛЕНИЕ НА ЕНЕРГИЯ (БГПЕ) {–} (1)}
Представлява определеното съгласно обследването за ЕЕ потребление на енергия за обекта, преди изпълнение на предвидените ЕСМ. \\

БГПЕ (1) {–} [●]XXXX kWh/година;

\subsection{ГАРАНТИРАНО ГОДИШНО ПОТРЕБЛЕНИЕ НА ЕНЕРГИЯ (ГГПЕ) {–} (2)}
Гарантираното годишно потребление на енергия, представлява гарантираното от Изпълнителя годишно потребление на енергия за обекта, след изпълнение на предвидените ЕСМ: \\

ГГПЕ (2) {–} [●]XXXX kWh/година;

\subsection{СТОЙНОСТ НА ГАРАНТИРАНОТО ГОДИШНО ПОТРЕБЛЕНИЕ НА ЕНЕРГИЯ
  (СГГПЕ) {-} [ (2)* (5)]}
Стойността на гарантираното годишно потребление на енергия в лева, представлява сумата от произведенията на гарантираното годишно потребление на топлинна енергия (ГГПТЕ) с определената в договора цена на топлинната енергия за обекта и гарантираното годишно потребление на електрическа енергия (ГГПЕЕ) с определената в договора цена на електрическа енергия за обекта: \\

СГГПЕ = ГГПТЕ* [●] лв./kWh + ГГПЕЕ* [●] лв./kWh = [●] kWh/год.* [●]лв./kWh + [●] kWh/год.* [●] лв./kWh \\

XXXXX лв./год. без ДДС \\
XXXXX лв./год. с ДДС

\subsection{ГАРАНТИРАНА ГОДИШНА ИКОНОМИЯ НА ЕНЕРГИЯ (ГГИЕ) {-} (3)}
Гарантираната годишна икономия на енергия представлява разликата между базовото годишно потребление на енергия и гарантираното годишно потребление на енергия за обекта. \\

ГГИЕ = БГПЕ - ГГПЕ = [●] – [●] = [●] kWh/година

\subsection{ОТЧИТАНЕ НА ГАРАНТИРАНИЯ РЕЗУЛТАТ}
Отчитането на гарантирания резултат се извършва с помощта на следната
таблица:

% table: baseline

\begin{center}
\begin{tabu}{|X[2]|X|X|X|X|X|X|} \tabucline{}
{{with $t := translate "bg" .Contract.Tables.baseline}}
	{{.Columns | column}} \\\tabucline{}
	{{range .Rows}} {{rowf $t .}} \\\tabucline{} {{end}} %chktex 26
{{end}}
\end{tabu}
\end{center}

\subsection{ДОСТИГНАТО ГОДИШНО ПОТРЕБЛЕНИЕ НА ЕНЕРГИЯ (ДГПЕ) {-} (12)}
Достигнатото годишно потребление на енергия от обекта е сумата от
потребената енергия в сградата от всички енергоносители, в kWh, за
една календарна година, след изпълнени ЕСМ.  При определянето на
достигнатото годишно потребление на топлинна и електрическа енергия
следва да се определи потребеното количество топлинна енергия и
потребено количество електрическа енергия, в kWh, за една календарна
година. Достигнатото годишно потребление на енергия (12) е сумата от
годишното енергийно потребление на всички енергийни ресурси.\\  В колона
(1) се записва посоченото в договора нормализирано базово потребление
на енергия.\\  В колона (12) се записва количеството потребена енергия
от доставения енергиен ресурс, удостоверено чрез фактура, издадена от
доставчик. В случай, че доставения ресурс не е напълно изразходван за
съответния отчетен период, остатъка по цени на закупуване следва да
бъде прехвърлен в следващия отчетен период. При цена на енергоносителя
0 лв./kWh не се изисква фактура.

\subsection{ДОСТИГНАТА ГОДИШНА ИКОНОМИЯ НА ЕНЕРГИЯ (ДГИЕ) {-} (13)}
Достигнатата годишна икономия на енергия за обекта представлява
разликата между Базовото годишно потребление на енергия и достигнатото
годишно потребление на енергия.

\subsection{ИЗЧИСЛЯВАНЕ НА КОЕФИЦИЕНТА НА ЕФЕКТИВНОСТ (КЕ)}
Коефициентът на ефективност е равен на отношението между достигнатата годишна икономия на енергия (13) и гарантираната годишна икономия на енергия (9) за обекта:\\

КЕ = (13)/(9) = [●] kWh/[●] kWh = [●] kWh \\

Гарантираният резултат е постигнат при изчислена стойност на КЕ равна или по-висока от 1.\\

Коефициентът на ефективност се изчислява ежегодно и на негова база се определя необходимостта от плащане на балансово плащане.\\

\subsection{КАЛИБРИРАНЕ}
Калибрирането на базовото (7) и гарантираното (8) годишно потребление на енергия има за цел отчитане на влиянието на климатичните фактори върху разхода на енергия в сградата.\\

За целта е необходимо да бъдат определени реалните годишни денградуси (DDC) за сградата, като се използват данни за реалните температури на околната среда и помещенията записани от системата за мониторинг.\\

% table: baseyear

\begin{center}
\begin{tabu}{|X[2]|X|X|X|X|X|X|X|X|X|X|X|X|} \tabucline{} \rowfont[c]\bfseries
	& \multicolumn{4}{c|}{ {{baseyear "bg" 2}} } & \multicolumn{4}{c|}{ {{baseyear "bg" 1}} } & \multicolumn{4}{c|}{ {{baseyear "bg" 0}} } \\
	& \multicolumn{4}{c|}{ {{.Contract.Tables.baseyear_n_2.Title}} } & \multicolumn{4}{c|}{ {{.Contract.Tables.baseyear_n_1.Title}} } & \multicolumn{4}{c|}{ {{.Contract.Tables.baseyear_n.Title}} } \\\tabucline{}\rowfont[c]\bfseries
  	{{$tt := join_tables .Contract.Tables.baseyear_n_2 .Contract.Tables.baseyear_n_1 .Contract.Tables.baseyear_n }} %chktex 25
  	{{with $t := translate "bg" $tt}}
	{{$t.Columns | column_sideways}} \\\tabucline{} \rowfont[]\bfseries %chktex 25
	{{range $t.Headers}}
	{{.|row}} \\\tabucline{}
	{{end}}
	{{range $t.Rows}}
	{{row .}} \\\tabucline{} %chktex 26
	{{end}}

	\bfseries {{average $t }} \\\tabucline{}
	\bfseries {{total $t}} \\\tabucline{}
{{end}}
\end{tabu}
\end{center}

\iffalse
%% TODO make nice legend.
%% comment region.
където:
nB – брой на дните за базовия период
ϴiB – средна обемна температура на помещенията за базовия период, ˚С
ϴmB – средна температура на външния въздух за базовия период, ˚С
DDB – денградуси на базовата година
nC – брой на дните за текущия период
ϴiC – средна обемна температура на помещенията за текущия период, ˚С
ϴmC – средна температура на външния въздух за текущия период, ˚С
DDC – денградуси на текущата година
DDB/DDC – коефициент за корекция на протреблението
DDB=nB*(ӨiB-ӨmB)
DDC=nC*(ӨiC-ӨmC)
Калибриране на енергиите (1,2) се извършва като базовия им еквивалент (X, kWh) се умножи с коефициента за корекция на съответната година:

Equation 1 Коефициент за корекция
X1,2,4C=X1,2,4B*DDBDDC
\fi

% table: baseconditions

\begin{center}
\begin{tabu}{|X[2]|X|X|X|X|X|X|X|X|X|X|X|X|} \tabucline{} \rowfont[c]\bfseries
	& \multicolumn{4}{c|}{ {{baseyear "bg" 2}} } & \multicolumn{4}{c|}{ {{baseyear "bg" 1}} } & \multicolumn{4}{c|}{ {{baseyear "bg" 0}} } \\
	& \multicolumn{4}{c|}{ {{.Contract.Tables.baseyear_n_2.Title}} } & \multicolumn{4}{c|}{ {{.Contract.Tables.baseyear_n_1.Title}} } & \multicolumn{4}{c|}{ {{.Contract.Tables.baseyear_n.Title}} } \\\tabucline{}\rowfont[c]\bfseries
  	{{$tt := join_tables .Contract.Tables.baseconditions_n_2 .Contract.Tables.baseconditions_n_1 .Contract.Tables.baseconditions_n }} %chktex 25
  	{{with $t := translate "bg" $tt}}
	{{$t.Columns | column_sideways}} \\\tabucline{} \rowfont[]\bfseries %chktex 25
	{{range $t.Headers}}
	{{.|row}} \\\tabucline{}
	{{end}}
	{{range $t.Rows}}
	{{row .}} \\\tabucline{} %chktex 26
	{{end}}

	\bfseries {{average $t }} \\\tabucline{}
	\bfseries {{total $t}} \\\tabucline{}
{{end}}
\end{tabu}
\end{center}

\subsection{Изложената методика се прилага при предназначение и режим на експлоатация на обекта, съгласно посочените от в обследването за енергийната ефективност в Приложение 1 {-} Доклад и резюме от обследване за ЕЕ за сградата (вж. Приложение 1 ); на договора.}

\subsection{При доказано безспорно неизпълнение на инструкциите за поддържане и експлоатация на сградата от обекта се приема, че “коефициента на ефективност” е равен на 1 /едно/ и гарантираният резултат за съответната мониторингова година е постигнат.}
